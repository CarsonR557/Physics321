%% Generated by Sphinx.
\def\sphinxdocclass{report}
\documentclass[letterpaper,10pt,english]{sphinxmanual}
\ifdefined\pdfpxdimen
   \let\sphinxpxdimen\pdfpxdimen\else\newdimen\sphinxpxdimen
\fi \sphinxpxdimen=.75bp\relax

\PassOptionsToPackage{warn}{textcomp}
\usepackage[utf8]{inputenc}
\ifdefined\DeclareUnicodeCharacter
% support both utf8 and utf8x syntaxes
  \ifdefined\DeclareUnicodeCharacterAsOptional
    \def\sphinxDUC#1{\DeclareUnicodeCharacter{"#1}}
  \else
    \let\sphinxDUC\DeclareUnicodeCharacter
  \fi
  \sphinxDUC{00A0}{\nobreakspace}
  \sphinxDUC{2500}{\sphinxunichar{2500}}
  \sphinxDUC{2502}{\sphinxunichar{2502}}
  \sphinxDUC{2514}{\sphinxunichar{2514}}
  \sphinxDUC{251C}{\sphinxunichar{251C}}
  \sphinxDUC{2572}{\textbackslash}
\fi
\usepackage{cmap}
\usepackage[T1]{fontenc}
\usepackage{amsmath,amssymb,amstext}
\usepackage{babel}



\usepackage{times}
\expandafter\ifx\csname T@LGR\endcsname\relax
\else
% LGR was declared as font encoding
  \substitutefont{LGR}{\rmdefault}{cmr}
  \substitutefont{LGR}{\sfdefault}{cmss}
  \substitutefont{LGR}{\ttdefault}{cmtt}
\fi
\expandafter\ifx\csname T@X2\endcsname\relax
  \expandafter\ifx\csname T@T2A\endcsname\relax
  \else
  % T2A was declared as font encoding
    \substitutefont{T2A}{\rmdefault}{cmr}
    \substitutefont{T2A}{\sfdefault}{cmss}
    \substitutefont{T2A}{\ttdefault}{cmtt}
  \fi
\else
% X2 was declared as font encoding
  \substitutefont{X2}{\rmdefault}{cmr}
  \substitutefont{X2}{\sfdefault}{cmss}
  \substitutefont{X2}{\ttdefault}{cmtt}
\fi


\usepackage[Bjarne]{fncychap}
\usepackage[,numfigreset=1,mathnumfig]{sphinx}

\fvset{fontsize=\small}
\usepackage{geometry}


% Include hyperref last.
\usepackage{hyperref}
% Fix anchor placement for figures with captions.
\usepackage{hypcap}% it must be loaded after hyperref.
% Set up styles of URL: it should be placed after hyperref.
\urlstyle{same}


\usepackage{sphinxmessages}




\title{Classical Mechanics}
\date{Dec 16, 2020}
\release{}
\author{Morten Hjorth\sphinxhyphen{}Jensen}
\newcommand{\sphinxlogo}{\vbox{}}
\renewcommand{\releasename}{}
\makeindex
\begin{document}

\pagestyle{empty}
\sphinxmaketitle
\pagestyle{plain}
\sphinxtableofcontents
\pagestyle{normal}
\phantomsection\label{\detokenize{chapter1::doc}}








\sphinxstylestrong{\sphinxhref{http://mhjgit.github.io/info/doc/web/}{Morten Hjorth\sphinxhyphen{}Jensen}}, Department of Physics and Astronomy and National Superconducting Cyclotron Laboratory, Michigan State University, USA and Department of Physics, University of Oslo, Norway









Date: \sphinxstylestrong{Dec 16, 2020}

Copyright 1999\sphinxhyphen{}2020, \sphinxhref{http://mhjgit.github.io/info/doc/web/}{Morten Hjorth\sphinxhyphen{}Jensen}. Released under CC Attribution\sphinxhyphen{}NonCommercial 4.0 license


\chapter{Introduction}
\label{\detokenize{chapter1:introduction}}
Classical mechanics is a topic which has been taught intensively over
several centuries. It is, with its many variants and ways of
presenting the educational material, normally the first \sphinxstylestrong{real} physics
course many of us meet and it lays the foundation for further physics
studies. Many of the equations and ways of reasoning about the
underlying laws of motion and pertinent forces, shape our approaches and understanding
of the scientific method and discourse, as well as the way we develop our insights
and deeper understanding about physical systems.

There is a wealth of
well\sphinxhyphen{}tested (from both a physics point of view and a pedagogical
standpoint) exercises and problems which can be solved
analytically. However, many of these problems represent idealized and
less realistic situations.  The large majority of these problems are
solved by paper and pencil and are traditionally aimed
at what we normally refer to as continuous models from which we may find an analytical solution.  As a consequence,
when teaching mechanics, it implies that we can seldomly venture beyond an idealized case
in order to develop our understandings and insights about the
underlying forces and laws of motion.

On the other hand, numerical algorithms call for approximate discrete
models and much of the development of methods for continuous models
are nowadays being replaced by methods for discrete models in science and
industry, simply because \sphinxstylestrong{much larger classes of problems can be addressed} with discrete models, often  by simpler and more
generic methodologies.

As we will see below, when properly scaling the equations at hand,
discrete models open up for more advanced abstractions and the possibility to
study  real life systems, with the added bonus that we can explore and
deepen our basic understanding of various physical systems

Analytical solutions are as important as before. In addition, such
solutions provide us with invaluable benchmarks and tests for our
discrete models. Such benchmarks, as we will see below, allow us
to discuss possible sources of errors and their behaviors.  And
finally, since most of our models are based on various algorithms from
numerical mathematics, we have a unique oppotunity to gain a deeper
understanding of the mathematical approaches we are using.

With computing and data science as important elements in essentially
all aspects of a modern society, we could  then try to define Computing as
\sphinxstylestrong{solving scientific problems using all possible tools, including
symbolic computing, computers and numerical algorithms, and analytical
paper and pencil solutions}.
Computing provides us with the tools to develope our own understanding of the scientific method by enhancing algorithmic thinking.

The way we will teach this course reflects
this definition of computing. The course contains both classical paper
and pencil exercises as well as  computational projects and exercises. The
hope is that this will allow you to explore the physics of systems
governed by the degrees of freedom of classical mechanics at a deeper
level, and that these insights about the scientific method will help
you to develop a better understanding of how the underlying forces and
equations of motion and how they impact a given system. Furthermore, by introducing various numerical methods
via computational projects and exercises, we aim at developing your competences and skills about these topics.

These competences will enable you to
\begin{itemize}
\item {} 
understand how algorithms are used to solve mathematical problems,

\item {} 
derive, verify, and implement algorithms,

\item {} 
understand what can go wrong with algorithms,

\item {} 
use these algorithms to construct reproducible scientific outcomes and to engage in science in ethical ways, and

\item {} 
think algorithmically for the purposes of gaining deeper insights about scientific problems.

\end{itemize}

All these elements are central for maturing and gaining a better understanding of the modern scientific process \sphinxstyleemphasis{per se}.

The power of the scientific method lies in identifying a given problem
as a special case of an abstract class of problems, identifying
general solution methods for this class of problems, and applying a
general method to the specific problem (applying means, in the case of
computing, calculations by pen and paper, symbolic computing, or
numerical computing by ready\sphinxhyphen{}made and/or self\sphinxhyphen{}written software). This
generic view on problems and methods is particularly important for
understanding how to apply available, generic software to solve a
particular problem.

\sphinxstyleemphasis{However, verification of algorithms and understanding their limitations requires much of the classical knowledge about continuous models.}


\section{A well\sphinxhyphen{}known examples to illustrate many of the above concepts}
\label{\detokenize{chapter1:a-well-known-examples-to-illustrate-many-of-the-above-concepts}}
Before we venture into a reminder on Python and mechanics relevant applications, let us briefly outline some of the
abovementioned topics using an example many of you may have seen before in for example CMSE201.
A simple algorithm for integration is the Trapezoidal rule.
Integration of a function \(f(x)\) by the Trapezoidal Rule is given by following algorithm for an interval \(x \in [a,b]\)
\begin{equation*}
\begin{split}
\int_a^b(f(x) dx = \frac{1}{2}\left [f(a)+2f(a+h)+\dots+2f(b-h)+f(b)\right] +O(h^2),
\end{split}
\end{equation*}
where \(h\) is the so\sphinxhyphen{}called stepsize defined by the number of integration points \(N\) as \(h=(b-a)/(n)\).
Python offers an  extremely versatile programming  environment, allowing for
the inclusion of analytical studies in a numerical program. Here we show an
example code with the \sphinxstylestrong{trapezoidal rule}. We use also \sphinxstylestrong{SymPy} to evaluate the exact value of the integral and compute the absolute error
with respect to the numerically evaluated one of the integral
\(\int_0^1 dx x^2 = 1/3\).
The following code for  the trapezoidal rule allows you  to plot the relative error by comparing with the exact result. By increasing to \(10^8\) points one arrives at a region where numerical errors start to accumulate.

\begin{sphinxVerbatim}[commandchars=\\\{\}]
\PYG{o}{\PYGZpc{}}\PYG{k}{matplotlib} inline

\PYG{k+kn}{from} \PYG{n+nn}{math} \PYG{k+kn}{import} \PYG{n}{log10}
\PYG{k+kn}{import} \PYG{n+nn}{numpy} \PYG{k}{as} \PYG{n+nn}{np}
\PYG{k+kn}{from} \PYG{n+nn}{sympy} \PYG{k+kn}{import} \PYG{n}{Symbol}\PYG{p}{,} \PYG{n}{integrate}
\PYG{k+kn}{import} \PYG{n+nn}{matplotlib}\PYG{n+nn}{.}\PYG{n+nn}{pyplot} \PYG{k}{as} \PYG{n+nn}{plt}
\PYG{c+c1}{\PYGZsh{} function for the trapezoidal rule}
\PYG{k}{def} \PYG{n+nf}{Trapez}\PYG{p}{(}\PYG{n}{a}\PYG{p}{,}\PYG{n}{b}\PYG{p}{,}\PYG{n}{f}\PYG{p}{,}\PYG{n}{n}\PYG{p}{)}\PYG{p}{:}
   \PYG{n}{h} \PYG{o}{=} \PYG{p}{(}\PYG{n}{b}\PYG{o}{\PYGZhy{}}\PYG{n}{a}\PYG{p}{)}\PYG{o}{/}\PYG{n+nb}{float}\PYG{p}{(}\PYG{n}{n}\PYG{p}{)}
   \PYG{n}{s} \PYG{o}{=} \PYG{l+m+mi}{0}
   \PYG{n}{x} \PYG{o}{=} \PYG{n}{a}
   \PYG{k}{for} \PYG{n}{i} \PYG{o+ow}{in} \PYG{n+nb}{range}\PYG{p}{(}\PYG{l+m+mi}{1}\PYG{p}{,}\PYG{n}{n}\PYG{p}{,}\PYG{l+m+mi}{1}\PYG{p}{)}\PYG{p}{:}
       \PYG{n}{x} \PYG{o}{=} \PYG{n}{x}\PYG{o}{+}\PYG{n}{h}
       \PYG{n}{s} \PYG{o}{=} \PYG{n}{s}\PYG{o}{+} \PYG{n}{f}\PYG{p}{(}\PYG{n}{x}\PYG{p}{)}
   \PYG{n}{s} \PYG{o}{=} \PYG{l+m+mf}{0.5}\PYG{o}{*}\PYG{p}{(}\PYG{n}{f}\PYG{p}{(}\PYG{n}{a}\PYG{p}{)}\PYG{o}{+}\PYG{n}{f}\PYG{p}{(}\PYG{n}{b}\PYG{p}{)}\PYG{p}{)} \PYG{o}{+}\PYG{n}{s}
   \PYG{k}{return} \PYG{n}{h}\PYG{o}{*}\PYG{n}{s}
\PYG{c+c1}{\PYGZsh{}  function to compute pi}
\PYG{k}{def} \PYG{n+nf}{function}\PYG{p}{(}\PYG{n}{x}\PYG{p}{)}\PYG{p}{:}
    \PYG{k}{return} \PYG{n}{x}\PYG{o}{*}\PYG{n}{x}
\PYG{c+c1}{\PYGZsh{} define integration limits}
\PYG{n}{a} \PYG{o}{=} \PYG{l+m+mf}{0.0}\PYG{p}{;}  \PYG{n}{b} \PYG{o}{=} \PYG{l+m+mf}{1.0}\PYG{p}{;}
\PYG{c+c1}{\PYGZsh{} find result from sympy}
\PYG{c+c1}{\PYGZsh{} define x as a symbol to be used by sympy}
\PYG{n}{x} \PYG{o}{=} \PYG{n}{Symbol}\PYG{p}{(}\PYG{l+s+s1}{\PYGZsq{}}\PYG{l+s+s1}{x}\PYG{l+s+s1}{\PYGZsq{}}\PYG{p}{)}
\PYG{n}{exact} \PYG{o}{=} \PYG{n}{integrate}\PYG{p}{(}\PYG{n}{function}\PYG{p}{(}\PYG{n}{x}\PYG{p}{)}\PYG{p}{,} \PYG{p}{(}\PYG{n}{x}\PYG{p}{,} \PYG{n}{a}\PYG{p}{,} \PYG{n}{b}\PYG{p}{)}\PYG{p}{)}
\PYG{c+c1}{\PYGZsh{} set up the arrays for plotting the relative error}
\PYG{n}{n} \PYG{o}{=} \PYG{n}{np}\PYG{o}{.}\PYG{n}{zeros}\PYG{p}{(}\PYG{l+m+mi}{9}\PYG{p}{)}\PYG{p}{;} \PYG{n}{y} \PYG{o}{=} \PYG{n}{np}\PYG{o}{.}\PYG{n}{zeros}\PYG{p}{(}\PYG{l+m+mi}{9}\PYG{p}{)}\PYG{p}{;}
\PYG{c+c1}{\PYGZsh{} find the relative error as function of integration points}
\PYG{k}{for} \PYG{n}{i} \PYG{o+ow}{in} \PYG{n+nb}{range}\PYG{p}{(}\PYG{l+m+mi}{1}\PYG{p}{,} \PYG{l+m+mi}{8}\PYG{p}{,} \PYG{l+m+mi}{1}\PYG{p}{)}\PYG{p}{:}
    \PYG{n}{npts} \PYG{o}{=} \PYG{l+m+mi}{10}\PYG{o}{*}\PYG{o}{*}\PYG{n}{i}
    \PYG{n}{result} \PYG{o}{=} \PYG{n}{Trapez}\PYG{p}{(}\PYG{n}{a}\PYG{p}{,}\PYG{n}{b}\PYG{p}{,}\PYG{n}{function}\PYG{p}{,}\PYG{n}{npts}\PYG{p}{)}
    \PYG{n}{RelativeError} \PYG{o}{=} \PYG{n+nb}{abs}\PYG{p}{(}\PYG{p}{(}\PYG{n}{exact}\PYG{o}{\PYGZhy{}}\PYG{n}{result}\PYG{p}{)}\PYG{o}{/}\PYG{n}{exact}\PYG{p}{)}
    \PYG{n}{n}\PYG{p}{[}\PYG{n}{i}\PYG{p}{]} \PYG{o}{=} \PYG{n}{log10}\PYG{p}{(}\PYG{n}{npts}\PYG{p}{)}\PYG{p}{;} \PYG{n}{y}\PYG{p}{[}\PYG{n}{i}\PYG{p}{]} \PYG{o}{=} \PYG{n}{log10}\PYG{p}{(}\PYG{n}{RelativeError}\PYG{p}{)}\PYG{p}{;}
\PYG{n}{plt}\PYG{o}{.}\PYG{n}{plot}\PYG{p}{(}\PYG{n}{n}\PYG{p}{,}\PYG{n}{y}\PYG{p}{,} \PYG{l+s+s1}{\PYGZsq{}}\PYG{l+s+s1}{ro}\PYG{l+s+s1}{\PYGZsq{}}\PYG{p}{)}
\PYG{n}{plt}\PYG{o}{.}\PYG{n}{xlabel}\PYG{p}{(}\PYG{l+s+s1}{\PYGZsq{}}\PYG{l+s+s1}{n}\PYG{l+s+s1}{\PYGZsq{}}\PYG{p}{)}
\PYG{n}{plt}\PYG{o}{.}\PYG{n}{ylabel}\PYG{p}{(}\PYG{l+s+s1}{\PYGZsq{}}\PYG{l+s+s1}{Relative error}\PYG{l+s+s1}{\PYGZsq{}}\PYG{p}{)}
\PYG{n}{plt}\PYG{o}{.}\PYG{n}{show}\PYG{p}{(}\PYG{p}{)}
\end{sphinxVerbatim}

\noindent\sphinxincludegraphics{{chapter1_3_0}.png}

This example shows the potential of combining numerical algorithms with symbolic calculations, allowing us to
\begin{itemize}
\item {} 
Validate and verify  their  algorithms.

\item {} 
Including concepts like unit testing, one has the possibility to test and test several or all parts of the code.

\item {} 
Validation and verification are then included \sphinxstyleemphasis{naturally} and one can develop a better attitude to what is meant with an ethically sound scientific approach.

\item {} 
The above example allows the student to also test the mathematical error of the algorithm for the trapezoidal rule by changing the number of integration points. The students get \sphinxstylestrong{trained from day one to think error analysis}.

\item {} 
With a Jupyter notebook you can keep exploring similar examples and turn them in as your own notebooks.

\end{itemize}

In this process we can easily bake in
\begin{enumerate}
\sphinxsetlistlabels{\arabic}{enumi}{enumii}{}{.}%
\item {} 
How to structure a code in terms of functions

\item {} 
How to make a module

\item {} 
How to read input data flexibly from the command line

\item {} 
How to create graphical/web user interfaces

\item {} 
How to write unit tests (test functions or doctests)

\item {} 
How to refactor code in terms of classes (instead of functions only)

\item {} 
How to conduct and automate large\sphinxhyphen{}scale numerical experiments

\item {} 
How to write scientific reports in various formats (LaTeX, HTML)

\end{enumerate}

The conventions and techniques outlined here will save you a lot of time when you incrementally extend software over time from simpler to more complicated problems. In particular, you will benefit from many good habits:
\begin{enumerate}
\sphinxsetlistlabels{\arabic}{enumi}{enumii}{}{.}%
\item {} 
New code is added in a modular fashion to a library (modules)

\item {} 
Programs are run through convenient user interfaces

\item {} 
It takes one quick command to let all your code undergo heavy testing

\item {} 
Tedious manual work with running programs is automated,

\item {} 
Your scientific investigations are reproducible, scientific reports with top quality typesetting are produced both for paper and electronic devices.

\end{enumerate}


\chapter{Space, Time, Motion, Reference Frames  and Reminder on vectors and other mathematical quantities}
\label{\detokenize{chapter1:space-time-motion-reference-frames-and-reminder-on-vectors-and-other-mathematical-quantities}}
Our studies will start with the motion of different types of objects
such as a falling ball, a runner, a bicycle etc etc. It means that an
object’s position in space varies with time.
In order to study such systems we need to define
\begin{itemize}
\item {} 
choice of origin

\item {} 
choice of the direction of the axes

\item {} 
choice of positive direction (left\sphinxhyphen{}handed or right\sphinxhyphen{}handed system of reference)

\item {} 
choice of units and dimensions

\end{itemize}

These choices lead to some important questions such as
\begin{itemize}
\item {} 
is the  physics of a system independent of the origin of the axes?

\item {} 
is the  physics independent of the directions of the axes, that is are there privileged axes?

\item {} 
is the physics independent of the orientation of system?

\item {} 
is the physics independent of the scale of the length?

\end{itemize}


\section{Dimension, units and labels}
\label{\detokenize{chapter1:dimension-units-and-labels}}
Throughout this course we will use the standardized SI units. The standard unit for length is thus one meter 1m, for mass
one kilogram 1kg, for time one second 1s, for force one Newton 1kgm/s\(^2\) and for energy 1 Joule 1kgm\(^2\)s\(^{-2}\).

We will use the following notations for various variables (vectors are always boldfaced in these lecture notes):
\begin{itemize}
\item {} 
position \(\boldsymbol{r}\), in one dimention we will normally just use \(x\),

\item {} 
mass \(m\),

\item {} 
time \(t\),

\item {} 
velocity \(\boldsymbol{v}\) or just \(v\) in one dimension,

\item {} 
acceleration \(\boldsymbol{a}\) or just \(a\) in one dimension,

\item {} 
momentum \(\boldsymbol{p}\) or just \(p\) in one dimension,

\item {} 
kinetic energy \(K\),

\item {} 
potential energy \(V\) and

\item {} 
frequency \(\omega\).

\end{itemize}

More variables will be defined as we need them.

It is also important to keep track of dimensionalities. Don’t mix this up with a chosen unit for a given variable. We mark the dimensionality in these lectures as \([a]\), where \(a\) is the quantity we are interested in. Thus
\begin{itemize}
\item {} 
\([\boldsymbol{r}]=\) length

\item {} 
\([m]=\) mass

\item {} 
\([K]=\) energy

\item {} 
\([t]=\) time

\item {} 
\([\boldsymbol{v}]=\) length over time

\item {} 
\([\boldsymbol{a}]=\) length over time squared

\item {} 
\([\boldsymbol{p}]=\) mass times length over time

\item {} 
\([\omega]=\) 1/time

\end{itemize}


\section{Elements of Vector Algebra}
\label{\detokenize{chapter1:elements-of-vector-algebra}}
\sphinxstylestrong{Note}: This section is under revision

In these lectures we will use boldfaced lower\sphinxhyphen{}case letters to label a vector. A vector \(\boldsymbol{a}\) in three dimensions is thus defined as
\begin{equation*}
\begin{split}
\boldsymbol{a} =(a_x,a_y, a_z),
\end{split}
\end{equation*}
and using the unit vectors in a cartesian system we have
\begin{equation*}
\begin{split}
\boldsymbol{a} = a_x\boldsymbol{e}_x+a_y\boldsymbol{e}_y+a_z\boldsymbol{e}_z,
\end{split}
\end{equation*}
where the unit vectors have magnitude \(\vert\boldsymbol{e}_i\vert = 1\) with \(i=x,y,z\).

Using the fact that multiplication of reals is distributive we can show that
\begin{equation*}
\begin{split}
\boldsymbol{a}(\boldsymbol{b}+\boldsymbol{c})=\boldsymbol{a}\boldsymbol{b}+\boldsymbol{a}\boldsymbol{c},
\end{split}
\end{equation*}
Similarly we can also show that (using product rule for differentiating reals)
\begin{equation*}
\begin{split}
\frac{d}{dt}(\boldsymbol{a}\boldsymbol{b})=\boldsymbol{a}\frac{d\boldsymbol{b}}{dt}+\boldsymbol{b}\frac{d\boldsymbol{a}}{dt}.
\end{split}
\end{equation*}
We can repeat these operations for the cross products and show that they are distribuitive
\begin{equation*}
\begin{split}
\boldsymbol{a}\times(\boldsymbol{b}+\boldsymbol{c})=\boldsymbol{a}\times\boldsymbol{b}+\boldsymbol{a}\times\boldsymbol{c}.
\end{split}
\end{equation*}
We have also that
\begin{equation*}
\begin{split}
\frac{d}{dt}(\boldsymbol{a}\times\boldsymbol{b})=\boldsymbol{a}\times\frac{d\boldsymbol{b}}{dt}+\boldsymbol{b}\times\frac{d\boldsymbol{a}}{dt}.
\end{split}
\end{equation*}
The rotation of a three\sphinxhyphen{}dimensional  vector \(\boldsymbol{a}=(a_x,a_y,a_z)\) in the \(xy\) plane around an angle \(\phi\) results in a new vector \(\boldsymbol{b}=(b_x,b_y,b_z)\).  This operation can be expressed in terms of linear algebra as a matrix (the rotation matrix) multiplied with a vector. We can write this as
\begin{equation*}
\begin{split}
\begin{bmatrix} b_x \\ b_y \\ b_z \end{bmatrix} = \begin{bmatrix} \cos{\phi} & \sin{\phi} & 0 \\ -\sin{\phi} & \cos{\phi} & 0 \\ 0 & 0 & 1\end{bmatrix}\begin{bmatrix} a_x \\ a_y \\ a_z \end{bmatrix}.
\end{split}
\end{equation*}
We can write this in a more compact form as \(\boldsymbol{b} = \boldsymbol{R}\boldsymbol{a}\), where the rotation matrix is defined as
\begin{equation*}
\begin{split}
\boldsymbol{R} = \begin{bmatrix} \cos{\phi} & \sin{\phi} & 0 \\ -\sin{\phi} & \cos{\phi} & 0 \\ 0 & 0 & 1\end{bmatrix}.
\end{split}
\end{equation*}

\section{Falling baseball in one dimension}
\label{\detokenize{chapter1:falling-baseball-in-one-dimension}}
We anticipate the mathematical model to come and assume that we have a
model for the motion of a falling baseball without air resistance.
Our system (the baseball) is at an initial height \(y_0\) (which we will
specify in the program below) at the initial time \(t_0=0\). In our program example here we will plot the position in steps of \(\Delta t\) up to a final time \(t_f\).
The mathematical formula for the position \(y(t)\) as function of time \(t\) is
\begin{equation*}
\begin{split}
y(t) = y_0-\frac{1}{2}gt^2,
\end{split}
\end{equation*}
where \(g=9.80665=0.980655\times 10^1\)m/s\({}^2\) is a constant representing the standard acceleration due to gravity.
We have here adopted the conventional standard value. This does not take into account other effects, such as buoyancy or drag.
Furthermore, we stop when the ball hits the ground, which takes place at
\begin{equation*}
\begin{split}
y(t) = 0= y_0-\frac{1}{2}gt^2,
\end{split}
\end{equation*}
which gives us a final time \(t_f=\sqrt{2y_0/g}\).

As of now we simply assume that   we know the formula for the falling object. Afterwards, we will derive it.


\section{Our Python Encounter}
\label{\detokenize{chapter1:our-python-encounter}}
We start with preparing folders for storing our calculations, figures and if needed, specific data files we use as input or output files.

\begin{sphinxVerbatim}[commandchars=\\\{\}]
\PYG{c+c1}{\PYGZsh{} Common imports}
\PYG{k+kn}{import} \PYG{n+nn}{numpy} \PYG{k}{as} \PYG{n+nn}{np}
\PYG{k+kn}{import} \PYG{n+nn}{pandas} \PYG{k}{as} \PYG{n+nn}{pd}
\PYG{k+kn}{import} \PYG{n+nn}{matplotlib}\PYG{n+nn}{.}\PYG{n+nn}{pyplot} \PYG{k}{as} \PYG{n+nn}{plt}
\PYG{k+kn}{import} \PYG{n+nn}{os}

\PYG{c+c1}{\PYGZsh{} Where to save the figures and data files}
\PYG{n}{PROJECT\PYGZus{}ROOT\PYGZus{}DIR} \PYG{o}{=} \PYG{l+s+s2}{\PYGZdq{}}\PYG{l+s+s2}{Results}\PYG{l+s+s2}{\PYGZdq{}}
\PYG{n}{FIGURE\PYGZus{}ID} \PYG{o}{=} \PYG{l+s+s2}{\PYGZdq{}}\PYG{l+s+s2}{Results/FigureFiles}\PYG{l+s+s2}{\PYGZdq{}}
\PYG{n}{DATA\PYGZus{}ID} \PYG{o}{=} \PYG{l+s+s2}{\PYGZdq{}}\PYG{l+s+s2}{DataFiles/}\PYG{l+s+s2}{\PYGZdq{}}

\PYG{k}{if} \PYG{o+ow}{not} \PYG{n}{os}\PYG{o}{.}\PYG{n}{path}\PYG{o}{.}\PYG{n}{exists}\PYG{p}{(}\PYG{n}{PROJECT\PYGZus{}ROOT\PYGZus{}DIR}\PYG{p}{)}\PYG{p}{:}
    \PYG{n}{os}\PYG{o}{.}\PYG{n}{mkdir}\PYG{p}{(}\PYG{n}{PROJECT\PYGZus{}ROOT\PYGZus{}DIR}\PYG{p}{)}

\PYG{k}{if} \PYG{o+ow}{not} \PYG{n}{os}\PYG{o}{.}\PYG{n}{path}\PYG{o}{.}\PYG{n}{exists}\PYG{p}{(}\PYG{n}{FIGURE\PYGZus{}ID}\PYG{p}{)}\PYG{p}{:}
    \PYG{n}{os}\PYG{o}{.}\PYG{n}{makedirs}\PYG{p}{(}\PYG{n}{FIGURE\PYGZus{}ID}\PYG{p}{)}

\PYG{k}{if} \PYG{o+ow}{not} \PYG{n}{os}\PYG{o}{.}\PYG{n}{path}\PYG{o}{.}\PYG{n}{exists}\PYG{p}{(}\PYG{n}{DATA\PYGZus{}ID}\PYG{p}{)}\PYG{p}{:}
    \PYG{n}{os}\PYG{o}{.}\PYG{n}{makedirs}\PYG{p}{(}\PYG{n}{DATA\PYGZus{}ID}\PYG{p}{)}

\PYG{k}{def} \PYG{n+nf}{image\PYGZus{}path}\PYG{p}{(}\PYG{n}{fig\PYGZus{}id}\PYG{p}{)}\PYG{p}{:}
    \PYG{k}{return} \PYG{n}{os}\PYG{o}{.}\PYG{n}{path}\PYG{o}{.}\PYG{n}{join}\PYG{p}{(}\PYG{n}{FIGURE\PYGZus{}ID}\PYG{p}{,} \PYG{n}{fig\PYGZus{}id}\PYG{p}{)}

\PYG{k}{def} \PYG{n+nf}{data\PYGZus{}path}\PYG{p}{(}\PYG{n}{dat\PYGZus{}id}\PYG{p}{)}\PYG{p}{:}
    \PYG{k}{return} \PYG{n}{os}\PYG{o}{.}\PYG{n}{path}\PYG{o}{.}\PYG{n}{join}\PYG{p}{(}\PYG{n}{DATA\PYGZus{}ID}\PYG{p}{,} \PYG{n}{dat\PYGZus{}id}\PYG{p}{)}

\PYG{k}{def} \PYG{n+nf}{save\PYGZus{}fig}\PYG{p}{(}\PYG{n}{fig\PYGZus{}id}\PYG{p}{)}\PYG{p}{:}
    \PYG{n}{plt}\PYG{o}{.}\PYG{n}{savefig}\PYG{p}{(}\PYG{n}{image\PYGZus{}path}\PYG{p}{(}\PYG{n}{fig\PYGZus{}id}\PYG{p}{)} \PYG{o}{+} \PYG{l+s+s2}{\PYGZdq{}}\PYG{l+s+s2}{.png}\PYG{l+s+s2}{\PYGZdq{}}\PYG{p}{,} \PYG{n+nb}{format}\PYG{o}{=}\PYG{l+s+s1}{\PYGZsq{}}\PYG{l+s+s1}{png}\PYG{l+s+s1}{\PYGZsq{}}\PYG{p}{)}

\PYG{c+c1}{\PYGZsh{}in case we have an input file we wish to read in}
\PYG{c+c1}{\PYGZsh{}infile = open(data\PYGZus{}path(\PYGZdq{}MassEval2016.dat\PYGZdq{}),\PYGZsq{}r\PYGZsq{})}
\end{sphinxVerbatim}

You could also define a function for making our plots. You
can obviously avoid this and simply set up various \sphinxstylestrong{matplotlib}
commands every time you need them. You may however find it convenient
to collect all such commands in one function and simply call this
function.

\begin{sphinxVerbatim}[commandchars=\\\{\}]
\PYG{k+kn}{from} \PYG{n+nn}{pylab} \PYG{k+kn}{import} \PYG{n}{plt}\PYG{p}{,} \PYG{n}{mpl}
\PYG{n}{plt}\PYG{o}{.}\PYG{n}{style}\PYG{o}{.}\PYG{n}{use}\PYG{p}{(}\PYG{l+s+s1}{\PYGZsq{}}\PYG{l+s+s1}{seaborn}\PYG{l+s+s1}{\PYGZsq{}}\PYG{p}{)}
\PYG{n}{mpl}\PYG{o}{.}\PYG{n}{rcParams}\PYG{p}{[}\PYG{l+s+s1}{\PYGZsq{}}\PYG{l+s+s1}{font.family}\PYG{l+s+s1}{\PYGZsq{}}\PYG{p}{]} \PYG{o}{=} \PYG{l+s+s1}{\PYGZsq{}}\PYG{l+s+s1}{serif}\PYG{l+s+s1}{\PYGZsq{}}

\PYG{k}{def} \PYG{n+nf}{MakePlot}\PYG{p}{(}\PYG{n}{x}\PYG{p}{,}\PYG{n}{y}\PYG{p}{,} \PYG{n}{styles}\PYG{p}{,} \PYG{n}{labels}\PYG{p}{,} \PYG{n}{axlabels}\PYG{p}{)}\PYG{p}{:}
    \PYG{n}{plt}\PYG{o}{.}\PYG{n}{figure}\PYG{p}{(}\PYG{n}{figsize}\PYG{o}{=}\PYG{p}{(}\PYG{l+m+mi}{10}\PYG{p}{,}\PYG{l+m+mi}{6}\PYG{p}{)}\PYG{p}{)}
    \PYG{k}{for} \PYG{n}{i} \PYG{o+ow}{in} \PYG{n+nb}{range}\PYG{p}{(}\PYG{n+nb}{len}\PYG{p}{(}\PYG{n}{x}\PYG{p}{)}\PYG{p}{)}\PYG{p}{:}
        \PYG{n}{plt}\PYG{o}{.}\PYG{n}{plot}\PYG{p}{(}\PYG{n}{x}\PYG{p}{[}\PYG{n}{i}\PYG{p}{]}\PYG{p}{,} \PYG{n}{y}\PYG{p}{[}\PYG{n}{i}\PYG{p}{]}\PYG{p}{,} \PYG{n}{styles}\PYG{p}{[}\PYG{n}{i}\PYG{p}{]}\PYG{p}{,} \PYG{n}{label} \PYG{o}{=} \PYG{n}{labels}\PYG{p}{[}\PYG{n}{i}\PYG{p}{]}\PYG{p}{)}
        \PYG{n}{plt}\PYG{o}{.}\PYG{n}{xlabel}\PYG{p}{(}\PYG{n}{axlabels}\PYG{p}{[}\PYG{l+m+mi}{0}\PYG{p}{]}\PYG{p}{)}
        \PYG{n}{plt}\PYG{o}{.}\PYG{n}{ylabel}\PYG{p}{(}\PYG{n}{axlabels}\PYG{p}{[}\PYG{l+m+mi}{1}\PYG{p}{]}\PYG{p}{)}
    \PYG{n}{plt}\PYG{o}{.}\PYG{n}{legend}\PYG{p}{(}\PYG{n}{loc}\PYG{o}{=}\PYG{l+m+mi}{0}\PYG{p}{)}
\end{sphinxVerbatim}

Thereafter we start setting up the code for the falling object.

\begin{sphinxVerbatim}[commandchars=\\\{\}]
\PYG{o}{\PYGZpc{}}\PYG{k}{matplotlib} inline
\PYG{k+kn}{import} \PYG{n+nn}{matplotlib}\PYG{n+nn}{.}\PYG{n+nn}{patches} \PYG{k}{as} \PYG{n+nn}{mpatches}

\PYG{n}{g} \PYG{o}{=} \PYG{l+m+mf}{9.80655} \PYG{c+c1}{\PYGZsh{}m/s\PYGZca{}2}
\PYG{n}{y\PYGZus{}0} \PYG{o}{=} \PYG{l+m+mf}{10.0} \PYG{c+c1}{\PYGZsh{} initial position in meters}
\PYG{n}{DeltaT} \PYG{o}{=} \PYG{l+m+mf}{0.1}  \PYG{c+c1}{\PYGZsh{} time step}
\PYG{c+c1}{\PYGZsh{} final time when y = 0, t = sqrt(2*10/g)}
\PYG{n}{tfinal} \PYG{o}{=} \PYG{n}{np}\PYG{o}{.}\PYG{n}{sqrt}\PYG{p}{(}\PYG{l+m+mf}{2.0}\PYG{o}{*}\PYG{n}{y\PYGZus{}0}\PYG{o}{/}\PYG{n}{g}\PYG{p}{)}
\PYG{c+c1}{\PYGZsh{}set up arrays }
\PYG{n}{t} \PYG{o}{=} \PYG{n}{np}\PYG{o}{.}\PYG{n}{arange}\PYG{p}{(}\PYG{l+m+mi}{0}\PYG{p}{,}\PYG{n}{tfinal}\PYG{p}{,}\PYG{n}{DeltaT}\PYG{p}{)}
\PYG{n}{y} \PYG{o}{=}\PYG{n}{y\PYGZus{}0} \PYG{o}{\PYGZhy{}}\PYG{n}{g}\PYG{o}{*}\PYG{o}{.}\PYG{l+m+mi}{5}\PYG{o}{*}\PYG{n}{t}\PYG{o}{*}\PYG{o}{*}\PYG{l+m+mi}{2}
\PYG{c+c1}{\PYGZsh{} Then make a nice printout in table form using Pandas}
\PYG{k+kn}{import} \PYG{n+nn}{pandas} \PYG{k}{as} \PYG{n+nn}{pd}
\PYG{k+kn}{from} \PYG{n+nn}{IPython}\PYG{n+nn}{.}\PYG{n+nn}{display} \PYG{k+kn}{import} \PYG{n}{display}
\PYG{n}{data} \PYG{o}{=} \PYG{p}{\PYGZob{}}\PYG{l+s+s1}{\PYGZsq{}}\PYG{l+s+s1}{t[s]}\PYG{l+s+s1}{\PYGZsq{}}\PYG{p}{:} \PYG{n}{t}\PYG{p}{,}
        \PYG{l+s+s1}{\PYGZsq{}}\PYG{l+s+s1}{y[m]}\PYG{l+s+s1}{\PYGZsq{}}\PYG{p}{:} \PYG{n}{y}
        \PYG{p}{\PYGZcb{}}
\PYG{n}{RawData} \PYG{o}{=} \PYG{n}{pd}\PYG{o}{.}\PYG{n}{DataFrame}\PYG{p}{(}\PYG{n}{data}\PYG{p}{)}
\PYG{n}{display}\PYG{p}{(}\PYG{n}{RawData}\PYG{p}{)}
\PYG{n}{plt}\PYG{o}{.}\PYG{n}{style}\PYG{o}{.}\PYG{n}{use}\PYG{p}{(}\PYG{l+s+s1}{\PYGZsq{}}\PYG{l+s+s1}{ggplot}\PYG{l+s+s1}{\PYGZsq{}}\PYG{p}{)}
\PYG{n}{plt}\PYG{o}{.}\PYG{n}{figure}\PYG{p}{(}\PYG{n}{figsize}\PYG{o}{=}\PYG{p}{(}\PYG{l+m+mi}{8}\PYG{p}{,}\PYG{l+m+mi}{8}\PYG{p}{)}\PYG{p}{)}
\PYG{n}{plt}\PYG{o}{.}\PYG{n}{scatter}\PYG{p}{(}\PYG{n}{t}\PYG{p}{,} \PYG{n}{y}\PYG{p}{,} \PYG{n}{color} \PYG{o}{=} \PYG{l+s+s1}{\PYGZsq{}}\PYG{l+s+s1}{b}\PYG{l+s+s1}{\PYGZsq{}}\PYG{p}{)}
\PYG{n}{blue\PYGZus{}patch} \PYG{o}{=} \PYG{n}{mpatches}\PYG{o}{.}\PYG{n}{Patch}\PYG{p}{(}\PYG{n}{color} \PYG{o}{=} \PYG{l+s+s1}{\PYGZsq{}}\PYG{l+s+s1}{b}\PYG{l+s+s1}{\PYGZsq{}}\PYG{p}{,} \PYG{n}{label} \PYG{o}{=} \PYG{l+s+s1}{\PYGZsq{}}\PYG{l+s+s1}{Height y as function of  time t}\PYG{l+s+s1}{\PYGZsq{}}\PYG{p}{)}
\PYG{n}{plt}\PYG{o}{.}\PYG{n}{legend}\PYG{p}{(}\PYG{n}{handles}\PYG{o}{=}\PYG{p}{[}\PYG{n}{blue\PYGZus{}patch}\PYG{p}{]}\PYG{p}{)}
\PYG{n}{plt}\PYG{o}{.}\PYG{n}{xlabel}\PYG{p}{(}\PYG{l+s+s2}{\PYGZdq{}}\PYG{l+s+s2}{t[s]}\PYG{l+s+s2}{\PYGZdq{}}\PYG{p}{)}
\PYG{n}{plt}\PYG{o}{.}\PYG{n}{ylabel}\PYG{p}{(}\PYG{l+s+s2}{\PYGZdq{}}\PYG{l+s+s2}{y[m]}\PYG{l+s+s2}{\PYGZdq{}}\PYG{p}{)}
\PYG{n}{save\PYGZus{}fig}\PYG{p}{(}\PYG{l+s+s2}{\PYGZdq{}}\PYG{l+s+s2}{FallingBaseball}\PYG{l+s+s2}{\PYGZdq{}}\PYG{p}{)}
\PYG{n}{plt}\PYG{o}{.}\PYG{n}{show}\PYG{p}{(}\PYG{p}{)}
\end{sphinxVerbatim}

\begin{sphinxVerbatim}[commandchars=\\\{\}]
    t[s]       y[m]
0    0.0  10.000000
1    0.1   9.950967
2    0.2   9.803869
3    0.3   9.558705
4    0.4   9.215476
5    0.5   8.774181
6    0.6   8.234821
7    0.7   7.597395
8    0.8   6.861904
9    0.9   6.028347
10   1.0   5.096725
11   1.1   4.067037
12   1.2   2.939284
13   1.3   1.713465
14   1.4   0.389581
\end{sphinxVerbatim}

\noindent\sphinxincludegraphics{{chapter1_29_1}.png}

Here we used \sphinxstylestrong{pandas} (see below) to systemize the output of the position as function of time.


\section{Average quantities}
\label{\detokenize{chapter1:average-quantities}}
We define now the average velocity as
\begin{equation*}
\begin{split}
\overline{v}(t) = \frac{y(t+\Delta t)-y(t)}{\Delta t}.
\end{split}
\end{equation*}
In the code we have set the time step \(\Delta t\) to a given value. We could define it in terms of the number of points \(n\) as
\begin{equation*}
\begin{split}
\Delta t = \frac{t_{\mathrm{final}-}t_{\mathrm{initial}}}{n+1}.
\end{split}
\end{equation*}
Since we have discretized the variables, we introduce the counter \(i\) and let \(y(t)\rightarrow y(t_i)=y_i\) and \(t\rightarrow t_i\)
with \(i=0,1,\dots, n\). This gives us the following shorthand notations that we will use for the rest of this course. We define
\begin{equation*}
\begin{split}
y_i = y(t_i),\hspace{0.2cm} i=0,1,2,\dots,n.
\end{split}
\end{equation*}
This applies to other variables which depend on say time. Examples are the velocities, accelerations, momenta etc.
Furthermore we use the shorthand
\begin{equation*}
\begin{split}
y_{i\pm 1} = y(t_i\pm \Delta t),\hspace{0.12cm} i=0,1,2,\dots,n.
\end{split}
\end{equation*}

\section{Compact equations}
\label{\detokenize{chapter1:compact-equations}}
We can then rewrite in a more compact form the average velocity as
\begin{equation*}
\begin{split}
\overline{v}_i = \frac{y_{i+1}-y_{i}}{\Delta t}.
\end{split}
\end{equation*}
The velocity is defined as the change in position per unit time.
In the limit \(\Delta t \rightarrow 0\) this defines the instantaneous velocity, which is nothing but the slope of the position at a time \(t\).
We have thus
\begin{equation*}
\begin{split}
v(t) = \frac{dy}{dt}=\lim_{\Delta t \rightarrow 0}\frac{y(t+\Delta t)-y(t)}{\Delta t}.
\end{split}
\end{equation*}
Similarly, we can define the average acceleration as the change in velocity per unit time as
\begin{equation*}
\begin{split}
\overline{a}_i = \frac{v_{i+1}-v_{i}}{\Delta t},
\end{split}
\end{equation*}
resulting in the instantaneous acceleration
\begin{equation*}
\begin{split}
a(t) = \frac{dv}{dt}=\lim_{\Delta t\rightarrow 0}\frac{v(t+\Delta t)-v(t)}{\Delta t}.
\end{split}
\end{equation*}
\sphinxstylestrong{A note on notations}: When writing for example the velocity as \(v(t)\) we are then referring to the continuous and instantaneous value. A subscript like
\(v_i\) refers always to the discretized values.


\section{A differential equation}
\label{\detokenize{chapter1:a-differential-equation}}
We can rewrite the instantaneous acceleration as
\begin{equation*}
\begin{split}
a(t) = \frac{dv}{dt}=\frac{d}{dt}\frac{dy}{dt}=\frac{d^2y}{dt^2}.
\end{split}
\end{equation*}
This forms the starting point for our definition of forces later. It is a famous second\sphinxhyphen{}order differential equation. If the acceleration is constant we can now recover the formula for the falling ball we started with.
The acceleration can depend on the position and the velocity. To be more formal we should then write the above differential equation as
\begin{equation*}
\begin{split}
\frac{d^2y}{dt^2}=a(t,y(t),\frac{dy}{dt}).
\end{split}
\end{equation*}
With given initial conditions for \(y(t_0)\) and \(v(t_0)\) we can then
integrate the above equation and find the velocities and positions at
a given time \(t\).

If we multiply with mass, we have one of the famous expressions for Newton’s second law,
\begin{equation*}
\begin{split}
F(y,v,t)=m\frac{d^2y}{dt^2}=ma(t,y(t),\frac{dy}{dt}),
\end{split}
\end{equation*}
where \(F\) is the force acting on an object with mass \(m\). We see that it also has the right dimension, mass times length divided by time squared.
We will come back to this soon.


\section{Integrating our equations}
\label{\detokenize{chapter1:integrating-our-equations}}
Formally we can then, starting with the acceleration (suppose we have measured it, how could we do that?)
compute say the height of a building.  To see this we perform the following integrations from an initial time \(t_0\)  to a given time \(t\)
\begin{equation*}
\begin{split}
\int_{t_0}^t dt a(t) = \int_{t_0}^t dt \frac{dv}{dt} = v(t)-v(t_0),
\end{split}
\end{equation*}
or as
\begin{equation*}
\begin{split}
v(t)=v(t_0)+\int_{t_0}^t dt a(t).
\end{split}
\end{equation*}
When we know the velocity as function of time, we can find the position as function of time starting from the defintion of velocity as the derivative with respect to time, that is we have
\begin{equation*}
\begin{split}
\int_{t_0}^t dt v(t) = \int_{t_0}^t dt \frac{dy}{dt} = y(t)-y(t_0),
\end{split}
\end{equation*}
or as
\begin{equation*}
\begin{split}
y(t)=y(t_0)+\int_{t_0}^t dt v(t).
\end{split}
\end{equation*}
These equations define what is called the integration method for
finding the position and the velocity as functions of time. There is
no loss of generality if we extend these equations to more than one
spatial dimension.


\section{Constant acceleration case, the velocity}
\label{\detokenize{chapter1:constant-acceleration-case-the-velocity}}
Let us compute the velocity using the constant value for the acceleration given by \(-g\). We have
\begin{equation*}
\begin{split}
v(t)=v(t_0)+\int_{t_0}^t dt a(t)=v(t_0)+\int_{t_0}^t dt (-g).
\end{split}
\end{equation*}
Using our initial time as \(t_0=0\)s and setting the initial velocity \(v(t_0)=v_0=0\)m/s we get when integrating
\begin{equation*}
\begin{split}
v(t)=-gt.
\end{split}
\end{equation*}
The more general case is
\begin{equation*}
\begin{split}
v(t)=v_0-g(t-t_0).
\end{split}
\end{equation*}
We can then integrate the velocity and obtain the final formula for the position as function of time through
\begin{equation*}
\begin{split}
y(t)=y(t_0)+\int_{t_0}^t dt v(t)=y_0+\int_{t_0}^t dt v(t)=y_0+\int_{t_0}^t dt (-gt),
\end{split}
\end{equation*}
With \(y_0=10\)m and \(t_0=0\)s, we obtain the equation we started with
\begin{equation*}
\begin{split}
y(t)=10-\frac{1}{2}gt^2.
\end{split}
\end{equation*}

\section{Computing the averages}
\label{\detokenize{chapter1:computing-the-averages}}
After this mathematical background we are now ready to compute the mean velocity using our data.

\begin{sphinxVerbatim}[commandchars=\\\{\}]
\PYG{c+c1}{\PYGZsh{} Now we can compute the mean velocity using our data}
\PYG{c+c1}{\PYGZsh{} We define first an array Vaverage}
\PYG{n}{n} \PYG{o}{=} \PYG{n}{np}\PYG{o}{.}\PYG{n}{size}\PYG{p}{(}\PYG{n}{t}\PYG{p}{)}
\PYG{n}{Vaverage} \PYG{o}{=} \PYG{n}{np}\PYG{o}{.}\PYG{n}{zeros}\PYG{p}{(}\PYG{n}{n}\PYG{p}{)}
\PYG{k}{for} \PYG{n}{i} \PYG{o+ow}{in} \PYG{n+nb}{range}\PYG{p}{(}\PYG{l+m+mi}{1}\PYG{p}{,}\PYG{n}{n}\PYG{o}{\PYGZhy{}}\PYG{l+m+mi}{1}\PYG{p}{)}\PYG{p}{:}
    \PYG{n}{Vaverage}\PYG{p}{[}\PYG{n}{i}\PYG{p}{]} \PYG{o}{=} \PYG{p}{(}\PYG{n}{y}\PYG{p}{[}\PYG{n}{i}\PYG{o}{+}\PYG{l+m+mi}{1}\PYG{p}{]}\PYG{o}{\PYGZhy{}}\PYG{n}{y}\PYG{p}{[}\PYG{n}{i}\PYG{p}{]}\PYG{p}{)}\PYG{o}{/}\PYG{n}{DeltaT}
\PYG{c+c1}{\PYGZsh{} Now we can compute the mean accelearatio using our data}
\PYG{c+c1}{\PYGZsh{} We define first an array Aaverage}
\PYG{n}{n} \PYG{o}{=} \PYG{n}{np}\PYG{o}{.}\PYG{n}{size}\PYG{p}{(}\PYG{n}{t}\PYG{p}{)}
\PYG{n}{Aaverage} \PYG{o}{=} \PYG{n}{np}\PYG{o}{.}\PYG{n}{zeros}\PYG{p}{(}\PYG{n}{n}\PYG{p}{)}
\PYG{n}{Aaverage}\PYG{p}{[}\PYG{l+m+mi}{0}\PYG{p}{]} \PYG{o}{=} \PYG{o}{\PYGZhy{}}\PYG{n}{g}
\PYG{k}{for} \PYG{n}{i} \PYG{o+ow}{in} \PYG{n+nb}{range}\PYG{p}{(}\PYG{l+m+mi}{1}\PYG{p}{,}\PYG{n}{n}\PYG{o}{\PYGZhy{}}\PYG{l+m+mi}{1}\PYG{p}{)}\PYG{p}{:}
    \PYG{n}{Aaverage}\PYG{p}{[}\PYG{n}{i}\PYG{p}{]} \PYG{o}{=} \PYG{p}{(}\PYG{n}{Vaverage}\PYG{p}{[}\PYG{n}{i}\PYG{o}{+}\PYG{l+m+mi}{1}\PYG{p}{]}\PYG{o}{\PYGZhy{}}\PYG{n}{Vaverage}\PYG{p}{[}\PYG{n}{i}\PYG{p}{]}\PYG{p}{)}\PYG{o}{/}\PYG{n}{DeltaT}
\PYG{n}{data} \PYG{o}{=} \PYG{p}{\PYGZob{}}\PYG{l+s+s1}{\PYGZsq{}}\PYG{l+s+s1}{t[s]}\PYG{l+s+s1}{\PYGZsq{}}\PYG{p}{:} \PYG{n}{t}\PYG{p}{,}
        \PYG{l+s+s1}{\PYGZsq{}}\PYG{l+s+s1}{y[m]}\PYG{l+s+s1}{\PYGZsq{}}\PYG{p}{:} \PYG{n}{y}\PYG{p}{,}
        \PYG{l+s+s1}{\PYGZsq{}}\PYG{l+s+s1}{v[m/s]}\PYG{l+s+s1}{\PYGZsq{}}\PYG{p}{:} \PYG{n}{Vaverage}\PYG{p}{,}
        \PYG{l+s+s1}{\PYGZsq{}}\PYG{l+s+s1}{a[m/s\PYGZca{}2]}\PYG{l+s+s1}{\PYGZsq{}}\PYG{p}{:} \PYG{n}{Aaverage}
        \PYG{p}{\PYGZcb{}}
\PYG{n}{NewData} \PYG{o}{=} \PYG{n}{pd}\PYG{o}{.}\PYG{n}{DataFrame}\PYG{p}{(}\PYG{n}{data}\PYG{p}{)}
\PYG{n}{display}\PYG{p}{(}\PYG{n}{NewData}\PYG{p}{[}\PYG{l+m+mi}{0}\PYG{p}{:}\PYG{n}{n}\PYG{o}{\PYGZhy{}}\PYG{l+m+mi}{2}\PYG{p}{]}\PYG{p}{)}
\end{sphinxVerbatim}

\begin{sphinxVerbatim}[commandchars=\\\{\}]
    t[s]       y[m]     v[m/s]  a[m/s\PYGZca{}2]
0    0.0  10.000000   0.000000  \PYGZhy{}9.80655
1    0.1   9.950967  \PYGZhy{}1.470982  \PYGZhy{}9.80655
2    0.2   9.803869  \PYGZhy{}2.451638  \PYGZhy{}9.80655
3    0.3   9.558705  \PYGZhy{}3.432292  \PYGZhy{}9.80655
4    0.4   9.215476  \PYGZhy{}4.412948  \PYGZhy{}9.80655
5    0.5   8.774181  \PYGZhy{}5.393602  \PYGZhy{}9.80655
6    0.6   8.234821  \PYGZhy{}6.374258  \PYGZhy{}9.80655
7    0.7   7.597395  \PYGZhy{}7.354913  \PYGZhy{}9.80655
8    0.8   6.861904  \PYGZhy{}8.335567  \PYGZhy{}9.80655
9    0.9   6.028347  \PYGZhy{}9.316222  \PYGZhy{}9.80655
10   1.0   5.096725 \PYGZhy{}10.296878  \PYGZhy{}9.80655
11   1.1   4.067037 \PYGZhy{}11.277533  \PYGZhy{}9.80655
12   1.2   2.939284 \PYGZhy{}12.258187  \PYGZhy{}9.80655
\end{sphinxVerbatim}

Note that we don’t print the last values!


\section{Including Air Resistance in our model}
\label{\detokenize{chapter1:including-air-resistance-in-our-model}}
In our discussions till now of the falling baseball, we have ignored
air resistance and simply assumed that our system is only influenced
by the gravitational force.  We will postpone the derivation of air
resistance till later, after our discussion of Newton’s laws and
forces.

For our discussions here it suffices to state that the accelerations is now modified to
\begin{equation*}
\begin{split}
\boldsymbol{a}(t) = -g +D\boldsymbol{v}(t)\vert v(t)\vert,
\end{split}
\end{equation*}
where \(\vert v(t)\vert\) is the absolute value of the velocity and \(D\) is a constant which pertains to the specific object we are studying.
Since we are dealing with motion in one dimension, we can simplify the above to
\begin{equation*}
\begin{split}
a(t) = -g +Dv^2(t).
\end{split}
\end{equation*}
We can rewrite this as a differential equation
\begin{equation*}
\begin{split}
a(t) = \frac{dv}{dt}=\frac{d^2y}{dt^2}= -g +Dv^2(t).
\end{split}
\end{equation*}
Using the integral equations discussed above we can integrate twice
and obtain first the velocity as function of time and thereafter the
position as function of time.

For this particular case, we can actually obtain an analytical
solution for the velocity and for the position. Here we will first
compute the solutions analytically, thereafter we will derive Euler’s
method for solving these differential equations numerically.


\section{Analytical solutions}
\label{\detokenize{chapter1:analytical-solutions}}
For simplicity let us just write \(v(t)\) as \(v\). We have
\begin{equation*}
\begin{split}
\frac{dv}{dt}= -g +Dv^2(t).
\end{split}
\end{equation*}
We can solve this using the technique of separation of variables. We
isolate on the left all terms that involve \(v\) and on the right all
terms that involve time. We get then
\begin{equation*}
\begin{split}
\frac{dv}{g -Dv^2(t) }= -dt,
\end{split}
\end{equation*}
We scale now the equation to the left by introducing a constant
\(v_T=\sqrt{g/D}\). This constant has dimension length/time. Can you
show this?

Next we integrate the left\sphinxhyphen{}hand side (lhs) from \(v_0=0\) m/s to \(v\) and
the right\sphinxhyphen{}hand side (rhs) from \(t_0=0\) to \(t\) and obtain
\begin{equation*}
\begin{split}
\int_{0}^v\frac{dv}{g -Dv^2(t) }= \frac{v_T}{g}\mathrm{arctanh}(\frac{v}{v_T})  =-\int_0^tdt = -t.
\end{split}
\end{equation*}
We can reorganize these equations as
\begin{equation*}
\begin{split}
v_T\mathrm{arctanh}(\frac{v}{v_T})  =-gt,
\end{split}
\end{equation*}
which gives us \(v\) as function of time
\begin{equation*}
\begin{split}
v(t)=v_T\tanh{-(\frac{gt}{v_T})}.
\end{split}
\end{equation*}

\section{Finding the final height}
\label{\detokenize{chapter1:finding-the-final-height}}
With the velocity we can then find the height \(y(t)\) by integrating yet another time, that is
\begin{equation*}
\begin{split}
y(t)=y(t_0)+\int_{t_0}^t dt v(t)=\int_{0}^t dt[v_T\tanh{-(\frac{gt}{v_T})}].
\end{split}
\end{equation*}
This integral is a little bit trickier but we can look it up in a table over
known integrals and we get
\begin{equation*}
\begin{split}
y(t)=y(t_0)-\frac{v_T^2}{g}\log{[\cosh{(\frac{gt}{v_T})}]}.
\end{split}
\end{equation*}
Alternatively we could have used the symbolic Python package \sphinxstylestrong{Sympy}  (example will be inserted later).

In most cases however, we need to revert to numerical solutions.


\section{Our first attempt at solving differential equations}
\label{\detokenize{chapter1:our-first-attempt-at-solving-differential-equations}}
Here we will try the simplest possible approach to solving the second\sphinxhyphen{}order differential
equation
\begin{equation*}
\begin{split}
a(t) =\frac{d^2y}{dt^2}= -g +Dv^2(t).
\end{split}
\end{equation*}
We rewrite it as two coupled first\sphinxhyphen{}order equations (this is a standard approach)
\begin{equation*}
\begin{split}
\frac{dy}{dt} = v(t),
\end{split}
\end{equation*}
with initial condition \(y(t_0)=y_0\) and
\begin{equation*}
\begin{split}
a(t) =\frac{dv}{dt}= -g +Dv^2(t),
\end{split}
\end{equation*}
with initial condition \(v(t_0)=v_0\).

Many of the algorithms for solving differential equations start with simple Taylor equations.
If we now Taylor expand \(y\) and \(v\) around a value \(t+\Delta t\) we have
\begin{equation*}
\begin{split}
y(t+\Delta t) = y(t)+\Delta t \frac{dy}{dt}+\frac{\Delta t^2}{2!} \frac{d^2y}{dt^2}+O(\Delta t^3),
\end{split}
\end{equation*}
and
\begin{equation*}
\begin{split}
v(t+\Delta t) = v(t)+\Delta t \frac{dv}{dt}+\frac{\Delta t^2}{2!} \frac{d^2v}{dt^2}+O(\Delta t^3).
\end{split}
\end{equation*}
Using the fact that \(dy/dt = v\) and \(dv/dt=a\) and keeping only terms up to \(\Delta t\) we have
\begin{equation*}
\begin{split}
y(t+\Delta t) = y(t)+\Delta t v(t)+O(\Delta t^2),
\end{split}
\end{equation*}
and
\begin{equation*}
\begin{split}
v(t+\Delta t) = v(t)+\Delta t a(t)+O(\Delta t^2).
\end{split}
\end{equation*}

\section{Discretizing our equations}
\label{\detokenize{chapter1:discretizing-our-equations}}
Using our discretized versions of the equations with for example
\(y_{i}=y(t_i)\) and \(y_{i\pm 1}=y(t_i+\Delta t)\), we can rewrite the
above equations as (and truncating at \(\Delta t\))
\begin{equation*}
\begin{split}
y_{i+1} = y_i+\Delta t v_i,
\end{split}
\end{equation*}
and
\begin{equation*}
\begin{split}
v_{i+1} = v_i+\Delta t a_i.
\end{split}
\end{equation*}
These are the famous Euler equations (forward Euler).

To solve these equations numerically we start at a time \(t_0\) and simply integrate up these equations to a final time \(t_f\),
The step size \(\Delta t\) is an input  parameter in our code.
You can define it directly in the code below as

\begin{sphinxVerbatim}[commandchars=\\\{\}]
\PYG{n}{DeltaT} \PYG{o}{=} \PYG{l+m+mf}{0.1}
\end{sphinxVerbatim}

With a given final time \sphinxstylestrong{tfinal}  we can then find the number of integration points via the \sphinxstylestrong{ceil} function included in the \sphinxstylestrong{math} package of Python
as

\begin{sphinxVerbatim}[commandchars=\\\{\}]
\PYG{c+c1}{\PYGZsh{}define final time, assuming that initial time is zero}
\PYG{k+kn}{from} \PYG{n+nn}{math} \PYG{k+kn}{import} \PYG{n}{ceil}
\PYG{n}{tfinal} \PYG{o}{=} \PYG{l+m+mf}{0.5}
\PYG{n}{n} \PYG{o}{=} \PYG{n}{ceil}\PYG{p}{(}\PYG{n}{tfinal}\PYG{o}{/}\PYG{n}{DeltaT}\PYG{p}{)}
\PYG{n+nb}{print}\PYG{p}{(}\PYG{n}{n}\PYG{p}{)}
\end{sphinxVerbatim}

\begin{sphinxVerbatim}[commandchars=\\\{\}]
5
\end{sphinxVerbatim}

The \sphinxstylestrong{ceil} function returns the smallest integer not less than the input in say

\begin{sphinxVerbatim}[commandchars=\\\{\}]
\PYG{n}{x} \PYG{o}{=} \PYG{l+m+mf}{21.15}
\PYG{n+nb}{print}\PYG{p}{(}\PYG{n}{ceil}\PYG{p}{(}\PYG{n}{x}\PYG{p}{)}\PYG{p}{)}
\end{sphinxVerbatim}

\begin{sphinxVerbatim}[commandchars=\\\{\}]
22
\end{sphinxVerbatim}

which in the case here is 22.

\begin{sphinxVerbatim}[commandchars=\\\{\}]
\PYG{n}{x} \PYG{o}{=} \PYG{l+m+mf}{21.75}
\PYG{n+nb}{print}\PYG{p}{(}\PYG{n}{ceil}\PYG{p}{(}\PYG{n}{x}\PYG{p}{)}\PYG{p}{)}
\end{sphinxVerbatim}

\begin{sphinxVerbatim}[commandchars=\\\{\}]
22
\end{sphinxVerbatim}

which also yields 22. The  \sphinxstylestrong{floor} function in the \sphinxstylestrong{math} package
is used to return the closest integer value which is less than or equal to the specified expression or value.
Compare the previous result to the usage of \sphinxstylestrong{floor}

\begin{sphinxVerbatim}[commandchars=\\\{\}]
\PYG{k+kn}{from} \PYG{n+nn}{math} \PYG{k+kn}{import} \PYG{n}{floor}
\PYG{n}{x} \PYG{o}{=} \PYG{l+m+mf}{21.75}
\PYG{n+nb}{print}\PYG{p}{(}\PYG{n}{floor}\PYG{p}{(}\PYG{n}{x}\PYG{p}{)}\PYG{p}{)}
\end{sphinxVerbatim}

\begin{sphinxVerbatim}[commandchars=\\\{\}]
21
\end{sphinxVerbatim}

Alternatively, we can define ourselves the number of integration(mesh) points. In this case we could have

\begin{sphinxVerbatim}[commandchars=\\\{\}]
\PYG{n}{n} \PYG{o}{=} \PYG{l+m+mi}{10}
\PYG{n}{tinitial} \PYG{o}{=} \PYG{l+m+mf}{0.0}
\PYG{n}{tfinal} \PYG{o}{=} \PYG{l+m+mf}{0.5}
\PYG{n}{DeltaT} \PYG{o}{=} \PYG{p}{(}\PYG{n}{tfinal}\PYG{o}{\PYGZhy{}}\PYG{n}{tinitial}\PYG{p}{)}\PYG{o}{/}\PYG{p}{(}\PYG{n}{n}\PYG{p}{)}
\PYG{n+nb}{print}\PYG{p}{(}\PYG{n}{DeltaT}\PYG{p}{)}
\end{sphinxVerbatim}

\begin{sphinxVerbatim}[commandchars=\\\{\}]
0.05
\end{sphinxVerbatim}

Since we will set up one\sphinxhyphen{}dimensional arrays that contain the values of
various variables like time, position, velocity, acceleration etc, we
need to know the value of \(n\), the number of data points (or
integration or mesh points).  With \(n\) we can initialize a given array
by setting all elelements to zero, as done here

\begin{sphinxVerbatim}[commandchars=\\\{\}]
\PYG{c+c1}{\PYGZsh{} define array a}
\PYG{n}{a} \PYG{o}{=} \PYG{n}{np}\PYG{o}{.}\PYG{n}{zeros}\PYG{p}{(}\PYG{n}{n}\PYG{p}{)}
\PYG{n+nb}{print}\PYG{p}{(}\PYG{n}{a}\PYG{p}{)}
\end{sphinxVerbatim}

\begin{sphinxVerbatim}[commandchars=\\\{\}]
[0. 0. 0. 0. 0. 0. 0. 0. 0. 0.]
\end{sphinxVerbatim}


\section{Code for implementing Euler’s method}
\label{\detokenize{chapter1:code-for-implementing-euler-s-method}}
In the code here we implement this simple Eurler scheme choosing a value for \(D=0.0245\) m/s.

\begin{sphinxVerbatim}[commandchars=\\\{\}]
\PYG{c+c1}{\PYGZsh{} Common imports}
\PYG{k+kn}{import} \PYG{n+nn}{numpy} \PYG{k}{as} \PYG{n+nn}{np}
\PYG{k+kn}{import} \PYG{n+nn}{pandas} \PYG{k}{as} \PYG{n+nn}{pd}
\PYG{k+kn}{from} \PYG{n+nn}{math} \PYG{k+kn}{import} \PYG{o}{*}
\PYG{k+kn}{import} \PYG{n+nn}{matplotlib}\PYG{n+nn}{.}\PYG{n+nn}{pyplot} \PYG{k}{as} \PYG{n+nn}{plt}
\PYG{k+kn}{import} \PYG{n+nn}{os}

\PYG{c+c1}{\PYGZsh{} Where to save the figures and data files}
\PYG{n}{PROJECT\PYGZus{}ROOT\PYGZus{}DIR} \PYG{o}{=} \PYG{l+s+s2}{\PYGZdq{}}\PYG{l+s+s2}{Results}\PYG{l+s+s2}{\PYGZdq{}}
\PYG{n}{FIGURE\PYGZus{}ID} \PYG{o}{=} \PYG{l+s+s2}{\PYGZdq{}}\PYG{l+s+s2}{Results/FigureFiles}\PYG{l+s+s2}{\PYGZdq{}}
\PYG{n}{DATA\PYGZus{}ID} \PYG{o}{=} \PYG{l+s+s2}{\PYGZdq{}}\PYG{l+s+s2}{DataFiles/}\PYG{l+s+s2}{\PYGZdq{}}

\PYG{k}{if} \PYG{o+ow}{not} \PYG{n}{os}\PYG{o}{.}\PYG{n}{path}\PYG{o}{.}\PYG{n}{exists}\PYG{p}{(}\PYG{n}{PROJECT\PYGZus{}ROOT\PYGZus{}DIR}\PYG{p}{)}\PYG{p}{:}
    \PYG{n}{os}\PYG{o}{.}\PYG{n}{mkdir}\PYG{p}{(}\PYG{n}{PROJECT\PYGZus{}ROOT\PYGZus{}DIR}\PYG{p}{)}

\PYG{k}{if} \PYG{o+ow}{not} \PYG{n}{os}\PYG{o}{.}\PYG{n}{path}\PYG{o}{.}\PYG{n}{exists}\PYG{p}{(}\PYG{n}{FIGURE\PYGZus{}ID}\PYG{p}{)}\PYG{p}{:}
    \PYG{n}{os}\PYG{o}{.}\PYG{n}{makedirs}\PYG{p}{(}\PYG{n}{FIGURE\PYGZus{}ID}\PYG{p}{)}

\PYG{k}{if} \PYG{o+ow}{not} \PYG{n}{os}\PYG{o}{.}\PYG{n}{path}\PYG{o}{.}\PYG{n}{exists}\PYG{p}{(}\PYG{n}{DATA\PYGZus{}ID}\PYG{p}{)}\PYG{p}{:}
    \PYG{n}{os}\PYG{o}{.}\PYG{n}{makedirs}\PYG{p}{(}\PYG{n}{DATA\PYGZus{}ID}\PYG{p}{)}

\PYG{k}{def} \PYG{n+nf}{image\PYGZus{}path}\PYG{p}{(}\PYG{n}{fig\PYGZus{}id}\PYG{p}{)}\PYG{p}{:}
    \PYG{k}{return} \PYG{n}{os}\PYG{o}{.}\PYG{n}{path}\PYG{o}{.}\PYG{n}{join}\PYG{p}{(}\PYG{n}{FIGURE\PYGZus{}ID}\PYG{p}{,} \PYG{n}{fig\PYGZus{}id}\PYG{p}{)}

\PYG{k}{def} \PYG{n+nf}{data\PYGZus{}path}\PYG{p}{(}\PYG{n}{dat\PYGZus{}id}\PYG{p}{)}\PYG{p}{:}
    \PYG{k}{return} \PYG{n}{os}\PYG{o}{.}\PYG{n}{path}\PYG{o}{.}\PYG{n}{join}\PYG{p}{(}\PYG{n}{DATA\PYGZus{}ID}\PYG{p}{,} \PYG{n}{dat\PYGZus{}id}\PYG{p}{)}

\PYG{k}{def} \PYG{n+nf}{save\PYGZus{}fig}\PYG{p}{(}\PYG{n}{fig\PYGZus{}id}\PYG{p}{)}\PYG{p}{:}
    \PYG{n}{plt}\PYG{o}{.}\PYG{n}{savefig}\PYG{p}{(}\PYG{n}{image\PYGZus{}path}\PYG{p}{(}\PYG{n}{fig\PYGZus{}id}\PYG{p}{)} \PYG{o}{+} \PYG{l+s+s2}{\PYGZdq{}}\PYG{l+s+s2}{.png}\PYG{l+s+s2}{\PYGZdq{}}\PYG{p}{,} \PYG{n+nb}{format}\PYG{o}{=}\PYG{l+s+s1}{\PYGZsq{}}\PYG{l+s+s1}{png}\PYG{l+s+s1}{\PYGZsq{}}\PYG{p}{)}


\PYG{n}{g} \PYG{o}{=} \PYG{l+m+mf}{9.80655} \PYG{c+c1}{\PYGZsh{}m/s\PYGZca{}2}
\PYG{n}{D} \PYG{o}{=} \PYG{l+m+mf}{0.00245} \PYG{c+c1}{\PYGZsh{}m/s}
\PYG{n}{DeltaT} \PYG{o}{=} \PYG{l+m+mf}{0.1}
\PYG{c+c1}{\PYGZsh{}set up arrays }
\PYG{n}{tfinal} \PYG{o}{=} \PYG{l+m+mf}{0.5}
\PYG{n}{n} \PYG{o}{=} \PYG{n}{ceil}\PYG{p}{(}\PYG{n}{tfinal}\PYG{o}{/}\PYG{n}{DeltaT}\PYG{p}{)}
\PYG{c+c1}{\PYGZsh{} define scaling constant vT}
\PYG{n}{vT} \PYG{o}{=} \PYG{n}{sqrt}\PYG{p}{(}\PYG{n}{g}\PYG{o}{/}\PYG{n}{D}\PYG{p}{)}
\PYG{c+c1}{\PYGZsh{} set up arrays for t, a, v, and y and we can compare our results with analytical ones}
\PYG{n}{t} \PYG{o}{=} \PYG{n}{np}\PYG{o}{.}\PYG{n}{zeros}\PYG{p}{(}\PYG{n}{n}\PYG{p}{)}
\PYG{n}{a} \PYG{o}{=} \PYG{n}{np}\PYG{o}{.}\PYG{n}{zeros}\PYG{p}{(}\PYG{n}{n}\PYG{p}{)}
\PYG{n}{v} \PYG{o}{=} \PYG{n}{np}\PYG{o}{.}\PYG{n}{zeros}\PYG{p}{(}\PYG{n}{n}\PYG{p}{)}
\PYG{n}{y} \PYG{o}{=} \PYG{n}{np}\PYG{o}{.}\PYG{n}{zeros}\PYG{p}{(}\PYG{n}{n}\PYG{p}{)}
\PYG{n}{yanalytic} \PYG{o}{=} \PYG{n}{np}\PYG{o}{.}\PYG{n}{zeros}\PYG{p}{(}\PYG{n}{n}\PYG{p}{)}
\PYG{c+c1}{\PYGZsh{} Initial conditions}
\PYG{n}{v}\PYG{p}{[}\PYG{l+m+mi}{0}\PYG{p}{]} \PYG{o}{=} \PYG{l+m+mf}{0.0}  \PYG{c+c1}{\PYGZsh{}m/s}
\PYG{n}{y}\PYG{p}{[}\PYG{l+m+mi}{0}\PYG{p}{]} \PYG{o}{=} \PYG{l+m+mf}{10.0} \PYG{c+c1}{\PYGZsh{}m}
\PYG{n}{yanalytic}\PYG{p}{[}\PYG{l+m+mi}{0}\PYG{p}{]} \PYG{o}{=} \PYG{n}{y}\PYG{p}{[}\PYG{l+m+mi}{0}\PYG{p}{]}
\PYG{c+c1}{\PYGZsh{} Start integrating using Euler\PYGZsq{}s method}
\PYG{k}{for} \PYG{n}{i} \PYG{o+ow}{in} \PYG{n+nb}{range}\PYG{p}{(}\PYG{n}{n}\PYG{o}{\PYGZhy{}}\PYG{l+m+mi}{1}\PYG{p}{)}\PYG{p}{:}
    \PYG{c+c1}{\PYGZsh{} expression for acceleration}
    \PYG{n}{a}\PYG{p}{[}\PYG{n}{i}\PYG{p}{]} \PYG{o}{=} \PYG{o}{\PYGZhy{}}\PYG{n}{g} \PYG{o}{+} \PYG{n}{D}\PYG{o}{*}\PYG{n}{v}\PYG{p}{[}\PYG{n}{i}\PYG{p}{]}\PYG{o}{*}\PYG{n}{v}\PYG{p}{[}\PYG{n}{i}\PYG{p}{]}
    \PYG{c+c1}{\PYGZsh{} update velocity and position}
    \PYG{n}{y}\PYG{p}{[}\PYG{n}{i}\PYG{o}{+}\PYG{l+m+mi}{1}\PYG{p}{]} \PYG{o}{=} \PYG{n}{y}\PYG{p}{[}\PYG{n}{i}\PYG{p}{]} \PYG{o}{+} \PYG{n}{DeltaT}\PYG{o}{*}\PYG{n}{v}\PYG{p}{[}\PYG{n}{i}\PYG{p}{]}
    \PYG{n}{v}\PYG{p}{[}\PYG{n}{i}\PYG{o}{+}\PYG{l+m+mi}{1}\PYG{p}{]} \PYG{o}{=} \PYG{n}{v}\PYG{p}{[}\PYG{n}{i}\PYG{p}{]} \PYG{o}{+} \PYG{n}{DeltaT}\PYG{o}{*}\PYG{n}{a}\PYG{p}{[}\PYG{n}{i}\PYG{p}{]}
    \PYG{c+c1}{\PYGZsh{} update time to next time step and compute analytical answer}
    \PYG{n}{t}\PYG{p}{[}\PYG{n}{i}\PYG{o}{+}\PYG{l+m+mi}{1}\PYG{p}{]} \PYG{o}{=} \PYG{n}{t}\PYG{p}{[}\PYG{n}{i}\PYG{p}{]} \PYG{o}{+} \PYG{n}{DeltaT}
    \PYG{n}{yanalytic}\PYG{p}{[}\PYG{n}{i}\PYG{o}{+}\PYG{l+m+mi}{1}\PYG{p}{]} \PYG{o}{=} \PYG{n}{y}\PYG{p}{[}\PYG{l+m+mi}{0}\PYG{p}{]}\PYG{o}{\PYGZhy{}}\PYG{p}{(}\PYG{n}{vT}\PYG{o}{*}\PYG{n}{vT}\PYG{o}{/}\PYG{n}{g}\PYG{p}{)}\PYG{o}{*}\PYG{n}{log}\PYG{p}{(}\PYG{n}{cosh}\PYG{p}{(}\PYG{n}{g}\PYG{o}{*}\PYG{n}{t}\PYG{p}{[}\PYG{n}{i}\PYG{o}{+}\PYG{l+m+mi}{1}\PYG{p}{]}\PYG{o}{/}\PYG{n}{vT}\PYG{p}{)}\PYG{p}{)}
    \PYG{k}{if} \PYG{p}{(} \PYG{n}{y}\PYG{p}{[}\PYG{n}{i}\PYG{o}{+}\PYG{l+m+mi}{1}\PYG{p}{]} \PYG{o}{\PYGZlt{}} \PYG{l+m+mf}{0.0}\PYG{p}{)}\PYG{p}{:}
        \PYG{k}{break}
\PYG{n}{a}\PYG{p}{[}\PYG{n}{n}\PYG{o}{\PYGZhy{}}\PYG{l+m+mi}{1}\PYG{p}{]} \PYG{o}{=} \PYG{o}{\PYGZhy{}}\PYG{n}{g} \PYG{o}{+} \PYG{n}{D}\PYG{o}{*}\PYG{n}{v}\PYG{p}{[}\PYG{n}{n}\PYG{o}{\PYGZhy{}}\PYG{l+m+mi}{1}\PYG{p}{]}\PYG{o}{*}\PYG{n}{v}\PYG{p}{[}\PYG{n}{n}\PYG{o}{\PYGZhy{}}\PYG{l+m+mi}{1}\PYG{p}{]}
\PYG{n}{data} \PYG{o}{=} \PYG{p}{\PYGZob{}}\PYG{l+s+s1}{\PYGZsq{}}\PYG{l+s+s1}{t[s]}\PYG{l+s+s1}{\PYGZsq{}}\PYG{p}{:} \PYG{n}{t}\PYG{p}{,}
        \PYG{l+s+s1}{\PYGZsq{}}\PYG{l+s+s1}{y[m]}\PYG{l+s+s1}{\PYGZsq{}}\PYG{p}{:} \PYG{n}{y}\PYG{o}{\PYGZhy{}}\PYG{n}{yanalytic}\PYG{p}{,}
        \PYG{l+s+s1}{\PYGZsq{}}\PYG{l+s+s1}{v[m/s]}\PYG{l+s+s1}{\PYGZsq{}}\PYG{p}{:} \PYG{n}{v}\PYG{p}{,}
        \PYG{l+s+s1}{\PYGZsq{}}\PYG{l+s+s1}{a[m/s\PYGZca{}2]}\PYG{l+s+s1}{\PYGZsq{}}\PYG{p}{:} \PYG{n}{a}
        \PYG{p}{\PYGZcb{}}
\PYG{n}{NewData} \PYG{o}{=} \PYG{n}{pd}\PYG{o}{.}\PYG{n}{DataFrame}\PYG{p}{(}\PYG{n}{data}\PYG{p}{)}
\PYG{n}{display}\PYG{p}{(}\PYG{n}{NewData}\PYG{p}{)}
\PYG{c+c1}{\PYGZsh{}finally we plot the data}
\PYG{n}{fig}\PYG{p}{,} \PYG{n}{axs} \PYG{o}{=} \PYG{n}{plt}\PYG{o}{.}\PYG{n}{subplots}\PYG{p}{(}\PYG{l+m+mi}{3}\PYG{p}{,} \PYG{l+m+mi}{1}\PYG{p}{)}
\PYG{n}{axs}\PYG{p}{[}\PYG{l+m+mi}{0}\PYG{p}{]}\PYG{o}{.}\PYG{n}{plot}\PYG{p}{(}\PYG{n}{t}\PYG{p}{,} \PYG{n}{y}\PYG{p}{,} \PYG{n}{t}\PYG{p}{,} \PYG{n}{yanalytic}\PYG{p}{)}
\PYG{n}{axs}\PYG{p}{[}\PYG{l+m+mi}{0}\PYG{p}{]}\PYG{o}{.}\PYG{n}{set\PYGZus{}xlim}\PYG{p}{(}\PYG{l+m+mi}{0}\PYG{p}{,} \PYG{n}{tfinal}\PYG{p}{)}
\PYG{n}{axs}\PYG{p}{[}\PYG{l+m+mi}{0}\PYG{p}{]}\PYG{o}{.}\PYG{n}{set\PYGZus{}ylabel}\PYG{p}{(}\PYG{l+s+s1}{\PYGZsq{}}\PYG{l+s+s1}{y and exact}\PYG{l+s+s1}{\PYGZsq{}}\PYG{p}{)}
\PYG{n}{axs}\PYG{p}{[}\PYG{l+m+mi}{1}\PYG{p}{]}\PYG{o}{.}\PYG{n}{plot}\PYG{p}{(}\PYG{n}{t}\PYG{p}{,} \PYG{n}{v}\PYG{p}{)}
\PYG{n}{axs}\PYG{p}{[}\PYG{l+m+mi}{1}\PYG{p}{]}\PYG{o}{.}\PYG{n}{set\PYGZus{}ylabel}\PYG{p}{(}\PYG{l+s+s1}{\PYGZsq{}}\PYG{l+s+s1}{v[m/s]}\PYG{l+s+s1}{\PYGZsq{}}\PYG{p}{)}
\PYG{n}{axs}\PYG{p}{[}\PYG{l+m+mi}{2}\PYG{p}{]}\PYG{o}{.}\PYG{n}{plot}\PYG{p}{(}\PYG{n}{t}\PYG{p}{,} \PYG{n}{a}\PYG{p}{)}
\PYG{n}{axs}\PYG{p}{[}\PYG{l+m+mi}{2}\PYG{p}{]}\PYG{o}{.}\PYG{n}{set\PYGZus{}xlabel}\PYG{p}{(}\PYG{l+s+s1}{\PYGZsq{}}\PYG{l+s+s1}{time[s]}\PYG{l+s+s1}{\PYGZsq{}}\PYG{p}{)}
\PYG{n}{axs}\PYG{p}{[}\PYG{l+m+mi}{2}\PYG{p}{]}\PYG{o}{.}\PYG{n}{set\PYGZus{}ylabel}\PYG{p}{(}\PYG{l+s+s1}{\PYGZsq{}}\PYG{l+s+s1}{a[m/s\PYGZca{}2]}\PYG{l+s+s1}{\PYGZsq{}}\PYG{p}{)}
\PYG{n}{fig}\PYG{o}{.}\PYG{n}{tight\PYGZus{}layout}\PYG{p}{(}\PYG{p}{)}
\PYG{n}{save\PYGZus{}fig}\PYG{p}{(}\PYG{l+s+s2}{\PYGZdq{}}\PYG{l+s+s2}{EulerIntegration}\PYG{l+s+s2}{\PYGZdq{}}\PYG{p}{)}
\PYG{n}{plt}\PYG{o}{.}\PYG{n}{show}\PYG{p}{(}\PYG{p}{)}
\end{sphinxVerbatim}

\begin{sphinxVerbatim}[commandchars=\\\{\}]
   t[s]      y[m]    v[m/s]  a[m/s\PYGZca{}2]
0   0.0  0.000000  0.000000 \PYGZhy{}9.806550
1   0.1  0.049031 \PYGZhy{}0.980655 \PYGZhy{}9.804194
2   0.2  0.098034 \PYGZhy{}1.961074 \PYGZhy{}9.797128
3   0.3  0.146963 \PYGZhy{}2.940787 \PYGZhy{}9.785362
4   0.4  0.195770 \PYGZhy{}3.919323 \PYGZhy{}9.768915
\end{sphinxVerbatim}

\noindent\sphinxincludegraphics{{chapter1_125_1}.png}

Try different values for \(\Delta t\) and study the difference between the exact solution and the numerical solution.


\section{Simple extension, the Euler\sphinxhyphen{}Cromer method}
\label{\detokenize{chapter1:simple-extension-the-euler-cromer-method}}
The Euler\sphinxhyphen{}Cromer method is a simple variant of the standard Euler
method. We use the newly updated velocity \(v_{i+1}\) as an input to the
new position, that is, instead of
\begin{equation*}
\begin{split}
y_{i+1} = y_i+\Delta t v_i,
\end{split}
\end{equation*}
and
\begin{equation*}
\begin{split}
v_{i+1} = v_i+\Delta t a_i,
\end{split}
\end{equation*}
we use now the newly calculate for \(v_{i+1}\) as input to \(y_{i+1}\), that is
we compute first
\begin{equation*}
\begin{split}
v_{i+1} = v_i+\Delta t a_i,
\end{split}
\end{equation*}
and then
\begin{equation*}
\begin{split}
y_{i+1} = y_i+\Delta t v_{i+1},
\end{split}
\end{equation*}
Implementing the Euler\sphinxhyphen{}Cromer method yields a simple change to the previous code. We only need to change the following line in the loop over time
steps

\begin{sphinxVerbatim}[commandchars=\\\{\}]
\PYG{k}{for} \PYG{n}{i} \PYG{o+ow}{in} \PYG{n+nb}{range}\PYG{p}{(}\PYG{n}{n}\PYG{o}{\PYGZhy{}}\PYG{l+m+mi}{1}\PYG{p}{)}\PYG{p}{:}
    \PYG{c+c1}{\PYGZsh{} more codes in between here}
    \PYG{n}{v}\PYG{p}{[}\PYG{n}{i}\PYG{o}{+}\PYG{l+m+mi}{1}\PYG{p}{]} \PYG{o}{=} \PYG{n}{v}\PYG{p}{[}\PYG{n}{i}\PYG{p}{]} \PYG{o}{+} \PYG{n}{DeltaT}\PYG{o}{*}\PYG{n}{a}\PYG{p}{[}\PYG{n}{i}\PYG{p}{]}
    \PYG{n}{y}\PYG{p}{[}\PYG{n}{i}\PYG{o}{+}\PYG{l+m+mi}{1}\PYG{p}{]} \PYG{o}{=} \PYG{n}{y}\PYG{p}{[}\PYG{n}{i}\PYG{p}{]} \PYG{o}{+} \PYG{n}{DeltaT}\PYG{o}{*}\PYG{n}{v}\PYG{p}{[}\PYG{n}{i}\PYG{o}{+}\PYG{l+m+mi}{1}\PYG{p}{]}
    \PYG{c+c1}{\PYGZsh{} more code}
\end{sphinxVerbatim}


\section{Python practicalities, Software and needed installations}
\label{\detokenize{chapter1:python-practicalities-software-and-needed-installations}}
We will make extensive use of Python as programming language and its
myriad of available libraries.  You will find
Jupyter notebooks invaluable in your work.

If you have Python installed (we strongly recommend Python3) and you feel
pretty familiar with installing different packages, we recommend that
you install the following Python packages via \sphinxstylestrong{pip} as
\begin{enumerate}
\sphinxsetlistlabels{\arabic}{enumi}{enumii}{}{.}%
\item {} 
pip install numpy scipy matplotlib ipython scikit\sphinxhyphen{}learn mglearn sympy pandas pillow

\end{enumerate}

For Python3, replace \sphinxstylestrong{pip} with \sphinxstylestrong{pip3}.

For OSX users we recommend, after having installed Xcode, to
install \sphinxstylestrong{brew}. Brew allows for a seamless installation of additional
software via for example
\begin{enumerate}
\sphinxsetlistlabels{\arabic}{enumi}{enumii}{}{.}%
\item {} 
brew install python3

\end{enumerate}

For Linux users, with its variety of distributions like for example the widely popular Ubuntu distribution,
you can use \sphinxstylestrong{pip} as well and simply install Python as
\begin{enumerate}
\sphinxsetlistlabels{\arabic}{enumi}{enumii}{}{.}%
\item {} 
sudo apt\sphinxhyphen{}get install python3  (or python for pyhton2.7)

\end{enumerate}

etc etc.


\section{Python installers}
\label{\detokenize{chapter1:python-installers}}
If you don’t want to perform these operations separately and venture
into the hassle of exploring how to set up dependencies and paths, we
recommend two widely used distrubutions which set up all relevant
dependencies for Python, namely
\begin{itemize}
\item {} 
\sphinxhref{https://docs.anaconda.com/}{Anaconda},

\end{itemize}

which is an open source
distribution of the Python and R programming languages for large\sphinxhyphen{}scale
data processing, predictive analytics, and scientific computing, that
aims to simplify package management and deployment. Package versions
are managed by the package management system \sphinxstylestrong{conda}.
\begin{itemize}
\item {} 
\sphinxhref{https://www.enthought.com/product/canopy/}{Enthought canopy}

\end{itemize}

is a Python
distribution for scientific and analytic computing distribution and
analysis environment, available for free and under a commercial
license.

Furthermore, \sphinxhref{https://colab.research.google.com/notebooks/welcome.ipynb}{Google’s Colab} is a free Jupyter notebook environment that requires
no setup and runs entirely in the cloud. Try it out!


\section{Useful Python libraries}
\label{\detokenize{chapter1:useful-python-libraries}}
Here we list several useful Python libraries we strongly recommend (if you use anaconda many of these are already there)
\begin{itemize}
\item {} 
\sphinxhref{https://www.numpy.org/}{NumPy} is a highly popular library for large, multi\sphinxhyphen{}dimensional arrays and matrices, along with a large collection of high\sphinxhyphen{}level mathematical functions to operate on these arrays

\item {} 
\sphinxhref{https://pandas.pydata.org/}{The pandas} library provides high\sphinxhyphen{}performance, easy\sphinxhyphen{}to\sphinxhyphen{}use data structures and data analysis tools

\item {} 
\sphinxhref{http://xarray.pydata.org/en/stable/}{Xarray} is a Python package that makes working with labelled multi\sphinxhyphen{}dimensional arrays simple, efficient, and fun!

\item {} 
\sphinxhref{https://www.scipy.org/}{Scipy} (pronounced “Sigh Pie”) is a Python\sphinxhyphen{}based ecosystem of open\sphinxhyphen{}source software for mathematics, science, and engineering.

\item {} 
\sphinxhref{https://matplotlib.org/}{Matplotlib} is a Python 2D plotting library which produces publication quality figures in a variety of hardcopy formats and interactive environments across platforms.

\item {} 
\sphinxhref{https://github.com/HIPS/autograd}{Autograd} can automatically differentiate native Python and Numpy code. It can handle a large subset of Python’s features, including loops, ifs, recursion and closures, and it can even take derivatives of derivatives of derivatives

\item {} 
\sphinxhref{https://www.sympy.org/en/index.html}{SymPy} is a Python library for symbolic mathematics.

\item {} 
\sphinxhref{https://scikit-learn.org/stable/}{scikit\sphinxhyphen{}learn} has simple and efficient tools for machine learning, data mining and data analysis

\item {} 
\sphinxhref{https://www.tensorflow.org/}{TensorFlow} is a Python library for fast numerical computing created and released by Google

\item {} 
\sphinxhref{https://keras.io/}{Keras} is a high\sphinxhyphen{}level neural networks API, written in Python and capable of running on top of TensorFlow, CNTK, or Theano

\item {} 
And many more such as \sphinxhref{https://pytorch.org/}{pytorch},  \sphinxhref{https://pypi.org/project/Theano/}{Theano} etc

\end{itemize}

Your jupyter notebook can easily be
converted into a nicely rendered \sphinxstylestrong{PDF} file or a Latex file for
further processing. For example, convert to latex as

\begin{sphinxVerbatim}[commandchars=\\\{\}]
    pycod jupyter nbconvert filename.ipynb \PYGZhy{}\PYGZhy{}to latex 
\end{sphinxVerbatim}

And to add more versatility, the Python package \sphinxhref{http://www.sympy.org/en/index.html}{SymPy} is a Python library for symbolic mathematics. It aims to become a full\sphinxhyphen{}featured computer algebra system (CAS)  and is entirely written in Python.


\section{Numpy examples and Important Matrix and vector handling packages}
\label{\detokenize{chapter1:numpy-examples-and-important-matrix-and-vector-handling-packages}}
There are several central software libraries for linear algebra and eigenvalue problems. Several of the more
popular ones have been wrapped into ofter software packages like those from the widely used text \sphinxstylestrong{Numerical Recipes}. The original source codes in many of the available packages are often taken from the widely used
software package LAPACK, which follows two other popular packages
developed in the 1970s, namely EISPACK and LINPACK.  We describe them shortly here.
\begin{itemize}
\item {} 
LINPACK: package for linear equations and least square problems.

\item {} 
LAPACK:package for solving symmetric, unsymmetric and generalized eigenvalue problems. From LAPACK’s website \sphinxurl{http://www.netlib.org} it is possible to download for free all source codes from this library. Both C/C++ and Fortran versions are available.

\item {} 
BLAS (I, II and III): (Basic Linear Algebra Subprograms) are routines that provide standard building blocks for performing basic vector and matrix operations. Blas I is vector operations, II vector\sphinxhyphen{}matrix operations and III matrix\sphinxhyphen{}matrix operations. Highly parallelized and efficient codes, all available for download from \sphinxurl{http://www.netlib.org}.

\end{itemize}


\section{Basic Matrix Features}
\label{\detokenize{chapter1:basic-matrix-features}}
\sphinxstylestrong{Matrix properties reminder.}
\begin{equation*}
\begin{split}
\mathbf{A} =
      \begin{bmatrix} a_{11} & a_{12} & a_{13} & a_{14} \\
                                 a_{21} & a_{22} & a_{23} & a_{24} \\
                                   a_{31} & a_{32} & a_{33} & a_{34} \\
                                  a_{41} & a_{42} & a_{43} & a_{44}
             \end{bmatrix}\qquad
\mathbf{I} =
      \begin{bmatrix} 1 & 0 & 0 & 0 \\
                                 0 & 1 & 0 & 0 \\
                                 0 & 0 & 1 & 0 \\
                                 0 & 0 & 0 & 1
             \end{bmatrix}
\end{split}
\end{equation*}
The inverse of a matrix is defined by
\begin{equation*}
\begin{split}
\mathbf{A}^{-1} \cdot \mathbf{A} = I
\end{split}
\end{equation*}



\subsection{Some famous Matrices}
\label{\detokenize{chapter1:some-famous-matrices}}\begin{itemize}
\item {} 
Diagonal if \(a_{ij}=0\) for \(i\ne j\)

\item {} 
Upper triangular if \(a_{ij}=0\) for \(i > j\)

\item {} 
Lower triangular if \(a_{ij}=0\) for \(i < j\)

\item {} 
Upper Hessenberg if \(a_{ij}=0\) for \(i > j+1\)

\item {} 
Lower Hessenberg if \(a_{ij}=0\) for \(i < j+1\)

\item {} 
Tridiagonal if \(a_{ij}=0\) for \(|i -j| > 1\)

\item {} 
Lower banded with bandwidth \(p\): \(a_{ij}=0\) for \(i > j+p\)

\item {} 
Upper banded with bandwidth \(p\): \(a_{ij}=0\) for \(i < j+p\)

\item {} 
Banded, block upper triangular, block lower triangular….

\end{itemize}


\subsection{More Basic Matrix Features}
\label{\detokenize{chapter1:more-basic-matrix-features}}
\sphinxstylestrong{Some Equivalent Statements.}

For an \(N\times N\) matrix  \(\mathbf{A}\) the following properties are all equivalent
\begin{itemize}
\item {} 
If the inverse of \(\mathbf{A}\) exists, \(\mathbf{A}\) is nonsingular.

\item {} 
The equation \(\mathbf{Ax}=0\) implies \(\mathbf{x}=0\).

\item {} 
The rows of \(\mathbf{A}\) form a basis of \(R^N\).

\item {} 
The columns of \(\mathbf{A}\) form a basis of \(R^N\).

\item {} 
\(\mathbf{A}\) is a product of elementary matrices.

\item {} 
\(0\) is not eigenvalue of \(\mathbf{A}\).

\end{itemize}


\section{Numpy and arrays}
\label{\detokenize{chapter1:numpy-and-arrays}}
\sphinxhref{http://www.numpy.org/}{Numpy} provides an easy way to handle arrays in Python. The standard way to import this library is as

\begin{sphinxVerbatim}[commandchars=\\\{\}]
\PYG{k+kn}{import} \PYG{n+nn}{numpy} \PYG{k}{as} \PYG{n+nn}{np}
\end{sphinxVerbatim}

Here follows a simple example where we set up an array of ten elements, all determined by random numbers drawn according to the normal distribution,

\begin{sphinxVerbatim}[commandchars=\\\{\}]
\PYG{n}{n} \PYG{o}{=} \PYG{l+m+mi}{10}
\PYG{n}{x} \PYG{o}{=} \PYG{n}{np}\PYG{o}{.}\PYG{n}{random}\PYG{o}{.}\PYG{n}{normal}\PYG{p}{(}\PYG{n}{size}\PYG{o}{=}\PYG{n}{n}\PYG{p}{)}
\PYG{n+nb}{print}\PYG{p}{(}\PYG{n}{x}\PYG{p}{)}
\end{sphinxVerbatim}

\begin{sphinxVerbatim}[commandchars=\\\{\}]
[ 0.4920762  \PYGZhy{}0.77808351  0.4639126   1.3043923   1.33905074 \PYGZhy{}0.49093788
 \PYGZhy{}0.9752277   0.0185915   0.91630589  1.52735265]
\end{sphinxVerbatim}

We defined a vector \(x\) with \(n=10\) elements with its values given by the Normal distribution \(N(0,1)\).
Another alternative is to declare a vector as follows

\begin{sphinxVerbatim}[commandchars=\\\{\}]
\PYG{k+kn}{import} \PYG{n+nn}{numpy} \PYG{k}{as} \PYG{n+nn}{np}
\PYG{n}{x} \PYG{o}{=} \PYG{n}{np}\PYG{o}{.}\PYG{n}{array}\PYG{p}{(}\PYG{p}{[}\PYG{l+m+mi}{1}\PYG{p}{,} \PYG{l+m+mi}{2}\PYG{p}{,} \PYG{l+m+mi}{3}\PYG{p}{]}\PYG{p}{)}
\PYG{n+nb}{print}\PYG{p}{(}\PYG{n}{x}\PYG{p}{)}
\end{sphinxVerbatim}

\begin{sphinxVerbatim}[commandchars=\\\{\}]
[1 2 3]
\end{sphinxVerbatim}

Here we have defined a vector with three elements, with \(x_0=1\), \(x_1=2\) and \(x_2=3\). Note that both Python and C++
start numbering array elements from \(0\) and on. This means that a vector with \(n\) elements has a sequence of entities \(x_0, x_1, x_2, \dots, x_{n-1}\). We could also let (recommended) Numpy to compute the logarithms of a specific array as

\begin{sphinxVerbatim}[commandchars=\\\{\}]
\PYG{k+kn}{import} \PYG{n+nn}{numpy} \PYG{k}{as} \PYG{n+nn}{np}
\PYG{n}{x} \PYG{o}{=} \PYG{n}{np}\PYG{o}{.}\PYG{n}{log}\PYG{p}{(}\PYG{n}{np}\PYG{o}{.}\PYG{n}{array}\PYG{p}{(}\PYG{p}{[}\PYG{l+m+mi}{4}\PYG{p}{,} \PYG{l+m+mi}{7}\PYG{p}{,} \PYG{l+m+mi}{8}\PYG{p}{]}\PYG{p}{)}\PYG{p}{)}
\PYG{n+nb}{print}\PYG{p}{(}\PYG{n}{x}\PYG{p}{)}
\end{sphinxVerbatim}

\begin{sphinxVerbatim}[commandchars=\\\{\}]
[1.38629436 1.94591015 2.07944154]
\end{sphinxVerbatim}

In the last example we used Numpy’s unary function \(np.log\). This function is
highly tuned to compute array elements since the code is vectorized
and does not require looping. We normaly recommend that you use the
Numpy intrinsic functions instead of the corresponding \sphinxstylestrong{log} function
from Python’s \sphinxstylestrong{math} module. The looping is done explicitely by the
\sphinxstylestrong{np.log} function. The alternative, and slower way to compute the
logarithms of a vector would be to write

\begin{sphinxVerbatim}[commandchars=\\\{\}]
\PYG{k+kn}{import} \PYG{n+nn}{numpy} \PYG{k}{as} \PYG{n+nn}{np}
\PYG{k+kn}{from} \PYG{n+nn}{math} \PYG{k+kn}{import} \PYG{n}{log}
\PYG{n}{x} \PYG{o}{=} \PYG{n}{np}\PYG{o}{.}\PYG{n}{array}\PYG{p}{(}\PYG{p}{[}\PYG{l+m+mi}{4}\PYG{p}{,} \PYG{l+m+mi}{7}\PYG{p}{,} \PYG{l+m+mi}{8}\PYG{p}{]}\PYG{p}{)}
\PYG{k}{for} \PYG{n}{i} \PYG{o+ow}{in} \PYG{n+nb}{range}\PYG{p}{(}\PYG{l+m+mi}{0}\PYG{p}{,} \PYG{n+nb}{len}\PYG{p}{(}\PYG{n}{x}\PYG{p}{)}\PYG{p}{)}\PYG{p}{:}
    \PYG{n}{x}\PYG{p}{[}\PYG{n}{i}\PYG{p}{]} \PYG{o}{=} \PYG{n}{log}\PYG{p}{(}\PYG{n}{x}\PYG{p}{[}\PYG{n}{i}\PYG{p}{]}\PYG{p}{)}
\PYG{n+nb}{print}\PYG{p}{(}\PYG{n}{x}\PYG{p}{)}
\end{sphinxVerbatim}

\begin{sphinxVerbatim}[commandchars=\\\{\}]
[1 1 2]
\end{sphinxVerbatim}

We note that our code is much longer already and we need to import the \sphinxstylestrong{log} function from the \sphinxstylestrong{math} module.
The attentive reader will also notice that the output is \([1, 1, 2]\). Python interprets automagically our numbers as integers (like the \sphinxstylestrong{automatic} keyword in C++). To change this we could define our array elements to be double precision numbers as

\begin{sphinxVerbatim}[commandchars=\\\{\}]
\PYG{k+kn}{import} \PYG{n+nn}{numpy} \PYG{k}{as} \PYG{n+nn}{np}
\PYG{n}{x} \PYG{o}{=} \PYG{n}{np}\PYG{o}{.}\PYG{n}{log}\PYG{p}{(}\PYG{n}{np}\PYG{o}{.}\PYG{n}{array}\PYG{p}{(}\PYG{p}{[}\PYG{l+m+mi}{4}\PYG{p}{,} \PYG{l+m+mi}{7}\PYG{p}{,} \PYG{l+m+mi}{8}\PYG{p}{]}\PYG{p}{,} \PYG{n}{dtype} \PYG{o}{=} \PYG{n}{np}\PYG{o}{.}\PYG{n}{float64}\PYG{p}{)}\PYG{p}{)}
\PYG{n+nb}{print}\PYG{p}{(}\PYG{n}{x}\PYG{p}{)}
\end{sphinxVerbatim}

\begin{sphinxVerbatim}[commandchars=\\\{\}]
[1.38629436 1.94591015 2.07944154]
\end{sphinxVerbatim}

or simply write them as double precision numbers (Python uses 64 bits as default for floating point type variables), that is

\begin{sphinxVerbatim}[commandchars=\\\{\}]
\PYG{k+kn}{import} \PYG{n+nn}{numpy} \PYG{k}{as} \PYG{n+nn}{np}
\PYG{n}{x} \PYG{o}{=} \PYG{n}{np}\PYG{o}{.}\PYG{n}{log}\PYG{p}{(}\PYG{n}{np}\PYG{o}{.}\PYG{n}{array}\PYG{p}{(}\PYG{p}{[}\PYG{l+m+mf}{4.0}\PYG{p}{,} \PYG{l+m+mf}{7.0}\PYG{p}{,} \PYG{l+m+mf}{8.0}\PYG{p}{]}\PYG{p}{)}
\PYG{n+nb}{print}\PYG{p}{(}\PYG{n}{x}\PYG{p}{)}
\end{sphinxVerbatim}

\begin{sphinxVerbatim}[commandchars=\\\{\}]
\PYG{g+gt}{  File}\PYG{n+nn}{ \PYGZdq{}\PYGZlt{}ipython\PYGZhy{}input\PYGZhy{}21\PYGZhy{}f6d7a289d493\PYGZgt{}\PYGZdq{}}\PYG{g+gt}{, line }\PYG{l+m+mi}{3}
    \PYG{n+nb}{print}\PYG{p}{(}\PYG{n}{x}\PYG{p}{)}
    \PYG{o}{\PYGZca{}}
\PYG{n+ne}{SyntaxError}: invalid syntax
\end{sphinxVerbatim}

To check the number of bytes (remember that one byte contains eight bits for double precision variables), you can use simple use the \sphinxstylestrong{itemsize} functionality (the array \(x\) is actually an object which inherits the functionalities defined in Numpy) as

\begin{sphinxVerbatim}[commandchars=\\\{\}]
\PYG{k+kn}{import} \PYG{n+nn}{numpy} \PYG{k}{as} \PYG{n+nn}{np}
\PYG{n}{x} \PYG{o}{=} \PYG{n}{np}\PYG{o}{.}\PYG{n}{log}\PYG{p}{(}\PYG{n}{np}\PYG{o}{.}\PYG{n}{array}\PYG{p}{(}\PYG{p}{[}\PYG{l+m+mf}{4.0}\PYG{p}{,} \PYG{l+m+mf}{7.0}\PYG{p}{,} \PYG{l+m+mf}{8.0}\PYG{p}{]}\PYG{p}{)}
\PYG{n+nb}{print}\PYG{p}{(}\PYG{n}{x}\PYG{o}{.}\PYG{n}{itemsize}\PYG{p}{)}
\end{sphinxVerbatim}


\section{Matrices in Python}
\label{\detokenize{chapter1:matrices-in-python}}
Having defined vectors, we are now ready to try out matrices. We can
define a \(3 \times 3 \) real matrix \(\hat{A}\) as (recall that we user
lowercase letters for vectors and uppercase letters for matrices)

\begin{sphinxVerbatim}[commandchars=\\\{\}]
\PYG{k+kn}{import} \PYG{n+nn}{numpy} \PYG{k}{as} \PYG{n+nn}{np}
\PYG{n}{A} \PYG{o}{=} \PYG{n}{np}\PYG{o}{.}\PYG{n}{log}\PYG{p}{(}\PYG{n}{np}\PYG{o}{.}\PYG{n}{array}\PYG{p}{(}\PYG{p}{[} \PYG{p}{[}\PYG{l+m+mf}{4.0}\PYG{p}{,} \PYG{l+m+mf}{7.0}\PYG{p}{,} \PYG{l+m+mf}{8.0}\PYG{p}{]}\PYG{p}{,} \PYG{p}{[}\PYG{l+m+mf}{3.0}\PYG{p}{,} \PYG{l+m+mf}{10.0}\PYG{p}{,} \PYG{l+m+mf}{11.0}\PYG{p}{]}\PYG{p}{,} \PYG{p}{[}\PYG{l+m+mf}{4.0}\PYG{p}{,} \PYG{l+m+mf}{5.0}\PYG{p}{,} \PYG{l+m+mf}{7.0}\PYG{p}{]} \PYG{p}{]}\PYG{p}{)}\PYG{p}{)}
\PYG{n+nb}{print}\PYG{p}{(}\PYG{n}{A}\PYG{p}{)}
\end{sphinxVerbatim}

If we use the \sphinxstylestrong{shape} function we would get \((3, 3)\) as output, that is verifying that our matrix is a \(3\times 3\) matrix. We can slice the matrix and print for example the first column (Python organized matrix elements in a row\sphinxhyphen{}major order, see below) as

\begin{sphinxVerbatim}[commandchars=\\\{\}]
\PYG{k+kn}{import} \PYG{n+nn}{numpy} \PYG{k}{as} \PYG{n+nn}{np}
\PYG{n}{A} \PYG{o}{=} \PYG{n}{np}\PYG{o}{.}\PYG{n}{log}\PYG{p}{(}\PYG{n}{np}\PYG{o}{.}\PYG{n}{array}\PYG{p}{(}\PYG{p}{[} \PYG{p}{[}\PYG{l+m+mf}{4.0}\PYG{p}{,} \PYG{l+m+mf}{7.0}\PYG{p}{,} \PYG{l+m+mf}{8.0}\PYG{p}{]}\PYG{p}{,} \PYG{p}{[}\PYG{l+m+mf}{3.0}\PYG{p}{,} \PYG{l+m+mf}{10.0}\PYG{p}{,} \PYG{l+m+mf}{11.0}\PYG{p}{]}\PYG{p}{,} \PYG{p}{[}\PYG{l+m+mf}{4.0}\PYG{p}{,} \PYG{l+m+mf}{5.0}\PYG{p}{,} \PYG{l+m+mf}{7.0}\PYG{p}{]} \PYG{p}{]}\PYG{p}{)}\PYG{p}{)}
\PYG{c+c1}{\PYGZsh{} print the first column, row\PYGZhy{}major order and elements start with 0}
\PYG{n+nb}{print}\PYG{p}{(}\PYG{n}{A}\PYG{p}{[}\PYG{p}{:}\PYG{p}{,}\PYG{l+m+mi}{0}\PYG{p}{]}\PYG{p}{)}
\end{sphinxVerbatim}

We can continue this was by printing out other columns or rows. The example here prints out the second column

\begin{sphinxVerbatim}[commandchars=\\\{\}]
\PYG{k+kn}{import} \PYG{n+nn}{numpy} \PYG{k}{as} \PYG{n+nn}{np}
\PYG{n}{A} \PYG{o}{=} \PYG{n}{np}\PYG{o}{.}\PYG{n}{log}\PYG{p}{(}\PYG{n}{np}\PYG{o}{.}\PYG{n}{array}\PYG{p}{(}\PYG{p}{[} \PYG{p}{[}\PYG{l+m+mf}{4.0}\PYG{p}{,} \PYG{l+m+mf}{7.0}\PYG{p}{,} \PYG{l+m+mf}{8.0}\PYG{p}{]}\PYG{p}{,} \PYG{p}{[}\PYG{l+m+mf}{3.0}\PYG{p}{,} \PYG{l+m+mf}{10.0}\PYG{p}{,} \PYG{l+m+mf}{11.0}\PYG{p}{]}\PYG{p}{,} \PYG{p}{[}\PYG{l+m+mf}{4.0}\PYG{p}{,} \PYG{l+m+mf}{5.0}\PYG{p}{,} \PYG{l+m+mf}{7.0}\PYG{p}{]} \PYG{p}{]}\PYG{p}{)}\PYG{p}{)}
\PYG{c+c1}{\PYGZsh{} print the first column, row\PYGZhy{}major order and elements start with 0}
\PYG{n+nb}{print}\PYG{p}{(}\PYG{n}{A}\PYG{p}{[}\PYG{l+m+mi}{1}\PYG{p}{,}\PYG{p}{:}\PYG{p}{]}\PYG{p}{)}
\end{sphinxVerbatim}

Numpy contains many other functionalities that allow us to slice, subdivide etc etc arrays. We strongly recommend that you look up the \sphinxhref{http://www.numpy.org/}{Numpy website for more details}. Useful functions when defining a matrix are the \sphinxstylestrong{np.zeros} function which declares a matrix of a given dimension and sets all elements to zero

\begin{sphinxVerbatim}[commandchars=\\\{\}]
\PYG{k+kn}{import} \PYG{n+nn}{numpy} \PYG{k}{as} \PYG{n+nn}{np}
\PYG{n}{n} \PYG{o}{=} \PYG{l+m+mi}{10}
\PYG{c+c1}{\PYGZsh{} define a matrix of dimension 10 x 10 and set all elements to zero}
\PYG{n}{A} \PYG{o}{=} \PYG{n}{np}\PYG{o}{.}\PYG{n}{zeros}\PYG{p}{(} \PYG{p}{(}\PYG{n}{n}\PYG{p}{,} \PYG{n}{n}\PYG{p}{)} \PYG{p}{)}
\PYG{n+nb}{print}\PYG{p}{(}\PYG{n}{A}\PYG{p}{)}
\end{sphinxVerbatim}

or initializing all elements to

\begin{sphinxVerbatim}[commandchars=\\\{\}]
\PYG{k+kn}{import} \PYG{n+nn}{numpy} \PYG{k}{as} \PYG{n+nn}{np}
\PYG{n}{n} \PYG{o}{=} \PYG{l+m+mi}{10}
\PYG{c+c1}{\PYGZsh{} define a matrix of dimension 10 x 10 and set all elements to one}
\PYG{n}{A} \PYG{o}{=} \PYG{n}{np}\PYG{o}{.}\PYG{n}{ones}\PYG{p}{(} \PYG{p}{(}\PYG{n}{n}\PYG{p}{,} \PYG{n}{n}\PYG{p}{)} \PYG{p}{)}
\PYG{n+nb}{print}\PYG{p}{(}\PYG{n}{A}\PYG{p}{)}
\end{sphinxVerbatim}

or as unitarily distributed random numbers (see the material on random number generators in the statistics part)

\begin{sphinxVerbatim}[commandchars=\\\{\}]
\PYG{k+kn}{import} \PYG{n+nn}{numpy} \PYG{k}{as} \PYG{n+nn}{np}
\PYG{n}{n} \PYG{o}{=} \PYG{l+m+mi}{10}
\PYG{c+c1}{\PYGZsh{} define a matrix of dimension 10 x 10 and set all elements to random numbers with x \PYGZbs{}in [0, 1]}
\PYG{n}{A} \PYG{o}{=} \PYG{n}{np}\PYG{o}{.}\PYG{n}{random}\PYG{o}{.}\PYG{n}{rand}\PYG{p}{(}\PYG{n}{n}\PYG{p}{,} \PYG{n}{n}\PYG{p}{)}
\PYG{n+nb}{print}\PYG{p}{(}\PYG{n}{A}\PYG{p}{)}
\end{sphinxVerbatim}


\section{Meet the Pandas}
\label{\detokenize{chapter1:meet-the-pandas}}








Another useful Python package is
\sphinxhref{https://pandas.pydata.org/}{pandas}, which is an open source library
providing high\sphinxhyphen{}performance, easy\sphinxhyphen{}to\sphinxhyphen{}use data structures and data
analysis tools for Python. \sphinxstylestrong{pandas} stands for panel data, a term borrowed from econometrics and is an efficient library for data analysis with an emphasis on tabular data.
\sphinxstylestrong{pandas} has two major classes, the \sphinxstylestrong{DataFrame} class with two\sphinxhyphen{}dimensional data objects and tabular data organized in columns and the class \sphinxstylestrong{Series} with a focus on one\sphinxhyphen{}dimensional data objects. Both classes allow you to index data easily as we will see in the examples below.
\sphinxstylestrong{pandas} allows you also to perform mathematical operations on the data, spanning from simple reshapings of vectors and matrices to statistical operations.

The following simple example shows how we can, in an easy way make tables of our data. Here we define a data set which includes names, place of birth and date of birth, and displays the data in an easy to read way. We will see repeated use of \sphinxstylestrong{pandas}, in particular in connection with classification of data.

\begin{sphinxVerbatim}[commandchars=\\\{\}]
\PYG{k+kn}{import} \PYG{n+nn}{pandas} \PYG{k}{as} \PYG{n+nn}{pd}
\PYG{k+kn}{from} \PYG{n+nn}{IPython}\PYG{n+nn}{.}\PYG{n+nn}{display} \PYG{k+kn}{import} \PYG{n}{display}
\PYG{n}{data} \PYG{o}{=} \PYG{p}{\PYGZob{}}\PYG{l+s+s1}{\PYGZsq{}}\PYG{l+s+s1}{First Name}\PYG{l+s+s1}{\PYGZsq{}}\PYG{p}{:} \PYG{p}{[}\PYG{l+s+s2}{\PYGZdq{}}\PYG{l+s+s2}{Frodo}\PYG{l+s+s2}{\PYGZdq{}}\PYG{p}{,} \PYG{l+s+s2}{\PYGZdq{}}\PYG{l+s+s2}{Bilbo}\PYG{l+s+s2}{\PYGZdq{}}\PYG{p}{,} \PYG{l+s+s2}{\PYGZdq{}}\PYG{l+s+s2}{Aragorn II}\PYG{l+s+s2}{\PYGZdq{}}\PYG{p}{,} \PYG{l+s+s2}{\PYGZdq{}}\PYG{l+s+s2}{Samwise}\PYG{l+s+s2}{\PYGZdq{}}\PYG{p}{]}\PYG{p}{,}
        \PYG{l+s+s1}{\PYGZsq{}}\PYG{l+s+s1}{Last Name}\PYG{l+s+s1}{\PYGZsq{}}\PYG{p}{:} \PYG{p}{[}\PYG{l+s+s2}{\PYGZdq{}}\PYG{l+s+s2}{Baggins}\PYG{l+s+s2}{\PYGZdq{}}\PYG{p}{,} \PYG{l+s+s2}{\PYGZdq{}}\PYG{l+s+s2}{Baggins}\PYG{l+s+s2}{\PYGZdq{}}\PYG{p}{,}\PYG{l+s+s2}{\PYGZdq{}}\PYG{l+s+s2}{Elessar}\PYG{l+s+s2}{\PYGZdq{}}\PYG{p}{,}\PYG{l+s+s2}{\PYGZdq{}}\PYG{l+s+s2}{Gamgee}\PYG{l+s+s2}{\PYGZdq{}}\PYG{p}{]}\PYG{p}{,}
        \PYG{l+s+s1}{\PYGZsq{}}\PYG{l+s+s1}{Place of birth}\PYG{l+s+s1}{\PYGZsq{}}\PYG{p}{:} \PYG{p}{[}\PYG{l+s+s2}{\PYGZdq{}}\PYG{l+s+s2}{Shire}\PYG{l+s+s2}{\PYGZdq{}}\PYG{p}{,} \PYG{l+s+s2}{\PYGZdq{}}\PYG{l+s+s2}{Shire}\PYG{l+s+s2}{\PYGZdq{}}\PYG{p}{,} \PYG{l+s+s2}{\PYGZdq{}}\PYG{l+s+s2}{Eriador}\PYG{l+s+s2}{\PYGZdq{}}\PYG{p}{,} \PYG{l+s+s2}{\PYGZdq{}}\PYG{l+s+s2}{Shire}\PYG{l+s+s2}{\PYGZdq{}}\PYG{p}{]}\PYG{p}{,}
        \PYG{l+s+s1}{\PYGZsq{}}\PYG{l+s+s1}{Date of Birth T.A.}\PYG{l+s+s1}{\PYGZsq{}}\PYG{p}{:} \PYG{p}{[}\PYG{l+m+mi}{2968}\PYG{p}{,} \PYG{l+m+mi}{2890}\PYG{p}{,} \PYG{l+m+mi}{2931}\PYG{p}{,} \PYG{l+m+mi}{2980}\PYG{p}{]}
        \PYG{p}{\PYGZcb{}}
\PYG{n}{data\PYGZus{}pandas} \PYG{o}{=} \PYG{n}{pd}\PYG{o}{.}\PYG{n}{DataFrame}\PYG{p}{(}\PYG{n}{data}\PYG{p}{)}
\PYG{n}{display}\PYG{p}{(}\PYG{n}{data\PYGZus{}pandas}\PYG{p}{)}
\end{sphinxVerbatim}

In the above we have imported \sphinxstylestrong{pandas} with the shorthand \sphinxstylestrong{pd}, the latter has become the standard way we import \sphinxstylestrong{pandas}. We make then a list of various variables
and reorganize the above lists into a \sphinxstylestrong{DataFrame} and then print out  a neat table with specific column labels as \sphinxstyleemphasis{Name}, \sphinxstyleemphasis{place of birth} and \sphinxstyleemphasis{date of birth}.
Displaying these results, we see that the indices are given by the default numbers from zero to three.
\sphinxstylestrong{pandas} is extremely flexible and we can easily change the above indices by defining a new type of indexing as

\begin{sphinxVerbatim}[commandchars=\\\{\}]
\PYG{n}{data\PYGZus{}pandas} \PYG{o}{=} \PYG{n}{pd}\PYG{o}{.}\PYG{n}{DataFrame}\PYG{p}{(}\PYG{n}{data}\PYG{p}{,}\PYG{n}{index}\PYG{o}{=}\PYG{p}{[}\PYG{l+s+s1}{\PYGZsq{}}\PYG{l+s+s1}{Frodo}\PYG{l+s+s1}{\PYGZsq{}}\PYG{p}{,}\PYG{l+s+s1}{\PYGZsq{}}\PYG{l+s+s1}{Bilbo}\PYG{l+s+s1}{\PYGZsq{}}\PYG{p}{,}\PYG{l+s+s1}{\PYGZsq{}}\PYG{l+s+s1}{Aragorn}\PYG{l+s+s1}{\PYGZsq{}}\PYG{p}{,}\PYG{l+s+s1}{\PYGZsq{}}\PYG{l+s+s1}{Sam}\PYG{l+s+s1}{\PYGZsq{}}\PYG{p}{]}\PYG{p}{)}
\PYG{n}{display}\PYG{p}{(}\PYG{n}{data\PYGZus{}pandas}\PYG{p}{)}
\end{sphinxVerbatim}

Thereafter we display the content of the row which begins with the index \sphinxstylestrong{Aragorn}

\begin{sphinxVerbatim}[commandchars=\\\{\}]
\PYG{n}{display}\PYG{p}{(}\PYG{n}{data\PYGZus{}pandas}\PYG{o}{.}\PYG{n}{loc}\PYG{p}{[}\PYG{l+s+s1}{\PYGZsq{}}\PYG{l+s+s1}{Aragorn}\PYG{l+s+s1}{\PYGZsq{}}\PYG{p}{]}\PYG{p}{)}
\end{sphinxVerbatim}

We can easily append data to this, for example

\begin{sphinxVerbatim}[commandchars=\\\{\}]
\PYG{n}{new\PYGZus{}hobbit} \PYG{o}{=} \PYG{p}{\PYGZob{}}\PYG{l+s+s1}{\PYGZsq{}}\PYG{l+s+s1}{First Name}\PYG{l+s+s1}{\PYGZsq{}}\PYG{p}{:} \PYG{p}{[}\PYG{l+s+s2}{\PYGZdq{}}\PYG{l+s+s2}{Peregrin}\PYG{l+s+s2}{\PYGZdq{}}\PYG{p}{]}\PYG{p}{,}
              \PYG{l+s+s1}{\PYGZsq{}}\PYG{l+s+s1}{Last Name}\PYG{l+s+s1}{\PYGZsq{}}\PYG{p}{:} \PYG{p}{[}\PYG{l+s+s2}{\PYGZdq{}}\PYG{l+s+s2}{Took}\PYG{l+s+s2}{\PYGZdq{}}\PYG{p}{]}\PYG{p}{,}
              \PYG{l+s+s1}{\PYGZsq{}}\PYG{l+s+s1}{Place of birth}\PYG{l+s+s1}{\PYGZsq{}}\PYG{p}{:} \PYG{p}{[}\PYG{l+s+s2}{\PYGZdq{}}\PYG{l+s+s2}{Shire}\PYG{l+s+s2}{\PYGZdq{}}\PYG{p}{]}\PYG{p}{,}
              \PYG{l+s+s1}{\PYGZsq{}}\PYG{l+s+s1}{Date of Birth T.A.}\PYG{l+s+s1}{\PYGZsq{}}\PYG{p}{:} \PYG{p}{[}\PYG{l+m+mi}{2990}\PYG{p}{]}
              \PYG{p}{\PYGZcb{}}
\PYG{n}{data\PYGZus{}pandas}\PYG{o}{=}\PYG{n}{data\PYGZus{}pandas}\PYG{o}{.}\PYG{n}{append}\PYG{p}{(}\PYG{n}{pd}\PYG{o}{.}\PYG{n}{DataFrame}\PYG{p}{(}\PYG{n}{new\PYGZus{}hobbit}\PYG{p}{,} \PYG{n}{index}\PYG{o}{=}\PYG{p}{[}\PYG{l+s+s1}{\PYGZsq{}}\PYG{l+s+s1}{Pippin}\PYG{l+s+s1}{\PYGZsq{}}\PYG{p}{]}\PYG{p}{)}\PYG{p}{)}
\PYG{n}{display}\PYG{p}{(}\PYG{n}{data\PYGZus{}pandas}\PYG{p}{)}
\end{sphinxVerbatim}

Here are other examples where we use the \sphinxstylestrong{DataFrame} functionality to handle arrays, now with more interesting features for us, namely numbers. We set up a matrix
of dimensionality \(10\times 5\) and compute the mean value and standard deviation of each column. Similarly, we can perform mathematial operations like squaring the matrix elements and many other operations.

\begin{sphinxVerbatim}[commandchars=\\\{\}]
\PYG{k+kn}{import} \PYG{n+nn}{numpy} \PYG{k}{as} \PYG{n+nn}{np}
\PYG{k+kn}{import} \PYG{n+nn}{pandas} \PYG{k}{as} \PYG{n+nn}{pd}
\PYG{k+kn}{from} \PYG{n+nn}{IPython}\PYG{n+nn}{.}\PYG{n+nn}{display} \PYG{k+kn}{import} \PYG{n}{display}
\PYG{n}{np}\PYG{o}{.}\PYG{n}{random}\PYG{o}{.}\PYG{n}{seed}\PYG{p}{(}\PYG{l+m+mi}{100}\PYG{p}{)}
\PYG{c+c1}{\PYGZsh{} setting up a 10 x 5 matrix}
\PYG{n}{rows} \PYG{o}{=} \PYG{l+m+mi}{10}
\PYG{n}{cols} \PYG{o}{=} \PYG{l+m+mi}{5}
\PYG{n}{a} \PYG{o}{=} \PYG{n}{np}\PYG{o}{.}\PYG{n}{random}\PYG{o}{.}\PYG{n}{randn}\PYG{p}{(}\PYG{n}{rows}\PYG{p}{,}\PYG{n}{cols}\PYG{p}{)}
\PYG{n}{df} \PYG{o}{=} \PYG{n}{pd}\PYG{o}{.}\PYG{n}{DataFrame}\PYG{p}{(}\PYG{n}{a}\PYG{p}{)}
\PYG{n}{display}\PYG{p}{(}\PYG{n}{df}\PYG{p}{)}
\PYG{n+nb}{print}\PYG{p}{(}\PYG{n}{df}\PYG{o}{.}\PYG{n}{mean}\PYG{p}{(}\PYG{p}{)}\PYG{p}{)}
\PYG{n+nb}{print}\PYG{p}{(}\PYG{n}{df}\PYG{o}{.}\PYG{n}{std}\PYG{p}{(}\PYG{p}{)}\PYG{p}{)}
\PYG{n}{display}\PYG{p}{(}\PYG{n}{df}\PYG{o}{*}\PYG{o}{*}\PYG{l+m+mi}{2}\PYG{p}{)}
\end{sphinxVerbatim}

Thereafter we can select specific columns only and plot final results

\begin{sphinxVerbatim}[commandchars=\\\{\}]
\PYG{n}{df}\PYG{o}{.}\PYG{n}{columns} \PYG{o}{=} \PYG{p}{[}\PYG{l+s+s1}{\PYGZsq{}}\PYG{l+s+s1}{First}\PYG{l+s+s1}{\PYGZsq{}}\PYG{p}{,} \PYG{l+s+s1}{\PYGZsq{}}\PYG{l+s+s1}{Second}\PYG{l+s+s1}{\PYGZsq{}}\PYG{p}{,} \PYG{l+s+s1}{\PYGZsq{}}\PYG{l+s+s1}{Third}\PYG{l+s+s1}{\PYGZsq{}}\PYG{p}{,} \PYG{l+s+s1}{\PYGZsq{}}\PYG{l+s+s1}{Fourth}\PYG{l+s+s1}{\PYGZsq{}}\PYG{p}{,} \PYG{l+s+s1}{\PYGZsq{}}\PYG{l+s+s1}{Fifth}\PYG{l+s+s1}{\PYGZsq{}}\PYG{p}{]}
\PYG{n}{df}\PYG{o}{.}\PYG{n}{index} \PYG{o}{=} \PYG{n}{np}\PYG{o}{.}\PYG{n}{arange}\PYG{p}{(}\PYG{l+m+mi}{10}\PYG{p}{)}

\PYG{n}{display}\PYG{p}{(}\PYG{n}{df}\PYG{p}{)}
\PYG{n+nb}{print}\PYG{p}{(}\PYG{n}{df}\PYG{p}{[}\PYG{l+s+s1}{\PYGZsq{}}\PYG{l+s+s1}{Second}\PYG{l+s+s1}{\PYGZsq{}}\PYG{p}{]}\PYG{o}{.}\PYG{n}{mean}\PYG{p}{(}\PYG{p}{)} \PYG{p}{)}
\PYG{n+nb}{print}\PYG{p}{(}\PYG{n}{df}\PYG{o}{.}\PYG{n}{info}\PYG{p}{(}\PYG{p}{)}\PYG{p}{)}
\PYG{n+nb}{print}\PYG{p}{(}\PYG{n}{df}\PYG{o}{.}\PYG{n}{describe}\PYG{p}{(}\PYG{p}{)}\PYG{p}{)}

\PYG{k+kn}{from} \PYG{n+nn}{pylab} \PYG{k+kn}{import} \PYG{n}{plt}\PYG{p}{,} \PYG{n}{mpl}
\PYG{n}{plt}\PYG{o}{.}\PYG{n}{style}\PYG{o}{.}\PYG{n}{use}\PYG{p}{(}\PYG{l+s+s1}{\PYGZsq{}}\PYG{l+s+s1}{seaborn}\PYG{l+s+s1}{\PYGZsq{}}\PYG{p}{)}
\PYG{n}{mpl}\PYG{o}{.}\PYG{n}{rcParams}\PYG{p}{[}\PYG{l+s+s1}{\PYGZsq{}}\PYG{l+s+s1}{font.family}\PYG{l+s+s1}{\PYGZsq{}}\PYG{p}{]} \PYG{o}{=} \PYG{l+s+s1}{\PYGZsq{}}\PYG{l+s+s1}{serif}\PYG{l+s+s1}{\PYGZsq{}}

\PYG{n}{df}\PYG{o}{.}\PYG{n}{cumsum}\PYG{p}{(}\PYG{p}{)}\PYG{o}{.}\PYG{n}{plot}\PYG{p}{(}\PYG{n}{lw}\PYG{o}{=}\PYG{l+m+mf}{2.0}\PYG{p}{,} \PYG{n}{figsize}\PYG{o}{=}\PYG{p}{(}\PYG{l+m+mi}{10}\PYG{p}{,}\PYG{l+m+mi}{6}\PYG{p}{)}\PYG{p}{)}
\PYG{n}{plt}\PYG{o}{.}\PYG{n}{show}\PYG{p}{(}\PYG{p}{)}


\PYG{n}{df}\PYG{o}{.}\PYG{n}{plot}\PYG{o}{.}\PYG{n}{bar}\PYG{p}{(}\PYG{n}{figsize}\PYG{o}{=}\PYG{p}{(}\PYG{l+m+mi}{10}\PYG{p}{,}\PYG{l+m+mi}{6}\PYG{p}{)}\PYG{p}{,} \PYG{n}{rot}\PYG{o}{=}\PYG{l+m+mi}{15}\PYG{p}{)}
\PYG{n}{plt}\PYG{o}{.}\PYG{n}{show}\PYG{p}{(}\PYG{p}{)}
\end{sphinxVerbatim}

We can produce a \(4\times 4\) matrix

\begin{sphinxVerbatim}[commandchars=\\\{\}]
\PYG{n}{b} \PYG{o}{=} \PYG{n}{np}\PYG{o}{.}\PYG{n}{arange}\PYG{p}{(}\PYG{l+m+mi}{16}\PYG{p}{)}\PYG{o}{.}\PYG{n}{reshape}\PYG{p}{(}\PYG{p}{(}\PYG{l+m+mi}{4}\PYG{p}{,}\PYG{l+m+mi}{4}\PYG{p}{)}\PYG{p}{)}
\PYG{n+nb}{print}\PYG{p}{(}\PYG{n}{b}\PYG{p}{)}
\PYG{n}{df1} \PYG{o}{=} \PYG{n}{pd}\PYG{o}{.}\PYG{n}{DataFrame}\PYG{p}{(}\PYG{n}{b}\PYG{p}{)}
\PYG{n+nb}{print}\PYG{p}{(}\PYG{n}{df1}\PYG{p}{)}
\end{sphinxVerbatim}

and many other operations.

The \sphinxstylestrong{Series} class is another important class included in
\sphinxstylestrong{pandas}. You can view it as a specialization of \sphinxstylestrong{DataFrame} but where
we have just a single column of data. It shares many of the same features as \_DataFrame. As with \sphinxstylestrong{DataFrame},
most operations are vectorized, achieving thereby a high performance when dealing with computations of arrays, in particular labeled arrays.
As we will see below it leads also to a very concice code close to the mathematical operations we may be interested in.
For multidimensional arrays, we recommend strongly \sphinxhref{http://xarray.pydata.org/en/stable/}{xarray}. \sphinxstylestrong{xarray} has much of the same flexibility as \sphinxstylestrong{pandas}, but allows for the extension to higher dimensions than two.


\chapter{Basic Steps of Scientific Investigations}
\label{\detokenize{chapter1:basic-steps-of-scientific-investigations}}
An overarching aim in this course is to give you a deeper
understanding of the scientific method. The problems we study will all
involve cases where we can apply classical mechanics. In our previous
material we already assumed that we had a model for the motion of an
object.  Alternatively we could have data from experiment (like Usain
Bolt’s 100m world record run in 2008).  Or we could have performed
ourselves an experiment and we want to understand which forces are at
play and whether these forces can be understood in terms of
fundamental forces.

Our first step consists in identifying the problem. What we sketch
here may include a mix of experiment and theoretical simulations, or
just experiment or only theory.


\section{Identifying our System}
\label{\detokenize{chapter1:identifying-our-system}}
Here we can ask questions like
\begin{enumerate}
\sphinxsetlistlabels{\arabic}{enumi}{enumii}{}{.}%
\item {} 
What kind of object is moving

\item {} 
What kind of data do we have

\item {} 
How do we measure position, velocity, acceleration etc

\item {} 
Which initial conditions influence our system

\item {} 
Other aspects which allow us to identify the system

\end{enumerate}


\section{Defining a Model}
\label{\detokenize{chapter1:defining-a-model}}
With our eventual data and observations we would now like to develop a
model for the system. In the end we want obviously to be able to
understand which forces are at play and how they influence our
specific system. That is, can we extract some deeper insights about a
system?

We need then to
\begin{enumerate}
\sphinxsetlistlabels{\arabic}{enumi}{enumii}{}{.}%
\item {} 
Find the forces that act on our system

\item {} 
Introduce models for the forces

\item {} 
Identify the equations which can govern the system (Newton’s second law for example)

\item {} 
More elements we deem important for defining our model

\end{enumerate}


\section{Solving the Equations}
\label{\detokenize{chapter1:solving-the-equations}}
With the model at hand, we can then solve the equations. In classical mechanics we normally end up  with solving sets of coupled ordinary differential equations or partial differential equations.
\begin{enumerate}
\sphinxsetlistlabels{\arabic}{enumi}{enumii}{}{.}%
\item {} 
Using Newton’s second law we have equations of the type \(\boldsymbol{F}=m\boldsymbol{a}=md\boldsymbol{v}/dt\)

\item {} 
We need to  define the initial conditions (typically the initial velocity and position as functions of time) and/or initial conditions and boundary conditions

\item {} 
The solution of the equations give us then the position, the velocity and other time\sphinxhyphen{}dependent quantities which may specify the motion of a given object.

\end{enumerate}

We are not yet done. With our lovely solvers, we need to start thinking.

Now it is time to ask the big questions. What do our results mean? Can we give a simple interpretation in terms of fundamental laws?  What do our results mean? Are they correct?
Thus, typical questions we may ask are
\begin{enumerate}
\sphinxsetlistlabels{\arabic}{enumi}{enumii}{}{.}%
\item {} 
Are our results for say \(\boldsymbol{r}(t)\) valid?  Do we trust what we did?  Can you validate and verify the correctness of your results?

\item {} 
Evaluate the answers and their implications

\item {} 
Compare with experimental data if possible. Does our model make sense?

\item {} 
and obviously many other questions.

\end{enumerate}

The analysis stage feeds back to the first stage. It may happen that
the data we had were not good enough, there could be large statistical
uncertainties. We may need to collect more data or perhaps we did a
sloppy job in identifying the degrees of freedom.

All these steps are essential elements in a scientific
enquiry. Hopefully, through a mix of numerical simulations, analytical
calculations and experiments we may gain a deeper insight about the
physics of a specific system.

Let us now remind ourselves of Newton’s laws, since these are the laws of motion we will study in this course.


\section{Newton’s Laws}
\label{\detokenize{chapter1:newton-s-laws}}
When analyzing a physical system we normally start with distinguishing between the object we are studying (we will label this in more general terms as our \sphinxstylestrong{system}) and how this system interacts with the environment (which often means everything else!)

In our investigations we will thus analyze a specific physics problem in terms of the system and the environment.
In doing so we need to identify the forces that act on the system and assume that the
forces acting on the system must have a source, an identifiable cause in
the environment.

A force acting on for example a falling object must be related to an interaction with something in the environment.
This also means that we do not consider internal forces. The latter are forces between
one part of the object and another part. In this course we will mainly focus on external forces.

Forces are either contact forces or long\sphinxhyphen{}range forces.

Contact forces, as evident from the name, are forces that occur at the contact between
the system and the environment. Well\sphinxhyphen{}known long\sphinxhyphen{}range forces are the gravitional force and the electromagnetic force.


\section{Setting up a model for forces acting on an object}
\label{\detokenize{chapter1:setting-up-a-model-for-forces-acting-on-an-object}}
In order to set up the forces which act on an object, the following steps may be useful
\begin{enumerate}
\sphinxsetlistlabels{\arabic}{enumi}{enumii}{}{.}%
\item {} 
Divide the problem into system and environment.

\item {} 
Draw a figure of the object and everything in contact with the object.

\item {} 
Draw a closed curve around the system.

\item {} 
Find contact points—these are the points where contact forces may act.

\item {} 
Give names and symbols to all the contact forces.

\item {} 
Identify the long\sphinxhyphen{}range forces.

\item {} 
Make a drawing of the object. Draw the forces as arrows, vectors, starting from where the force is acting. The direction of the vector(s) indicates the (positive) direction of the force. Try to make the length of the arrow indicate the relative magnitude of the forces.

\item {} 
Draw in the axes of the coordinate system. It is often convenient to make one axis parallel to the direction of motion. When you choose the direction of the axis you also choose the positive direction for the axis.

\end{enumerate}


\section{Newton’s Laws, the Second one first}
\label{\detokenize{chapter1:newton-s-laws-the-second-one-first}}
Newton’s second law of motion: The force \(\boldsymbol{F}\) on an object of inertial mass \(m\)
is related to the acceleration a of the object through
\begin{equation*}
\begin{split}
\boldsymbol{F} = m\boldsymbol{a},
\end{split}
\end{equation*}
where \(\boldsymbol{a}\) is the acceleration.

Newton’s laws of motion are laws of nature that have been found by experimental
investigations and have been shown to hold up to continued experimental investigations.
Newton’s laws are valid over a wide range of length\sphinxhyphen{} and time\sphinxhyphen{}scales. We
use Newton’s laws of motion to describe everything from the motion of atoms to the
motion of galaxies.

The second law is a vector equation with the acceleration having the same
direction as the force. The acceleration is proportional to the force via the mass \(m\) of the system under study.

Newton’s second law introduces a new property of an object, the so\sphinxhyphen{}called
inertial mass \(m\). We determine the inertial mass of an object by measuring the
acceleration for a given applied force.


\section{Then the First Law}
\label{\detokenize{chapter1:then-the-first-law}}
What happens if the net external force on a body is zero? Applying Newton’s second
law, we find:
\begin{equation*}
\begin{split}
\boldsymbol{F} = 0 = m\boldsymbol{a},
\end{split}
\end{equation*}
which gives using the definition of the acceleration
\begin{equation*}
\begin{split}
\boldsymbol{a} = \frac{d\boldsymbol{v}}{dt}=0.
\end{split}
\end{equation*}
The acceleration is zero, which means that the velocity of the object is constant. This
is often referred to as Newton’s first law. An object in a state of uniform motion tends to remain in
that state unless an external force changes its state of motion.
Why do we need a separate law for this? Is it not simply a special case of Newton’s
second law? Yes, Newton’s first law can be deduced from the second law as we have
illustrated. However, the first law is often used for a different purpose: Newton’s
First Law tells us about the limit of applicability of Newton’s Second law. Newton’s
Second law can only be used in reference systems where the First law is obeyed. But
is not the First law always valid? No! The First law is only valid in reference systems
that are not accelerated. If you observe the motion of a ball from an accelerating
car, the ball will appear to accelerate even if there are no forces acting on it. We call
systems that are not accelerating inertial systems, and Newton’s first law is often
called the law of inertia. Newton’s first and second laws of motion are only valid in
inertial systems.

A system is an inertial system if it is not accelerated. It means that the reference system
must not be accelerating linearly or rotating. Unfortunately, this means that most
systems we know are not really inertial systems. For example, the surface of the
Earth is clearly not an inertial system, because the Earth is rotating. The Earth is also
not an inertial system, because it ismoving in a curved path around the Sun. However,
even if the surface of the Earth is not strictly an inertial system, it may be considered
to be approximately an inertial system for many laboratory\sphinxhyphen{}size experiments.


\section{And finally the Third Law}
\label{\detokenize{chapter1:and-finally-the-third-law}}
If there is a force from object A on object B, there is also a force from object B on object A.
This fundamental principle of interactions is called Newton’s third law. We do not
know of any force that do not obey this law: All forces appear in pairs. Newton’s
third law is usually formulated as: For every action there is an equal and opposite
reaction.


\section{Motion of a Single Object}
\label{\detokenize{chapter1:motion-of-a-single-object}}
Here we consider the motion of a single particle moving under
the influence of some set of forces.  We will consider some problems where
the force does not depend on the position. In that case Newton’s law
\(m\dot{\boldsymbol{v}}=\boldsymbol{F}(\boldsymbol{v})\) is a first\sphinxhyphen{}order differential
equation and one solves for \(\boldsymbol{v}(t)\), then moves on to integrate
\(\boldsymbol{v}\) to get the position. In essentially all of these cases we cna find an analytical solution.


\section{Air Resistance in One Dimension}
\label{\detokenize{chapter1:air-resistance-in-one-dimension}}
Air resistance tends to scale as the square of the velocity. This is
in contrast to many problems chosen for textbooks, where it is linear
in the velocity. The choice of a linear dependence is motivated by
mathematical simplicity (it keeps the differential equation linear)
rather than by physics. One can see that the force should be quadratic
in velocity by considering the momentum imparted on the air
molecules. If an object sweeps through a volume \(dV\) of air in time
\(dt\), the momentum imparted on the air is




\begin{equation*}
\begin{split}
\begin{equation}
dP=\rho_m dV v,
\label{_auto1} \tag{1}
\end{equation}
\end{split}
\end{equation*}
where \(v\) is the velocity of the object and \(\rho_m\) is the mass
density of the air. If the molecules bounce back as opposed to stop
you would double the size of the term. The opposite value of the
momentum is imparted onto the object itself. Geometrically, the
differential volume is




\begin{equation*}
\begin{split}
\begin{equation}
dV=Avdt,
\label{_auto2} \tag{2}
\end{equation}
\end{split}
\end{equation*}
where \(A\) is the cross\sphinxhyphen{}sectional area and \(vdt\) is the distance the
object moved in time \(dt\).


\section{Resulting Acceleration}
\label{\detokenize{chapter1:resulting-acceleration}}
Plugging this into the expression above,




\begin{equation*}
\begin{split}
\begin{equation}
\frac{dP}{dt}=-\rho_m A v^2.
\label{_auto3} \tag{3}
\end{equation}
\end{split}
\end{equation*}
This is the force felt by the particle, and is opposite to its
direction of motion. Now, because air doesn’t stop when it hits an
object, but flows around the best it can, the actual force is reduced
by a dimensionless factor \(c_W\), called the drag coefficient.




\begin{equation*}
\begin{split}
\begin{equation}
F_{\rm drag}=-c_W\rho_m Av^2,
\label{_auto4} \tag{4}
\end{equation}
\end{split}
\end{equation*}
and the acceleration is
\begin{equation*}
\begin{split}
\begin{eqnarray}
\frac{dv}{dt}=-\frac{c_W\rho_mA}{m}v^2.
\end{eqnarray}
\end{split}
\end{equation*}
For a particle with initial velocity \(v_0\), one can separate the \(dt\)
to one side of the equation, and move everything with \(v\)s to the
other side. We did this in our discussion of simple motion and will not repeat it here.

On more general terms,
for many systems, e.g. an automobile, there are multiple sources of
resistance. In addition to wind resistance, where the force is
proportional to \(v^2\), there are dissipative effects of the tires on
the pavement, and in the axel and drive train. These other forces can
have components that scale proportional to \(v\), and components that
are independent of \(v\). Those independent of \(v\), e.g. the usual
\(f=\mu_K N\) frictional force you consider in your first Physics courses, only set in
once the object is actually moving. As speeds become higher, the \(v^2\)
components begin to dominate relative to the others. For automobiles
at freeway speeds, the \(v^2\) terms are largely responsible for the
loss of efficiency. To travel a distance \(L\) at fixed speed \(v\), the
energy/work required to overcome the dissipative forces are \(fL\),
which for a force of the form \(f=\alpha v^n\) becomes
\begin{equation*}
\begin{split}
\begin{eqnarray}
W=\int dx~f=\alpha v^n L.
\end{eqnarray}
\end{split}
\end{equation*}
For \(n=0\) the work is
independent of speed, but for the wind resistance, where \(n=2\),
slowing down is essential if one wishes to reduce fuel consumption. It
is also important to consider that engines are designed to be most
efficient at a chosen range of power output. Thus, some cars will get
better mileage at higher speeds (They perform better at 50 mph than at
5 mph) despite the considerations mentioned above.


\section{Going Ballistic, Projectile Motion or a Softer Approach, Falling Raindrops}
\label{\detokenize{chapter1:going-ballistic-projectile-motion-or-a-softer-approach-falling-raindrops}}
As an example of Newton’s Laws we consider projectile motion (or a
falling raindrop or a ball we throw up in the air) with a drag force. Even though air resistance is
largely proportional to the square of the velocity, we will consider
the drag force to be linear to the velocity, \(\boldsymbol{F}=-m\gamma\boldsymbol{v}\),
for the purposes of this exercise. The acceleration for a projectile moving upwards,
\(\boldsymbol{a}=\boldsymbol{F}/m\), becomes
\begin{equation*}
\begin{split}
\begin{eqnarray}
\frac{dv_x}{dt}=-\gamma v_x,\\
\nonumber
\frac{dv_y}{dt}=-\gamma v_y-g,
\end{eqnarray}
\end{split}
\end{equation*}
and \(\gamma\) has dimensions of inverse time.

If you on the other hand have a falling raindrop, how do these equations change? See for example Figure 2.1 in Taylor.
Let us stay with a ball which is thrown up in the air at \(t=0\).


\section{Ways of solving these equations}
\label{\detokenize{chapter1:ways-of-solving-these-equations}}
We will go over two different ways to solve this equation. The first
by direct integration, and the second as a differential equation. To
do this by direct integration, one simply multiplies both sides of the
equations above by \(dt\), then divide by the appropriate factors so
that the \(v\)s are all on one side of the equation and the \(dt\) is on
the other. For the \(x\) motion one finds an easily integrable equation,
\begin{equation*}
\begin{split}
\begin{eqnarray}
\frac{dv_x}{v_x}&=&-\gamma dt,\\
\nonumber
\int_{v_{0x}}^{v_{x}}\frac{dv_x}{v_x}&=&-\gamma\int_0^{t}dt,\\
\nonumber
\ln\left(\frac{v_{x}}{v_{0x}}\right)&=&-\gamma t,\\
\nonumber
v_{x}(t)&=&v_{0x}e^{-\gamma t}.
\end{eqnarray}
\end{split}
\end{equation*}
This is very much the result you would have written down
by inspection. For the \(y\)\sphinxhyphen{}component of the velocity,
\begin{equation*}
\begin{split}
\begin{eqnarray}
\frac{dv_y}{v_y+g/\gamma}&=&-\gamma dt\\
\nonumber
\ln\left(\frac{v_{y}+g/\gamma}{v_{0y}-g/\gamma}\right)&=&-\gamma t_f,\\
\nonumber
v_{fy}&=&-\frac{g}{\gamma}+\left(v_{0y}+\frac{g}{\gamma}\right)e^{-\gamma t}.
\end{eqnarray}
\end{split}
\end{equation*}
Whereas \(v_x\) starts at some value and decays
exponentially to zero, \(v_y\) decays exponentially to the terminal
velocity, \(v_t=-g/\gamma\).


\section{Solving as differential equations}
\label{\detokenize{chapter1:solving-as-differential-equations}}
Although this direct integration is simpler than the method we invoke
below, the method below will come in useful for some slightly more
difficult differential equations in the future. The differential
equation for \(v_x\) is straight\sphinxhyphen{}forward to solve. Because it is first
order there is one arbitrary constant, \(A\), and by inspection the
solution is




\begin{equation*}
\begin{split}
\begin{equation}
v_x=Ae^{-\gamma t}.
\label{_auto5} \tag{5}
\end{equation}
\end{split}
\end{equation*}
The arbitrary constants for equations of motion are usually determined
by the initial conditions, or more generally boundary conditions. By
inspection \(A=v_{0x}\), the initial \(x\) component of the velocity.


\section{Differential Equations, contn}
\label{\detokenize{chapter1:differential-equations-contn}}
The differential equation for \(v_y\) is a bit more complicated due to
the presence of \(g\). Differential equations where all the terms are
linearly proportional to a function, in this case \(v_y\), or to
derivatives of the function, e.g., \(v_y\), \(dv_y/dt\),
\(d^2v_y/dt^2\cdots\), are called linear differential equations. If
there are terms proportional to \(v^2\), as would happen if the drag
force were proportional to the square of the velocity, the
differential equation is not longer linear. Because this expression
has only one derivative in \(v\) it is a first\sphinxhyphen{}order linear differential
equation. If a term were added proportional to \(d^2v/dt^2\) it would be
a second\sphinxhyphen{}order differential equation.  In this case we have a term
completely independent of \(v\), the gravitational acceleration \(g\), and
the usual strategy is to first rewrite the equation with all the
linear terms on one side of the equal sign,




\begin{equation*}
\begin{split}
\begin{equation}
\frac{dv_y}{dt}+\gamma v_y=-g.
\label{_auto6} \tag{6}
\end{equation}
\end{split}
\end{equation*}

\section{Splitting into two parts}
\label{\detokenize{chapter1:splitting-into-two-parts}}
Now, the solution to the equation can be broken into two
parts. Because this is a first\sphinxhyphen{}order differential equation we know
that there will be one arbitrary constant. Physically, the arbitrary
constant will be determined by setting the initial velocity, though it
could be determined by setting the velocity at any given time. Like
most differential equations, solutions are not “solved”. Instead,
one guesses at a form, then shows the guess is correct. For these
types of equations, one first tries to find a single solution,
i.e. one with no arbitrary constants. This is called the \{\textbackslash{}it
particular\} solution, \(y_p(t)\), though it should really be called
“a” particular solution because there are an infinite number of such
solutions. One then finds a solution to the \{\textbackslash{}it homogenous\} equation,
which is the equation with zero on the right\sphinxhyphen{}hand side,




\begin{equation*}
\begin{split}
\begin{equation}
\frac{dv_{y,h}}{dt}+\gamma v_{y,h}=0.
\label{_auto7} \tag{7}
\end{equation}
\end{split}
\end{equation*}
Homogenous solutions will have arbitrary constants.

The particular solution will solve the same equation as the original
general equation




\begin{equation*}
\begin{split}
\begin{equation}
\frac{dv_{y,p}}{dt}+\gamma v_{y,p}=-g.
\label{_auto8} \tag{8}
\end{equation}
\end{split}
\end{equation*}
However, we don’t need find one with arbitrary constants. Hence, it is
called a \sphinxstylestrong{particular} solution.

The sum of the two,




\begin{equation*}
\begin{split}
\begin{equation}
v_y=v_{y,p}+v_{y,h},
\label{_auto9} \tag{9}
\end{equation}
\end{split}
\end{equation*}
is a solution of the total equation because of the linear nature of
the differential equation. One has now found a \sphinxstyleemphasis{general} solution
encompassing all solutions, because it both satisfies the general
equation (like the particular solution), and has an arbitrary constant
that can be adjusted to fit any initial condition (like the homogneous
solution). If the equation were not linear, e.g if there were a term
such as \(v_y^2\) or \(v_y\dot{v}_y\), this technique would not work.


\section{More details}
\label{\detokenize{chapter1:more-details}}
Returning to the example above, the homogenous solution is the same as
that for \(v_x\), because there was no gravitational acceleration in
that case,




\begin{equation*}
\begin{split}
\begin{equation}
v_{y,h}=Be^{-\gamma t}.
\label{_auto10} \tag{10}
\end{equation}
\end{split}
\end{equation*}
In this case a particular solution is one with constant velocity,




\begin{equation*}
\begin{split}
\begin{equation}
v_{y,p}=-g/\gamma.
\label{_auto11} \tag{11}
\end{equation}
\end{split}
\end{equation*}
Note that this is the terminal velocity of a particle falling from a
great height. The general solution is thus,




\begin{equation*}
\begin{split}
\begin{equation}
v_y=Be^{-\gamma t}-g/\gamma,
\label{_auto12} \tag{12}
\end{equation}
\end{split}
\end{equation*}
and one can find \(B\) from the initial velocity,




\begin{equation*}
\begin{split}
\begin{equation}
v_{0y}=B-g/\gamma,~~~B=v_{0y}+g/\gamma.
\label{_auto13} \tag{13}
\end{equation}
\end{split}
\end{equation*}
Plugging in the expression for \(B\) gives the \(y\) motion given the initial velocity,




\begin{equation*}
\begin{split}
\begin{equation}
v_y=(v_{0y}+g/\gamma)e^{-\gamma t}-g/\gamma.
\label{_auto14} \tag{14}
\end{equation}
\end{split}
\end{equation*}
It is easy to see that this solution has \(v_y=v_{0y}\) when \(t=0\) and
\(v_y=-g/\gamma\) when \(t\rightarrow\infty\).

One can also integrate the two equations to find the coordinates \(x\)
and \(y\) as functions of \(t\),
\begin{equation*}
\begin{split}
\begin{eqnarray}
x&=&\int_0^t dt'~v_{0x}(t')=\frac{v_{0x}}{\gamma}\left(1-e^{-\gamma t}\right),\\
\nonumber
y&=&\int_0^t dt'~v_{0y}(t')=-\frac{gt}{\gamma}+\frac{v_{0y}+g/\gamma}{\gamma}\left(1-e^{-\gamma t}\right).
\end{eqnarray}
\end{split}
\end{equation*}
If the question was to find the position at a time \(t\), we would be
finished. However, the more common goal in a projectile equation
problem is to find the range, i.e. the distance \(x\) at which \(y\)
returns to zero. For the case without a drag force this was much
simpler. The solution for the \(y\) coordinate would have been
\(y=v_{0y}t-gt^2/2\). One would solve for \(t\) to make \(y=0\), which would
be \(t=2v_{0y}/g\), then plug that value for \(t\) into \(x=v_{0x}t\) to
find \(x=2v_{0x}v_{0y}/g=v_0\sin(2\theta_0)/g\). One follows the same
steps here, except that the expression for \(y(t)\) is more
complicated. Searching for the time where \(y=0\), and we get




\begin{equation*}
\begin{split}
\begin{equation}
0=-\frac{gt}{\gamma}+\frac{v_{0y}+g/\gamma}{\gamma}\left(1-e^{-\gamma t}\right).
\label{_auto15} \tag{15}
\end{equation}
\end{split}
\end{equation*}
This cannot be inverted into a simple expression \(t=\cdots\). Such
expressions are known as “transcendental equations”, and are not the
rare instance, but are the norm. In the days before computers, one
might plot the right\sphinxhyphen{}hand side of the above graphically as
a function of time, then find the point where it crosses zero.

Now, the most common way to solve for an equation of the above type
would be to apply Newton’s method numerically. This involves the
following algorithm for finding solutions of some equation \(F(t)=0\).
\begin{enumerate}
\sphinxsetlistlabels{\arabic}{enumi}{enumii}{}{.}%
\item {} 
First guess a value for the time, \(t_{\rm guess}\).

\item {} 
Calculate \(F\) and its derivative, \(F(t_{\rm guess})\) and \(F'(t_{\rm guess})\).

\item {} 
Unless you guessed perfectly, \(F\ne 0\), and assuming that \(\Delta F\approx F'\Delta t\), one would choose

\item {} 
\(\Delta t=-F(t_{\rm guess})/F'(t_{\rm guess})\).

\item {} 
Now repeat step 1, but with \(t_{\rm guess}\rightarrow t_{\rm guess}+\Delta t\).

\end{enumerate}

If the \(F(t)\) were perfectly linear in \(t\), one would find \(t\) in one
step. Instead, one typically finds a value of \(t\) that is closer to
the final answer than \(t_{\rm guess}\). One breaks the loop once one
finds \(F\) within some acceptable tolerance of zero. A program to do
this will be added shortly.


\section{Motion in a Magnetic Field}
\label{\detokenize{chapter1:motion-in-a-magnetic-field}}
Another example of a velocity\sphinxhyphen{}dependent force is magnetism,
\begin{equation*}
\begin{split}
\begin{eqnarray}
\boldsymbol{F}&=&q\boldsymbol{v}\times\boldsymbol{B},\\
\nonumber
F_i&=&q\sum_{jk}\epsilon_{ijk}v_jB_k.
\end{eqnarray}
\end{split}
\end{equation*}
For a uniform field in the \(z\) direction \(\boldsymbol{B}=B\hat{z}\), the force can only have \(x\) and \(y\) components,
\begin{equation*}
\begin{split}
\begin{eqnarray}
F_x&=&qBv_y\\
\nonumber
F_y&=&-qBv_x.
\end{eqnarray}
\end{split}
\end{equation*}
The differential equations are
\begin{equation*}
\begin{split}
\begin{eqnarray}
\dot{v}_x&=&\omega_c v_y,\omega_c= qB/m\\
\nonumber
\dot{v}_y&=&-\omega_c v_x.
\end{eqnarray}
\end{split}
\end{equation*}
One can solve the equations by taking time derivatives of either equation, then substituting into the other equation,
\begin{equation*}
\begin{split}
\begin{eqnarray}
\ddot{v}_x=\omega_c\dot{v_y}=-\omega_c^2v_x,\\
\nonumber
\ddot{v}_y&=&-\omega_c\dot{v}_x=-\omega_cv_y.
\end{eqnarray}
\end{split}
\end{equation*}
The solution to these equations can be seen by inspection,
\begin{equation*}
\begin{split}
\begin{eqnarray}
v_x&=&A\sin(\omega_ct+\phi),\\
\nonumber
v_y&=&A\cos(\omega_ct+\phi).
\end{eqnarray}
\end{split}
\end{equation*}
One can integrate the equations to find the positions as a function of time,
\begin{equation*}
\begin{split}
\begin{eqnarray}
x-x_0&=&\int_{x_0}^x dx=\int_0^t dt v(t)\\
\nonumber
&=&\frac{-A}{\omega_c}\cos(\omega_ct+\phi),\\
\nonumber
y-y_0&=&\frac{A}{\omega_c}\sin(\omega_ct+\phi).
\end{eqnarray}
\end{split}
\end{equation*}
The trajectory is a circle centered at \(x_0,y_0\) with amplitude \(A\) rotating in the clockwise direction.

The equations of motion for the \(z\) motion are




\begin{equation*}
\begin{split}
\begin{equation}
\dot{v_z}=0,
\label{_auto16} \tag{16}
\end{equation}
\end{split}
\end{equation*}
which leads to




\begin{equation*}
\begin{split}
\begin{equation}
z-z_0=V_zt.
\label{_auto17} \tag{17}
\end{equation}
\end{split}
\end{equation*}
Added onto the circle, the motion is helical.

Note that the kinetic energy,




\begin{equation*}
\begin{split}
\begin{equation}
T=\frac{1}{2}m(v_x^2+v_y^2+v_z^2)=\frac{1}{2}m(\omega_c^2A^2+V_z^2),
\label{_auto18} \tag{18}
\end{equation}
\end{split}
\end{equation*}
is constant. This is because the force is perpendicular to the
velocity, so that in any differential time element \(dt\) the work done
on the particle \(\boldsymbol{F}\cdot{dr}=dt\boldsymbol{F}\cdot{v}=0\).

One should think about the implications of a velocity dependent
force. Suppose one had a constant magnetic field in deep space. If a
particle came through with velocity \(v_0\), it would undergo cyclotron
motion with radius \(R=v_0/\omega_c\). However, if it were still its
motion would remain fixed. Now, suppose an observer looked at the
particle in one reference frame where the particle was moving, then
changed their velocity so that the particle’s velocity appeared to be
zero. The motion would change from circular to fixed. Is this
possible?

The solution to the puzzle above relies on understanding
relativity. Imagine that the first observer believes \(\boldsymbol{B}\ne 0\) and
that the electric field \(\boldsymbol{E}=0\). If the observer then changes
reference frames by accelerating to a velocity \(\boldsymbol{v}\), in the new
frame \(\boldsymbol{B}\) and \(\boldsymbol{E}\) both change. If the observer moved to the
frame where the charge, originally moving with a small velocity \(v\),
is now at rest, the new electric field is indeed \(\boldsymbol{v}\times\boldsymbol{B}\),
which then leads to the same acceleration as one had before. If the
velocity is not small compared to the speed of light, additional
\(\gamma\) factors come into play,
\(\gamma=1/\sqrt{1-(v/c)^2}\). Relativistic motion will not be
considered in this course.


\section{Sliding Block tied to a Wall}
\label{\detokenize{chapter1:sliding-block-tied-to-a-wall}}
Another classical case is that of simple harmonic oscillations, here represented by a block sliding on a horizontal frictionless surface. The block is tied to a wall with a spring. If the spring is not compressed or stretched too far, the force on the block at a given position \(x\) is
\begin{equation*}
\begin{split}
F=-kx.
\end{split}
\end{equation*}
The negative sign means that the force acts to restore the object to an equilibrium position. Newton’s equation of motion for this idealized system is then
\begin{equation*}
\begin{split}
m\frac{d^2x}{dt^2}=-kx,
\end{split}
\end{equation*}
or we could rephrase it as




\begin{equation*}
\begin{split}
\frac{d^2x}{dt^2}=-\frac{k}{m}x=-\omega_0^2x,
\label{eq:newton1} \tag{19}
\end{split}
\end{equation*}
with the angular frequency \(\omega_0^2=k/m\).

The above differential equation has the advantage that it can be solved  analytically with solutions on the form
\begin{equation*}
\begin{split}
x(t)=Acos(\omega_0t+\nu),
\end{split}
\end{equation*}
where \(A\) is the amplitude and \(\nu\) the phase constant.   This provides in turn an important test for the numerical
solution and the development of a program for more complicated cases which cannot be solved analytically.

With the position \(x(t)\) and the velocity  \(v(t)=dx/dt\) we can reformulate Newton’s equation in the following way
\begin{equation*}
\begin{split}
\frac{dx(t)}{dt}=v(t),
\end{split}
\end{equation*}
and
\begin{equation*}
\begin{split}
\frac{dv(t)}{dt}=-\omega_0^2x(t).
\end{split}
\end{equation*}
We are now going to solve these equations using first the standard forward Euler  method. Later we will try to improve upon this.

Before proceeding however, it is important to note that in addition to the exact solution, we have at least two further tests which can be used to check our solution.

Since functions like \(cos\) are periodic with a period \(2\pi\), then the solution \(x(t)\) has also to be periodic. This means that
\begin{equation*}
\begin{split}
x(t+T)=x(t),
\end{split}
\end{equation*}
with \(T\) the period defined as
\begin{equation*}
\begin{split}
T=\frac{2\pi}{\omega_0}=\frac{2\pi}{\sqrt{k/m}}.
\end{split}
\end{equation*}
Observe that \(T\) depends only on \(k/m\) and not on the amplitude of the solution.

In addition to the periodicity test, the total energy has also to be conserved.

Suppose we choose the initial conditions
\begin{equation*}
\begin{split}
x(t=0)=1\hspace{0.1cm} \mathrm{m}\hspace{1cm} v(t=0)=0\hspace{0.1cm}\mathrm{m/s},
\end{split}
\end{equation*}
meaning that block is at rest at \(t=0\) but with a potential energy
\begin{equation*}
\begin{split}
E_0=\frac{1}{2}kx(t=0)^2=\frac{1}{2}k.
\end{split}
\end{equation*}
The total energy at any time \(t\) has however to be conserved, meaning that our solution has to fulfil the condition
\begin{equation*}
\begin{split}
E_0=\frac{1}{2}kx(t)^2+\frac{1}{2}mv(t)^2.
\end{split}
\end{equation*}
We will derive this equation in our discussion on \sphinxhref{https://mhjensen.github.io/Physics321/doc/pub/energyconserv/html/energyconserv.html}{energy conservation}.

An algorithm which implements these equations is included below.
\begin{itemize}
\item {} 
Choose the initial position and speed, with the most common choice \(v(t=0)=0\) and some fixed value for the position.

\item {} 
Choose the method you wish to employ in solving the problem.

\item {} 
Subdivide the time interval \([t_i,t_f] \) into a grid with step size

\end{itemize}
\begin{equation*}
\begin{split}
h=\frac{t_f-t_i}{N},
\end{split}
\end{equation*}
where \(N\) is the number of mesh points.
\begin{itemize}
\item {} 
Calculate now the total energy given by

\end{itemize}
\begin{equation*}
\begin{split}
E_0=\frac{1}{2}kx(t=0)^2=\frac{1}{2}k.
\end{split}
\end{equation*}\begin{itemize}
\item {} 
Choose ODE solver to obtain \(x_{i+1}\) and \(v_{i+1}\) starting from the previous values \(x_i\) and \(v_i\).

\item {} 
When we have computed \(x(v)_{i+1}\) we upgrade  \(t_{i+1}=t_i+h\).

\item {} 
This iterative  process continues till we reach the maximum time \(t_f\).

\item {} 
The results are checked against the exact solution. Furthermore, one has to check the stability of the numerical solution against the chosen number of mesh points \(N\).

\end{itemize}

The following python program ( code will be added shortly)

\begin{sphinxVerbatim}[commandchars=\\\{\}]
\PYG{c+c1}{\PYGZsh{}}
\PYG{c+c1}{\PYGZsh{} This program solves Newtons equation for a block sliding on}
\PYG{c+c1}{\PYGZsh{} an horizontal frictionless surface.}
\PYG{c+c1}{\PYGZsh{} The block is tied to the wall with a spring, so N\PYGZsq{}s eq takes the form:}
\PYG{c+c1}{\PYGZsh{}}
\PYG{c+c1}{\PYGZsh{}  m d\PYGZca{}2x/dt\PYGZca{}2 = \PYGZhy{} kx}
\PYG{c+c1}{\PYGZsh{}}
\PYG{c+c1}{\PYGZsh{} In order to make the solution dimless, we set k/m = 1.}
\PYG{c+c1}{\PYGZsh{} This results in two coupled diff. eq\PYGZsq{}s that may be written as:}
\PYG{c+c1}{\PYGZsh{}}
\PYG{c+c1}{\PYGZsh{}  dx/dt = v}
\PYG{c+c1}{\PYGZsh{}  dv/dt = \PYGZhy{}x}
\PYG{c+c1}{\PYGZsh{}}
\PYG{c+c1}{\PYGZsh{} The user has to specify the initial velocity and position,}
\PYG{c+c1}{\PYGZsh{} and the number of steps. The time interval is fixed to}
\PYG{c+c1}{\PYGZsh{} t \PYGZbs{}in [0, 4\PYGZbs{}pi) (two periods)}
\PYG{c+c1}{\PYGZsh{}}
\end{sphinxVerbatim}


\section{The classical pendulum and scaling the equations}
\label{\detokenize{chapter1:the-classical-pendulum-and-scaling-the-equations}}
The angular equation of motion of the pendulum is given by
Newton’s equation and with no external force it reads




\begin{equation*}
\begin{split}
\begin{equation}
  ml\frac{d^2\theta}{dt^2}+mgsin(\theta)=0,
\label{_auto19} \tag{20}
\end{equation}
\end{split}
\end{equation*}
with an angular velocity and acceleration given by




\begin{equation*}
\begin{split}
\begin{equation}
     v=l\frac{d\theta}{dt},
\label{_auto20} \tag{21}
\end{equation}
\end{split}
\end{equation*}
and




\begin{equation*}
\begin{split}
\begin{equation}
     a=l\frac{d^2\theta}{dt^2}.
\label{_auto21} \tag{22}
\end{equation}
\end{split}
\end{equation*}

\section{More on the Pendulum}
\label{\detokenize{chapter1:more-on-the-pendulum}}
We do however expect that the motion will gradually come to an end due a viscous drag torque acting on the pendulum.
In the presence of the drag, the above equation becomes




\begin{equation*}
\begin{split}
\begin{equation}
   ml\frac{d^2\theta}{dt^2}+\nu\frac{d\theta}{dt}  +mgsin(\theta)=0, \label{eq:pend1} \tag{23}
\end{equation}
\end{split}
\end{equation*}
where \(\nu\) is now a positive constant parameterizing the viscosity
of the medium in question. In order to maintain the motion against
viscosity, it is necessary to add some external driving force.
We choose here a periodic driving force. The last equation becomes then




\begin{equation*}
\begin{split}
\begin{equation}
   ml\frac{d^2\theta}{dt^2}+\nu\frac{d\theta}{dt}  +mgsin(\theta)=Asin(\omega t), \label{eq:pend2} \tag{24}
\end{equation}
\end{split}
\end{equation*}
with \(A\) and \(\omega\) two constants representing the amplitude and
the angular frequency respectively. The latter is called the driving frequency.


\section{More on the Pendulum}
\label{\detokenize{chapter1:id1}}
We define
\begin{equation*}
\begin{split}
\omega_0=\sqrt{g/l},
\end{split}
\end{equation*}
the so\sphinxhyphen{}called natural frequency and the new dimensionless quantities
\begin{equation*}
\begin{split}
\hat{t}=\omega_0t,
\end{split}
\end{equation*}
with the dimensionless driving frequency
\begin{equation*}
\begin{split}
\hat{\omega}=\frac{\omega}{\omega_0},
\end{split}
\end{equation*}
and introducing the quantity \(Q\), called the \sphinxstyleemphasis{quality factor},
\begin{equation*}
\begin{split}
Q=\frac{mg}{\omega_0\nu},
\end{split}
\end{equation*}
and the dimensionless amplitude
\begin{equation*}
\begin{split}
\hat{A}=\frac{A}{mg}
\end{split}
\end{equation*}
We have
\begin{equation*}
\begin{split}
\frac{d^2\theta}{d\hat{t}^2}+\frac{1}{Q}\frac{d\theta}{d\hat{t}}  
     +sin(\theta)=\hat{A}cos(\hat{\omega}\hat{t}).
\end{split}
\end{equation*}
This equation can in turn be recast in terms of two coupled first\sphinxhyphen{}order differential equations as follows
\begin{equation*}
\begin{split}
\frac{d\theta}{d\hat{t}}=\hat{v},
\end{split}
\end{equation*}
and
\begin{equation*}
\begin{split}
\frac{d\hat{v}}{d\hat{t}}=-\frac{\hat{v}}{Q}-sin(\theta)+\hat{A}cos(\hat{\omega}\hat{t}).
\end{split}
\end{equation*}
These are the equations to be solved.  The factor \(Q\) represents the number of oscillations of the undriven system that must occur before  its energy is significantly reduced due to the viscous drag. The  amplitude \(\hat{A}\) is measured in units of the maximum possible  gravitational torque while \(\hat{\omega}\) is the angular frequency of the external torque measured in units of the pendulum’s natural frequency.


\chapter{Work, Energy, Momentum and Conservation laws}
\label{\detokenize{chapter1:work-energy-momentum-and-conservation-laws}}
Energy conservation is most convenient as a strategy for addressing
problems where time does not appear. For example, a particle goes
from position \(x_0\) with speed \(v_0\), to position \(x_f\); what is its
new speed? However, it can also be applied to problems where time
does appear, such as in solving for the trajectory \(x(t)\), or
equivalently \(t(x)\).

More material to be added here.


\section{Energy Conservation}
\label{\detokenize{chapter1:energy-conservation}}
Energy is conserved in the case where the potential energy, \(V(\boldsymbol{r})\), depends only on position, and not on time. The force is determined by \(V\),




\begin{equation*}
\begin{split}
\begin{equation}
\boldsymbol{F}(\boldsymbol{r})=-\nabla V(\boldsymbol{r}).
\label{_auto22} \tag{25}
\end{equation}
\end{split}
\end{equation*}
The net energy, \(E=V+K\) where \(K\) is the kinetic energy, is then conserved,
\begin{equation*}
\begin{split}
\begin{eqnarray}
\frac{d}{dt}(K+V)&=&\frac{d}{dt}\left(\frac{m}{2}(v_x^2+v_y^2+v_z^2)+V(\boldsymbol{r})\right)\\
\nonumber
&=&m\left(v_x\frac{dv_x}{dt}+v_y\frac{dv_y}{dt}+v_z\frac{dv_z}{dt}\right)
+\partial_xV\frac{dx}{dt}+\partial_yV\frac{dy}{dt}+\partial_zV\frac{dz}{dt}\\
\nonumber
&=&v_xF_x+v_yF_y+v_zF_z-F_xv_x-F_yv_y-F_zv_z=0.
\end{eqnarray}
\end{split}
\end{equation*}
The same proof can be written more compactly with vector notation,
\begin{equation*}
\begin{split}
\begin{eqnarray}
\frac{d}{dt}\left(\frac{m}{2}v^2+V(\boldsymbol{r})\right)
&=&m\boldsymbol{v}\cdot\dot{\boldsymbol{v}}+\nabla V(\boldsymbol{r})\cdot\dot{\boldsymbol{r}}\\
\nonumber
&=&\boldsymbol{v}\cdot\boldsymbol{F}-\boldsymbol{F}\cdot\boldsymbol{v}=0.
\end{eqnarray}
\end{split}
\end{equation*}
Inverting the expression for kinetic energy,




\begin{equation*}
\begin{split}
\begin{equation}
v=\sqrt{2K/m}=\sqrt{2(E-V)/m},
\label{_auto23} \tag{26}
\end{equation}
\end{split}
\end{equation*}
allows one to solve for the one\sphinxhyphen{}dimensional trajectory \(x(t)\), by finding \(t(x)\),




\begin{equation*}
\begin{split}
\begin{equation}
t=\int_{x_0}^x \frac{dx'}{v(x')}=\int_{x_0}^x\frac{dx'}{\sqrt{2(E-V(x'))/m}}.
\label{_auto24} \tag{27}
\end{equation}
\end{split}
\end{equation*}
Note this would be much more difficult in higher dimensions, because
you would have to determine which points, \(x,y,z\), the particles might
reach in the trajectory, whereas in one dimension you can typically
tell by simply seeing whether the kinetic energy is positive at every
point between the old position and the new position.

Consider a simple harmonic oscillator potential, \(V(x)=kx^2/2\), with a particle emitted from \(x=0\) with velocity \(v_0\). Solve for the trajectory \(t(x)\),
\begin{equation*}
\begin{split}
\begin{eqnarray}
t&=&\int_{0}^x \frac{dx'}{\sqrt{2(E-kx^2/2)/m}}\\
\nonumber
&=&\sqrt{m/k}\int_0^x~\frac{dx'}{\sqrt{x_{\rm max}^2-x^{\prime 2}}},~~~x_{\rm max}^2=2E/k.
\end{eqnarray}
\end{split}
\end{equation*}
Here \(E=mv_0^2/2\) and \(x_{\rm max}\) is defined as the maximum
displacement before the particle turns around. This integral is done
by the substitution \(\sin\theta=x/x_{\rm max}\).
\begin{equation*}
\begin{split}
\begin{eqnarray}
(k/m)^{1/2}t&=&\sin^{-1}(x/x_{\rm max}),\\
\nonumber
x&=&x_{\rm max}\sin\omega t,~~~\omega=\sqrt{k/m}.
\end{eqnarray}
\end{split}
\end{equation*}

\section{Conservation of Momentum}
\label{\detokenize{chapter1:conservation-of-momentum}}
Newton’s third law which we met earlier states that \sphinxstylestrong{For every action there is an equal and opposite reaction}, is more accurately stated as
\sphinxstylestrong{If two bodies exert forces on each other, these forces are equal in magnitude and opposite in direction}.

This means that for two bodies \(i\) and \(j\), if the force on \(i\) due to \(j\) is called \(\boldsymbol{F}_{ij}\), then




\begin{equation*}
\begin{split}
\begin{equation}
\boldsymbol{F}_{ij}=-\boldsymbol{F}_{ji}. 
\label{_auto25} \tag{28}
\end{equation}
\end{split}
\end{equation*}
Newton’s second law, \(\boldsymbol{F}=m\boldsymbol{a}\), can be written for a particle \(i\) as




\begin{equation*}
\begin{split}
\begin{equation}
\boldsymbol{F}_i=\sum_{j\ne i} \boldsymbol{F}_{ij}=m_i\boldsymbol{a}_i,
\label{_auto26} \tag{29}
\end{equation}
\end{split}
\end{equation*}
where \(\boldsymbol{F}_i\) (a single subscript) denotes the net force acting on \(i\). Because the mass of \(i\) is fixed, one can see that




\begin{equation*}
\begin{split}
\begin{equation}
\boldsymbol{F}_i=\frac{d}{dt}m_i\boldsymbol{v}_i=\sum_{j\ne i}\boldsymbol{F}_{ij}.
\label{_auto27} \tag{30}
\end{equation}
\end{split}
\end{equation*}
Now, one can sum over all the particles and obtain
\begin{equation*}
\begin{split}
\begin{eqnarray}
\frac{d}{dt}\sum_i m_iv_i&=&\sum_{ij, i\ne j}\boldsymbol{F}_{ij}\\
\nonumber
&=&0.
\end{eqnarray}
\end{split}
\end{equation*}
The last step made use of the fact that for every term \(ij\), there is
an equivalent term \(ji\) with opposite force. Because the momentum is
defined as \(m\boldsymbol{v}\), for a system of particles,




\begin{equation*}
\begin{split}
\begin{equation}
\frac{d}{dt}\sum_im_i\boldsymbol{v}_i=0,~~{\rm for~isolated~particles}.
\label{_auto28} \tag{31}
\end{equation}
\end{split}
\end{equation*}
By “isolated” one means that the only force acting on any particle \(i\)
are those originating from other particles in the sum, i.e. “no
external” forces. Thus, Newton’s third law leads to the conservation
of total momentum,
\begin{equation*}
\begin{split}
\begin{eqnarray}
\boldsymbol{P}&=&\sum_i m_i\boldsymbol{v}_i,\\
\nonumber
\frac{d}{dt}\boldsymbol{P}&=&0.
\end{eqnarray}
\end{split}
\end{equation*}
Consider the rocket of mass \(M\) moving with velocity \(v\). After a
brief instant, the velocity of the rocket is \(v+\Delta v\) and the mass
is \(M-\Delta M\). Momentum conservation gives
\begin{equation*}
\begin{split}
\begin{eqnarray*}
Mv&=&(M-\Delta M)(v+\Delta v)+\Delta M(v-v_e)\\
0&=&-\Delta Mv+M\Delta v+\Delta M(v-v_e),\\
0&=&M\Delta v-\Delta Mv_e.
\end{eqnarray*}
\end{split}
\end{equation*}
In the second step we ignored the term \(\Delta M\Delta v\) because it is doubly small. The last equation gives
\begin{equation*}
\begin{split}
\begin{eqnarray}
\Delta v&=&\frac{v_e}{M}\Delta M,\\
\nonumber
\frac{dv}{dt}&=&\frac{v_e}{M}\frac{dM}{dt}.
\end{eqnarray}
\end{split}
\end{equation*}
Integrating the expression with lower limits \(v_0=0\) and \(M_0\), one finds
\begin{equation*}
\begin{split}
\begin{eqnarray*}
v&=&v_e\int_{M_0}^M \frac{dM'}{M'}\\
v&=&-v_e\ln(M/M_0)\\
&=&-v_e\ln[(M_0-\alpha t)/M_0].
\end{eqnarray*}
\end{split}
\end{equation*}
Because the total momentum of an isolated system is constant, one can
also quickly see that the center of mass of an isolated system is also
constant. The center of mass is the average position of a set of
masses weighted by the mass,




\begin{equation*}
\begin{split}
\begin{equation}
\bar{x}=\frac{\sum_im_ix_i}{\sum_i m_i}.
\label{_auto29} \tag{32}
\end{equation}
\end{split}
\end{equation*}
The rate of change of \(\bar{x}\) is
\begin{equation*}
\begin{split}
\begin{eqnarray}
\dot{\bar{x}}&=&\frac{1}{M}\sum_i m_i\dot{x}_i=\frac{1}{M}P_x.
\end{eqnarray}
\end{split}
\end{equation*}
Thus if the total momentum is constant the center of mass moves at a
constant velocity, and if the total momentum is zero the center of
mass is fixed.


\section{Conservation of Angular Momentum}
\label{\detokenize{chapter1:conservation-of-angular-momentum}}
Consider a case where the force always points radially,




\begin{equation*}
\begin{split}
\begin{equation}
\boldsymbol{F}(\boldsymbol{r})=F(r)\hat{r},
\label{_auto30} \tag{33}
\end{equation}
\end{split}
\end{equation*}
where \(\hat{r}\) is a unit vector pointing outward from the origin. The angular momentum is defined as




\begin{equation*}
\begin{split}
\begin{equation}
\boldsymbol{L}=\boldsymbol{r}\times\boldsymbol{p}=m\boldsymbol{r}\times\boldsymbol{v}.
\label{_auto31} \tag{34}
\end{equation}
\end{split}
\end{equation*}
The rate of change of the angular momentum is
\begin{equation*}
\begin{split}
\begin{eqnarray}
\frac{d\boldsymbol{L}}{dt}&=&m\boldsymbol{v}\times\boldsymbol{v}+m\boldsymbol{r}\times\dot{\boldsymbol{v}}\\
\nonumber
&=&m\boldsymbol{v}\times\boldsymbol{v}+\boldsymbol{r}\times{\boldsymbol{F}}=0.
\end{eqnarray}
\end{split}
\end{equation*}
The first term is zero because \(\boldsymbol{v}\) is parallel to itself, and the
second term is zero because \(\boldsymbol{F}\) is parallel to \(\boldsymbol{r}\).

As an aside, one can see from the Levi\sphinxhyphen{}Civita symbol that the cross
product of a vector with itself is zero. Here, we consider a vector
\begin{equation*}
\begin{split}
\begin{eqnarray}
\boldsymbol{V}&=&\boldsymbol{A}\times\boldsymbol{A},\\
\nonumber
V_i&=&(\boldsymbol{A}\times\boldsymbol{A})_i=\sum_{jk}\epsilon_{ijk}A_jA_k.
\end{eqnarray}
\end{split}
\end{equation*}
For any term \(i\), there are two contributions. For example, for \(i\)
denoting the \(x\) direction, either \(j\) denotes the \(y\) direction and
\(k\) denotes the \(z\) direction, or vice versa, so




\begin{equation*}
\begin{split}
\begin{equation}
V_1=\epsilon_{123}A_2A_3+\epsilon_{132}A_3A_2.
\label{_auto32} \tag{35}
\end{equation}
\end{split}
\end{equation*}
This is zero by the antisymmetry of \(\epsilon\) under permutations.

If the force is not radial, \(\boldsymbol{r}\times\boldsymbol{F}\ne 0\) as above, and angular momentum is no longer conserved,




\begin{equation*}
\begin{split}
\begin{equation}
\frac{d\boldsymbol{L}}{dt}=\boldsymbol{r}\times\boldsymbol{F}\equiv\boldsymbol{\tau},
\label{_auto33} \tag{36}
\end{equation}
\end{split}
\end{equation*}
where \(\boldsymbol{\tau}\) is the torque.

For a system of isolated particles, one can write
\begin{equation*}
\begin{split}
\begin{eqnarray}
\frac{d}{dt}\sum_i\boldsymbol{L}_i&=&\sum_{i\ne j}\boldsymbol{r}_i\times \boldsymbol{F}_{ij}\\
\nonumber
&=&\frac{1}{2}\sum_{i\ne j} \boldsymbol{r}_i\times \boldsymbol{F}_{ij}+\boldsymbol{r}_j\times\boldsymbol{F}_{ji}\\
\nonumber
&=&\frac{1}{2}\sum_{i\ne j} (\boldsymbol{r}_i-\boldsymbol{r}_j)\times\boldsymbol{F}_{ij}=0,
\end{eqnarray}
\end{split}
\end{equation*}
where the last step used Newton’s third law,
\(\boldsymbol{F}_{ij}=-\boldsymbol{F}_{ji}\). If the forces between the particles are
radial, i.e. \(\boldsymbol{F}_{ij} ~||~ (\boldsymbol{r}_i-\boldsymbol{r}_j)\), then each term in
the sum is zero and the net angular momentum is fixed. Otherwise, you
could imagine an isolated system that would start spinning
spontaneously.

One can write the torque about a given axis, which we will denote as \(\hat{z}\), in polar coordinates, where
\begin{equation*}
\begin{split}
\begin{eqnarray}
x&=&r\sin\theta\cos\phi,~~y=r\sin\theta\cos\phi,~~z=r\cos\theta,
\end{eqnarray}
\end{split}
\end{equation*}
to find the \(z\) component of the torque,
\begin{equation*}
\begin{split}
\begin{eqnarray}
\tau_z&=&xF_y-yF_x\\
\nonumber
&=&-r\sin\theta\left\{\cos\phi \partial_y-\sin\phi \partial_x\right\}V(x,y,z).
\end{eqnarray}
\end{split}
\end{equation*}
One can use the chain rule to write the partial derivative w.r.t. \(\phi\) (keeping \(r\) and \(\theta\) fixed),
\begin{equation*}
\begin{split}
\begin{eqnarray}
\partial_\phi&=&\frac{\partial x}{\partial\phi}\partial_x+\frac{\partial_y}{\partial\phi}\partial_y
+\frac{\partial z}{\partial\phi}\partial_z\\
\nonumber
&=&-r\sin\theta\sin\phi\partial_x+\sin\theta\cos\phi\partial_y.
\end{eqnarray}
\end{split}
\end{equation*}
Combining the two equations,
\begin{equation*}
\begin{split}
\begin{eqnarray}
\tau_z&=&-\partial_\phi V(r,\theta,\phi).
\end{eqnarray}
\end{split}
\end{equation*}
Thus, if the potential is independent of the azimuthal angle \(\phi\),
there is no torque about the \(z\) axis and \(L_z\) is conserved.


\section{Symmetries and Conservation Laws}
\label{\detokenize{chapter1:symmetries-and-conservation-laws}}
When we derived the conservation of energy, we assumed that the
potential depended only on position, not on time. If it depended
explicitly on time, one can quickly see that the energy would have
changed at a rate \(\partial_tV(x,y,z,t)\). Note that if there is no
explicit dependence on time, i.e. \(V(x,y,z)\), the potential energy can
depend on time through the variations of \(x,y,z\) with time. However,
that variation does not lead to energy non\sphinxhyphen{}conservation. Further, we
just saw that if a potential does not depend on the azimuthal angle
about some axis, \(\phi\), that the angular momentum about that axis is
conserved.

Now, we relate momentum conservation to translational
invariance. Considering a system of particles with positions,
\(\boldsymbol{r}_i\), if one changed the coordinate system by a translation by a
differential distance \(\boldsymbol{\epsilon}\), the net potential would change
by
\begin{equation*}
\begin{split}
\begin{eqnarray}
\delta V(\boldsymbol{r}_1,\boldsymbol{r}_2\cdots)&=&\sum_i \boldsymbol{\epsilon}\cdot\nabla_i V(\boldsymbol{r}_1,\boldsymbol{r}_2,\cdots)\\
\nonumber
&=&-\sum_i \boldsymbol{\epsilon}\cdot\boldsymbol{F}_i\\
\nonumber
&=&-\frac{d}{dt}\sum_i \boldsymbol{\epsilon}\cdot\boldsymbol{p}_i.
\end{eqnarray}
\end{split}
\end{equation*}
Thus, if the potential is unchanged by a translation of the coordinate
system, the total momentum is conserved. If the potential is
translationally invariant in a given direction, defined by a unit
vector, \(\hat{\epsilon}\) in the \(\boldsymbol{\epsilon}\) direction, one can see
that
\begin{equation*}
\begin{split}
\begin{eqnarray}
\hat{\epsilon}\cdot\nabla_i V(\boldsymbol{r}_i)&=&0.
\end{eqnarray}
\end{split}
\end{equation*}
The component of the total momentum along that axis is conserved. This
is rather obvious for a single particle. If \(V(\boldsymbol{r})\) does not
depend on some coordinate \(x\), then the force in the \(x\) direction is
\(F_x=-\partial_xV=0\), and momentum along the \(x\) direction is
constant.

We showed how the total momentum of an isolated system of particle was conserved, even if the particles feel internal forces in all directions. In that case the potential energy could be written
\begin{equation*}
\begin{split}
\begin{eqnarray}
V=\sum_{i,j\le i}V_{ij}(\boldsymbol{r}_i-\boldsymbol{r}_j).
\end{eqnarray}
\end{split}
\end{equation*}
In this case, a translation leads to \(\boldsymbol{r}_i\rightarrow
\boldsymbol{r}_i+\boldsymbol{\epsilon}\), with the translation equally affecting the
coordinates of each particle. Because the potential depends only on
the relative coordinates, \(\delta V\) is manifestly zero. If one were
to go through the exercise of calculating \(\delta V\) for small
\(\boldsymbol{\epsilon}\), one would find that the term
\(\nabla_i V(\boldsymbol{r}_i-\boldsymbol{r}_j)\) would be canceled by the term
\(\nabla_jV(\boldsymbol{r}_i-\boldsymbol{r}_j)\).

The relation between symmetries of the potential and conserved
quantities (also called constants of motion) is one of the most
profound concepts one should gain from this course. It plays a
critical role in all fields of physics. This is especially true in
quantum mechanics, where a quantity \(A\) is conserved if its operator
commutes with the Hamiltonian. For example if the momentum operator
\(-i\hbar\partial_x\) commutes with the Hamiltonian, momentum is
conserved, and clearly this operator commutes if the Hamiltonian
(which represents the total energy, not just the potential) does not
depend on \(x\). Also in quantum mechanics the angular momentum operator
is \(L_z=-i\hbar\partial_\phi\). In fact, if the potential is unchanged
by rotations about some axis, angular momentum about that axis is
conserved. We return to this concept, from a more formal perspective,
later in the course when Lagrangian mechanics is presented.


\section{Bulding a code for the Earth\sphinxhyphen{}Sun system}
\label{\detokenize{chapter1:bulding-a-code-for-the-earth-sun-system}}
We will now venture into a study of a system which is energy
conserving. The aim is to see if we (since it is not possible to solve
the general equations analytically) we can develop stable numerical
algorithms whose results we can trust!

We solve the equations of motion numerically. We will also compute
quantities like the energy numerically.

We start with a simpler case first, the Earth\sphinxhyphen{}Sun system  in two dimensions only.  The gravitational force \(F_G\) on the earth from the sun is
\begin{equation*}
\begin{split}
\boldsymbol{F}_G=-\frac{GM_{\odot}M_E}{r^3}\boldsymbol{r},
\end{split}
\end{equation*}
where \(G\) is the gravitational constant,
\begin{equation*}
\begin{split}
M_E=6\times 10^{24}\mathrm{Kg},
\end{split}
\end{equation*}
the mass of Earth,
\begin{equation*}
\begin{split}
M_{\odot}=2\times 10^{30}\mathrm{Kg},
\end{split}
\end{equation*}
the mass of the Sun and
\begin{equation*}
\begin{split}
r=1.5\times 10^{11}\mathrm{m},
\end{split}
\end{equation*}
is the distance between Earth and the Sun. The latter defines what we call an astronomical unit \sphinxstylestrong{AU}.
From Newton’s second law we have then for the \(x\) direction
\begin{equation*}
\begin{split}
\frac{d^2x}{dt^2}=-\frac{F_{x}}{M_E},
\end{split}
\end{equation*}
and
\begin{equation*}
\begin{split}
\frac{d^2y}{dt^2}=-\frac{F_{y}}{M_E},
\end{split}
\end{equation*}
for the \(y\) direction.

Here we will use  that  \(x=r\cos{(\theta)}\), \(y=r\sin{(\theta)}\) and
\begin{equation*}
\begin{split}
r = \sqrt{x^2+y^2}.
\end{split}
\end{equation*}
We can rewrite
\begin{equation*}
\begin{split}
F_{x}=-\frac{GM_{\odot}M_E}{r^2}\cos{(\theta)}=-\frac{GM_{\odot}M_E}{r^3}x,
\end{split}
\end{equation*}
and
\begin{equation*}
\begin{split}
F_{y}=-\frac{GM_{\odot}M_E}{r^2}\sin{(\theta)}=-\frac{GM_{\odot}M_E}{r^3}y,
\end{split}
\end{equation*}
for the \(y\) direction.

We can rewrite these two equations
\begin{equation*}
\begin{split}
F_{x}=-\frac{GM_{\odot}M_E}{r^2}\cos{(\theta)}=-\frac{GM_{\odot}M_E}{r^3}x,
\end{split}
\end{equation*}
and
\begin{equation*}
\begin{split}
F_{y}=-\frac{GM_{\odot}M_E}{r^2}\sin{(\theta)}=-\frac{GM_{\odot}M_E}{r^3}y,
\end{split}
\end{equation*}
as four first\sphinxhyphen{}order coupled differential equations

1
5
8

\textless{}
\textless{}
\textless{}
!
!
M
A
T
H
\_
B
L
O
C
K

1
5
9

\textless{}
\textless{}
\textless{}
!
!
M
A
T
H
\_
B
L
O
C
K

1
6
0

\textless{}
\textless{}
\textless{}
!
!
M
A
T
H
\_
B
L
O
C
K
\begin{equation*}
\begin{split}
\frac{dy}{dt}=v_y.
\end{split}
\end{equation*}

\section{Building a code for the solar system, final coupled equations}
\label{\detokenize{chapter1:building-a-code-for-the-solar-system-final-coupled-equations}}
The four coupled differential equations

1
6
2

\textless{}
\textless{}
\textless{}
!
!
M
A
T
H
\_
B
L
O
C
K

1
6
3

\textless{}
\textless{}
\textless{}
!
!
M
A
T
H
\_
B
L
O
C
K

1
6
4

\textless{}
\textless{}
\textless{}
!
!
M
A
T
H
\_
B
L
O
C
K
\begin{equation*}
\begin{split}
\frac{dy}{dt}=v_y,
\end{split}
\end{equation*}
can be turned into dimensionless equations or we can introduce astronomical units with \(1\) AU = \(1.5\times 10^{11}\).

Using the equations from circular motion (with \(r =1\mathrm{AU}\))
\begin{equation*}
\begin{split}
\frac{M_E v^2}{r} = F = \frac{GM_{\odot}M_E}{r^2},
\end{split}
\end{equation*}
we have
\begin{equation*}
\begin{split}
GM_{\odot}=v^2r,
\end{split}
\end{equation*}
and using that the velocity of Earth (assuming circular motion) is
\(v = 2\pi r/\mathrm{yr}=2\pi\mathrm{AU}/\mathrm{yr}\), we have
\begin{equation*}
\begin{split}
GM_{\odot}= v^2r = 4\pi^2 \frac{(\mathrm{AU})^3}{\mathrm{yr}^2}.
\end{split}
\end{equation*}

\section{Building a code for the solar system, discretized equations}
\label{\detokenize{chapter1:building-a-code-for-the-solar-system-discretized-equations}}
The four coupled differential equations can then be discretized using Euler’s method as (with step length \(h\))

1
6
9

\textless{}
\textless{}
\textless{}
!
!
M
A
T
H
\_
B
L
O
C
K

1
7
0

\textless{}
\textless{}
\textless{}
!
!
M
A
T
H
\_
B
L
O
C
K

1
7
1

\textless{}
\textless{}
\textless{}
!
!
M
A
T
H
\_
B
L
O
C
K
\begin{equation*}
\begin{split}
y_{i+1}=y_i+hv_{y,i},
\end{split}
\end{equation*}

\section{Code Example with Euler’s Method}
\label{\detokenize{chapter1:code-example-with-euler-s-method}}
The code here implements Euler’s method for the Earth\sphinxhyphen{}Sun system using a more compact way of representing the vectors. Alternatively, you could have spelled out all the variables \(v_x\), \(v_y\), \(x\) and \(y\) as one\sphinxhyphen{}dimensional arrays.

\begin{sphinxVerbatim}[commandchars=\\\{\}]
\PYG{c+c1}{\PYGZsh{} Common imports}
\PYG{k+kn}{import} \PYG{n+nn}{numpy} \PYG{k}{as} \PYG{n+nn}{np}
\PYG{k+kn}{import} \PYG{n+nn}{pandas} \PYG{k}{as} \PYG{n+nn}{pd}
\PYG{k+kn}{from} \PYG{n+nn}{math} \PYG{k+kn}{import} \PYG{o}{*}
\PYG{k+kn}{import} \PYG{n+nn}{matplotlib}\PYG{n+nn}{.}\PYG{n+nn}{pyplot} \PYG{k}{as} \PYG{n+nn}{plt}
\PYG{k+kn}{import} \PYG{n+nn}{os}

\PYG{c+c1}{\PYGZsh{} Where to save the figures and data files}
\PYG{n}{PROJECT\PYGZus{}ROOT\PYGZus{}DIR} \PYG{o}{=} \PYG{l+s+s2}{\PYGZdq{}}\PYG{l+s+s2}{Results}\PYG{l+s+s2}{\PYGZdq{}}
\PYG{n}{FIGURE\PYGZus{}ID} \PYG{o}{=} \PYG{l+s+s2}{\PYGZdq{}}\PYG{l+s+s2}{Results/FigureFiles}\PYG{l+s+s2}{\PYGZdq{}}
\PYG{n}{DATA\PYGZus{}ID} \PYG{o}{=} \PYG{l+s+s2}{\PYGZdq{}}\PYG{l+s+s2}{DataFiles/}\PYG{l+s+s2}{\PYGZdq{}}

\PYG{k}{if} \PYG{o+ow}{not} \PYG{n}{os}\PYG{o}{.}\PYG{n}{path}\PYG{o}{.}\PYG{n}{exists}\PYG{p}{(}\PYG{n}{PROJECT\PYGZus{}ROOT\PYGZus{}DIR}\PYG{p}{)}\PYG{p}{:}
    \PYG{n}{os}\PYG{o}{.}\PYG{n}{mkdir}\PYG{p}{(}\PYG{n}{PROJECT\PYGZus{}ROOT\PYGZus{}DIR}\PYG{p}{)}

\PYG{k}{if} \PYG{o+ow}{not} \PYG{n}{os}\PYG{o}{.}\PYG{n}{path}\PYG{o}{.}\PYG{n}{exists}\PYG{p}{(}\PYG{n}{FIGURE\PYGZus{}ID}\PYG{p}{)}\PYG{p}{:}
    \PYG{n}{os}\PYG{o}{.}\PYG{n}{makedirs}\PYG{p}{(}\PYG{n}{FIGURE\PYGZus{}ID}\PYG{p}{)}

\PYG{k}{if} \PYG{o+ow}{not} \PYG{n}{os}\PYG{o}{.}\PYG{n}{path}\PYG{o}{.}\PYG{n}{exists}\PYG{p}{(}\PYG{n}{DATA\PYGZus{}ID}\PYG{p}{)}\PYG{p}{:}
    \PYG{n}{os}\PYG{o}{.}\PYG{n}{makedirs}\PYG{p}{(}\PYG{n}{DATA\PYGZus{}ID}\PYG{p}{)}

\PYG{k}{def} \PYG{n+nf}{image\PYGZus{}path}\PYG{p}{(}\PYG{n}{fig\PYGZus{}id}\PYG{p}{)}\PYG{p}{:}
    \PYG{k}{return} \PYG{n}{os}\PYG{o}{.}\PYG{n}{path}\PYG{o}{.}\PYG{n}{join}\PYG{p}{(}\PYG{n}{FIGURE\PYGZus{}ID}\PYG{p}{,} \PYG{n}{fig\PYGZus{}id}\PYG{p}{)}

\PYG{k}{def} \PYG{n+nf}{data\PYGZus{}path}\PYG{p}{(}\PYG{n}{dat\PYGZus{}id}\PYG{p}{)}\PYG{p}{:}
    \PYG{k}{return} \PYG{n}{os}\PYG{o}{.}\PYG{n}{path}\PYG{o}{.}\PYG{n}{join}\PYG{p}{(}\PYG{n}{DATA\PYGZus{}ID}\PYG{p}{,} \PYG{n}{dat\PYGZus{}id}\PYG{p}{)}

\PYG{k}{def} \PYG{n+nf}{save\PYGZus{}fig}\PYG{p}{(}\PYG{n}{fig\PYGZus{}id}\PYG{p}{)}\PYG{p}{:}
    \PYG{n}{plt}\PYG{o}{.}\PYG{n}{savefig}\PYG{p}{(}\PYG{n}{image\PYGZus{}path}\PYG{p}{(}\PYG{n}{fig\PYGZus{}id}\PYG{p}{)} \PYG{o}{+} \PYG{l+s+s2}{\PYGZdq{}}\PYG{l+s+s2}{.png}\PYG{l+s+s2}{\PYGZdq{}}\PYG{p}{,} \PYG{n+nb}{format}\PYG{o}{=}\PYG{l+s+s1}{\PYGZsq{}}\PYG{l+s+s1}{png}\PYG{l+s+s1}{\PYGZsq{}}\PYG{p}{)}


\PYG{n}{DeltaT} \PYG{o}{=} \PYG{l+m+mf}{0.001}
\PYG{c+c1}{\PYGZsh{}set up arrays }
\PYG{n}{tfinal} \PYG{o}{=} \PYG{l+m+mi}{10} \PYG{c+c1}{\PYGZsh{} in years}
\PYG{n}{n} \PYG{o}{=} \PYG{n}{ceil}\PYG{p}{(}\PYG{n}{tfinal}\PYG{o}{/}\PYG{n}{DeltaT}\PYG{p}{)}
\PYG{c+c1}{\PYGZsh{} set up arrays for t, a, v, and x}
\PYG{n}{t} \PYG{o}{=} \PYG{n}{np}\PYG{o}{.}\PYG{n}{zeros}\PYG{p}{(}\PYG{n}{n}\PYG{p}{)}
\PYG{n}{v} \PYG{o}{=} \PYG{n}{np}\PYG{o}{.}\PYG{n}{zeros}\PYG{p}{(}\PYG{p}{(}\PYG{n}{n}\PYG{p}{,}\PYG{l+m+mi}{2}\PYG{p}{)}\PYG{p}{)}
\PYG{n}{r} \PYG{o}{=} \PYG{n}{np}\PYG{o}{.}\PYG{n}{zeros}\PYG{p}{(}\PYG{p}{(}\PYG{n}{n}\PYG{p}{,}\PYG{l+m+mi}{2}\PYG{p}{)}\PYG{p}{)}
\PYG{c+c1}{\PYGZsh{} Initial conditions as compact 2\PYGZhy{}dimensional arrays}
\PYG{n}{r0} \PYG{o}{=} \PYG{n}{np}\PYG{o}{.}\PYG{n}{array}\PYG{p}{(}\PYG{p}{[}\PYG{l+m+mf}{1.0}\PYG{p}{,}\PYG{l+m+mf}{0.0}\PYG{p}{]}\PYG{p}{)}
\PYG{n}{v0} \PYG{o}{=} \PYG{n}{np}\PYG{o}{.}\PYG{n}{array}\PYG{p}{(}\PYG{p}{[}\PYG{l+m+mf}{0.0}\PYG{p}{,}\PYG{l+m+mi}{2}\PYG{o}{*}\PYG{n}{pi}\PYG{p}{]}\PYG{p}{)}
\PYG{n}{r}\PYG{p}{[}\PYG{l+m+mi}{0}\PYG{p}{]} \PYG{o}{=} \PYG{n}{r0}
\PYG{n}{v}\PYG{p}{[}\PYG{l+m+mi}{0}\PYG{p}{]} \PYG{o}{=} \PYG{n}{v0}
\PYG{n}{Fourpi2} \PYG{o}{=} \PYG{l+m+mi}{4}\PYG{o}{*}\PYG{n}{pi}\PYG{o}{*}\PYG{n}{pi}
\PYG{c+c1}{\PYGZsh{} Start integrating using Euler\PYGZsq{}s method}
\PYG{k}{for} \PYG{n}{i} \PYG{o+ow}{in} \PYG{n+nb}{range}\PYG{p}{(}\PYG{n}{n}\PYG{o}{\PYGZhy{}}\PYG{l+m+mi}{1}\PYG{p}{)}\PYG{p}{:}
    \PYG{c+c1}{\PYGZsh{} Set up the acceleration}
    \PYG{c+c1}{\PYGZsh{} Here you could have defined your own function for this}
    \PYG{n}{rabs} \PYG{o}{=} \PYG{n}{sqrt}\PYG{p}{(}\PYG{n+nb}{sum}\PYG{p}{(}\PYG{n}{r}\PYG{p}{[}\PYG{n}{i}\PYG{p}{]}\PYG{o}{*}\PYG{n}{r}\PYG{p}{[}\PYG{n}{i}\PYG{p}{]}\PYG{p}{)}\PYG{p}{)}
    \PYG{n}{a} \PYG{o}{=}  \PYG{o}{\PYGZhy{}}\PYG{n}{Fourpi2}\PYG{o}{*}\PYG{n}{r}\PYG{p}{[}\PYG{n}{i}\PYG{p}{]}\PYG{o}{/}\PYG{p}{(}\PYG{n}{rabs}\PYG{o}{*}\PYG{o}{*}\PYG{l+m+mi}{3}\PYG{p}{)}
    \PYG{c+c1}{\PYGZsh{} update velocity, time and position using Euler\PYGZsq{}s forward method}
    \PYG{n}{v}\PYG{p}{[}\PYG{n}{i}\PYG{o}{+}\PYG{l+m+mi}{1}\PYG{p}{]} \PYG{o}{=} \PYG{n}{v}\PYG{p}{[}\PYG{n}{i}\PYG{p}{]} \PYG{o}{+} \PYG{n}{DeltaT}\PYG{o}{*}\PYG{n}{a}
    \PYG{n}{r}\PYG{p}{[}\PYG{n}{i}\PYG{o}{+}\PYG{l+m+mi}{1}\PYG{p}{]} \PYG{o}{=} \PYG{n}{r}\PYG{p}{[}\PYG{n}{i}\PYG{p}{]} \PYG{o}{+} \PYG{n}{DeltaT}\PYG{o}{*}\PYG{n}{v}\PYG{p}{[}\PYG{n}{i}\PYG{p}{]}
    \PYG{n}{t}\PYG{p}{[}\PYG{n}{i}\PYG{o}{+}\PYG{l+m+mi}{1}\PYG{p}{]} \PYG{o}{=} \PYG{n}{t}\PYG{p}{[}\PYG{n}{i}\PYG{p}{]} \PYG{o}{+} \PYG{n}{DeltaT}
\PYG{c+c1}{\PYGZsh{} Plot position as function of time    }
\PYG{n}{fig}\PYG{p}{,} \PYG{n}{ax} \PYG{o}{=} \PYG{n}{plt}\PYG{o}{.}\PYG{n}{subplots}\PYG{p}{(}\PYG{p}{)}
\PYG{c+c1}{\PYGZsh{}ax.set\PYGZus{}xlim(0, tfinal)}
\PYG{n}{ax}\PYG{o}{.}\PYG{n}{set\PYGZus{}ylabel}\PYG{p}{(}\PYG{l+s+s1}{\PYGZsq{}}\PYG{l+s+s1}{x[m]}\PYG{l+s+s1}{\PYGZsq{}}\PYG{p}{)}
\PYG{n}{ax}\PYG{o}{.}\PYG{n}{set\PYGZus{}xlabel}\PYG{p}{(}\PYG{l+s+s1}{\PYGZsq{}}\PYG{l+s+s1}{y[m]}\PYG{l+s+s1}{\PYGZsq{}}\PYG{p}{)}
\PYG{n}{ax}\PYG{o}{.}\PYG{n}{plot}\PYG{p}{(}\PYG{n}{r}\PYG{p}{[}\PYG{p}{:}\PYG{p}{,}\PYG{l+m+mi}{0}\PYG{p}{]}\PYG{p}{,} \PYG{n}{r}\PYG{p}{[}\PYG{p}{:}\PYG{p}{,}\PYG{l+m+mi}{1}\PYG{p}{]}\PYG{p}{)}
\PYG{n}{fig}\PYG{o}{.}\PYG{n}{tight\PYGZus{}layout}\PYG{p}{(}\PYG{p}{)}
\PYG{n}{save\PYGZus{}fig}\PYG{p}{(}\PYG{l+s+s2}{\PYGZdq{}}\PYG{l+s+s2}{EarthSunEuler}\PYG{l+s+s2}{\PYGZdq{}}\PYG{p}{)}
\PYG{n}{plt}\PYG{o}{.}\PYG{n}{show}\PYG{p}{(}\PYG{p}{)}
\end{sphinxVerbatim}


\section{Problems with Euler’s Method}
\label{\detokenize{chapter1:problems-with-euler-s-method}}
We notice here that Euler’s method doesn’t give a stable orbit. It
means that we cannot trust Euler’s method. In a deeper way, as we will
see in homework 5, Euler’s method does not conserve energy. It is an
example of an integrator which is not
\sphinxhref{https://en.wikipedia.org/wiki/Symplectic\_integrator}{symplectic}.

Here we present thus two methods, which with simple changes allow us to avoid these pitfalls. The simplest possible extension is the so\sphinxhyphen{}called Euler\sphinxhyphen{}Cromer method.
The changes we need to make to our code are indeed marginal here.
We need simply to replace

\begin{sphinxVerbatim}[commandchars=\\\{\}]
    \PYG{n}{r}\PYG{p}{[}\PYG{n}{i}\PYG{o}{+}\PYG{l+m+mi}{1}\PYG{p}{]} \PYG{o}{=} \PYG{n}{r}\PYG{p}{[}\PYG{n}{i}\PYG{p}{]} \PYG{o}{+} \PYG{n}{DeltaT}\PYG{o}{*}\PYG{n}{v}\PYG{p}{[}\PYG{n}{i}\PYG{p}{]}
\end{sphinxVerbatim}

in the above code with the velocity at the new time \(t_{i+1}\)

\begin{sphinxVerbatim}[commandchars=\\\{\}]
    \PYG{n}{r}\PYG{p}{[}\PYG{n}{i}\PYG{o}{+}\PYG{l+m+mi}{1}\PYG{p}{]} \PYG{o}{=} \PYG{n}{r}\PYG{p}{[}\PYG{n}{i}\PYG{p}{]} \PYG{o}{+} \PYG{n}{DeltaT}\PYG{o}{*}\PYG{n}{v}\PYG{p}{[}\PYG{n}{i}\PYG{o}{+}\PYG{l+m+mi}{1}\PYG{p}{]}
\end{sphinxVerbatim}

By this simple caveat we get stable orbits.
Below we derive the Euler\sphinxhyphen{}Cromer method as well as one of the most utlized algorithms for sovling the above type of problems, the so\sphinxhyphen{}called Velocity\sphinxhyphen{}Verlet method.


\section{Deriving the Euler\sphinxhyphen{}Cromer Method}
\label{\detokenize{chapter1:deriving-the-euler-cromer-method}}
Let us repeat Euler’s method.
We have a differential equation




\begin{equation*}
\begin{split}
\begin{equation}
y'(t_i)=f(t_i,y_i)   
\label{_auto34} \tag{37}
\end{equation}
\end{split}
\end{equation*}
and if we truncate at the first derivative, we have from the Taylor expansion




\begin{equation*}
\begin{split}
\begin{equation}
y_{i+1}=y(t_i) + (\Delta t) f(t_i,y_i) + O(\Delta t^2), \label{eq:euler} \tag{38}
\end{equation}
\end{split}
\end{equation*}
which when complemented with \(t_{i+1}=t_i+\Delta t\) forms
the algorithm for the well\sphinxhyphen{}known Euler method.
Note that at every step we make an approximation error
of the order of \(O(\Delta t^2)\), however the total error is the sum over all
steps \(N=(b-a)/(\Delta t)\) for \(t\in [a,b]\), yielding thus a global error which goes like
\(NO(\Delta t^2)\approx O(\Delta t)\).

To make Euler’s method more precise we can obviously
decrease \(\Delta t\) (increase \(N\)), but this can lead to loss of numerical precision.
Euler’s method is not recommended for precision calculation,
although it is handy to use in order to get a first
view on how a solution may look like.

Euler’s method is asymmetric in time, since it uses information about the derivative at the beginning
of the time interval. This means that we evaluate the position at \(y_1\) using the velocity
at \(v_0\). A simple variation is to determine \(x_{n+1}\) using the velocity at
\(v_{n+1}\), that is (in a slightly more generalized form)




\begin{equation*}
\begin{split}
\begin{equation} 
y_{n+1}=y_{n}+ v_{n+1}+O(\Delta t^2)
\label{_auto35} \tag{39}
\end{equation}
\end{split}
\end{equation*}
and




\begin{equation*}
\begin{split}
\begin{equation}
v_{n+1}=v_{n}+(\Delta t) a_{n}+O(\Delta t^2).
\label{_auto36} \tag{40}
\end{equation}
\end{split}
\end{equation*}
The acceleration \(a_n\) is a function of \(a_n(y_n, v_n, t_n)\) and needs to be evaluated
as well. This is the Euler\sphinxhyphen{}Cromer method.

\sphinxstylestrong{Exercise}: go back to the above code with Euler’s method and add the Euler\sphinxhyphen{}Cromer method.


\section{Deriving the Velocity\sphinxhyphen{}Verlet Method}
\label{\detokenize{chapter1:deriving-the-velocity-verlet-method}}
Let us stay with \(x\) (position) and \(v\) (velocity) as the quantities we are interested in.

We have the Taylor expansion for the position given by
\begin{equation*}
\begin{split}
x_{i+1} = x_i+(\Delta t)v_i+\frac{(\Delta t)^2}{2}a_i+O((\Delta t)^3).
\end{split}
\end{equation*}
The corresponding expansion for the velocity is
\begin{equation*}
\begin{split}
v_{i+1} = v_i+(\Delta t)a_i+\frac{(\Delta t)^2}{2}v^{(2)}_i+O((\Delta t)^3).
\end{split}
\end{equation*}
Via Newton’s second law we have normally an analytical expression for the derivative of the velocity, namely
\begin{equation*}
\begin{split}
a_i= \frac{d^2 x}{dt^2}\vert_{i}=\frac{d v}{dt}\vert_{i}= \frac{F(x_i,v_i,t_i)}{m}.
\end{split}
\end{equation*}
If we add to this the corresponding expansion for the derivative of the velocity
\begin{equation*}
\begin{split}
v^{(1)}_{i+1} = a_{i+1}= a_i+(\Delta t)v^{(2)}_i+O((\Delta t)^2)=a_i+(\Delta t)v^{(2)}_i+O((\Delta t)^2),
\end{split}
\end{equation*}
and retain only terms up to the second derivative of the velocity since our error goes as \(O(h^3)\), we have
\begin{equation*}
\begin{split}
(\Delta t)v^{(2)}_i\approx a_{i+1}-a_i.
\end{split}
\end{equation*}
We can then rewrite the Taylor expansion for the velocity as
\begin{equation*}
\begin{split}
v_{i+1} = v_i+\frac{(\Delta t)}{2}\left( a_{i+1}+a_{i}\right)+O((\Delta t)^3).
\end{split}
\end{equation*}

\section{The velocity Verlet method}
\label{\detokenize{chapter1:the-velocity-verlet-method}}
Our final equations for the position and the velocity become then
\begin{equation*}
\begin{split}
x_{i+1} = x_i+(\Delta t)v_i+\frac{(\Delta t)^2}{2}a_{i}+O((\Delta t)^3),
\end{split}
\end{equation*}
and
\begin{equation*}
\begin{split}
v_{i+1} = v_i+\frac{(\Delta t)}{2}\left(a_{i+1}+a_{i}\right)+O((\Delta t)^3).
\end{split}
\end{equation*}
Note well that the term \(a_{i+1}\) depends on the position at \(x_{i+1}\). This means that you need to calculate
the position at the updated time \(t_{i+1}\) before the computing the next velocity.  Note also that the derivative of the velocity at the time
\(t_i\) used in the updating of the position can be reused in the calculation of the velocity update as well.


\section{Adding the Velocity\sphinxhyphen{}Verlet Method}
\label{\detokenize{chapter1:adding-the-velocity-verlet-method}}
We can now easily add the Verlet method to our original code as

\begin{sphinxVerbatim}[commandchars=\\\{\}]
\PYG{n}{DeltaT} \PYG{o}{=} \PYG{l+m+mf}{0.01}
\PYG{c+c1}{\PYGZsh{}set up arrays }
\PYG{n}{tfinal} \PYG{o}{=} \PYG{l+m+mi}{10}
\PYG{n}{n} \PYG{o}{=} \PYG{n}{ceil}\PYG{p}{(}\PYG{n}{tfinal}\PYG{o}{/}\PYG{n}{DeltaT}\PYG{p}{)}
\PYG{c+c1}{\PYGZsh{} set up arrays for t, a, v, and x}
\PYG{n}{t} \PYG{o}{=} \PYG{n}{np}\PYG{o}{.}\PYG{n}{zeros}\PYG{p}{(}\PYG{n}{n}\PYG{p}{)}
\PYG{n}{v} \PYG{o}{=} \PYG{n}{np}\PYG{o}{.}\PYG{n}{zeros}\PYG{p}{(}\PYG{p}{(}\PYG{n}{n}\PYG{p}{,}\PYG{l+m+mi}{2}\PYG{p}{)}\PYG{p}{)}
\PYG{n}{r} \PYG{o}{=} \PYG{n}{np}\PYG{o}{.}\PYG{n}{zeros}\PYG{p}{(}\PYG{p}{(}\PYG{n}{n}\PYG{p}{,}\PYG{l+m+mi}{2}\PYG{p}{)}\PYG{p}{)}
\PYG{c+c1}{\PYGZsh{} Initial conditions as compact 2\PYGZhy{}dimensional arrays}
\PYG{n}{r0} \PYG{o}{=} \PYG{n}{np}\PYG{o}{.}\PYG{n}{array}\PYG{p}{(}\PYG{p}{[}\PYG{l+m+mf}{1.0}\PYG{p}{,}\PYG{l+m+mf}{0.0}\PYG{p}{]}\PYG{p}{)}
\PYG{n}{v0} \PYG{o}{=} \PYG{n}{np}\PYG{o}{.}\PYG{n}{array}\PYG{p}{(}\PYG{p}{[}\PYG{l+m+mf}{0.0}\PYG{p}{,}\PYG{l+m+mi}{2}\PYG{o}{*}\PYG{n}{pi}\PYG{p}{]}\PYG{p}{)}
\PYG{n}{r}\PYG{p}{[}\PYG{l+m+mi}{0}\PYG{p}{]} \PYG{o}{=} \PYG{n}{r0}
\PYG{n}{v}\PYG{p}{[}\PYG{l+m+mi}{0}\PYG{p}{]} \PYG{o}{=} \PYG{n}{v0}
\PYG{n}{Fourpi2} \PYG{o}{=} \PYG{l+m+mi}{4}\PYG{o}{*}\PYG{n}{pi}\PYG{o}{*}\PYG{n}{pi}
\PYG{c+c1}{\PYGZsh{} Start integrating using the Velocity\PYGZhy{}Verlet  method}
\PYG{k}{for} \PYG{n}{i} \PYG{o+ow}{in} \PYG{n+nb}{range}\PYG{p}{(}\PYG{n}{n}\PYG{o}{\PYGZhy{}}\PYG{l+m+mi}{1}\PYG{p}{)}\PYG{p}{:}
    \PYG{c+c1}{\PYGZsh{} Set up forces, air resistance FD, note now that we need the norm of the vecto}
    \PYG{c+c1}{\PYGZsh{} Here you could have defined your own function for this}
    \PYG{n}{rabs} \PYG{o}{=} \PYG{n}{sqrt}\PYG{p}{(}\PYG{n+nb}{sum}\PYG{p}{(}\PYG{n}{r}\PYG{p}{[}\PYG{n}{i}\PYG{p}{]}\PYG{o}{*}\PYG{n}{r}\PYG{p}{[}\PYG{n}{i}\PYG{p}{]}\PYG{p}{)}\PYG{p}{)}
    \PYG{n}{a} \PYG{o}{=}  \PYG{o}{\PYGZhy{}}\PYG{n}{Fourpi2}\PYG{o}{*}\PYG{n}{r}\PYG{p}{[}\PYG{n}{i}\PYG{p}{]}\PYG{o}{/}\PYG{p}{(}\PYG{n}{rabs}\PYG{o}{*}\PYG{o}{*}\PYG{l+m+mi}{3}\PYG{p}{)}
    \PYG{c+c1}{\PYGZsh{} update velocity, time and position using the Velocity\PYGZhy{}Verlet method}
    \PYG{n}{r}\PYG{p}{[}\PYG{n}{i}\PYG{o}{+}\PYG{l+m+mi}{1}\PYG{p}{]} \PYG{o}{=} \PYG{n}{r}\PYG{p}{[}\PYG{n}{i}\PYG{p}{]} \PYG{o}{+} \PYG{n}{DeltaT}\PYG{o}{*}\PYG{n}{v}\PYG{p}{[}\PYG{n}{i}\PYG{p}{]}\PYG{o}{+}\PYG{l+m+mf}{0.5}\PYG{o}{*}\PYG{p}{(}\PYG{n}{DeltaT}\PYG{o}{*}\PYG{o}{*}\PYG{l+m+mi}{2}\PYG{p}{)}\PYG{o}{*}\PYG{n}{a}
    \PYG{n}{rabs} \PYG{o}{=} \PYG{n}{sqrt}\PYG{p}{(}\PYG{n+nb}{sum}\PYG{p}{(}\PYG{n}{r}\PYG{p}{[}\PYG{n}{i}\PYG{o}{+}\PYG{l+m+mi}{1}\PYG{p}{]}\PYG{o}{*}\PYG{n}{r}\PYG{p}{[}\PYG{n}{i}\PYG{o}{+}\PYG{l+m+mi}{1}\PYG{p}{]}\PYG{p}{)}\PYG{p}{)}
    \PYG{n}{anew} \PYG{o}{=} \PYG{o}{\PYGZhy{}}\PYG{l+m+mi}{4}\PYG{o}{*}\PYG{p}{(}\PYG{n}{pi}\PYG{o}{*}\PYG{o}{*}\PYG{l+m+mi}{2}\PYG{p}{)}\PYG{o}{*}\PYG{n}{r}\PYG{p}{[}\PYG{n}{i}\PYG{o}{+}\PYG{l+m+mi}{1}\PYG{p}{]}\PYG{o}{/}\PYG{p}{(}\PYG{n}{rabs}\PYG{o}{*}\PYG{o}{*}\PYG{l+m+mi}{3}\PYG{p}{)}
    \PYG{n}{v}\PYG{p}{[}\PYG{n}{i}\PYG{o}{+}\PYG{l+m+mi}{1}\PYG{p}{]} \PYG{o}{=} \PYG{n}{v}\PYG{p}{[}\PYG{n}{i}\PYG{p}{]} \PYG{o}{+} \PYG{l+m+mf}{0.5}\PYG{o}{*}\PYG{n}{DeltaT}\PYG{o}{*}\PYG{p}{(}\PYG{n}{a}\PYG{o}{+}\PYG{n}{anew}\PYG{p}{)}
    \PYG{n}{t}\PYG{p}{[}\PYG{n}{i}\PYG{o}{+}\PYG{l+m+mi}{1}\PYG{p}{]} \PYG{o}{=} \PYG{n}{t}\PYG{p}{[}\PYG{n}{i}\PYG{p}{]} \PYG{o}{+} \PYG{n}{DeltaT}
\PYG{c+c1}{\PYGZsh{} Plot position as function of time    }
\PYG{n}{fig}\PYG{p}{,} \PYG{n}{ax} \PYG{o}{=} \PYG{n}{plt}\PYG{o}{.}\PYG{n}{subplots}\PYG{p}{(}\PYG{p}{)}
\PYG{n}{ax}\PYG{o}{.}\PYG{n}{set\PYGZus{}ylabel}\PYG{p}{(}\PYG{l+s+s1}{\PYGZsq{}}\PYG{l+s+s1}{x[m]}\PYG{l+s+s1}{\PYGZsq{}}\PYG{p}{)}
\PYG{n}{ax}\PYG{o}{.}\PYG{n}{set\PYGZus{}xlabel}\PYG{p}{(}\PYG{l+s+s1}{\PYGZsq{}}\PYG{l+s+s1}{y[m]}\PYG{l+s+s1}{\PYGZsq{}}\PYG{p}{)}
\PYG{n}{ax}\PYG{o}{.}\PYG{n}{plot}\PYG{p}{(}\PYG{n}{r}\PYG{p}{[}\PYG{p}{:}\PYG{p}{,}\PYG{l+m+mi}{0}\PYG{p}{]}\PYG{p}{,} \PYG{n}{r}\PYG{p}{[}\PYG{p}{:}\PYG{p}{,}\PYG{l+m+mi}{1}\PYG{p}{]}\PYG{p}{)}
\PYG{n}{fig}\PYG{o}{.}\PYG{n}{tight\PYGZus{}layout}\PYG{p}{(}\PYG{p}{)}
\PYG{n}{save\PYGZus{}fig}\PYG{p}{(}\PYG{l+s+s2}{\PYGZdq{}}\PYG{l+s+s2}{EarthSunVV}\PYG{l+s+s2}{\PYGZdq{}}\PYG{p}{)}
\PYG{n}{plt}\PYG{o}{.}\PYG{n}{show}\PYG{p}{(}\PYG{p}{)}
\end{sphinxVerbatim}

You can easily generalize the calculation of the forces by defining a function
which takes in as input the various variables. We leave this as a challenge to you.


\section{Studying Energy Conservation}
\label{\detokenize{chapter1:studying-energy-conservation}}
In order to study the conservation of energy, we will need to perform
a numerical integration, unless we can integrate analytically. Here we
present the Trapezoidal rule as a the simplest possible approximation.


\section{Numerical Integration}
\label{\detokenize{chapter1:numerical-integration}}
It is also useful to consider methods to integrate numerically.
Let us consider the following case.
We have  classical electron which moves in the \(x\)\sphinxhyphen{}direction along a surface. The force from the surface is
\begin{equation*}
\begin{split}
\boldsymbol{F}(x)=-F_0\sin{(\frac{2\pi x}{b})}\boldsymbol{e}_x.
\end{split}
\end{equation*}
The constant \(b\) represents the distance between atoms at the surface of the material, \(F_0\) is a constant and \(x\) is the position of the electron.
Using the work\sphinxhyphen{}energy theorem we can find the work \(W\) done when moving an electron from a position \(x_0\) to a final position \(x\) through the
integral
\begin{equation*}
\begin{split}
W=-\int_{x_0}^x \boldsymbol{F}(x')dx' =  \int_{x_0}^x F_0\sin{(\frac{2\pi x'}{b})} dx',
\end{split}
\end{equation*}
which results in
\begin{equation*}
\begin{split}
W=\frac{F_0b}{2\pi}\left[\cos{(\frac{2\pi x}{b})}-\cos{(\frac{2\pi x_0}{b})}\right].
\end{split}
\end{equation*}

\section{Numerical Integration}
\label{\detokenize{chapter1:id2}}
There are several numerical algorithms for finding an integral
numerically. The more familiar ones like the rectangular rule or the
trapezoidal rule have simple geometric interpretations.

Let us look at the mathematical details of what are called equal\sphinxhyphen{}step methods, also known as Newton\sphinxhyphen{}Cotes quadrature.


\section{Newton\sphinxhyphen{}Cotes Quadrature or equal\sphinxhyphen{}step methods}
\label{\detokenize{chapter1:newton-cotes-quadrature-or-equal-step-methods}}
The integral




\begin{equation*}
\begin{split}
\begin{equation}
   I=\int_a^bf(x) dx
\label{eq:integraldef} \tag{41}
\end{equation}
\end{split}
\end{equation*}
has a very simple meaning. The integral is the
area enscribed by the function \(f(x)\) starting from \(x=a\) to  \(x=b\). It is subdivided in several smaller areas whose evaluation is to  be approximated by different techniques. The areas under the curve can for example  be approximated by rectangular boxes or trapezoids.


\section{Basic philosophy of equal\sphinxhyphen{}step methods}
\label{\detokenize{chapter1:basic-philosophy-of-equal-step-methods}}
In considering equal step  methods, our basic approach is that of approximating
a function \(f(x)\) with a polynomial of at most
degree \(N-1\), given \(N\) integration points. If our polynomial is of degree \(1\),
the function will be approximated with \(f(x)\approx a_0+a_1x\).

The algorithm for these integration methods  is rather simple, and the number of approximations perhaps  unlimited!
\begin{itemize}
\item {} 
Choose a step size \(h=(b-a)/N\)  where \(N\) is the number of steps and \(a\) and \(b\) the lower and upper limits of integration.

\item {} 
With a given step length we rewrite the integral as

\end{itemize}
\begin{equation*}
\begin{split}
\int_a^bf(x) dx= \int_a^{a+h}f(x)dx + \int_{a+h}^{a+2h}f(x)dx+\dots \int_{b-h}^{b}f(x)dx.
\end{split}
\end{equation*}\begin{itemize}
\item {} 
The strategy then is to find a reliable polynomial approximation   for \(f(x)\) in the various intervals.  Choosing a given approximation for  \(f(x)\), we obtain a specific approximation to the  integral.

\item {} 
With this approximation to \(f(x)\) we perform the integration by computing the integrals over all subintervals.

\end{itemize}

One possible strategy then is to find a reliable polynomial expansion for \(f(x)\) in the smaller
subintervals. Consider for example evaluating
\begin{equation*}
\begin{split}
\int_a^{a+2h}f(x)dx,
\end{split}
\end{equation*}
which we rewrite as




\begin{equation*}
\begin{split}
\begin{equation}
\int_a^{a+2h}f(x)dx=\int_{x_0-h}^{x_0+h}f(x)dx.
\label{eq:hhint} \tag{42}
\end{equation}
\end{split}
\end{equation*}
We have chosen a midpoint \(x_0\) and have defined \(x_0=a+h\).


\section{The rectangle method}
\label{\detokenize{chapter1:the-rectangle-method}}
A very simple approach is the so\sphinxhyphen{}called midpoint or rectangle method.
In this case the integration area is split in a given number of rectangles with length \(h\) and height given by the mid\sphinxhyphen{}point value of the function.  This gives the following simple rule for approximating an integral




\begin{equation*}
\begin{split}
\begin{equation}
I=\int_a^bf(x) dx \approx  h\sum_{i=1}^N f(x_{i-1/2}), 
\label{eq:rectangle} \tag{43}
\end{equation}
\end{split}
\end{equation*}
where \(f(x_{i-1/2})\) is the midpoint value of \(f\) for a given rectangle. We will discuss its truncation
error below.  It is easy to implement this algorithm,  as shown below

The correct mathematical expression for the local error for the rectangular rule \(R_i(h)\) for element \(i\) is
\begin{equation*}
\begin{split}
\int_{-h}^hf(x)dx - R_i(h)=-\frac{h^3}{24}f^{(2)}(\xi),
\end{split}
\end{equation*}
and the global error reads
\begin{equation*}
\begin{split}
\int_a^bf(x)dx -R_h(f)=-\frac{b-a}{24}h^2f^{(2)}(\xi),
\end{split}
\end{equation*}
where \(R_h\) is the result obtained with rectangular rule and \(\xi \in [a,b]\).

We go back to our simple example above and set \(F_0=b=1\) and choose \(x_0=0\) and \(x=1/2\), and have
\begin{equation*}
\begin{split}
W=\frac{1}{\pi}.
\end{split}
\end{equation*}
The code here computes the integral using the rectangle rule and \(n=100\) integration points we have a relative error of
\(10^{-5}\).

\begin{sphinxVerbatim}[commandchars=\\\{\}]
\PYG{k+kn}{from} \PYG{n+nn}{math} \PYG{k+kn}{import} \PYG{n}{sin}\PYG{p}{,} \PYG{n}{pi}
\PYG{k+kn}{import} \PYG{n+nn}{numpy} \PYG{k}{as} \PYG{n+nn}{np}
\PYG{k+kn}{from} \PYG{n+nn}{sympy} \PYG{k+kn}{import} \PYG{n}{Symbol}\PYG{p}{,} \PYG{n}{integrate}
\PYG{c+c1}{\PYGZsh{} function for the Rectangular rule                                                                                        }
\PYG{k}{def} \PYG{n+nf}{Rectangular}\PYG{p}{(}\PYG{n}{a}\PYG{p}{,}\PYG{n}{b}\PYG{p}{,}\PYG{n}{f}\PYG{p}{,}\PYG{n}{n}\PYG{p}{)}\PYG{p}{:}
   \PYG{n}{h} \PYG{o}{=} \PYG{p}{(}\PYG{n}{b}\PYG{o}{\PYGZhy{}}\PYG{n}{a}\PYG{p}{)}\PYG{o}{/}\PYG{n+nb}{float}\PYG{p}{(}\PYG{n}{n}\PYG{p}{)}
   \PYG{n}{s} \PYG{o}{=} \PYG{l+m+mi}{0}
   \PYG{k}{for} \PYG{n}{i} \PYG{o+ow}{in} \PYG{n+nb}{range}\PYG{p}{(}\PYG{l+m+mi}{0}\PYG{p}{,}\PYG{n}{n}\PYG{p}{,}\PYG{l+m+mi}{1}\PYG{p}{)}\PYG{p}{:}
       \PYG{n}{x} \PYG{o}{=} \PYG{p}{(}\PYG{n}{i}\PYG{o}{+}\PYG{l+m+mf}{0.5}\PYG{p}{)}\PYG{o}{*}\PYG{n}{h}
       \PYG{n}{s} \PYG{o}{=} \PYG{n}{s}\PYG{o}{+} \PYG{n}{f}\PYG{p}{(}\PYG{n}{x}\PYG{p}{)}
   \PYG{k}{return} \PYG{n}{h}\PYG{o}{*}\PYG{n}{s}
\PYG{c+c1}{\PYGZsh{} function to integrate}
\PYG{k}{def} \PYG{n+nf}{function}\PYG{p}{(}\PYG{n}{x}\PYG{p}{)}\PYG{p}{:}
    \PYG{k}{return} \PYG{n}{sin}\PYG{p}{(}\PYG{l+m+mi}{2}\PYG{o}{*}\PYG{n}{pi}\PYG{o}{*}\PYG{n}{x}\PYG{p}{)}
\PYG{c+c1}{\PYGZsh{} define integration limits and integration points                                                                         }
\PYG{n}{a} \PYG{o}{=} \PYG{l+m+mf}{0.0}\PYG{p}{;} \PYG{n}{b} \PYG{o}{=} \PYG{l+m+mf}{0.5}\PYG{p}{;}
\PYG{n}{n} \PYG{o}{=} \PYG{l+m+mi}{100}
\PYG{n}{Exact} \PYG{o}{=} \PYG{l+m+mf}{1.}\PYG{o}{/}\PYG{n}{pi}
\PYG{n+nb}{print}\PYG{p}{(}\PYG{l+s+s2}{\PYGZdq{}}\PYG{l+s+s2}{Relative error= }\PYG{l+s+s2}{\PYGZdq{}}\PYG{p}{,} \PYG{n+nb}{abs}\PYG{p}{(} \PYG{p}{(}\PYG{n}{Rectangular}\PYG{p}{(}\PYG{n}{a}\PYG{p}{,}\PYG{n}{b}\PYG{p}{,}\PYG{n}{function}\PYG{p}{,}\PYG{n}{n}\PYG{p}{)}\PYG{o}{\PYGZhy{}}\PYG{n}{Exact}\PYG{p}{)}\PYG{o}{/}\PYG{n}{Exact}\PYG{p}{)}\PYG{p}{)}
\end{sphinxVerbatim}


\section{The trapezoidal rule}
\label{\detokenize{chapter1:the-trapezoidal-rule}}
The other integral gives
\begin{equation*}
\begin{split}
\int_{x_0-h}^{x_0}f(x)dx=\frac{h}{2}\left(f(x_0) + f(x_0-h)\right)+O(h^3),
\end{split}
\end{equation*}
and adding up we obtain




\begin{equation*}
\begin{split}
\begin{equation}
   \int_{x_0-h}^{x_0+h}f(x)dx=\frac{h}{2}\left(f(x_0+h) + 2f(x_0) + f(x_0-h)\right)+O(h^3),
\label{eq:trapez} \tag{44}
\end{equation}
\end{split}
\end{equation*}
which is the well\sphinxhyphen{}known trapezoidal rule.  Concerning the error in the approximation made,
\(O(h^3)=O((b-a)^3/N^3)\), you should  note
that this is the local error.  Since we are splitting the integral from
\(a\) to \(b\) in \(N\) pieces, we will have to perform approximately \(N\)
such operations.

This means that the \sphinxstyleemphasis{global error} goes like \(\approx O(h^2)\).
The trapezoidal reads then




\begin{equation*}
\begin{split}
\begin{equation}
   I=\int_a^bf(x) dx=h\left(f(a)/2 + f(a+h) +f(a+2h)+
                          \dots +f(b-h)+ f_{b}/2\right),
\label{eq:trapez1} \tag{45}
\end{equation}
\end{split}
\end{equation*}
with a global error which goes like \(O(h^2)\).

Hereafter we use the shorthand notations \(f_{-h}=f(x_0-h)\), \(f_{0}=f(x_0)\)
and \(f_{h}=f(x_0+h)\).

The correct mathematical expression for the local error for the trapezoidal rule is
\begin{equation*}
\begin{split}
\int_a^bf(x)dx -\frac{b-a}{2}\left[f(a)+f(b)\right]=-\frac{h^3}{12}f^{(2)}(\xi),
\end{split}
\end{equation*}
and the global error reads
\begin{equation*}
\begin{split}
\int_a^bf(x)dx -T_h(f)=-\frac{b-a}{12}h^2f^{(2)}(\xi),
\end{split}
\end{equation*}
where \(T_h\) is the trapezoidal result and \(\xi \in [a,b]\).


\section{Algorithm for the trapezoidal rule}
\label{\detokenize{chapter1:algorithm-for-the-trapezoidal-rule}}
The trapezoidal rule is easy to  implement numerically
through the following simple algorithm
\begin{itemize}
\item {} 
Choose the number of mesh points and fix the step length.

\item {} 
calculate \(f(a)\) and \(f(b)\) and multiply with \(h/2\).

\item {} 
Perform a loop over \(n=1\) to \(n-1\) (\(f(a)\) and \(f(b)\) are known) and sum up  the terms \(f(a+h) +f(a+2h)+f(a+3h)+\dots +f(b-h)\). Each step in the loop  corresponds to a given value \(a+nh\).

\item {} 
Multiply the final result by \(h\) and add \(hf(a)/2\) and \(hf(b)/2\).

\end{itemize}

We use the same function and integrate now using the trapoezoidal rule.

\begin{sphinxVerbatim}[commandchars=\\\{\}]
\PYG{k+kn}{import} \PYG{n+nn}{numpy} \PYG{k}{as} \PYG{n+nn}{np}
\PYG{k+kn}{from} \PYG{n+nn}{sympy} \PYG{k+kn}{import} \PYG{n}{Symbol}\PYG{p}{,} \PYG{n}{integrate}
\PYG{c+c1}{\PYGZsh{} function for the trapezoidal rule}
\PYG{k}{def} \PYG{n+nf}{Trapez}\PYG{p}{(}\PYG{n}{a}\PYG{p}{,}\PYG{n}{b}\PYG{p}{,}\PYG{n}{f}\PYG{p}{,}\PYG{n}{n}\PYG{p}{)}\PYG{p}{:}
   \PYG{n}{h} \PYG{o}{=} \PYG{p}{(}\PYG{n}{b}\PYG{o}{\PYGZhy{}}\PYG{n}{a}\PYG{p}{)}\PYG{o}{/}\PYG{n+nb}{float}\PYG{p}{(}\PYG{n}{n}\PYG{p}{)}
   \PYG{n}{s} \PYG{o}{=} \PYG{l+m+mi}{0}
   \PYG{n}{x} \PYG{o}{=} \PYG{n}{a}
   \PYG{k}{for} \PYG{n}{i} \PYG{o+ow}{in} \PYG{n+nb}{range}\PYG{p}{(}\PYG{l+m+mi}{1}\PYG{p}{,}\PYG{n}{n}\PYG{p}{,}\PYG{l+m+mi}{1}\PYG{p}{)}\PYG{p}{:}
       \PYG{n}{x} \PYG{o}{=} \PYG{n}{x}\PYG{o}{+}\PYG{n}{h}
       \PYG{n}{s} \PYG{o}{=} \PYG{n}{s}\PYG{o}{+} \PYG{n}{f}\PYG{p}{(}\PYG{n}{x}\PYG{p}{)}
   \PYG{n}{s} \PYG{o}{=} \PYG{l+m+mf}{0.5}\PYG{o}{*}\PYG{p}{(}\PYG{n}{f}\PYG{p}{(}\PYG{n}{a}\PYG{p}{)}\PYG{o}{+}\PYG{n}{f}\PYG{p}{(}\PYG{n}{b}\PYG{p}{)}\PYG{p}{)} \PYG{o}{+}\PYG{n}{s}
   \PYG{k}{return} \PYG{n}{h}\PYG{o}{*}\PYG{n}{s}
\PYG{c+c1}{\PYGZsh{} function to integrate}
\PYG{k}{def} \PYG{n+nf}{function}\PYG{p}{(}\PYG{n}{x}\PYG{p}{)}\PYG{p}{:}
    \PYG{k}{return} \PYG{n}{sin}\PYG{p}{(}\PYG{l+m+mi}{2}\PYG{o}{*}\PYG{n}{pi}\PYG{o}{*}\PYG{n}{x}\PYG{p}{)}
\PYG{c+c1}{\PYGZsh{} define integration limits and integration points                                                                         }
\PYG{n}{a} \PYG{o}{=} \PYG{l+m+mf}{0.0}\PYG{p}{;} \PYG{n}{b} \PYG{o}{=} \PYG{l+m+mf}{0.5}\PYG{p}{;}
\PYG{n}{n} \PYG{o}{=} \PYG{l+m+mi}{100}
\PYG{n}{Exact} \PYG{o}{=} \PYG{l+m+mf}{1.}\PYG{o}{/}\PYG{n}{pi}
\PYG{n+nb}{print}\PYG{p}{(}\PYG{l+s+s2}{\PYGZdq{}}\PYG{l+s+s2}{Relative error= }\PYG{l+s+s2}{\PYGZdq{}}\PYG{p}{,} \PYG{n+nb}{abs}\PYG{p}{(} \PYG{p}{(}\PYG{n}{Trapez}\PYG{p}{(}\PYG{n}{a}\PYG{p}{,}\PYG{n}{b}\PYG{p}{,}\PYG{n}{function}\PYG{p}{,}\PYG{n}{n}\PYG{p}{)}\PYG{o}{\PYGZhy{}}\PYG{n}{Exact}\PYG{p}{)}\PYG{o}{/}\PYG{n}{Exact}\PYG{p}{)}\PYG{p}{)}
\end{sphinxVerbatim}


\section{Simpsons’ rule}
\label{\detokenize{chapter1:simpsons-rule}}
Instead of using the above first\sphinxhyphen{}order polynomials
approximations for \(f\), we attempt at using a second\sphinxhyphen{}order polynomials.
In this case we need three points in order to define a second\sphinxhyphen{}order
polynomial approximation
\begin{equation*}
\begin{split}
f(x) \approx P_2(x)=a_0+a_1x+a_2x^2.
\end{split}
\end{equation*}
Using again Lagrange’s interpolation formula we have
\begin{equation*}
\begin{split}
P_2(x)=\frac{(x-x_0)(x-x_1)}{(x_2-x_0)(x_2-x_1)}y_2+
            \frac{(x-x_0)(x-x_2)}{(x_1-x_0)(x_1-x_2)}y_1+
            \frac{(x-x_1)(x-x_2)}{(x_0-x_1)(x_0-x_2)}y_0.
\end{split}
\end{equation*}
Inserting this formula in the integral of Eq.  ({\hyperref[\detokenize{chapter1:eq:hhint}]{\emph{42}}}) we obtain
\begin{equation*}
\begin{split}
\int_{-h}^{+h}f(x)dx=\frac{h}{3}\left(f_h + 4f_0 + f_{-h}\right)+O(h^5),
\end{split}
\end{equation*}
which is Simpson’s rule.

Note that the improved accuracy in the evaluation of
the derivatives gives a better error approximation, \(O(h^5)\) vs.\textbackslash{} \(O(h^3)\) .
But this is again the \sphinxstyleemphasis{local error approximation}.
Using Simpson’s rule we can easily compute
the integral     of Eq.  ({\hyperref[\detokenize{chapter1:eq:integraldef}]{\emph{41}}}) to be




\begin{equation*}
\begin{split}
\begin{equation}
   I=\int_a^bf(x) dx=\frac{h}{3}\left(f(a) + 4f(a+h) +2f(a+2h)+
                          \dots +4f(b-h)+ f_{b}\right),
\label{eq:simpson} \tag{46}
\end{equation}
\end{split}
\end{equation*}
with a global error which goes like \(O(h^4)\).

More formal expressions for the local and global errors are for the local error
\begin{equation*}
\begin{split}
\int_a^bf(x)dx -\frac{b-a}{6}\left[f(a)+4f((a+b)/2)+f(b)\right]=-\frac{h^5}{90}f^{(4)}(\xi),
\end{split}
\end{equation*}
and for the global error
\begin{equation*}
\begin{split}
\int_a^bf(x)dx -S_h(f)=-\frac{b-a}{180}h^4f^{(4)}(\xi).
\end{split}
\end{equation*}
with \(\xi\in[a,b]\) and \(S_h\) the results obtained with Simpson’s method.


\section{Algorithm for Simpson’s rule}
\label{\detokenize{chapter1:algorithm-for-simpson-s-rule}}
The method
can easily be implemented numerically through the following simple algorithm
\begin{itemize}
\item {} 
Choose the number of mesh points and fix the step.

\item {} 
calculate \(f(a)\) and \(f(b)\)

\item {} 
Perform a loop over \(n=1\) to \(n-1\) (\(f(a)\) and \(f(b)\) are known) and sum up   the terms \(4f(a+h) +2f(a+2h)+4f(a+3h)+\dots +4f(b-h)\). Each step in the loop  corresponds to a given value \(a+nh\). Odd values of \(n\) give \(4\) as factor  while even values yield \(2\) as factor.

\item {} 
Multiply the final result by \(\frac{h}{3}\).

\end{itemize}


\section{Code example}
\label{\detokenize{chapter1:code-example}}
\begin{sphinxVerbatim}[commandchars=\\\{\}]
\PYG{k+kn}{from} \PYG{n+nn}{math} \PYG{k+kn}{import} \PYG{n}{sin}\PYG{p}{,} \PYG{n}{pi}
\PYG{k+kn}{import} \PYG{n+nn}{numpy} \PYG{k}{as} \PYG{n+nn}{np}
\PYG{k+kn}{from} \PYG{n+nn}{sympy} \PYG{k+kn}{import} \PYG{n}{Symbol}\PYG{p}{,} \PYG{n}{integrate}
\PYG{c+c1}{\PYGZsh{} function for the trapezoidal rule                                                                                        }
\PYG{k}{def} \PYG{n+nf}{Simpson}\PYG{p}{(}\PYG{n}{a}\PYG{p}{,}\PYG{n}{b}\PYG{p}{,}\PYG{n}{f}\PYG{p}{,}\PYG{n}{n}\PYG{p}{)}\PYG{p}{:}
   \PYG{n}{h} \PYG{o}{=} \PYG{p}{(}\PYG{n}{b}\PYG{o}{\PYGZhy{}}\PYG{n}{a}\PYG{p}{)}\PYG{o}{/}\PYG{n+nb}{float}\PYG{p}{(}\PYG{n}{n}\PYG{p}{)}
   \PYG{n+nb}{sum} \PYG{o}{=} \PYG{n}{f}\PYG{p}{(}\PYG{n}{a}\PYG{p}{)}\PYG{o}{/}\PYG{n+nb}{float}\PYG{p}{(}\PYG{l+m+mi}{2}\PYG{p}{)}\PYG{p}{;}
   \PYG{k}{for} \PYG{n}{i} \PYG{o+ow}{in} \PYG{n+nb}{range}\PYG{p}{(}\PYG{l+m+mi}{1}\PYG{p}{,}\PYG{n}{n}\PYG{p}{)}\PYG{p}{:}
       \PYG{n+nb}{sum} \PYG{o}{=} \PYG{n+nb}{sum} \PYG{o}{+} \PYG{n}{f}\PYG{p}{(}\PYG{n}{a}\PYG{o}{+}\PYG{n}{i}\PYG{o}{*}\PYG{n}{h}\PYG{p}{)}\PYG{o}{*}\PYG{p}{(}\PYG{l+m+mi}{3}\PYG{o}{+}\PYG{p}{(}\PYG{o}{\PYGZhy{}}\PYG{l+m+mi}{1}\PYG{p}{)}\PYG{o}{*}\PYG{o}{*}\PYG{p}{(}\PYG{n}{i}\PYG{o}{+}\PYG{l+m+mi}{1}\PYG{p}{)}\PYG{p}{)}
   \PYG{n+nb}{sum} \PYG{o}{=} \PYG{n+nb}{sum} \PYG{o}{+} \PYG{n}{f}\PYG{p}{(}\PYG{n}{b}\PYG{p}{)}\PYG{o}{/}\PYG{n+nb}{float}\PYG{p}{(}\PYG{l+m+mi}{2}\PYG{p}{)}
   \PYG{k}{return} \PYG{n+nb}{sum}\PYG{o}{*}\PYG{n}{h}\PYG{o}{/}\PYG{l+m+mf}{3.0}
\PYG{c+c1}{\PYGZsh{} function to integrate                                                                                                    }
\PYG{k}{def} \PYG{n+nf}{function}\PYG{p}{(}\PYG{n}{x}\PYG{p}{)}\PYG{p}{:}
    \PYG{k}{return} \PYG{n}{sin}\PYG{p}{(}\PYG{l+m+mi}{2}\PYG{o}{*}\PYG{n}{pi}\PYG{o}{*}\PYG{n}{x}\PYG{p}{)}
\PYG{c+c1}{\PYGZsh{} define integration limits and integration points                                                                         }
\PYG{n}{a} \PYG{o}{=} \PYG{l+m+mf}{0.0}\PYG{p}{;} \PYG{n}{b} \PYG{o}{=} \PYG{l+m+mf}{0.5}\PYG{p}{;}
\PYG{n}{n} \PYG{o}{=} \PYG{l+m+mi}{100}
\PYG{n}{Exact} \PYG{o}{=} \PYG{l+m+mf}{1.}\PYG{o}{/}\PYG{n}{pi}
\PYG{n+nb}{print}\PYG{p}{(}\PYG{l+s+s2}{\PYGZdq{}}\PYG{l+s+s2}{Relative error= }\PYG{l+s+s2}{\PYGZdq{}}\PYG{p}{,} \PYG{n+nb}{abs}\PYG{p}{(} \PYG{p}{(}\PYG{n}{Simpson}\PYG{p}{(}\PYG{n}{a}\PYG{p}{,}\PYG{n}{b}\PYG{p}{,}\PYG{n}{function}\PYG{p}{,}\PYG{n}{n}\PYG{p}{)}\PYG{o}{\PYGZhy{}}\PYG{n}{Exact}\PYG{p}{)}\PYG{o}{/}\PYG{n}{Exact}\PYG{p}{)}\PYG{p}{)}
\end{sphinxVerbatim}

We see that Simpson’s rule gives a much better estimation of the relative error with the same amount of points as we had for the Rectangle rule and the Trapezoidal rule.


\chapter{Harmonic Oscillations}
\label{\detokenize{chapter1:harmonic-oscillations}}
The harmonic oscillator is omnipresent in physics. Although you may think
of this as being related to springs, it, or an equivalent
mathematical representation, appears in just about any problem where a
mode is sitting near its potential energy minimum. At that point,
\(\partial_x V(x)=0\), and the first non\sphinxhyphen{}zero term (aside from a
constant) in the potential energy is that of a harmonic oscillator. In
a solid, sound modes (phonons) are built on a picture of coupled
harmonic oscillators, and in relativistic field theory the fundamental
interactions are also built on coupled oscillators positioned
infinitesimally close to one another in space. The phenomena of a
resonance of an oscillator driven at a fixed frequency plays out
repeatedly in atomic, nuclear and high\sphinxhyphen{}energy physics, when quantum
mechanically the evolution of a state oscillates according to
\(e^{-iEt}\) and exciting discrete quantum states has very similar
mathematics as exciting discrete states of an oscillator.

The potential energy for a single particle as a function of its position \(x\) can be written as a Taylor expansion about some point \(x_0\)




\begin{equation*}
\begin{split}
\begin{equation}
V(x)=V(x_0)+(x-x_0)\left.\partial_xV(x)\right|_{x_0}+\frac{1}{2}(x-x_0)^2\left.\partial_x^2V(x)\right|_{x_0}
+\frac{1}{3!}\left.\partial_x^3V(x)\right|_{x_0}+\cdots
\label{_auto37} \tag{47}
\end{equation}
\end{split}
\end{equation*}
If the position \(x_0\) is at the minimum of the resonance, the first two non\sphinxhyphen{}zero terms of the potential are
\begin{equation*}
\begin{split}
\begin{eqnarray}
V(x)&\approx& V(x_0)+\frac{1}{2}(x-x_0)^2\left.\partial_x^2V(x)\right|_{x_0},\\
\nonumber
&=&V(x_0)+\frac{1}{2}k(x-x_0)^2,~~~~k\equiv \left.\partial_x^2V(x)\right|_{x_0},\\
\nonumber
F&=&-\partial_xV(x)=-k(x-x_0).
\end{eqnarray}
\end{split}
\end{equation*}
Put into Newton’s 2nd law (assuming \(x_0=0\)),
\begin{equation*}
\begin{split}
\begin{eqnarray}
m\ddot{x}&=&-kx,\\
x&=&A\cos(\omega_0 t-\phi),~~~\omega_0=\sqrt{k/m}.
\end{eqnarray}
\end{split}
\end{equation*}
Here \(A\) and \(\phi\) are arbitrary. Equivalently, one could have
written this as \(A\cos(\omega_0 t)+B\sin(\omega_0 t)\), or as the real
part of \(Ae^{i\omega_0 t}\). In this last case \(A\) could be an
arbitrary complex constant. Thus, there are 2 arbitrary constants
(either \(A\) and \(B\) or \(A\) and \(\phi\), or the real and imaginary part
of one complex constant. This is the expectation for a second order
differential equation, and also agrees with the physical expectation
that if you know a particle’s initial velocity and position you should
be able to define its future motion, and that those two arbitrary
conditions should translate to two arbitrary constants.

A key feature of harmonic motion is that the system repeats itself
after a time \(T=1/f\), where \(f\) is the frequency, and \(\omega=2\pi f\)
is the angular frequency. The period of the motion is independent of
the amplitude. However, this independence is only exact when one can
neglect higher terms of the potential, \(x^3, x^4\cdots\). Once can
neglect these terms for sufficiently small amplitudes, and for larger
amplitudes the motion is no longer purely sinusoidal, and even though
the motion repeats itself, the time for repeating the motion is no
longer independent of the amplitude.

One can also calculate the velocity and the kinetic energy as a function of time,
\begin{equation*}
\begin{split}
\begin{eqnarray}
\dot{x}&=&-\omega_0A\sin(\omega_0 t-\phi),\\
\nonumber
K&=&\frac{1}{2}m\dot{x}^2=\frac{m\omega_0^2A^2}{2}\sin^2(\omega_0t-\phi),\\
\nonumber
&=&\frac{k}{2}A^2\sin^2(\omega_0t-\phi).
\end{eqnarray}
\end{split}
\end{equation*}
The total energy is then




\begin{equation*}
\begin{split}
\begin{equation}
E=K+V=\frac{1}{2}m\dot{x}^2+\frac{1}{2}kx^2=\frac{1}{2}kA^2.
\label{_auto38} \tag{48}
\end{equation}
\end{split}
\end{equation*}
The total energy then goes as the square of the amplitude.

A pendulum is an example of a harmonic oscillator. By expanding the
kinetic and potential energies for small angles find the frequency for
a pendulum of length \(L\) with all the mass \(m\) centered at the end by
writing the eq.s of motion in the form of a harmonic oscillator.

The potential energy and kinetic energies are (for \(x\) being the displacement)
\begin{equation*}
\begin{split}
\begin{eqnarray*}
V&=&mgL(1-\cos\theta)\approx mgL\frac{x^2}{2L^2},\\
K&=&\frac{1}{2}mL^2\dot{\theta}^2\approx \frac{m}{2}\dot{x}^2.
\end{eqnarray*}
\end{split}
\end{equation*}
For small \(x\) Newton’s 2nd law becomes
\begin{equation*}
\begin{split}
m\ddot{x}=-\frac{mg}{L}x,
\end{split}
\end{equation*}
and the spring constant would appear to be \(k=mg/L\), which makes the
frequency equal to \(\omega_0=\sqrt{g/L}\). Note that the frequency is
independent of the mass.


\section{Damped Oscillators}
\label{\detokenize{chapter1:damped-oscillators}}
We consider only the case where the damping force is proportional to
the velocity. This is counter to dragging friction, where the force is
proportional in strength to the normal force and independent of
velocity, and is also inconsistent with wind resistance, where the
magnitude of the drag force is proportional the square of the
velocity. Rolling resistance does seem to be mainly proportional to
the velocity. However, the main motivation for considering damping
forces proportional to the velocity is that the math is more
friendly. This is because the differential equation is linear,
i.e. each term is of order \(x\), \(\dot{x}\), \(\ddot{x}\cdots\), or even
terms with no mention of \(x\), and there are no terms such as \(x^2\) or
\(x\ddot{x}\). The equations of motion for a spring with damping force
\(-b\dot{x}\) are




\begin{equation*}
\begin{split}
\begin{equation}
m\ddot{x}+b\dot{x}+kx=0.
\label{_auto39} \tag{49}
\end{equation}
\end{split}
\end{equation*}
Just to make the solution a bit less messy, we rewrite this equation as




\begin{equation*}
\begin{split}
\begin{equation}
\label{eq:dampeddiffyq} \tag{50}
\ddot{x}+2\beta\dot{x}+\omega_0^2x=0,~~~~\beta\equiv b/2m,~\omega_0\equiv\sqrt{k/m}.
\end{equation}
\end{split}
\end{equation*}
Both \(\beta\) and \(\omega\) have dimensions of inverse time. To find solutions (see appendix C in the text) you must make an educated guess at the form of the solution. To do this, first realize that the solution will need an arbitrary normalization \(A\) because the equation is linear. Secondly, realize that if the form is




\begin{equation*}
\begin{split}
\begin{equation}
x=Ae^{rt}
\label{_auto40} \tag{51}
\end{equation}
\end{split}
\end{equation*}
that each derivative simply brings out an extra power of \(r\). This
means that the \(Ae^{rt}\) factors out and one can simply solve for an
equation for \(r\). Plugging this form into Eq. ({\hyperref[\detokenize{chapter1:eq:dampeddiffyq}]{\emph{50}}}),




\begin{equation*}
\begin{split}
\begin{equation}
r^2+2\beta r+\omega_0^2=0.
\label{_auto41} \tag{52}
\end{equation}
\end{split}
\end{equation*}
Because this is a quadratic equation there will be two solutions,




\begin{equation*}
\begin{split}
\begin{equation}
r=-\beta\pm\sqrt{\beta^2-\omega_0^2}.
\label{_auto42} \tag{53}
\end{equation}
\end{split}
\end{equation*}
We refer to the two solutions as \(r_1\) and \(r_2\) corresponding to the
\(+\) and \(-\) roots. As expected, there should be two arbitrary
constants involved in the solution,




\begin{equation*}
\begin{split}
\begin{equation}
x=A_1e^{r_1t}+A_2e^{r_2t},
\label{_auto43} \tag{54}
\end{equation}
\end{split}
\end{equation*}
where the coefficients \(A_1\) and \(A_2\) are determined by initial
conditions.

The roots listed above, \(\sqrt{\omega_0^2-\beta_0^2}\), will be
imaginary if the damping is small and \(\beta<\omega_0\). In that case,
\(r\) is complex and the factor \(e{rt}\) will have some oscillatory
behavior. If the roots are real, there will only be exponentially
decaying solutions. There are three cases:


\subsection{Underdamped: \protect\(\beta<\omega_0\protect\)}
\label{\detokenize{chapter1:underdamped-beta-omega-0}}\begin{equation*}
\begin{split}
\begin{eqnarray}
x&=&A_1e^{-\beta t}e^{i\omega't}+A_2e^{-\beta t}e^{-i\omega't},~~\omega'\equiv\sqrt{\omega_0^2-\beta^2}\\
\nonumber
&=&(A_1+A_2)e^{-\beta t}\cos\omega't+i(A_1-A_2)e^{-\beta t}\sin\omega't.
\end{eqnarray}
\end{split}
\end{equation*}
Here we have made use of the identity
\(e^{i\omega't}=\cos\omega't+i\sin\omega't\). Because the constants are
arbitrary, and because the real and imaginary parts are both solutions
individually, we can simply consider the real part of the solution
alone:




\begin{equation*}
\begin{split}
\begin{eqnarray}
\label{eq:homogsolution} \tag{55}
x&=&B_1e^{-\beta t}\cos\omega't+B_2e^{-\beta t}\sin\omega't,\\
\nonumber 
\omega'&\equiv&\sqrt{\omega_0^2-\beta^2}.
\end{eqnarray}
\end{split}
\end{equation*}

\subsection{Critical dampling: \protect\(\beta=\omega_0\protect\)}
\label{\detokenize{chapter1:critical-dampling-beta-omega-0}}
In this case the two terms involving \(r_1\) and \(r_2\) are identical
because \(\omega'=0\). Because we need to arbitrary constants, there
needs to be another solution. This is found by simply guessing, or by
taking the limit of \(\omega'\rightarrow 0\) from the underdamped
solution. The solution is then




\begin{equation*}
\begin{split}
\begin{equation}
\label{eq:criticallydamped} \tag{56}
x=Ae^{-\beta t}+Bte^{-\beta t}.
\end{equation}
\end{split}
\end{equation*}
The critically damped solution is interesting because the solution
approaches zero quickly, but does not oscillate. For a problem with
zero initial velocity, the solution never crosses zero. This is a good
choice for designing shock absorbers or swinging doors.


\subsection{Overdamped: \protect\(\beta>\omega_0\protect\)}
\label{\detokenize{chapter1:overdamped-beta-omega-0}}\begin{equation*}
\begin{split}
\begin{eqnarray}
x&=&A_1\exp{-(\beta+\sqrt{\beta^2-\omega_0^2})t}+A_2\exp{-(\beta-\sqrt{\beta^2-\omega_0^2})t}
\end{eqnarray}
\end{split}
\end{equation*}
This solution will also never pass the origin more than once, and then
only if the initial velocity is strong and initially toward zero.

Given \(b\), \(m\) and \(\omega_0\), find \(x(t)\) for a particle whose
initial position is \(x=0\) and has initial velocity \(v_0\) (assuming an
underdamped solution).

The solution is of the form,
\begin{equation*}
\begin{split}
\begin{eqnarray*}
x&=&e^{-\beta t}\left[A_1\cos(\omega' t)+A_2\sin\omega't\right],\\
\dot{x}&=&-\beta x+\omega'e^{-\beta t}\left[-A_1\sin\omega't+A_2\cos\omega't\right].\\
\omega'&\equiv&\sqrt{\omega_0^2-\beta^2},~~~\beta\equiv b/2m.
\end{eqnarray*}
\end{split}
\end{equation*}
From the initial conditions, \(A_1=0\) because \(x(0)=0\) and \(\omega'A_2=v_0\). So
\begin{equation*}
\begin{split}
x=\frac{v_0}{\omega'}e^{-\beta t}\sin\omega't.
\end{split}
\end{equation*}
Here we study first the case without additional friction term and scale our equation
in terms of a dimensionless time \(\tau\).

Let us remind ourselves about the differential equation we want to solve (the general case with damping due to friction)
\begin{equation*}
\begin{split}
m\frac{d^2x}{dt^2} + b\frac{dx}{dt}+kx(t) =0.
\end{split}
\end{equation*}
We divide by \(m\) and introduce \(\omega_0^2=\sqrt{k/m}\) and obtain
\begin{equation*}
\begin{split}
\frac{d^2x}{dt^2} + \frac{b}{m}\frac{dx}{dt}+\omega_0^2x(t) =0.
\end{split}
\end{equation*}
Thereafter we introduce a dimensionless time \(\tau = t\omega_0\) (check
that the dimensionality is correct) and rewrite our equation as
\begin{equation*}
\begin{split}
\frac{d^2x}{d\tau^2} + \frac{b}{m\omega_0}\frac{dx}{d\tau}+x(\tau) =0,
\end{split}
\end{equation*}
which gives us
\begin{equation*}
\begin{split}
\frac{d^2x}{d\tau^2} + \frac{b}{m\omega_0}\frac{dx}{d\tau}+x(\tau) =0.
\end{split}
\end{equation*}
We then define \(\gamma = b/(2m\omega_0)\) and rewrite our equations as
\begin{equation*}
\begin{split}
\frac{d^2x}{d\tau^2} + 2\gamma\frac{dx}{d\tau}+x(\tau) =0.
\end{split}
\end{equation*}
This is the equation we will code below. The first version employs the Euler\sphinxhyphen{}Cromer method.

\begin{sphinxVerbatim}[commandchars=\\\{\}]
\PYG{c+c1}{\PYGZsh{} Common imports}
\PYG{k+kn}{import} \PYG{n+nn}{numpy} \PYG{k}{as} \PYG{n+nn}{np}
\PYG{k+kn}{import} \PYG{n+nn}{pandas} \PYG{k}{as} \PYG{n+nn}{pd}
\PYG{k+kn}{from} \PYG{n+nn}{math} \PYG{k+kn}{import} \PYG{o}{*}
\PYG{k+kn}{import} \PYG{n+nn}{matplotlib}\PYG{n+nn}{.}\PYG{n+nn}{pyplot} \PYG{k}{as} \PYG{n+nn}{plt}
\PYG{k+kn}{import} \PYG{n+nn}{os}

\PYG{c+c1}{\PYGZsh{} Where to save the figures and data files}
\PYG{n}{PROJECT\PYGZus{}ROOT\PYGZus{}DIR} \PYG{o}{=} \PYG{l+s+s2}{\PYGZdq{}}\PYG{l+s+s2}{Results}\PYG{l+s+s2}{\PYGZdq{}}
\PYG{n}{FIGURE\PYGZus{}ID} \PYG{o}{=} \PYG{l+s+s2}{\PYGZdq{}}\PYG{l+s+s2}{Results/FigureFiles}\PYG{l+s+s2}{\PYGZdq{}}
\PYG{n}{DATA\PYGZus{}ID} \PYG{o}{=} \PYG{l+s+s2}{\PYGZdq{}}\PYG{l+s+s2}{DataFiles/}\PYG{l+s+s2}{\PYGZdq{}}

\PYG{k}{if} \PYG{o+ow}{not} \PYG{n}{os}\PYG{o}{.}\PYG{n}{path}\PYG{o}{.}\PYG{n}{exists}\PYG{p}{(}\PYG{n}{PROJECT\PYGZus{}ROOT\PYGZus{}DIR}\PYG{p}{)}\PYG{p}{:}
    \PYG{n}{os}\PYG{o}{.}\PYG{n}{mkdir}\PYG{p}{(}\PYG{n}{PROJECT\PYGZus{}ROOT\PYGZus{}DIR}\PYG{p}{)}

\PYG{k}{if} \PYG{o+ow}{not} \PYG{n}{os}\PYG{o}{.}\PYG{n}{path}\PYG{o}{.}\PYG{n}{exists}\PYG{p}{(}\PYG{n}{FIGURE\PYGZus{}ID}\PYG{p}{)}\PYG{p}{:}
    \PYG{n}{os}\PYG{o}{.}\PYG{n}{makedirs}\PYG{p}{(}\PYG{n}{FIGURE\PYGZus{}ID}\PYG{p}{)}

\PYG{k}{if} \PYG{o+ow}{not} \PYG{n}{os}\PYG{o}{.}\PYG{n}{path}\PYG{o}{.}\PYG{n}{exists}\PYG{p}{(}\PYG{n}{DATA\PYGZus{}ID}\PYG{p}{)}\PYG{p}{:}
    \PYG{n}{os}\PYG{o}{.}\PYG{n}{makedirs}\PYG{p}{(}\PYG{n}{DATA\PYGZus{}ID}\PYG{p}{)}

\PYG{k}{def} \PYG{n+nf}{image\PYGZus{}path}\PYG{p}{(}\PYG{n}{fig\PYGZus{}id}\PYG{p}{)}\PYG{p}{:}
    \PYG{k}{return} \PYG{n}{os}\PYG{o}{.}\PYG{n}{path}\PYG{o}{.}\PYG{n}{join}\PYG{p}{(}\PYG{n}{FIGURE\PYGZus{}ID}\PYG{p}{,} \PYG{n}{fig\PYGZus{}id}\PYG{p}{)}

\PYG{k}{def} \PYG{n+nf}{data\PYGZus{}path}\PYG{p}{(}\PYG{n}{dat\PYGZus{}id}\PYG{p}{)}\PYG{p}{:}
    \PYG{k}{return} \PYG{n}{os}\PYG{o}{.}\PYG{n}{path}\PYG{o}{.}\PYG{n}{join}\PYG{p}{(}\PYG{n}{DATA\PYGZus{}ID}\PYG{p}{,} \PYG{n}{dat\PYGZus{}id}\PYG{p}{)}

\PYG{k}{def} \PYG{n+nf}{save\PYGZus{}fig}\PYG{p}{(}\PYG{n}{fig\PYGZus{}id}\PYG{p}{)}\PYG{p}{:}
    \PYG{n}{plt}\PYG{o}{.}\PYG{n}{savefig}\PYG{p}{(}\PYG{n}{image\PYGZus{}path}\PYG{p}{(}\PYG{n}{fig\PYGZus{}id}\PYG{p}{)} \PYG{o}{+} \PYG{l+s+s2}{\PYGZdq{}}\PYG{l+s+s2}{.png}\PYG{l+s+s2}{\PYGZdq{}}\PYG{p}{,} \PYG{n+nb}{format}\PYG{o}{=}\PYG{l+s+s1}{\PYGZsq{}}\PYG{l+s+s1}{png}\PYG{l+s+s1}{\PYGZsq{}}\PYG{p}{)}


\PYG{k+kn}{from} \PYG{n+nn}{pylab} \PYG{k+kn}{import} \PYG{n}{plt}\PYG{p}{,} \PYG{n}{mpl}
\PYG{n}{plt}\PYG{o}{.}\PYG{n}{style}\PYG{o}{.}\PYG{n}{use}\PYG{p}{(}\PYG{l+s+s1}{\PYGZsq{}}\PYG{l+s+s1}{seaborn}\PYG{l+s+s1}{\PYGZsq{}}\PYG{p}{)}
\PYG{n}{mpl}\PYG{o}{.}\PYG{n}{rcParams}\PYG{p}{[}\PYG{l+s+s1}{\PYGZsq{}}\PYG{l+s+s1}{font.family}\PYG{l+s+s1}{\PYGZsq{}}\PYG{p}{]} \PYG{o}{=} \PYG{l+s+s1}{\PYGZsq{}}\PYG{l+s+s1}{serif}\PYG{l+s+s1}{\PYGZsq{}}

\PYG{n}{DeltaT} \PYG{o}{=} \PYG{l+m+mf}{0.001}
\PYG{c+c1}{\PYGZsh{}set up arrays }
\PYG{n}{tfinal} \PYG{o}{=} \PYG{l+m+mi}{20} \PYG{c+c1}{\PYGZsh{} in years}
\PYG{n}{n} \PYG{o}{=} \PYG{n}{ceil}\PYG{p}{(}\PYG{n}{tfinal}\PYG{o}{/}\PYG{n}{DeltaT}\PYG{p}{)}
\PYG{c+c1}{\PYGZsh{} set up arrays for t, v, and x}
\PYG{n}{t} \PYG{o}{=} \PYG{n}{np}\PYG{o}{.}\PYG{n}{zeros}\PYG{p}{(}\PYG{n}{n}\PYG{p}{)}
\PYG{n}{v} \PYG{o}{=} \PYG{n}{np}\PYG{o}{.}\PYG{n}{zeros}\PYG{p}{(}\PYG{n}{n}\PYG{p}{)}
\PYG{n}{x} \PYG{o}{=} \PYG{n}{np}\PYG{o}{.}\PYG{n}{zeros}\PYG{p}{(}\PYG{n}{n}\PYG{p}{)}
\PYG{c+c1}{\PYGZsh{} Initial conditions as simple one\PYGZhy{}dimensional arrays of time}
\PYG{n}{x0} \PYG{o}{=}  \PYG{l+m+mf}{1.0} 
\PYG{n}{v0} \PYG{o}{=} \PYG{l+m+mf}{0.0}
\PYG{n}{x}\PYG{p}{[}\PYG{l+m+mi}{0}\PYG{p}{]} \PYG{o}{=} \PYG{n}{x0}
\PYG{n}{v}\PYG{p}{[}\PYG{l+m+mi}{0}\PYG{p}{]} \PYG{o}{=} \PYG{n}{v0}
\PYG{n}{gamma} \PYG{o}{=} \PYG{l+m+mf}{0.0}
\PYG{c+c1}{\PYGZsh{} Start integrating using Euler\PYGZhy{}Cromer\PYGZsq{}s method}
\PYG{k}{for} \PYG{n}{i} \PYG{o+ow}{in} \PYG{n+nb}{range}\PYG{p}{(}\PYG{n}{n}\PYG{o}{\PYGZhy{}}\PYG{l+m+mi}{1}\PYG{p}{)}\PYG{p}{:}
    \PYG{c+c1}{\PYGZsh{} Set up the acceleration}
    \PYG{c+c1}{\PYGZsh{} Here you could have defined your own function for this}
    \PYG{n}{a} \PYG{o}{=}  \PYG{o}{\PYGZhy{}}\PYG{l+m+mi}{2}\PYG{o}{*}\PYG{n}{gamma}\PYG{o}{*}\PYG{n}{v}\PYG{p}{[}\PYG{n}{i}\PYG{p}{]}\PYG{o}{\PYGZhy{}}\PYG{n}{x}\PYG{p}{[}\PYG{n}{i}\PYG{p}{]}
    \PYG{c+c1}{\PYGZsh{} update velocity, time and position}
    \PYG{n}{v}\PYG{p}{[}\PYG{n}{i}\PYG{o}{+}\PYG{l+m+mi}{1}\PYG{p}{]} \PYG{o}{=} \PYG{n}{v}\PYG{p}{[}\PYG{n}{i}\PYG{p}{]} \PYG{o}{+} \PYG{n}{DeltaT}\PYG{o}{*}\PYG{n}{a}
    \PYG{n}{x}\PYG{p}{[}\PYG{n}{i}\PYG{o}{+}\PYG{l+m+mi}{1}\PYG{p}{]} \PYG{o}{=} \PYG{n}{x}\PYG{p}{[}\PYG{n}{i}\PYG{p}{]} \PYG{o}{+} \PYG{n}{DeltaT}\PYG{o}{*}\PYG{n}{v}\PYG{p}{[}\PYG{n}{i}\PYG{o}{+}\PYG{l+m+mi}{1}\PYG{p}{]}
    \PYG{n}{t}\PYG{p}{[}\PYG{n}{i}\PYG{o}{+}\PYG{l+m+mi}{1}\PYG{p}{]} \PYG{o}{=} \PYG{n}{t}\PYG{p}{[}\PYG{n}{i}\PYG{p}{]} \PYG{o}{+} \PYG{n}{DeltaT}
\PYG{c+c1}{\PYGZsh{} Plot position as function of time    }
\PYG{n}{fig}\PYG{p}{,} \PYG{n}{ax} \PYG{o}{=} \PYG{n}{plt}\PYG{o}{.}\PYG{n}{subplots}\PYG{p}{(}\PYG{p}{)}
\PYG{c+c1}{\PYGZsh{}ax.set\PYGZus{}xlim(0, tfinal)}
\PYG{n}{ax}\PYG{o}{.}\PYG{n}{set\PYGZus{}ylabel}\PYG{p}{(}\PYG{l+s+s1}{\PYGZsq{}}\PYG{l+s+s1}{x[m]}\PYG{l+s+s1}{\PYGZsq{}}\PYG{p}{)}
\PYG{n}{ax}\PYG{o}{.}\PYG{n}{set\PYGZus{}xlabel}\PYG{p}{(}\PYG{l+s+s1}{\PYGZsq{}}\PYG{l+s+s1}{t[s]}\PYG{l+s+s1}{\PYGZsq{}}\PYG{p}{)}
\PYG{n}{ax}\PYG{o}{.}\PYG{n}{plot}\PYG{p}{(}\PYG{n}{t}\PYG{p}{,} \PYG{n}{x}\PYG{p}{)}
\PYG{n}{fig}\PYG{o}{.}\PYG{n}{tight\PYGZus{}layout}\PYG{p}{(}\PYG{p}{)}
\PYG{n}{save\PYGZus{}fig}\PYG{p}{(}\PYG{l+s+s2}{\PYGZdq{}}\PYG{l+s+s2}{BlockEulerCromer}\PYG{l+s+s2}{\PYGZdq{}}\PYG{p}{)}
\PYG{n}{plt}\PYG{o}{.}\PYG{n}{show}\PYG{p}{(}\PYG{p}{)}
\end{sphinxVerbatim}

When setting up the value of \(\gamma\) we see that for \(\gamma=0\) we get the simple oscillatory motion with no damping.
Choosing \(\gamma < 1\) leads to the classical underdamped case with oscillatory motion, but where the motion comes to an end.

Choosing \(\gamma =1\) leads to what normally is called critical damping and \(\gamma> 1\) leads to critical overdamping.
Try it out and try also to change the initial position and velocity. Setting \(\gamma=1\)
yields a situation, as discussed above, where the solution approaches quickly zero and does not oscillate. With zero initial velocity it will never cross zero.


\section{Sinusoidally Driven Oscillators}
\label{\detokenize{chapter1:sinusoidally-driven-oscillators}}
Here, we consider the force




\begin{equation*}
\begin{split}
\begin{equation}
F=-kx-b\dot{x}+F_0\cos\omega t,
\label{_auto44} \tag{57}
\end{equation}
\end{split}
\end{equation*}
which leads to the differential equation




\begin{equation*}
\begin{split}
\begin{equation}
\label{eq:drivenosc} \tag{58}
\ddot{x}+2\beta\dot{x}+\omega_0^2x=(F_0/m)\cos\omega t.
\end{equation}
\end{split}
\end{equation*}
Consider a single solution with no arbitrary constants, which we will
call a \{\textbackslash{}it particular solution\}, \(x_p(t)\). It should be emphasized
that this is \{\textbackslash{}bf A\} particular solution, because there exists an
infinite number of such solutions because the general solution should
have two arbitrary constants. Now consider solutions to the same
equation without the driving term, which include two arbitrary
constants. These are called either \{\textbackslash{}it homogenous solutions\} or \{\textbackslash{}it
complementary solutions\}, and were given in the previous section,
e.g. Eq. ({\hyperref[\detokenize{chapter1:eq:homogsolution}]{\emph{55}}}) for the underdamped case. The
homogenous solution already incorporates the two arbitrary constants,
so any sum of a homogenous solution and a particular solution will
represent the \{\textbackslash{}it general solution\} of the equation. The general
solution incorporates the two arbitrary constants \(A\) and \(B\) to
accommodate the two initial conditions. One could have picked a
different particular solution, i.e. the original particular solution
plus any homogenous solution with the arbitrary constants \(A_p\) and
\(B_p\) chosen at will. When one adds in the homogenous solution, which
has adjustable constants with arbitrary constants \(A'\) and \(B'\), to
the new particular solution, one can get the same general solution by
simply adjusting the new constants such that \(A'+A_p=A\) and
\(B'+B_p=B\). Thus, the choice of \(A_p\) and \(B_p\) are irrelevant, and
when choosing the particular solution it is best to make the simplest
choice possible.

To find a particular solution, one first guesses at the form,




\begin{equation*}
\begin{split}
\begin{equation}
\label{eq:partform} \tag{59}
x_p(t)=D\cos(\omega t-\delta),
\end{equation}
\end{split}
\end{equation*}
and rewrite the differential equation as




\begin{equation*}
\begin{split}
\begin{equation}
D\left\{-\omega^2\cos(\omega t-\delta)-2\beta\omega\sin(\omega t-\delta)+\omega_0^2\cos(\omega t-\delta)\right\}=\frac{F_0}{m}\cos(\omega t).
\label{_auto45} \tag{60}
\end{equation}
\end{split}
\end{equation*}
One can now use angle addition formulas to get
\begin{equation*}
\begin{split}
\begin{eqnarray}
D\left\{(-\omega^2\cos\delta+2\beta\omega\sin\delta+\omega_0^2\cos\delta)\cos(\omega t)\right.&&\\
\nonumber
\left.+(-\omega^2\sin\delta-2\beta\omega\cos\delta+\omega_0^2\sin\delta)\sin(\omega t)\right\}
&=&\frac{F_0}{m}\cos(\omega t).
\end{eqnarray}
\end{split}
\end{equation*}
Both the \(\cos\) and \(\sin\) terms need to equate if the expression is to hold at all times. Thus, this becomes two equations
\begin{equation*}
\begin{split}
\begin{eqnarray}
D\left\{-\omega^2\cos\delta+2\beta\omega\sin\delta+\omega_0^2\cos\delta\right\}&=&\frac{F_0}{m}\\
\nonumber
-\omega^2\sin\delta-2\beta\omega\cos\delta+\omega_0^2\sin\delta&=&0.
\end{eqnarray}
\end{split}
\end{equation*}
After dividing by \(\cos\delta\), the lower expression leads to




\begin{equation*}
\begin{split}
\begin{equation}
\tan\delta=\frac{2\beta\omega}{\omega_0^2-\omega^2}.
\label{_auto46} \tag{61}
\end{equation}
\end{split}
\end{equation*}
Using the identities \(\tan^2+1=\csc^2\) and \(\sin^2+\cos^2=1\), one can also express \(\sin\delta\) and \(\cos\delta\),
\begin{equation*}
\begin{split}
\begin{eqnarray}
\sin\delta&=&\frac{2\beta\omega}{\sqrt{(\omega_0^2-\omega^2)^2+4\omega^2\beta^2}},\\
\nonumber
\cos\delta&=&\frac{(\omega_0^2-\omega^2)}{\sqrt{(\omega_0^2-\omega^2)^2+4\omega^2\beta^2}}
\end{eqnarray}
\end{split}
\end{equation*}
Inserting the expressions for \(\cos\delta\) and \(\sin\delta\) into the expression for \(D\),




\begin{equation*}
\begin{split}
\begin{equation}
\label{eq:Ddrive} \tag{62}
D=\frac{F_0/m}{\sqrt{(\omega_0^2-\omega^2)^2+4\omega^2\beta^2}}.
\end{equation}
\end{split}
\end{equation*}
For a given initial condition, e.g. initial displacement and velocity,
one must add the homogenous solution then solve for the two arbitrary
constants. However, because the homogenous solutions decay with time
as \(e^{-\beta t}\), the particular solution is all that remains at
large times, and is therefore the steady state solution. Because the
arbitrary constants are all in the homogenous solution, all memory of
the initial conditions are lost at large times, \(t>>1/\beta\).

The amplitude of the motion, \(D\), is linearly proportional to the
driving force (\(F_0/m\)), but also depends on the driving frequency
\(\omega\). For small \(\beta\) the maximum will occur at
\(\omega=\omega_0\). This is referred to as a resonance. In the limit
\(\beta\rightarrow 0\) the amplitude at resonance approaches infinity.


\section{Alternative Derivation for Driven Oscillators}
\label{\detokenize{chapter1:alternative-derivation-for-driven-oscillators}}
Here, we derive the same expressions as in Equations ({\hyperref[\detokenize{chapter1:eq:partform}]{\emph{59}}}) and ({\hyperref[\detokenize{chapter1:eq:Ddrive}]{\emph{62}}}) but express the driving forces as
\begin{equation*}
\begin{split}
\begin{eqnarray}
F(t)&=&F_0e^{i\omega t},
\end{eqnarray}
\end{split}
\end{equation*}
rather than as \(F_0\cos\omega t\). The real part of \(F\) is the same as before. For the differential equation,




\begin{equation*}
\begin{split}
\begin{eqnarray}
\label{eq:compdrive} \tag{63}
\ddot{x}+2\beta\dot{x}+\omega_0^2x&=&\frac{F_0}{m}e^{i\omega t},
\end{eqnarray}
\end{split}
\end{equation*}
one can treat \(x(t)\) as an imaginary function. Because the operations
\(d^2/dt^2\) and \(d/dt\) are real and thus do not mix the real and
imaginary parts of \(x(t)\), Eq. ({\hyperref[\detokenize{chapter1:eq:compdrive}]{\emph{63}}}) is effectively 2
equations. Because \(e^{\omega t}=\cos\omega t+i\sin\omega t\), the real
part of the solution for \(x(t)\) gives the solution for a driving force
\(F_0\cos\omega t\), and the imaginary part of \(x\) corresponds to the
case where the driving force is \(F_0\sin\omega t\). It is rather easy
to solve for the complex \(x\) in this case, and by taking the real part
of the solution, one finds the answer for the \(\cos\omega t\) driving
force.

We assume a simple form for the particular solution




\begin{equation*}
\begin{split}
\begin{equation}
x_p=De^{i\omega t},
\label{_auto47} \tag{64}
\end{equation}
\end{split}
\end{equation*}
where \(D\) is a complex constant.

From Eq. ({\hyperref[\detokenize{chapter1:eq:compdrive}]{\emph{63}}}) one inserts the form for \(x_p\) above to get
\begin{equation*}
\begin{split}
\begin{eqnarray}
D\left\{-\omega^2+2i\beta\omega+\omega_0^2\right\}e^{i\omega t}=(F_0/m)e^{i\omega t},\\
\nonumber
D=\frac{F_0/m}{(\omega_0^2-\omega^2)+2i\beta\omega}.
\end{eqnarray}
\end{split}
\end{equation*}
The norm and phase for \(D=|D|e^{-i\delta}\) can be read by inspection,




\begin{equation*}
\begin{split}
\begin{equation}
|D|=\frac{F_0/m}{\sqrt{(\omega_0^2-\omega^2)^2+4\beta^2\omega^2}},~~~~\tan\delta=\frac{2\beta\omega}{\omega_0^2-\omega^2}.
\label{_auto48} \tag{65}
\end{equation}
\end{split}
\end{equation*}
This is the same expression for \(\delta\) as before. One then finds \(x_p(t)\),




\begin{equation*}
\begin{split}
\begin{eqnarray}
\label{eq:fastdriven1} \tag{66}
x_p(t)&=&\Re\frac{(F_0/m)e^{i\omega t-i\delta}}{\sqrt{(\omega_0^2-\omega^2)^2+4\beta^2\omega^2}}\\
\nonumber
&=&\frac{(F_0/m)\cos(\omega t-\delta)}{\sqrt{(\omega_0^2-\omega^2)^2+4\beta^2\omega^2}}.
\end{eqnarray}
\end{split}
\end{equation*}
This is the same answer as before.
If one wished to solve for the case where \(F(t)= F_0\sin\omega t\), the imaginary part of the solution would work




\begin{equation*}
\begin{split}
\begin{eqnarray}
\label{eq:fastdriven2} \tag{67}
x_p(t)&=&\Im\frac{(F_0/m)e^{i\omega t-i\delta}}{\sqrt{(\omega_0^2-\omega^2)^2+4\beta^2\omega^2}}\\
\nonumber
&=&\frac{(F_0/m)\sin(\omega t-\delta)}{\sqrt{(\omega_0^2-\omega^2)^2+4\beta^2\omega^2}}.
\end{eqnarray}
\end{split}
\end{equation*}
Consider the damped and driven harmonic oscillator worked out above. Given \(F_0, m,\beta\) and \(\omega_0\), solve for the complete solution \(x(t)\) for the case where \(F=F_0\sin\omega t\) with initial conditions \(x(t=0)=0\) and \(v(t=0)=0\). Assume the underdamped case.

The general solution including the arbitrary constants includes both the homogenous and particular solutions,
\begin{equation*}
\begin{split}
\begin{eqnarray*}
x(t)&=&\frac{F_0}{m}\frac{\sin(\omega t-\delta)}{\sqrt{(\omega_0^2-\omega^2)^2+4\beta^2\omega^2}}
+A\cos\omega't e^{-\beta t}+B\sin\omega't e^{-\beta t}.
\end{eqnarray*}
\end{split}
\end{equation*}
The quantities \(\delta\) and \(\omega'\) are given earlier in the
section, \(\omega'=\sqrt{\omega_0^2-\beta^2},
\delta=\tan^{-1}(2\beta\omega/(\omega_0^2-\omega^2)\). Here, solving
the problem means finding the arbitrary constants \(A\) and
\(B\). Satisfying the initial conditions for the initial position and
velocity:
\begin{equation*}
\begin{split}
\begin{eqnarray*}
x(t=0)=0&=&-\eta\sin\delta+A,\\
v(t=0)=0&=&\omega\eta\cos\delta-\beta A+\omega'B,\\
\eta&\equiv&\frac{F_0}{m}\frac{1}{\sqrt{(\omega_0^2-\omega^2)^2+4\beta^2\omega^2}}.
\end{eqnarray*}
\end{split}
\end{equation*}
The problem is now reduced to 2 equations and 2 unknowns, \(A\) and \(B\). The solution is
\begin{equation*}
\begin{split}
\begin{eqnarray}
A&=& \eta\sin\delta ,~~~B=\frac{-\omega\eta\cos\delta+\beta\eta\sin\delta}{\omega'}.
\end{eqnarray}
\end{split}
\end{equation*}

\section{Resonance Widths; the \protect\(Q\protect\) factor}
\label{\detokenize{chapter1:resonance-widths-the-q-factor}}
From the previous two sections, the particular solution for a driving force, \(F=F_0\cos\omega t\), is
\begin{equation*}
\begin{split}
\begin{eqnarray}
x_p(t)&=&\frac{F_0/m}{\sqrt{(\omega_0^2-\omega^2)^2+4\omega^2\beta^2}}\cos(\omega_t-\delta),\\
\nonumber
\delta&=&\tan^{-1}\left(\frac{2\beta\omega}{\omega_0^2-\omega^2}\right).
\end{eqnarray}
\end{split}
\end{equation*}
If one fixes the driving frequency \(\omega\) and adjusts the
fundamental frequency \(\omega_0=\sqrt{k/m}\), the maximum amplitude
occurs when \(\omega_0=\omega\) because that is when the term from the
denominator \((\omega_0^2-\omega^2)^2+4\omega^2\beta^2\) is at a
minimum. This is akin to dialing into a radio station. However, if one
fixes \(\omega_0\) and adjusts the driving frequency one minimize with
respect to \(\omega\), e.g. set




\begin{equation*}
\begin{split}
\begin{equation}
\frac{d}{d\omega}\left[(\omega_0^2-\omega^2)^2+4\omega^2\beta^2\right]=0,
\label{_auto49} \tag{68}
\end{equation}
\end{split}
\end{equation*}
and one finds that the maximum amplitude occurs when
\(\omega=\sqrt{\omega_0^2-2\beta^2}\). If \(\beta\) is small relative to
\(\omega_0\), one can simply state that the maximum amplitude is




\begin{equation*}
\begin{split}
\begin{equation}
x_{\rm max}\approx\frac{F_0}{2m\beta \omega_0}.
\label{_auto50} \tag{69}
\end{equation}
\end{split}
\end{equation*}\begin{equation*}
\begin{split}
\begin{eqnarray}
\frac{4\omega^2\beta^2}{(\omega_0^2-\omega^2)^2+4\omega^2\beta^2}=\frac{1}{2}.
\end{eqnarray}
\end{split}
\end{equation*}
For small damping this occurs when \(\omega=\omega_0\pm \beta\), so the \(FWHM\approx 2\beta\). For the purposes of tuning to a specific frequency, one wants the width to be as small as possible. The ratio of \(\omega_0\) to \(FWHM\) is known as the \{\textbackslash{}it quality\} factor, or \(Q\) factor,




\begin{equation*}
\begin{split}
\begin{equation}
Q\equiv \frac{\omega_0}{2\beta}.
\label{_auto51} \tag{70}
\end{equation}
\end{split}
\end{equation*}

\section{Numerical Studies of Driven Oscillations}
\label{\detokenize{chapter1:numerical-studies-of-driven-oscillations}}
Solving the problem of driven oscillations numerically gives us much
more flexibility to study different types of driving forces. We can
reuse our earlier code by simply adding a driving force. If we stay in
the \(x\)\sphinxhyphen{}direction only this can be easily done by adding a term
\(F_{\mathrm{ext}}(x,t)\). Note that we have kept it rather general
here, allowing for both a spatial and a temporal dependence.

Before we dive into the code, we need to briefly remind ourselves
about the equations we started with for the case with damping, namely
\begin{equation*}
\begin{split}
m\frac{d^2x}{dt^2} + b\frac{dx}{dt}+kx(t) =0,
\end{split}
\end{equation*}
with no external force applied to the system.

Let us now for simplicty assume that our external force is given by
\begin{equation*}
\begin{split}
F_{\mathrm{ext}}(t) = F_0\cos{(\omega t)},
\end{split}
\end{equation*}
where \(F_0\) is a constant (what is its dimension?) and \(\omega\) is the frequency of the applied external driving force.
\sphinxstylestrong{Small question:} would you expect energy to be conserved now?

Introducing the external force into our lovely differential equation
and dividing by \(m\) and introducing \(\omega_0^2=\sqrt{k/m}\) we have
\begin{equation*}
\begin{split}
\frac{d^2x}{dt^2} + \frac{b}{m}\frac{dx}{dt}+\omega_0^2x(t) =\frac{F_0}{m}\cos{(\omega t)},
\end{split}
\end{equation*}
Thereafter we introduce a dimensionless time \(\tau = t\omega_0\)
and a dimensionless frequency \(\tilde{\omega}=\omega/\omega_0\). We have then
\begin{equation*}
\begin{split}
\frac{d^2x}{d\tau^2} + \frac{b}{m\omega_0}\frac{dx}{d\tau}+x(\tau) =\frac{F_0}{m\omega_0^2}\cos{(\tilde{\omega}\tau)},
\end{split}
\end{equation*}
Introducing a new amplitude \(\tilde{F} =F_0/(m\omega_0^2)\) (check dimensionality again) we have
\begin{equation*}
\begin{split}
\frac{d^2x}{d\tau^2} + \frac{b}{m\omega_0}\frac{dx}{d\tau}+x(\tau) =\tilde{F}\cos{(\tilde{\omega}\tau)}.
\end{split}
\end{equation*}
Our final step, as we did in the case of various types of damping, is
to define \(\gamma = b/(2m\omega_0)\) and rewrite our equations as
\begin{equation*}
\begin{split}
\frac{d^2x}{d\tau^2} + 2\gamma\frac{dx}{d\tau}+x(\tau) =\tilde{F}\cos{(\tilde{\omega}\tau)}.
\end{split}
\end{equation*}
This is the equation we will code below using the Euler\sphinxhyphen{}Cromer method.

\begin{sphinxVerbatim}[commandchars=\\\{\}]
\PYG{n}{DeltaT} \PYG{o}{=} \PYG{l+m+mf}{0.001}
\PYG{c+c1}{\PYGZsh{}set up arrays }
\PYG{n}{tfinal} \PYG{o}{=} \PYG{l+m+mi}{20} \PYG{c+c1}{\PYGZsh{} in years}
\PYG{n}{n} \PYG{o}{=} \PYG{n}{ceil}\PYG{p}{(}\PYG{n}{tfinal}\PYG{o}{/}\PYG{n}{DeltaT}\PYG{p}{)}
\PYG{c+c1}{\PYGZsh{} set up arrays for t, v, and x}
\PYG{n}{t} \PYG{o}{=} \PYG{n}{np}\PYG{o}{.}\PYG{n}{zeros}\PYG{p}{(}\PYG{n}{n}\PYG{p}{)}
\PYG{n}{v} \PYG{o}{=} \PYG{n}{np}\PYG{o}{.}\PYG{n}{zeros}\PYG{p}{(}\PYG{n}{n}\PYG{p}{)}
\PYG{n}{x} \PYG{o}{=} \PYG{n}{np}\PYG{o}{.}\PYG{n}{zeros}\PYG{p}{(}\PYG{n}{n}\PYG{p}{)}
\PYG{c+c1}{\PYGZsh{} Initial conditions as one\PYGZhy{}dimensional arrays of time}
\PYG{n}{x0} \PYG{o}{=}  \PYG{l+m+mf}{1.0} 
\PYG{n}{v0} \PYG{o}{=} \PYG{l+m+mf}{0.0}
\PYG{n}{x}\PYG{p}{[}\PYG{l+m+mi}{0}\PYG{p}{]} \PYG{o}{=} \PYG{n}{x0}
\PYG{n}{v}\PYG{p}{[}\PYG{l+m+mi}{0}\PYG{p}{]} \PYG{o}{=} \PYG{n}{v0}
\PYG{n}{gamma} \PYG{o}{=} \PYG{l+m+mf}{0.2}
\PYG{n}{Omegatilde} \PYG{o}{=} \PYG{l+m+mf}{0.5}
\PYG{n}{Ftilde} \PYG{o}{=} \PYG{l+m+mf}{1.0}
\PYG{c+c1}{\PYGZsh{} Start integrating using Euler\PYGZhy{}Cromer\PYGZsq{}s method}
\PYG{k}{for} \PYG{n}{i} \PYG{o+ow}{in} \PYG{n+nb}{range}\PYG{p}{(}\PYG{n}{n}\PYG{o}{\PYGZhy{}}\PYG{l+m+mi}{1}\PYG{p}{)}\PYG{p}{:}
    \PYG{c+c1}{\PYGZsh{} Set up the acceleration}
    \PYG{c+c1}{\PYGZsh{} Here you could have defined your own function for this}
    \PYG{n}{a} \PYG{o}{=}  \PYG{o}{\PYGZhy{}}\PYG{l+m+mi}{2}\PYG{o}{*}\PYG{n}{gamma}\PYG{o}{*}\PYG{n}{v}\PYG{p}{[}\PYG{n}{i}\PYG{p}{]}\PYG{o}{\PYGZhy{}}\PYG{n}{x}\PYG{p}{[}\PYG{n}{i}\PYG{p}{]}\PYG{o}{+}\PYG{n}{Ftilde}\PYG{o}{*}\PYG{n}{cos}\PYG{p}{(}\PYG{n}{t}\PYG{p}{[}\PYG{n}{i}\PYG{p}{]}\PYG{o}{*}\PYG{n}{Omegatilde}\PYG{p}{)}
    \PYG{c+c1}{\PYGZsh{} update velocity, time and position}
    \PYG{n}{v}\PYG{p}{[}\PYG{n}{i}\PYG{o}{+}\PYG{l+m+mi}{1}\PYG{p}{]} \PYG{o}{=} \PYG{n}{v}\PYG{p}{[}\PYG{n}{i}\PYG{p}{]} \PYG{o}{+} \PYG{n}{DeltaT}\PYG{o}{*}\PYG{n}{a}
    \PYG{n}{x}\PYG{p}{[}\PYG{n}{i}\PYG{o}{+}\PYG{l+m+mi}{1}\PYG{p}{]} \PYG{o}{=} \PYG{n}{x}\PYG{p}{[}\PYG{n}{i}\PYG{p}{]} \PYG{o}{+} \PYG{n}{DeltaT}\PYG{o}{*}\PYG{n}{v}\PYG{p}{[}\PYG{n}{i}\PYG{o}{+}\PYG{l+m+mi}{1}\PYG{p}{]}
    \PYG{n}{t}\PYG{p}{[}\PYG{n}{i}\PYG{o}{+}\PYG{l+m+mi}{1}\PYG{p}{]} \PYG{o}{=} \PYG{n}{t}\PYG{p}{[}\PYG{n}{i}\PYG{p}{]} \PYG{o}{+} \PYG{n}{DeltaT}
\PYG{c+c1}{\PYGZsh{} Plot position as function of time    }
\PYG{n}{fig}\PYG{p}{,} \PYG{n}{ax} \PYG{o}{=} \PYG{n}{plt}\PYG{o}{.}\PYG{n}{subplots}\PYG{p}{(}\PYG{p}{)}
\PYG{n}{ax}\PYG{o}{.}\PYG{n}{set\PYGZus{}ylabel}\PYG{p}{(}\PYG{l+s+s1}{\PYGZsq{}}\PYG{l+s+s1}{x[m]}\PYG{l+s+s1}{\PYGZsq{}}\PYG{p}{)}
\PYG{n}{ax}\PYG{o}{.}\PYG{n}{set\PYGZus{}xlabel}\PYG{p}{(}\PYG{l+s+s1}{\PYGZsq{}}\PYG{l+s+s1}{t[s]}\PYG{l+s+s1}{\PYGZsq{}}\PYG{p}{)}
\PYG{n}{ax}\PYG{o}{.}\PYG{n}{plot}\PYG{p}{(}\PYG{n}{t}\PYG{p}{,} \PYG{n}{x}\PYG{p}{)}
\PYG{n}{fig}\PYG{o}{.}\PYG{n}{tight\PYGZus{}layout}\PYG{p}{(}\PYG{p}{)}
\PYG{n}{save\PYGZus{}fig}\PYG{p}{(}\PYG{l+s+s2}{\PYGZdq{}}\PYG{l+s+s2}{ForcedBlockEulerCromer}\PYG{l+s+s2}{\PYGZdq{}}\PYG{p}{)}
\PYG{n}{plt}\PYG{o}{.}\PYG{n}{show}\PYG{p}{(}\PYG{p}{)}
\end{sphinxVerbatim}

In the above example we have focused on the Euler\sphinxhyphen{}Cromer method. This
method has a local truncation error which is proportional to \(\Delta t^2\)
and thereby a global error which is proportional to \(\Delta t\).
We can improve this by using the Runge\sphinxhyphen{}Kutta family of
methods. The widely popular Runge\sphinxhyphen{}Kutta to fourth order or just \sphinxstylestrong{RK4}
has indeed a much better truncation error. The RK4 method has a global
error which is proportional to \(\Delta t\).

Let us revisit this method and see how we can implement it for the above example.


\section{Differential Equations, Runge\sphinxhyphen{}Kutta methods}
\label{\detokenize{chapter1:differential-equations-runge-kutta-methods}}
Runge\sphinxhyphen{}Kutta (RK) methods are based on Taylor expansion formulae, but yield
in general better algorithms for solutions of an ordinary differential equation.
The basic philosophy is that it provides an intermediate step in the computation of \(y_{i+1}\).

To see this, consider first the following definitions




\begin{equation*}
\begin{split}
\begin{equation}
\frac{dy}{dt}=f(t,y),  
\label{_auto52} \tag{71}
\end{equation}
\end{split}
\end{equation*}
and




\begin{equation*}
\begin{split}
\begin{equation}
y(t)=\int f(t,y) dt,  
\label{_auto53} \tag{72}
\end{equation}
\end{split}
\end{equation*}
and




\begin{equation*}
\begin{split}
\begin{equation}
y_{i+1}=y_i+ \int_{t_i}^{t_{i+1}} f(t,y) dt.
\label{_auto54} \tag{73}
\end{equation}
\end{split}
\end{equation*}
To demonstrate the philosophy behind RK methods, let us consider
the second\sphinxhyphen{}order RK method, RK2.
The first approximation consists in Taylor expanding \(f(t,y)\)
around the center of the integration interval \(t_i\) to \(t_{i+1}\),
that is, at \(t_i+h/2\), \(h\) being the step.
Using the midpoint formula for an integral,
defining \(y(t_i+h/2) = y_{i+1/2}\) and\(t_i+h/2 = t_{i+1/2}\), we obtain




\begin{equation*}
\begin{split}
\begin{equation}
\int_{t_i}^{t_{i+1}} f(t,y) dt \approx hf(t_{i+1/2},y_{i+1/2}) +O(h^3).
\label{_auto55} \tag{74}
\end{equation}
\end{split}
\end{equation*}
This means in turn that we have




\begin{equation*}
\begin{split}
\begin{equation}
y_{i+1}=y_i + hf(t_{i+1/2},y_{i+1/2}) +O(h^3).
\label{_auto56} \tag{75}
\end{equation}
\end{split}
\end{equation*}
However, we do not know the value of   \(y_{i+1/2}\). Here comes thus the next approximation, namely, we use Euler’s
method to approximate \(y_{i+1/2}\). We have then




\begin{equation*}
\begin{split}
\begin{equation}
y_{(i+1/2)}=y_i + \frac{h}{2}\frac{dy}{dt}=y(t_i) + \frac{h}{2}f(t_i,y_i).
\label{_auto57} \tag{76}
\end{equation}
\end{split}
\end{equation*}
This means that we can define the following algorithm for
the second\sphinxhyphen{}order Runge\sphinxhyphen{}Kutta method, RK2.

2
6
7

\textless{}
\textless{}
\textless{}
!
!
M
A
T
H
\_
B
L
O
C
K




\begin{equation*}
\begin{split}
\begin{equation}
k_2=hf(t_{i+1/2},y_i+k_1/2),
\label{_auto59} \tag{78}
\end{equation}
\end{split}
\end{equation*}
with the final value




\begin{equation*}
\begin{split}
\begin{equation} 
y_{i+i}\approx y_i + k_2 +O(h^3). 
\label{_auto60} \tag{79}
\end{equation}
\end{split}
\end{equation*}
The difference between the previous one\sphinxhyphen{}step methods
is that we now need an intermediate step in our evaluation,
namely \(t_i+h/2 = t_{(i+1/2)}\) where we evaluate the derivative \(f\).
This involves more operations, but the gain is a better stability
in the solution.

The fourth\sphinxhyphen{}order Runge\sphinxhyphen{}Kutta, RK4, has the following algorithm

2
7
0

\textless{}
\textless{}
\textless{}
!
!
M
A
T
H
\_
B
L
O
C
K
\begin{equation*}
\begin{split}
k_3=hf(t_i+h/2,y_i+k_2/2)\hspace{0.5cm}   k_4=hf(t_i+h,y_i+k_3)
\end{split}
\end{equation*}
with the final result
\begin{equation*}
\begin{split}
y_{i+1}=y_i +\frac{1}{6}\left( k_1 +2k_2+2k_3+k_4\right).
\end{split}
\end{equation*}
Thus, the algorithm consists in first calculating \(k_1\)
with \(t_i\), \(y_1\) and \(f\) as inputs. Thereafter, we increase the step
size by \(h/2\) and calculate \(k_2\), then \(k_3\) and finally \(k_4\). The global error goes as \(O(h^4)\).

However, at this stage, if we keep adding different methods in our
main program, the code will quickly become messy and ugly. Before we
proceed thus, we will now introduce functions that enbody the various
methods for solving differential equations. This means that we can
separate out these methods in own functions and files (and later as classes and more
generic functions) and simply call them when needed. Similarly, we
could easily encapsulate various forces or other quantities of
interest in terms of functions. To see this, let us bring up the code
we developed above for the simple sliding block, but now only with the simple forward Euler method. We introduce
two functions, one for the simple Euler method and one for the
force.

Note that here the forward Euler method does not know the specific force function to be called.
It receives just an input the name. We can easily change the force by adding another function.

\begin{sphinxVerbatim}[commandchars=\\\{\}]
\PYG{k}{def} \PYG{n+nf}{ForwardEuler}\PYG{p}{(}\PYG{n}{v}\PYG{p}{,}\PYG{n}{x}\PYG{p}{,}\PYG{n}{t}\PYG{p}{,}\PYG{n}{n}\PYG{p}{,}\PYG{n}{Force}\PYG{p}{)}\PYG{p}{:}
    \PYG{k}{for} \PYG{n}{i} \PYG{o+ow}{in} \PYG{n+nb}{range}\PYG{p}{(}\PYG{n}{n}\PYG{o}{\PYGZhy{}}\PYG{l+m+mi}{1}\PYG{p}{)}\PYG{p}{:}
        \PYG{n}{v}\PYG{p}{[}\PYG{n}{i}\PYG{o}{+}\PYG{l+m+mi}{1}\PYG{p}{]} \PYG{o}{=} \PYG{n}{v}\PYG{p}{[}\PYG{n}{i}\PYG{p}{]} \PYG{o}{+} \PYG{n}{DeltaT}\PYG{o}{*}\PYG{n}{Force}\PYG{p}{(}\PYG{n}{v}\PYG{p}{[}\PYG{n}{i}\PYG{p}{]}\PYG{p}{,}\PYG{n}{x}\PYG{p}{[}\PYG{n}{i}\PYG{p}{]}\PYG{p}{,}\PYG{n}{t}\PYG{p}{[}\PYG{n}{i}\PYG{p}{]}\PYG{p}{)}
        \PYG{n}{x}\PYG{p}{[}\PYG{n}{i}\PYG{o}{+}\PYG{l+m+mi}{1}\PYG{p}{]} \PYG{o}{=} \PYG{n}{x}\PYG{p}{[}\PYG{n}{i}\PYG{p}{]} \PYG{o}{+} \PYG{n}{DeltaT}\PYG{o}{*}\PYG{n}{v}\PYG{p}{[}\PYG{n}{i}\PYG{p}{]}
        \PYG{n}{t}\PYG{p}{[}\PYG{n}{i}\PYG{o}{+}\PYG{l+m+mi}{1}\PYG{p}{]} \PYG{o}{=} \PYG{n}{t}\PYG{p}{[}\PYG{n}{i}\PYG{p}{]} \PYG{o}{+} \PYG{n}{DeltaT}
\end{sphinxVerbatim}

\begin{sphinxVerbatim}[commandchars=\\\{\}]
\PYG{k}{def} \PYG{n+nf}{SpringForce}\PYG{p}{(}\PYG{n}{v}\PYG{p}{,}\PYG{n}{x}\PYG{p}{,}\PYG{n}{t}\PYG{p}{)}\PYG{p}{:}
\PYG{c+c1}{\PYGZsh{}   note here that we have divided by mass and we return the acceleration}
    \PYG{k}{return}  \PYG{o}{\PYGZhy{}}\PYG{l+m+mi}{2}\PYG{o}{*}\PYG{n}{gamma}\PYG{o}{*}\PYG{n}{v}\PYG{o}{\PYGZhy{}}\PYG{n}{x}\PYG{o}{+}\PYG{n}{Ftilde}\PYG{o}{*}\PYG{n}{cos}\PYG{p}{(}\PYG{n}{t}\PYG{o}{*}\PYG{n}{Omegatilde}\PYG{p}{)}
\end{sphinxVerbatim}

It is easy to add a new method like the Euler\sphinxhyphen{}Cromer

\begin{sphinxVerbatim}[commandchars=\\\{\}]
\PYG{k}{def} \PYG{n+nf}{ForwardEulerCromer}\PYG{p}{(}\PYG{n}{v}\PYG{p}{,}\PYG{n}{x}\PYG{p}{,}\PYG{n}{t}\PYG{p}{,}\PYG{n}{n}\PYG{p}{,}\PYG{n}{Force}\PYG{p}{)}\PYG{p}{:}
    \PYG{k}{for} \PYG{n}{i} \PYG{o+ow}{in} \PYG{n+nb}{range}\PYG{p}{(}\PYG{n}{n}\PYG{o}{\PYGZhy{}}\PYG{l+m+mi}{1}\PYG{p}{)}\PYG{p}{:}
        \PYG{n}{a} \PYG{o}{=} \PYG{n}{Force}\PYG{p}{(}\PYG{n}{v}\PYG{p}{[}\PYG{n}{i}\PYG{p}{]}\PYG{p}{,}\PYG{n}{x}\PYG{p}{[}\PYG{n}{i}\PYG{p}{]}\PYG{p}{,}\PYG{n}{t}\PYG{p}{[}\PYG{n}{i}\PYG{p}{]}\PYG{p}{)}
        \PYG{n}{v}\PYG{p}{[}\PYG{n}{i}\PYG{o}{+}\PYG{l+m+mi}{1}\PYG{p}{]} \PYG{o}{=} \PYG{n}{v}\PYG{p}{[}\PYG{n}{i}\PYG{p}{]} \PYG{o}{+} \PYG{n}{DeltaT}\PYG{o}{*}\PYG{n}{a}
        \PYG{n}{x}\PYG{p}{[}\PYG{n}{i}\PYG{o}{+}\PYG{l+m+mi}{1}\PYG{p}{]} \PYG{o}{=} \PYG{n}{x}\PYG{p}{[}\PYG{n}{i}\PYG{p}{]} \PYG{o}{+} \PYG{n}{DeltaT}\PYG{o}{*}\PYG{n}{v}\PYG{p}{[}\PYG{n}{i}\PYG{o}{+}\PYG{l+m+mi}{1}\PYG{p}{]}
        \PYG{n}{t}\PYG{p}{[}\PYG{n}{i}\PYG{o}{+}\PYG{l+m+mi}{1}\PYG{p}{]} \PYG{o}{=} \PYG{n}{t}\PYG{p}{[}\PYG{n}{i}\PYG{p}{]} \PYG{o}{+} \PYG{n}{DeltaT}
\end{sphinxVerbatim}

and the Velocity Verlet method (be careful with time\sphinxhyphen{}dependence here, it is not an ideal method for non\sphinxhyphen{}conservative forces))

\begin{sphinxVerbatim}[commandchars=\\\{\}]
\PYG{k}{def} \PYG{n+nf}{VelocityVerlet}\PYG{p}{(}\PYG{n}{v}\PYG{p}{,}\PYG{n}{x}\PYG{p}{,}\PYG{n}{t}\PYG{p}{,}\PYG{n}{n}\PYG{p}{,}\PYG{n}{Force}\PYG{p}{)}\PYG{p}{:}
    \PYG{k}{for} \PYG{n}{i} \PYG{o+ow}{in} \PYG{n+nb}{range}\PYG{p}{(}\PYG{n}{n}\PYG{o}{\PYGZhy{}}\PYG{l+m+mi}{1}\PYG{p}{)}\PYG{p}{:}
        \PYG{n}{a} \PYG{o}{=} \PYG{n}{Force}\PYG{p}{(}\PYG{n}{v}\PYG{p}{[}\PYG{n}{i}\PYG{p}{]}\PYG{p}{,}\PYG{n}{x}\PYG{p}{[}\PYG{n}{i}\PYG{p}{]}\PYG{p}{,}\PYG{n}{t}\PYG{p}{[}\PYG{n}{i}\PYG{p}{]}\PYG{p}{)}
        \PYG{n}{x}\PYG{p}{[}\PYG{n}{i}\PYG{o}{+}\PYG{l+m+mi}{1}\PYG{p}{]} \PYG{o}{=} \PYG{n}{x}\PYG{p}{[}\PYG{n}{i}\PYG{p}{]} \PYG{o}{+} \PYG{n}{DeltaT}\PYG{o}{*}\PYG{n}{v}\PYG{p}{[}\PYG{n}{i}\PYG{p}{]}\PYG{o}{+}\PYG{l+m+mf}{0.5}\PYG{o}{*}\PYG{n}{a}
        \PYG{n}{anew} \PYG{o}{=} \PYG{n}{Force}\PYG{p}{(}\PYG{n}{v}\PYG{p}{[}\PYG{n}{i}\PYG{p}{]}\PYG{p}{,}\PYG{n}{x}\PYG{p}{[}\PYG{n}{i}\PYG{o}{+}\PYG{l+m+mi}{1}\PYG{p}{]}\PYG{p}{,}\PYG{n}{t}\PYG{p}{[}\PYG{n}{i}\PYG{o}{+}\PYG{l+m+mi}{1}\PYG{p}{]}\PYG{p}{)}
        \PYG{n}{v}\PYG{p}{[}\PYG{n}{i}\PYG{o}{+}\PYG{l+m+mi}{1}\PYG{p}{]} \PYG{o}{=} \PYG{n}{v}\PYG{p}{[}\PYG{n}{i}\PYG{p}{]} \PYG{o}{+} \PYG{l+m+mf}{0.5}\PYG{o}{*}\PYG{n}{DeltaT}\PYG{o}{*}\PYG{p}{(}\PYG{n}{a}\PYG{o}{+}\PYG{n}{anew}\PYG{p}{)}
        \PYG{n}{t}\PYG{p}{[}\PYG{n}{i}\PYG{o}{+}\PYG{l+m+mi}{1}\PYG{p}{]} \PYG{o}{=} \PYG{n}{t}\PYG{p}{[}\PYG{n}{i}\PYG{p}{]} \PYG{o}{+} \PYG{n}{DeltaT}
\end{sphinxVerbatim}

Finally, we can now add the Runge\sphinxhyphen{}Kutta2 method via a new function

\begin{sphinxVerbatim}[commandchars=\\\{\}]
\PYG{k}{def} \PYG{n+nf}{RK2}\PYG{p}{(}\PYG{n}{v}\PYG{p}{,}\PYG{n}{x}\PYG{p}{,}\PYG{n}{t}\PYG{p}{,}\PYG{n}{n}\PYG{p}{,}\PYG{n}{Force}\PYG{p}{)}\PYG{p}{:}
    \PYG{k}{for} \PYG{n}{i} \PYG{o+ow}{in} \PYG{n+nb}{range}\PYG{p}{(}\PYG{n}{n}\PYG{o}{\PYGZhy{}}\PYG{l+m+mi}{1}\PYG{p}{)}\PYG{p}{:}
\PYG{c+c1}{\PYGZsh{} Setting up k1}
        \PYG{n}{k1x} \PYG{o}{=} \PYG{n}{DeltaT}\PYG{o}{*}\PYG{n}{v}\PYG{p}{[}\PYG{n}{i}\PYG{p}{]}
        \PYG{n}{k1v} \PYG{o}{=} \PYG{n}{DeltaT}\PYG{o}{*}\PYG{n}{Force}\PYG{p}{(}\PYG{n}{v}\PYG{p}{[}\PYG{n}{i}\PYG{p}{]}\PYG{p}{,}\PYG{n}{x}\PYG{p}{[}\PYG{n}{i}\PYG{p}{]}\PYG{p}{,}\PYG{n}{t}\PYG{p}{[}\PYG{n}{i}\PYG{p}{]}\PYG{p}{)}
\PYG{c+c1}{\PYGZsh{} Setting up k2}
        \PYG{n}{vv} \PYG{o}{=} \PYG{n}{v}\PYG{p}{[}\PYG{n}{i}\PYG{p}{]}\PYG{o}{+}\PYG{n}{k1v}\PYG{o}{*}\PYG{l+m+mf}{0.5}
        \PYG{n}{xx} \PYG{o}{=} \PYG{n}{x}\PYG{p}{[}\PYG{n}{i}\PYG{p}{]}\PYG{o}{+}\PYG{n}{k1x}\PYG{o}{*}\PYG{l+m+mf}{0.5}
        \PYG{n}{k2x} \PYG{o}{=} \PYG{n}{DeltaT}\PYG{o}{*}\PYG{n}{vv}
        \PYG{n}{k2v} \PYG{o}{=} \PYG{n}{DeltaT}\PYG{o}{*}\PYG{n}{Force}\PYG{p}{(}\PYG{n}{vv}\PYG{p}{,}\PYG{n}{xx}\PYG{p}{,}\PYG{n}{t}\PYG{p}{[}\PYG{n}{i}\PYG{p}{]}\PYG{o}{+}\PYG{n}{DeltaT}\PYG{o}{*}\PYG{l+m+mf}{0.5}\PYG{p}{)}
\PYG{c+c1}{\PYGZsh{} Final result}
        \PYG{n}{x}\PYG{p}{[}\PYG{n}{i}\PYG{o}{+}\PYG{l+m+mi}{1}\PYG{p}{]} \PYG{o}{=} \PYG{n}{x}\PYG{p}{[}\PYG{n}{i}\PYG{p}{]}\PYG{o}{+}\PYG{n}{k2x}
        \PYG{n}{v}\PYG{p}{[}\PYG{n}{i}\PYG{o}{+}\PYG{l+m+mi}{1}\PYG{p}{]} \PYG{o}{=} \PYG{n}{v}\PYG{p}{[}\PYG{n}{i}\PYG{p}{]}\PYG{o}{+}\PYG{n}{k2v}
	\PYG{n}{t}\PYG{p}{[}\PYG{n}{i}\PYG{o}{+}\PYG{l+m+mi}{1}\PYG{p}{]} \PYG{o}{=} \PYG{n}{t}\PYG{p}{[}\PYG{n}{i}\PYG{p}{]}\PYG{o}{+}\PYG{n}{DeltaT}
\end{sphinxVerbatim}

Finally, we can now add the Runge\sphinxhyphen{}Kutta2 method via a new function

\begin{sphinxVerbatim}[commandchars=\\\{\}]
\PYG{k}{def} \PYG{n+nf}{RK4}\PYG{p}{(}\PYG{n}{v}\PYG{p}{,}\PYG{n}{x}\PYG{p}{,}\PYG{n}{t}\PYG{p}{,}\PYG{n}{n}\PYG{p}{,}\PYG{n}{Force}\PYG{p}{)}\PYG{p}{:}
    \PYG{k}{for} \PYG{n}{i} \PYG{o+ow}{in} \PYG{n+nb}{range}\PYG{p}{(}\PYG{n}{n}\PYG{o}{\PYGZhy{}}\PYG{l+m+mi}{1}\PYG{p}{)}\PYG{p}{:}
\PYG{c+c1}{\PYGZsh{} Setting up k1}
        \PYG{n}{k1x} \PYG{o}{=} \PYG{n}{DeltaT}\PYG{o}{*}\PYG{n}{v}\PYG{p}{[}\PYG{n}{i}\PYG{p}{]}
        \PYG{n}{k1v} \PYG{o}{=} \PYG{n}{DeltaT}\PYG{o}{*}\PYG{n}{Force}\PYG{p}{(}\PYG{n}{v}\PYG{p}{[}\PYG{n}{i}\PYG{p}{]}\PYG{p}{,}\PYG{n}{x}\PYG{p}{[}\PYG{n}{i}\PYG{p}{]}\PYG{p}{,}\PYG{n}{t}\PYG{p}{[}\PYG{n}{i}\PYG{p}{]}\PYG{p}{)}
\PYG{c+c1}{\PYGZsh{} Setting up k2}
        \PYG{n}{vv} \PYG{o}{=} \PYG{n}{v}\PYG{p}{[}\PYG{n}{i}\PYG{p}{]}\PYG{o}{+}\PYG{n}{k1v}\PYG{o}{*}\PYG{l+m+mf}{0.5}
        \PYG{n}{xx} \PYG{o}{=} \PYG{n}{x}\PYG{p}{[}\PYG{n}{i}\PYG{p}{]}\PYG{o}{+}\PYG{n}{k1x}\PYG{o}{*}\PYG{l+m+mf}{0.5}
        \PYG{n}{k2x} \PYG{o}{=} \PYG{n}{DeltaT}\PYG{o}{*}\PYG{n}{vv}
        \PYG{n}{k2v} \PYG{o}{=} \PYG{n}{DeltaT}\PYG{o}{*}\PYG{n}{Force}\PYG{p}{(}\PYG{n}{vv}\PYG{p}{,}\PYG{n}{xx}\PYG{p}{,}\PYG{n}{t}\PYG{p}{[}\PYG{n}{i}\PYG{p}{]}\PYG{o}{+}\PYG{n}{DeltaT}\PYG{o}{*}\PYG{l+m+mf}{0.5}\PYG{p}{)}
\PYG{c+c1}{\PYGZsh{} Setting up k3}
        \PYG{n}{vv} \PYG{o}{=} \PYG{n}{v}\PYG{p}{[}\PYG{n}{i}\PYG{p}{]}\PYG{o}{+}\PYG{n}{k2v}\PYG{o}{*}\PYG{l+m+mf}{0.5}
        \PYG{n}{xx} \PYG{o}{=} \PYG{n}{x}\PYG{p}{[}\PYG{n}{i}\PYG{p}{]}\PYG{o}{+}\PYG{n}{k2x}\PYG{o}{*}\PYG{l+m+mf}{0.5}
        \PYG{n}{k3x} \PYG{o}{=} \PYG{n}{DeltaT}\PYG{o}{*}\PYG{n}{vv}
        \PYG{n}{k3v} \PYG{o}{=} \PYG{n}{DeltaT}\PYG{o}{*}\PYG{n}{Force}\PYG{p}{(}\PYG{n}{vv}\PYG{p}{,}\PYG{n}{xx}\PYG{p}{,}\PYG{n}{t}\PYG{p}{[}\PYG{n}{i}\PYG{p}{]}\PYG{o}{+}\PYG{n}{DeltaT}\PYG{o}{*}\PYG{l+m+mf}{0.5}\PYG{p}{)}
\PYG{c+c1}{\PYGZsh{} Setting up k4}
        \PYG{n}{vv} \PYG{o}{=} \PYG{n}{v}\PYG{p}{[}\PYG{n}{i}\PYG{p}{]}\PYG{o}{+}\PYG{n}{k3v}
        \PYG{n}{xx} \PYG{o}{=} \PYG{n}{x}\PYG{p}{[}\PYG{n}{i}\PYG{p}{]}\PYG{o}{+}\PYG{n}{k3x}
        \PYG{n}{k4x} \PYG{o}{=} \PYG{n}{DeltaT}\PYG{o}{*}\PYG{n}{vv}
        \PYG{n}{k4v} \PYG{o}{=} \PYG{n}{DeltaT}\PYG{o}{*}\PYG{n}{Force}\PYG{p}{(}\PYG{n}{vv}\PYG{p}{,}\PYG{n}{xx}\PYG{p}{,}\PYG{n}{t}\PYG{p}{[}\PYG{n}{i}\PYG{p}{]}\PYG{o}{+}\PYG{n}{DeltaT}\PYG{p}{)}
\PYG{c+c1}{\PYGZsh{} Final result}
        \PYG{n}{x}\PYG{p}{[}\PYG{n}{i}\PYG{o}{+}\PYG{l+m+mi}{1}\PYG{p}{]} \PYG{o}{=} \PYG{n}{x}\PYG{p}{[}\PYG{n}{i}\PYG{p}{]}\PYG{o}{+}\PYG{p}{(}\PYG{n}{k1x}\PYG{o}{+}\PYG{l+m+mi}{2}\PYG{o}{*}\PYG{n}{k2x}\PYG{o}{+}\PYG{l+m+mi}{2}\PYG{o}{*}\PYG{n}{k3x}\PYG{o}{+}\PYG{n}{k4x}\PYG{p}{)}\PYG{o}{/}\PYG{l+m+mf}{6.}
        \PYG{n}{v}\PYG{p}{[}\PYG{n}{i}\PYG{o}{+}\PYG{l+m+mi}{1}\PYG{p}{]} \PYG{o}{=} \PYG{n}{v}\PYG{p}{[}\PYG{n}{i}\PYG{p}{]}\PYG{o}{+}\PYG{p}{(}\PYG{n}{k1v}\PYG{o}{+}\PYG{l+m+mi}{2}\PYG{o}{*}\PYG{n}{k2v}\PYG{o}{+}\PYG{l+m+mi}{2}\PYG{o}{*}\PYG{n}{k3v}\PYG{o}{+}\PYG{n}{k4v}\PYG{p}{)}\PYG{o}{/}\PYG{l+m+mf}{6.}
        \PYG{n}{t}\PYG{p}{[}\PYG{n}{i}\PYG{o}{+}\PYG{l+m+mi}{1}\PYG{p}{]} \PYG{o}{=} \PYG{n}{t}\PYG{p}{[}\PYG{n}{i}\PYG{p}{]} \PYG{o}{+} \PYG{n}{DeltaT}
\end{sphinxVerbatim}

The Runge\sphinxhyphen{}Kutta family of methods are particularly useful when we have a time\sphinxhyphen{}dependent acceleration.
If we have forces which depend only the spatial degrees of freedom (no velocity and/or time\sphinxhyphen{}dependence), then energy conserving methods like the Velocity Verlet or the Euler\sphinxhyphen{}Cromer method are preferred. As soon as we introduce an explicit time\sphinxhyphen{}dependence and/or add dissipitave forces like friction or air resistance, then methods like the family of Runge\sphinxhyphen{}Kutta methods are well suited for this.
The code below uses the Runge\sphinxhyphen{}Kutta4 methods.

\begin{sphinxVerbatim}[commandchars=\\\{\}]
\PYG{n}{DeltaT} \PYG{o}{=} \PYG{l+m+mf}{0.001}
\PYG{c+c1}{\PYGZsh{}set up arrays }
\PYG{n}{tfinal} \PYG{o}{=} \PYG{l+m+mi}{20} \PYG{c+c1}{\PYGZsh{} in years}
\PYG{n}{n} \PYG{o}{=} \PYG{n}{ceil}\PYG{p}{(}\PYG{n}{tfinal}\PYG{o}{/}\PYG{n}{DeltaT}\PYG{p}{)}
\PYG{c+c1}{\PYGZsh{} set up arrays for t, v, and x}
\PYG{n}{t} \PYG{o}{=} \PYG{n}{np}\PYG{o}{.}\PYG{n}{zeros}\PYG{p}{(}\PYG{n}{n}\PYG{p}{)}
\PYG{n}{v} \PYG{o}{=} \PYG{n}{np}\PYG{o}{.}\PYG{n}{zeros}\PYG{p}{(}\PYG{n}{n}\PYG{p}{)}
\PYG{n}{x} \PYG{o}{=} \PYG{n}{np}\PYG{o}{.}\PYG{n}{zeros}\PYG{p}{(}\PYG{n}{n}\PYG{p}{)}
\PYG{c+c1}{\PYGZsh{} Initial conditions (can change to more than one dim)}
\PYG{n}{x0} \PYG{o}{=}  \PYG{l+m+mf}{1.0} 
\PYG{n}{v0} \PYG{o}{=} \PYG{l+m+mf}{0.0}
\PYG{n}{x}\PYG{p}{[}\PYG{l+m+mi}{0}\PYG{p}{]} \PYG{o}{=} \PYG{n}{x0}
\PYG{n}{v}\PYG{p}{[}\PYG{l+m+mi}{0}\PYG{p}{]} \PYG{o}{=} \PYG{n}{v0}
\PYG{n}{gamma} \PYG{o}{=} \PYG{l+m+mf}{0.2}
\PYG{n}{Omegatilde} \PYG{o}{=} \PYG{l+m+mf}{0.5}
\PYG{n}{Ftilde} \PYG{o}{=} \PYG{l+m+mf}{1.0}
\PYG{c+c1}{\PYGZsh{} Start integrating using Euler\PYGZsq{}s method}
\PYG{c+c1}{\PYGZsh{} Note that we define the force function as a SpringForce}
\PYG{n}{RK4}\PYG{p}{(}\PYG{n}{v}\PYG{p}{,}\PYG{n}{x}\PYG{p}{,}\PYG{n}{t}\PYG{p}{,}\PYG{n}{n}\PYG{p}{,}\PYG{n}{SpringForce}\PYG{p}{)}

\PYG{c+c1}{\PYGZsh{} Plot position as function of time    }
\PYG{n}{fig}\PYG{p}{,} \PYG{n}{ax} \PYG{o}{=} \PYG{n}{plt}\PYG{o}{.}\PYG{n}{subplots}\PYG{p}{(}\PYG{p}{)}
\PYG{n}{ax}\PYG{o}{.}\PYG{n}{set\PYGZus{}ylabel}\PYG{p}{(}\PYG{l+s+s1}{\PYGZsq{}}\PYG{l+s+s1}{x[m]}\PYG{l+s+s1}{\PYGZsq{}}\PYG{p}{)}
\PYG{n}{ax}\PYG{o}{.}\PYG{n}{set\PYGZus{}xlabel}\PYG{p}{(}\PYG{l+s+s1}{\PYGZsq{}}\PYG{l+s+s1}{t[s]}\PYG{l+s+s1}{\PYGZsq{}}\PYG{p}{)}
\PYG{n}{ax}\PYG{o}{.}\PYG{n}{plot}\PYG{p}{(}\PYG{n}{t}\PYG{p}{,} \PYG{n}{x}\PYG{p}{)}
\PYG{n}{fig}\PYG{o}{.}\PYG{n}{tight\PYGZus{}layout}\PYG{p}{(}\PYG{p}{)}
\PYG{n}{save\PYGZus{}fig}\PYG{p}{(}\PYG{l+s+s2}{\PYGZdq{}}\PYG{l+s+s2}{ForcedBlockRK4}\PYG{l+s+s2}{\PYGZdq{}}\PYG{p}{)}
\PYG{n}{plt}\PYG{o}{.}\PYG{n}{show}\PYG{p}{(}\PYG{p}{)}
\end{sphinxVerbatim}


\section{Principle of Superposition and Periodic Forces (Fourier Transforms)}
\label{\detokenize{chapter1:principle-of-superposition-and-periodic-forces-fourier-transforms}}
If one has several driving forces, \(F(t)=\sum_n F_n(t)\), one can find
the particular solution to each \(F_n\), \(x_{pn}(t)\), and the particular
solution for the entire driving force is




\begin{equation*}
\begin{split}
\begin{equation}
x_p(t)=\sum_nx_{pn}(t).
\label{_auto61} \tag{80}
\end{equation}
\end{split}
\end{equation*}
This is known as the principal of superposition. It only applies when
the homogenous equation is linear. If there were an anharmonic term
such as \(x^3\) in the homogenous equation, then when one summed various
solutions, \(x=(\sum_n x_n)^2\), one would get cross
terms. Superposition is especially useful when \(F(t)\) can be written
as a sum of sinusoidal terms, because the solutions for each
sinusoidal (sine or cosine)  term is analytic, as we saw above.

Driving forces are often periodic, even when they are not
sinusoidal. Periodicity implies that for some time \(\tau\)
\begin{equation*}
\begin{split}
\begin{eqnarray}
F(t+\tau)=F(t). 
\end{eqnarray}
\end{split}
\end{equation*}
One example of a non\sphinxhyphen{}sinusoidal periodic force is a square wave. Many
components in electric circuits are non\sphinxhyphen{}linear, e.g. diodes, which
makes many wave forms non\sphinxhyphen{}sinusoidal even when the circuits are being
driven by purely sinusoidal sources.

The code here shows a typical example of such a square wave generated using the functionality included in the \sphinxstylestrong{scipy} Python package. We have used a period of \(\tau=0.2\).

\begin{sphinxVerbatim}[commandchars=\\\{\}]
\PYG{k+kn}{import} \PYG{n+nn}{numpy} \PYG{k}{as} \PYG{n+nn}{np}
\PYG{k+kn}{import} \PYG{n+nn}{math}
\PYG{k+kn}{from} \PYG{n+nn}{scipy} \PYG{k+kn}{import} \PYG{n}{signal}
\PYG{k+kn}{import} \PYG{n+nn}{matplotlib}\PYG{n+nn}{.}\PYG{n+nn}{pyplot} \PYG{k}{as} \PYG{n+nn}{plt}

\PYG{c+c1}{\PYGZsh{} number of points                                                                                       }
\PYG{n}{n} \PYG{o}{=} \PYG{l+m+mi}{500}
\PYG{c+c1}{\PYGZsh{} start and final times                                                                                  }
\PYG{n}{t0} \PYG{o}{=} \PYG{l+m+mf}{0.0}
\PYG{n}{tn} \PYG{o}{=} \PYG{l+m+mf}{1.0}
\PYG{c+c1}{\PYGZsh{} Period                                                                                                 }
\PYG{n}{t} \PYG{o}{=} \PYG{n}{np}\PYG{o}{.}\PYG{n}{linspace}\PYG{p}{(}\PYG{n}{t0}\PYG{p}{,} \PYG{n}{tn}\PYG{p}{,} \PYG{n}{n}\PYG{p}{,} \PYG{n}{endpoint}\PYG{o}{=}\PYG{k+kc}{False}\PYG{p}{)}
\PYG{n}{SqrSignal} \PYG{o}{=} \PYG{n}{np}\PYG{o}{.}\PYG{n}{zeros}\PYG{p}{(}\PYG{n}{n}\PYG{p}{)}
\PYG{n}{SqrSignal} \PYG{o}{=} \PYG{l+m+mf}{1.0}\PYG{o}{+}\PYG{n}{signal}\PYG{o}{.}\PYG{n}{square}\PYG{p}{(}\PYG{l+m+mi}{2}\PYG{o}{*}\PYG{n}{np}\PYG{o}{.}\PYG{n}{pi}\PYG{o}{*}\PYG{l+m+mi}{5}\PYG{o}{*}\PYG{n}{t}\PYG{p}{)}
\PYG{n}{plt}\PYG{o}{.}\PYG{n}{plot}\PYG{p}{(}\PYG{n}{t}\PYG{p}{,} \PYG{n}{SqrSignal}\PYG{p}{)}
\PYG{n}{plt}\PYG{o}{.}\PYG{n}{ylim}\PYG{p}{(}\PYG{o}{\PYGZhy{}}\PYG{l+m+mf}{0.5}\PYG{p}{,} \PYG{l+m+mf}{2.5}\PYG{p}{)}
\PYG{n}{plt}\PYG{o}{.}\PYG{n}{show}\PYG{p}{(}\PYG{p}{)}
\end{sphinxVerbatim}

For the sinusoidal example studied in the previous subsections the
period is \(\tau=2\pi/\omega\). However, higher harmonics can also
satisfy the periodicity requirement. In general, any force that
satisfies the periodicity requirement can be expressed as a sum over
harmonics,




\begin{equation*}
\begin{split}
\begin{equation}
F(t)=\frac{f_0}{2}+\sum_{n>0} f_n\cos(2n\pi t/\tau)+g_n\sin(2n\pi t/\tau).
\label{_auto62} \tag{81}
\end{equation}
\end{split}
\end{equation*}
From the previous subsection, one can write down the answer for
\(x_{pn}(t)\), by substituting \(f_n/m\) or \(g_n/m\) for \(F_0/m\) into Eq.s
({\hyperref[\detokenize{chapter1:eq:fastdriven1}]{\emph{66}}}) or ({\hyperref[\detokenize{chapter1:eq:fastdriven2}]{\emph{67}}}) respectively. By
writing each factor \(2n\pi t/\tau\) as \(n\omega t\), with \(\omega\equiv
2\pi/\tau\),




\begin{equation*}
\begin{split}
\begin{equation}
\label{eq:fourierdef1} \tag{82}
F(t)=\frac{f_0}{2}+\sum_{n>0}f_n\cos(n\omega t)+g_n\sin(n\omega t).
\end{equation}
\end{split}
\end{equation*}
The solutions for \(x(t)\) then come from replacing \(\omega\) with
\(n\omega\) for each term in the particular solution in Equations
({\hyperref[\detokenize{chapter1:eq:partform}]{\emph{59}}}) and ({\hyperref[\detokenize{chapter1:eq:Ddrive}]{\emph{62}}}),
\begin{equation*}
\begin{split}
\begin{eqnarray}
x_p(t)&=&\frac{f_0}{2k}+\sum_{n>0} \alpha_n\cos(n\omega t-\delta_n)+\beta_n\sin(n\omega t-\delta_n),\\
\nonumber
\alpha_n&=&\frac{f_n/m}{\sqrt{((n\omega)^2-\omega_0^2)+4\beta^2n^2\omega^2}},\\
\nonumber
\beta_n&=&\frac{g_n/m}{\sqrt{((n\omega)^2-\omega_0^2)+4\beta^2n^2\omega^2}},\\
\nonumber
\delta_n&=&\tan^{-1}\left(\frac{2\beta n\omega}{\omega_0^2-n^2\omega^2}\right).
\end{eqnarray}
\end{split}
\end{equation*}
Because the forces have been applied for a long time, any non\sphinxhyphen{}zero
damping eliminates the homogenous parts of the solution, so one need
only consider the particular solution for each \(n\).

The problem will considered solved if one can find expressions for the
coefficients \(f_n\) and \(g_n\), even though the solutions are expressed
as an infinite sum. The coefficients can be extracted from the
function \(F(t)\) by




\begin{equation*}
\begin{split}
\begin{eqnarray}
\label{eq:fourierdef2} \tag{83}
f_n&=&\frac{2}{\tau}\int_{-\tau/2}^{\tau/2} dt~F(t)\cos(2n\pi t/\tau),\\
\nonumber
g_n&=&\frac{2}{\tau}\int_{-\tau/2}^{\tau/2} dt~F(t)\sin(2n\pi t/\tau).
\end{eqnarray}
\end{split}
\end{equation*}
To check the consistency of these expressions and to verify
Eq. ({\hyperref[\detokenize{chapter1:eq:fourierdef2}]{\emph{83}}}), one can insert the expansion of \(F(t)\) in
Eq. ({\hyperref[\detokenize{chapter1:eq:fourierdef1}]{\emph{82}}}) into the expression for the coefficients in
Eq. ({\hyperref[\detokenize{chapter1:eq:fourierdef2}]{\emph{83}}}) and see whether
\begin{equation*}
\begin{split}
\begin{eqnarray}
f_n&=?&\frac{2}{\tau}\int_{-\tau/2}^{\tau/2} dt~\left\{
\frac{f_0}{2}+\sum_{m>0}f_m\cos(m\omega t)+g_m\sin(m\omega t)
\right\}\cos(n\omega t).
\end{eqnarray}
\end{split}
\end{equation*}
Immediately, one can throw away all the terms with \(g_m\) because they
convolute an even and an odd function. The term with \(f_0/2\)
disappears because \(\cos(n\omega t)\) is equally positive and negative
over the interval and will integrate to zero. For all the terms
\(f_m\cos(m\omega t)\) appearing in the sum, one can use angle addition
formulas to see that \(\cos(m\omega t)\cos(n\omega
t)=(1/2)(\cos[(m+n)\omega t]+\cos[(m-n)\omega t]\). This will integrate
to zero unless \(m=n\). In that case the \(m=n\) term gives




\begin{equation*}
\begin{split}
\begin{equation}
\int_{-\tau/2}^{\tau/2}dt~\cos^2(m\omega t)=\frac{\tau}{2},
\label{_auto63} \tag{84}
\end{equation}
\end{split}
\end{equation*}
and
\begin{equation*}
\begin{split}
\begin{eqnarray}
f_n&=?&\frac{2}{\tau}\int_{-\tau/2}^{\tau/2} dt~f_n/2\\
\nonumber
&=&f_n~\checkmark.
\end{eqnarray}
\end{split}
\end{equation*}
The same method can be used to check for the consistency of \(g_n\).

Consider the driving force:




\begin{equation*}
\begin{split}
\begin{equation}
F(t)=At/\tau,~~-\tau/2<t<\tau/2,~~~F(t+\tau)=F(t).
\label{_auto64} \tag{85}
\end{equation}
\end{split}
\end{equation*}
Find the Fourier coefficients \(f_n\) and \(g_n\) for all \(n\) using Eq. ({\hyperref[\detokenize{chapter1:eq:fourierdef2}]{\emph{83}}}).

Only the odd coefficients enter by symmetry, i.e. \(f_n=0\). One can find \(g_n\) integrating by parts,




\begin{equation*}
\begin{split}
\begin{eqnarray}
\label{eq:fouriersolution} \tag{86}
g_n&=&\frac{2}{\tau}\int_{-\tau/2}^{\tau/2}dt~\sin(n\omega t) \frac{At}{\tau}\\
\nonumber
u&=&t,~dv=\sin(n\omega t)dt,~v=-\cos(n\omega t)/(n\omega),\\
\nonumber
g_n&=&\frac{-2A}{n\omega \tau^2}\int_{-\tau/2}^{\tau/2}dt~\cos(n\omega t)
+\left.2A\frac{-t\cos(n\omega t)}{n\omega\tau^2}\right|_{-\tau/2}^{\tau/2}.
\end{eqnarray}
\end{split}
\end{equation*}
The first term is zero because \(\cos(n\omega t)\) will be equally
positive and negative over the interval. Using the fact that
\(\omega\tau=2\pi\),
\begin{equation*}
\begin{split}
\begin{eqnarray}
g_n&=&-\frac{2A}{2n\pi}\cos(n\omega\tau/2)\\
\nonumber
&=&-\frac{A}{n\pi}\cos(n\pi)\\
\nonumber
&=&\frac{A}{n\pi}(-1)^{n+1}.
\end{eqnarray}
\end{split}
\end{equation*}

\section{Fourier Series}
\label{\detokenize{chapter1:fourier-series}}
More text will come here, chpater 5.7\sphinxhyphen{}5.8 of Taylor are discussed
during the lectures. The code here uses the Fourier series discussed
in chapter 5.7 for a square wave signal. The equations for the
coefficients are are discussed in Taylor section 5.7, see Example
5.4. The code here visualizes the various approximations given by
Fourier series compared with a square wave with period \(T=0.2\), witth
\(0.1\) and max value \(F=2\). We see that when we increase the number of
components in the Fourier series, the Fourier series approximation gets closes and closes to the square wave signal.

\begin{sphinxVerbatim}[commandchars=\\\{\}]
\PYG{k+kn}{import} \PYG{n+nn}{numpy} \PYG{k}{as} \PYG{n+nn}{np}
\PYG{k+kn}{import} \PYG{n+nn}{math}
\PYG{k+kn}{from} \PYG{n+nn}{scipy} \PYG{k+kn}{import} \PYG{n}{signal}
\PYG{k+kn}{import} \PYG{n+nn}{matplotlib}\PYG{n+nn}{.}\PYG{n+nn}{pyplot} \PYG{k}{as} \PYG{n+nn}{plt}

\PYG{c+c1}{\PYGZsh{} number of points                                                                                       }
\PYG{n}{n} \PYG{o}{=} \PYG{l+m+mi}{500}
\PYG{c+c1}{\PYGZsh{} start and final times                                                                                  }
\PYG{n}{t0} \PYG{o}{=} \PYG{l+m+mf}{0.0}
\PYG{n}{tn} \PYG{o}{=} \PYG{l+m+mf}{1.0}
\PYG{c+c1}{\PYGZsh{} Period                                                                                                 }
\PYG{n}{T} \PYG{o}{=}\PYG{l+m+mf}{0.2}
\PYG{c+c1}{\PYGZsh{} Max value of square signal                                                                             }
\PYG{n}{Fmax}\PYG{o}{=} \PYG{l+m+mf}{2.0}
\PYG{c+c1}{\PYGZsh{} Width of signal                                                                                        }
\PYG{n}{Width} \PYG{o}{=} \PYG{l+m+mf}{0.1}
\PYG{n}{t} \PYG{o}{=} \PYG{n}{np}\PYG{o}{.}\PYG{n}{linspace}\PYG{p}{(}\PYG{n}{t0}\PYG{p}{,} \PYG{n}{tn}\PYG{p}{,} \PYG{n}{n}\PYG{p}{,} \PYG{n}{endpoint}\PYG{o}{=}\PYG{k+kc}{False}\PYG{p}{)}
\PYG{n}{SqrSignal} \PYG{o}{=} \PYG{n}{np}\PYG{o}{.}\PYG{n}{zeros}\PYG{p}{(}\PYG{n}{n}\PYG{p}{)}
\PYG{n}{FourierSeriesSignal} \PYG{o}{=} \PYG{n}{np}\PYG{o}{.}\PYG{n}{zeros}\PYG{p}{(}\PYG{n}{n}\PYG{p}{)}
\PYG{n}{SqrSignal} \PYG{o}{=} \PYG{l+m+mf}{1.0}\PYG{o}{+}\PYG{n}{signal}\PYG{o}{.}\PYG{n}{square}\PYG{p}{(}\PYG{l+m+mi}{2}\PYG{o}{*}\PYG{n}{np}\PYG{o}{.}\PYG{n}{pi}\PYG{o}{*}\PYG{l+m+mi}{5}\PYG{o}{*}\PYG{n}{t}\PYG{o}{+}\PYG{n}{np}\PYG{o}{.}\PYG{n}{pi}\PYG{o}{*}\PYG{n}{Width}\PYG{o}{/}\PYG{n}{T}\PYG{p}{)}
\PYG{n}{a0} \PYG{o}{=} \PYG{n}{Fmax}\PYG{o}{*}\PYG{n}{Width}\PYG{o}{/}\PYG{n}{T}
\PYG{n}{FourierSeriesSignal} \PYG{o}{=} \PYG{n}{a0}
\PYG{n}{Factor} \PYG{o}{=} \PYG{l+m+mf}{2.0}\PYG{o}{*}\PYG{n}{Fmax}\PYG{o}{/}\PYG{n}{np}\PYG{o}{.}\PYG{n}{pi}
\PYG{k}{for} \PYG{n}{i} \PYG{o+ow}{in} \PYG{n+nb}{range}\PYG{p}{(}\PYG{l+m+mi}{1}\PYG{p}{,}\PYG{l+m+mi}{500}\PYG{p}{)}\PYG{p}{:}
    \PYG{n}{FourierSeriesSignal} \PYG{o}{+}\PYG{o}{=} \PYG{n}{Factor}\PYG{o}{/}\PYG{p}{(}\PYG{n}{i}\PYG{p}{)}\PYG{o}{*}\PYG{n}{np}\PYG{o}{.}\PYG{n}{sin}\PYG{p}{(}\PYG{n}{np}\PYG{o}{.}\PYG{n}{pi}\PYG{o}{*}\PYG{n}{i}\PYG{o}{*}\PYG{n}{Width}\PYG{o}{/}\PYG{n}{T}\PYG{p}{)}\PYG{o}{*}\PYG{n}{np}\PYG{o}{.}\PYG{n}{cos}\PYG{p}{(}\PYG{n}{i}\PYG{o}{*}\PYG{n}{t}\PYG{o}{*}\PYG{l+m+mi}{2}\PYG{o}{*}\PYG{n}{np}\PYG{o}{.}\PYG{n}{pi}\PYG{o}{/}\PYG{n}{T}\PYG{p}{)}
\PYG{n}{plt}\PYG{o}{.}\PYG{n}{plot}\PYG{p}{(}\PYG{n}{t}\PYG{p}{,} \PYG{n}{SqrSignal}\PYG{p}{)}
\PYG{n}{plt}\PYG{o}{.}\PYG{n}{plot}\PYG{p}{(}\PYG{n}{t}\PYG{p}{,} \PYG{n}{FourierSeriesSignal}\PYG{p}{)}
\PYG{n}{plt}\PYG{o}{.}\PYG{n}{ylim}\PYG{p}{(}\PYG{o}{\PYGZhy{}}\PYG{l+m+mf}{0.5}\PYG{p}{,} \PYG{l+m+mf}{2.5}\PYG{p}{)}
\PYG{n}{plt}\PYG{o}{.}\PYG{n}{show}\PYG{p}{(}\PYG{p}{)}
\end{sphinxVerbatim}


\section{Solving differential equations with Fouries series}
\label{\detokenize{chapter1:solving-differential-equations-with-fouries-series}}
The material here was discussed during the lecture of February 19 and 21.
It is also covered by Taylor in section 5.8.

Consider a particle at rest in the bottom of an underdamped harmonic
oscillator, that then feels a sudden impulse, or change in momentum,
\(I=F\Delta t\) at \(t=0\). This increases the velocity immediately by an
amount \(v_0=I/m\) while not changing the position. One can then solve
the trajectory by solving Eq. ({\hyperref[\detokenize{chapter1:eq:homogsolution}]{\emph{55}}}) with initial
conditions \(v_0=I/m\) and \(x_0=0\). This gives




\begin{equation*}
\begin{split}
\begin{equation}
x(t)=\frac{I}{m\omega'}e^{-\beta t}\sin\omega't, ~~t>0.
\label{_auto65} \tag{87}
\end{equation}
\end{split}
\end{equation*}
Here, \(\omega'=\sqrt{\omega_0^2-\beta^2}\). For an impulse \(I_i\) that
occurs at time \(t_i\) the trajectory would be




\begin{equation*}
\begin{split}
\begin{equation}
x(t)=\frac{I_i}{m\omega'}e^{-\beta (t-t_i)}\sin[\omega'(t-t_i)] \Theta(t-t_i),
\label{_auto66} \tag{88}
\end{equation}
\end{split}
\end{equation*}
where \(\Theta(t-t_i)\) is a step function, i.e. \(\Theta(x)\) is zero for
\(x<0\) and unity for \(x>0\). If there were several impulses linear
superposition tells us that we can sum over each contribution,




\begin{equation*}
\begin{split}
\begin{equation}
x(t)=\sum_i\frac{I_i}{m\omega'}e^{-\beta(t-t_i)}\sin[\omega'(t-t_i)]\Theta(t-t_i)
\label{_auto67} \tag{89}
\end{equation}
\end{split}
\end{equation*}
Now one can consider a series of impulses at times separated by
\(\Delta t\), where each impulse is given by \(F_i\Delta t\). The sum
above now becomes an integral,




\begin{equation*}
\begin{split}
\begin{eqnarray}\label{eq:Greeny} \tag{90}
x(t)&=&\int_{-\infty}^\infty dt'~F(t')\frac{e^{-\beta(t-t')}\sin[\omega'(t-t')]}{m\omega'}\Theta(t-t')\\
\nonumber
&=&\int_{-\infty}^\infty dt'~F(t')G(t-t'),\\
\nonumber
G(\Delta t)&=&\frac{e^{-\beta\Delta t}\sin[\omega' \Delta t]}{m\omega'}\Theta(\Delta t)
\end{eqnarray}
\end{split}
\end{equation*}
The quantity
\(e^{-\beta(t-t')}\sin[\omega'(t-t')]/m\omega'\Theta(t-t')\) is called a
Green’s function, \(G(t-t')\). It describes the response at \(t\) due to a
force applied at a time \(t'\), and is a function of \(t-t'\). The step
function ensures that the response does not occur before the force is
applied. One should remember that the form for \(G\) would change if the
oscillator were either critically\sphinxhyphen{} or over\sphinxhyphen{}damped.

When performing the integral in Eq. ({\hyperref[\detokenize{chapter1:eq:Greeny}]{\emph{90}}}) one can use
angle addition formulas to factor out the part with the \(t'\)
dependence in the integrand,




\begin{equation*}
\begin{split}
\begin{eqnarray}
\label{eq:Greeny2} \tag{91}
x(t)&=&\frac{1}{m\omega'}e^{-\beta t}\left[I_c(t)\sin(\omega't)-I_s(t)\cos(\omega't)\right],\\
\nonumber
I_c(t)&\equiv&\int_{-\infty}^t dt'~F(t')e^{\beta t'}\cos(\omega't'),\\
\nonumber
I_s(t)&\equiv&\int_{-\infty}^t dt'~F(t')e^{\beta t'}\sin(\omega't').
\end{eqnarray}
\end{split}
\end{equation*}
If the time \(t\) is beyond any time at which the force acts,
\(F(t'>t)=0\), the coefficients \(I_c\) and \(I_s\) become independent of
\(t\).

Consider an undamped oscillator (\(\beta\rightarrow 0\)), with
characteristic frequency \(\omega_0\) and mass \(m\), that is at rest
until it feels a force described by a Gaussian form,
\begin{equation*}
\begin{split}
\begin{eqnarray*}
F(t)&=&F_0 \exp\left\{\frac{-t^2}{2\tau^2}\right\}.
\end{eqnarray*}
\end{split}
\end{equation*}
For large times (\(t>>\tau\)), where the force has died off, find
\(x(t)\).\textbackslash{} Solve for the coefficients \(I_c\) and \(I_s\) in
Eq. ({\hyperref[\detokenize{chapter1:eq:Greeny2}]{\emph{91}}}). Because the Gaussian is an even function,
\(I_s=0\), and one need only solve for \(I_c\),
\begin{equation*}
\begin{split}
\begin{eqnarray*}
I_c&=&F_0\int_{-\infty}^\infty dt'~e^{-t^{\prime 2}/(2\tau^2)}\cos(\omega_0 t')\\
&=&\Re F_0 \int_{-\infty}^\infty dt'~e^{-t^{\prime 2}/(2\tau^2)}e^{i\omega_0 t'}\\
&=&\Re F_0 \int_{-\infty}^\infty dt'~e^{-(t'-i\omega_0\tau^2)^2/(2\tau^2)}e^{-\omega_0^2\tau^2/2}\\
&=&F_0\tau \sqrt{2\pi} e^{-\omega_0^2\tau^2/2}.
\end{eqnarray*}
\end{split}
\end{equation*}
The third step involved completing the square, and the final step used the fact that the integral
\begin{equation*}
\begin{split}
\begin{eqnarray*}
\int_{-\infty}^\infty dx~e^{-x^2/2}&=&\sqrt{2\pi}.
\end{eqnarray*}
\end{split}
\end{equation*}
To see that this integral is true, consider the square of the integral, which you can change to polar coordinates,
\begin{equation*}
\begin{split}
\begin{eqnarray*}
I&=&\int_{-\infty}^\infty dx~e^{-x^2/2}\\
I^2&=&\int_{-\infty}^\infty dxdy~e^{-(x^2+y^2)/2}\\
&=&2\pi\int_0^\infty rdr~e^{-r^2/2}\\
&=&2\pi.
\end{eqnarray*}
\end{split}
\end{equation*}
Finally, the expression for \(x\) from Eq. ({\hyperref[\detokenize{chapter1:eq:Greeny2}]{\emph{91}}}) is
\begin{equation*}
\begin{split}
\begin{eqnarray*}
x(t>>\tau)&=&\frac{F_0\tau}{m\omega_0} \sqrt{2\pi} e^{-\omega_0^2\tau^2/2}\sin(\omega_0t).
\end{eqnarray*}
\end{split}
\end{equation*}

\section{The classical pendulum and scaling the equations}
\label{\detokenize{chapter1:id3}}
Let us end our discussion of oscillations with another classical case, the pendulum.

The angular equation of motion of the pendulum is given by
Newton’s equation and with no external force it reads




\begin{equation*}
\begin{split}
\begin{equation}
  ml\frac{d^2\theta}{dt^2}+mgsin(\theta)=0,
\label{_auto68} \tag{92}
\end{equation}
\end{split}
\end{equation*}
with an angular velocity and acceleration given by




\begin{equation*}
\begin{split}
\begin{equation}
     v=l\frac{d\theta}{dt},
\label{_auto69} \tag{93}
\end{equation}
\end{split}
\end{equation*}
and




\begin{equation*}
\begin{split}
\begin{equation}
     a=l\frac{d^2\theta}{dt^2}.
\label{_auto70} \tag{94}
\end{equation}
\end{split}
\end{equation*}
We do however expect that the motion will gradually come to an end due a viscous drag torque acting on the pendulum.
In the presence of the drag, the above equation becomes




\begin{equation*}
\begin{split}
\begin{equation}
   ml\frac{d^2\theta}{dt^2}+\nu\frac{d\theta}{dt}  +mgsin(\theta)=0, 
\label{_auto71} \tag{95}
\end{equation}
\end{split}
\end{equation*}
where \(\nu\) is now a positive constant parameterizing the viscosity
of the medium in question. In order to maintain the motion against
viscosity, it is necessary to add some external driving force.
We choose here a periodic driving force. The last equation becomes then




\begin{equation*}
\begin{split}
\begin{equation}
   ml\frac{d^2\theta}{dt^2}+\nu\frac{d\theta}{dt}  +mgsin(\theta)=Asin(\omega t), 
\label{_auto72} \tag{96}
\end{equation}
\end{split}
\end{equation*}
with \(A\) and \(\omega\) two constants representing the amplitude and
the angular frequency respectively. The latter is called the driving frequency.

We define
\begin{equation*}
\begin{split}
\omega_0=\sqrt{g/l},
\end{split}
\end{equation*}
the so\sphinxhyphen{}called natural frequency and the new dimensionless quantities
\begin{equation*}
\begin{split}
\hat{t}=\omega_0t,
\end{split}
\end{equation*}
with the dimensionless driving frequency
\begin{equation*}
\begin{split}
\hat{\omega}=\frac{\omega}{\omega_0},
\end{split}
\end{equation*}
and introducing the quantity \(Q\), called the \sphinxstyleemphasis{quality factor},
\begin{equation*}
\begin{split}
Q=\frac{mg}{\omega_0\nu},
\end{split}
\end{equation*}
and the dimensionless amplitude
\begin{equation*}
\begin{split}
\hat{A}=\frac{A}{mg}
\end{split}
\end{equation*}
We have
\begin{equation*}
\begin{split}
\frac{d^2\theta}{d\hat{t}^2}+\frac{1}{Q}\frac{d\theta}{d\hat{t}}  
     +sin(\theta)=\hat{A}cos(\hat{\omega}\hat{t}).
\end{split}
\end{equation*}
This equation can in turn be recast in terms of two coupled first\sphinxhyphen{}order differential equations as follows
\begin{equation*}
\begin{split}
\frac{d\theta}{d\hat{t}}=\hat{v},
\end{split}
\end{equation*}
and
\begin{equation*}
\begin{split}
\frac{d\hat{v}}{d\hat{t}}=-\frac{\hat{v}}{Q}-sin(\theta)+\hat{A}cos(\hat{\omega}\hat{t}).
\end{split}
\end{equation*}
These are the equations to be solved.  The factor \(Q\) represents the
number of oscillations of the undriven system that must occur before
its energy is significantly reduced due to the viscous drag. The
amplitude \(\hat{A}\) is measured in units of the maximum possible
gravitational torque while \(\hat{\omega}\) is the angular frequency of
the external torque measured in units of the pendulum’s natural
frequency.


\chapter{Gravity and Central Forces}
\label{\detokenize{chapter1:gravity-and-central-forces}}
The gravitational potential energy and forces involving two masses \(a\) and \(b\) are
\begin{equation*}
\begin{split}
\begin{eqnarray}
U_{ab}&=&-\frac{Gm_am_b}{|\boldsymbol{r}_a-\boldsymbol{r}_b|},\\
\nonumber
F_{ba}&=&-\frac{Gm_am_b}{|\boldsymbol{r}_a-\boldsymbol{r}_b|^2}\hat{r}_{ab},\\
\nonumber
\hat{r}_{ab}&=&\frac{\boldsymbol{r}_b-\boldsymbol{r}_a}{|\boldsymbol{r}_a-\boldsymbol{r}_b|}.
\end{eqnarray}
\end{split}
\end{equation*}
Here \(G=6.67\times 10^{-11}\) Nm\(^2\)/kg\(^2\), and \(F_{ba}\) is the force
on \(b\) due to \(a\). By inspection, one can see that the force on \(b\)
due to \(a\) and the force on \(a\) due to \(b\) are equal and opposite. The
net potential energy for a large number of masses would be




\begin{equation*}
\begin{split}
\begin{equation}
U=\sum_{a<b}U_{ab}=\frac{1}{2}\sum_{a\ne b}U_{ab}.
\label{_auto73} \tag{97}
\end{equation}
\end{split}
\end{equation*}

\section{Relative and Center of Mass Motion}
\label{\detokenize{chapter1:relative-and-center-of-mass-motion}}
Thus far, we have considered the trajectory as if the force is
centered around a fixed point. For two bodies interacting only with
one another, both masses circulate around the center of mass. One
might think that solutions would become more complex when both
particles move, but we will see here that the problem can be reduced
to one with a single body moving according to a fixed force by
expressing the trajectories for \(\boldsymbol{r}_1\) and \(\boldsymbol{r}_2\) into the
center\sphinxhyphen{}of\sphinxhyphen{}mass coordinate \(\boldsymbol{R}_{\rm cm}\) and the relative
coordinate \(\boldsymbol{r}\),
\begin{equation*}
\begin{split}
\begin{eqnarray}
\boldsymbol{R}_{\rm cm}&\equiv&\frac{m_1\boldsymbol{r}_1+m_2\boldsymbol{r}_2}{m_1+m_2},\\
\nonumber
\boldsymbol{r}&\equiv&\boldsymbol{r}_1-\boldsymbol{r_2}.
\end{eqnarray}
\end{split}
\end{equation*}
Here, we assume the two particles interact only with one another, so
\(\boldsymbol{F}_{12}=-\boldsymbol{F}_{21}\) (where \(\boldsymbol{F}_{ij}\) is the force on \(i\)
due to \(j\). The equations of motion then become
\begin{equation*}
\begin{split}
\begin{eqnarray}
\ddot{\boldsymbol{R}}_{\rm cm}&=&\frac{1}{m_1+m_2}\left\{m_1\ddot{\boldsymbol{r}}_1+m_2\ddot{\boldsymbol{r}}_2\right\}\\
\nonumber
&=&\frac{1}{m_1+m_2}\left\{\boldsymbol{F}_{12}+\boldsymbol{F}_{21}\right\}=0.\\
\ddot{\boldsymbol{r}}&=&\ddot{\boldsymbol{r}}_1-\ddot{\boldsymbol{r}}_2=\left(\frac{\boldsymbol{F}_{12}}{m_1}-\frac{\boldsymbol{F}_{21}}{m_2}\right)\\
\nonumber
&=&\left(\frac{1}{m_1}+\frac{1}{m_2}\right)\boldsymbol{F}_{12}.
\end{eqnarray}
\end{split}
\end{equation*}
The first expression simply states that the center of mass coordinate
\(\boldsymbol{R}_{\rm cm}\) moves at a fixed velocity. The second expression
can be rewritten in terms of the reduced mass \(\mu\).
\begin{equation*}
\begin{split}
\begin{eqnarray}
\mu \ddot{\boldsymbol{r}}&=&\boldsymbol{F}_{12},\\
\frac{1}{\mu}&=&\frac{1}{m_1}+\frac{1}{m_2},~~~~\mu=\frac{m_1m_2}{m_1+m_2}.
\end{eqnarray}
\end{split}
\end{equation*}
Thus, one can treat the trajectory as a one\sphinxhyphen{}body problem where the
reduced mass is \(\mu\), and a second trivial problem for the center of
mass. The reduced mass is especially convenient when one is
considering gravitational problems because then
\begin{equation*}
\begin{split}
\begin{eqnarray}
\mu \ddot{r}&=&-\frac{Gm_1m_2}{r^2}\hat{r}\\
\nonumber
&=&-\frac{GM\mu}{r^2}\hat{r},~~~M\equiv m_1+m_2.
\end{eqnarray}
\end{split}
\end{equation*}
For the gravitational problem, the reduced mass then falls out and the
trajectory depends only on the total mass \(M\).

The kinetic energy and momenta also have analogues in center\sphinxhyphen{}of\sphinxhyphen{}mass
coordinates. The total and relative momenta are
\begin{equation*}
\begin{split}
\begin{eqnarray}
\boldsymbol{P}&\equiv&\boldsymbol{p}_1+\boldsymbol{p}_2=M\dot{\boldsymbol{R}}_{\rm cm},\\
\nonumber
\boldsymbol{q}&\equiv&\mu\dot{\boldsymbol{r}}.
\end{eqnarray}
\end{split}
\end{equation*}
With these definitions, a little algebra shows that the kinetic energy becomes
\begin{equation*}
\begin{split}
\begin{eqnarray}
T&=&\frac{1}{2}m_1|\boldsymbol{v}_1|^2+\frac{1}{2}m_2|\boldsymbol{v}_2|^2\\
\nonumber
&=&\frac{1}{2}M|\dot{\boldsymbol{R}}_{\rm cm}|^2
+\frac{1}{2}\mu|\dot{\boldsymbol{r}}|^2\\
\nonumber
&=&\frac{P^2}{2M}+\frac{q^2}{2\mu}.
\end{eqnarray}
\end{split}
\end{equation*}
The standard strategy is to transform into the center of mass frame,
then treat the problem as one of a single particle of mass \(\mu\)
undergoing a force \(\boldsymbol{F}_{12}\). Scattering angles can also be
expressed in this frame, then transformed into the lab frame. In
practice, one sees examples in the literature where \(d\sigma/d\Omega\)
expressed in both the “center\sphinxhyphen{}of\sphinxhyphen{}mass” and in the “laboratory”
frame.


\section{Deriving Elliptical Orbits}
\label{\detokenize{chapter1:deriving-elliptical-orbits}}
Kepler’s laws state that a gravitational orbit should be an ellipse
with the source of the gravitational field at one focus. Deriving this
is surprisingly messy. To do this, we first use angular momentum
conservation to transform the equations of motion so that it is in
terms of \(r\) and \(\theta\) instead of \(r\) and \(t\). The overall strategy
is to
\begin{enumerate}
\sphinxsetlistlabels{\arabic}{enumi}{enumii}{}{.}%
\item {} 
Find equations of motion for \(r\) and \(t\) with no angle (\(\theta\)) mentioned, i.e. \(d^2r/dt^2=\cdots\). Angular momentum conservation will be used, and the equation will involve the angular momentum \(L\).

\item {} 
Use angular momentum conservation to find an expression for \(\dot{\theta}\) in terms of \(r\).

\item {} 
Use the chain rule to convert the equations of motions for \(r\), an expression involving \(r,\dot{r}\) and \(\ddot{r}\), to one involving \(r,dr/d\theta\) and \(d^2r/d\theta^2\). This is quitecomplicated because the expressions will also involve a substitution \(u=1/r\) so that one finds an expression in terms of \(u\) and \(\theta\).

\item {} 
Once \(u(\theta)\) is found, you need to show that this can be converted to the familiar form for an ellipse.

\end{enumerate}

The equations of motion give




\begin{equation*}
\begin{split}
\begin{eqnarray}
\label{eq:radialeqofmotion} \tag{98}
\frac{d}{dt}r^2&=&\frac{d}{dt}(x^2+y^2)=2x\dot{x}+2y\dot{y}=2r\dot{r},\\
\nonumber
\dot{r}&=&\frac{x}{r}\dot{x}+\frac{y}{r}\dot{y},\\
\nonumber
\ddot{r}&=&\frac{x}{r}\ddot{x}+\frac{y}{r}\ddot{y}
+\frac{\dot{x}^2+\dot{y}^2}{r}
-\frac{\dot{r}^2}{r}.
\end{eqnarray}
\end{split}
\end{equation*}
Recognizing that the numerator of the third term is the velocity squared, and that it can be written in polar coordinates,




\begin{equation*}
\begin{split}
\begin{equation}
v^2=\dot{x}^2+\dot{y}^2=\dot{r}^2+r^2\dot{\theta}^2,
\label{_auto74} \tag{99}
\end{equation}
\end{split}
\end{equation*}
one can write \(\ddot{r}\) as




\begin{equation*}
\begin{split}
\begin{eqnarray}
\label{eq:radialeqofmotion2} \tag{100}
\ddot{r}&=&\frac{F_x\cos\theta+F_y\sin\theta}{m}+\frac{\dot{r}^2+r^2\dot{\theta}^2}{r}-\frac{\dot{r}^2}{r}\\
\nonumber
&=&\frac{F}{m}+\frac{r^2\dot{\theta}^2}{r}\\
\nonumber
m\ddot{r}&=&F+\frac{L^2}{mr^3}.
\end{eqnarray}
\end{split}
\end{equation*}
This derivation used the fact that the force was radial,
\(F=F_r=F_x\cos\theta+F_y\sin\theta\), and that angular momentum is
\(L=mrv_{\theta}=mr^2\dot{\theta}\). The term \(L^2/mr^3=mv^2/r\) behaves
like an additional force. Sometimes this is referred to as a
centrifugal force, but it is not a force. Instead, it is the
consequence of considering the motion in a rotating (and therefore
accelerating) frame.

Now, we switch to the particular case of an attractive inverse square
force, \(F=-\alpha/r^2\), and show that the trajectory, \(r(\theta)\), is
an ellipse. To do this we transform derivatives w.r.t. time to
derivatives w.r.t. \(\theta\) using the chain rule combined with angular
momentum conservation, \(\dot{\theta}=L/mr^2\).




\begin{equation*}
\begin{split}
\begin{eqnarray}
\label{eq:rtotheta} \tag{101}
\dot{r}&=&\frac{dr}{d\theta}\dot{\theta}=\frac{dr}{d\theta}\frac{L}{mr^2},\\
\nonumber
\ddot{r}&=&\frac{d^2r}{d\theta^2}\dot{\theta}^2
+\frac{dr}{d\theta}\left(\frac{d}{dr}\frac{L}{mr^2}\right)\dot{r}\\
\nonumber
&=&\frac{d^2r}{d\theta^2}\left(\frac{L}{mr^2}\right)^2
-2\frac{dr}{d\theta}\frac{L}{mr^3}\dot{r}\\
\nonumber
&=&\frac{d^2r}{d\theta^2}\left(\frac{L}{mr^2}\right)^2
-\frac{2}{r}\left(\frac{dr}{d\theta}\right)^2\left(\frac{L}{mr^2}\right)^2
\end{eqnarray}
\end{split}
\end{equation*}
Equating the two expressions for \(\ddot{r}\) in Eq.s ({\hyperref[\detokenize{chapter1:eq:radialeqofmotion2}]{\emph{100}}}) and ({\hyperref[\detokenize{chapter1:eq:rtotheta}]{\emph{101}}}) eliminates all the derivatives w.r.t. time, and provides a differential equation with only derivatives w.r.t. \(\theta\),




\begin{equation*}
\begin{split}
\begin{equation}
\label{eq:rdotdot} \tag{102}
\frac{d^2r}{d\theta^2}\left(\frac{L}{mr^2}\right)^2
-\frac{2}{r}\left(\frac{dr}{d\theta}\right)^2\left(\frac{L}{mr^2}\right)^2
=\frac{F}{m}+\frac{L^2}{m^2r^3},
\end{equation}
\end{split}
\end{equation*}
that when solved yields the trajectory, i.e. \(r(\theta)\). Up to this
point the expressions work for any radial force, not just forces that
fall as \(1/r^2\).

The trick to simplifying this differential equation for the inverse
square problems is to make a substitution, \(u\equiv 1/r\), and rewrite
the differential equation for \(u(\theta)\).
\begin{equation*}
\begin{split}
\begin{eqnarray}
r&=&1/u,\\
\nonumber
\frac{dr}{d\theta}&=&-\frac{1}{u^2}\frac{du}{d\theta},\\
\nonumber
\frac{d^2r}{d\theta^2}&=&\frac{2}{u^3}\left(\frac{du}{d\theta}\right)^2-\frac{1}{u^2}\frac{d^2u}{d\theta^2}.
\end{eqnarray}
\end{split}
\end{equation*}
Plugging these expressions into Eq. ({\hyperref[\detokenize{chapter1:eq:rdotdot}]{\emph{102}}}) gives an
expression in terms of \(u\), \(du/d\theta\), and \(d^2u/d\theta^2\). After
some tedious algebra,




\begin{equation*}
\begin{split}
\begin{equation}
\frac{d^2u}{d\theta^2}=-u-\frac{F m}{L^2u^2}.
\label{_auto75} \tag{103}
\end{equation}
\end{split}
\end{equation*}
For the attractive inverse square law force, \(F=-\alpha u^2\),




\begin{equation*}
\begin{split}
\begin{equation}
\frac{d^2u}{d\theta^2}=-u+\frac{m\alpha}{L^2}.
\label{_auto76} \tag{104}
\end{equation}
\end{split}
\end{equation*}
The solution has two arbitrary constants, \(A\) and \(\theta_0\),




\begin{equation*}
\begin{split}
\begin{eqnarray}
\label{eq:Ctrajectory} \tag{105}
u&=&\frac{m\alpha}{L^2}+A\cos(\theta-\theta_0),\\
\nonumber
r&=&\frac{1}{(m\alpha/L^2)+A\cos(\theta-\theta_0)}.
\end{eqnarray}
\end{split}
\end{equation*}
The radius will be at a minimum when \(\theta=\theta_0\) and at a
maximum when \(\theta=\theta_0+\pi\). The constant \(A\) is related to the
eccentricity of the orbit. When \(A=0\) the radius is a constant
\(r=L^2/(m\alpha)\), and the motion is circular. If one solved the
expression \(mv^2/r=-\alpha/r^2\) for a circular orbit, using the
substitution \(v=L/(mr)\), one would reproduce the expression
\(r=L^2/(m\alpha)\).

The form describing the elliptical trajectory in
Eq. ({\hyperref[\detokenize{chapter1:eq:Ctrajectory}]{\emph{105}}}) can be identified as an ellipse with one
focus being the center of the ellipse by considering the definition of
an ellipse as being the points such that the sum of the two distances
between the two foci are a constant. Making that distance \(2D\), the
distance between the two foci as \(2a\), and putting one focus at the
origin,
\begin{equation*}
\begin{split}
\begin{eqnarray}
2D&=&r+\sqrt{(r\cos\theta-2a)^2+r^2\sin^2\theta},\\
\nonumber
4D^2+r^2-4Dr&=&r^2+4a^2-4ar\cos\theta,\\
\nonumber
r&=&\frac{D^2-a^2}{D+a\cos\theta}=\frac{1}{D/(D^2-a^2)-a\cos\theta/(D^2-a^2)}.
\end{eqnarray}
\end{split}
\end{equation*}
By inspection, this is the same form as Eq. ({\hyperref[\detokenize{chapter1:eq:Ctrajectory}]{\emph{105}}}) with \(D/(D^2-a^2)=m\alpha/L^2\) and \(a/(D^2-a^2)=A\).

Let us remind ourselves about what an ellipse is before we proceed.

\begin{sphinxVerbatim}[commandchars=\\\{\}]
\PYG{k+kn}{import} \PYG{n+nn}{numpy} \PYG{k}{as} \PYG{n+nn}{np}
\PYG{k+kn}{from} \PYG{n+nn}{matplotlib} \PYG{k+kn}{import} \PYG{n}{pyplot} \PYG{k}{as} \PYG{n}{plt}
\PYG{k+kn}{from} \PYG{n+nn}{math} \PYG{k+kn}{import} \PYG{n}{pi}

\PYG{n}{u}\PYG{o}{=}\PYG{l+m+mf}{1.}     \PYG{c+c1}{\PYGZsh{}x\PYGZhy{}position of the center}
\PYG{n}{v}\PYG{o}{=}\PYG{l+m+mf}{0.5}    \PYG{c+c1}{\PYGZsh{}y\PYGZhy{}position of the center}
\PYG{n}{a}\PYG{o}{=}\PYG{l+m+mf}{2.}     \PYG{c+c1}{\PYGZsh{}radius on the x\PYGZhy{}axis}
\PYG{n}{b}\PYG{o}{=}\PYG{l+m+mf}{1.5}    \PYG{c+c1}{\PYGZsh{}radius on the y\PYGZhy{}axis}

\PYG{n}{t} \PYG{o}{=} \PYG{n}{np}\PYG{o}{.}\PYG{n}{linspace}\PYG{p}{(}\PYG{l+m+mi}{0}\PYG{p}{,} \PYG{l+m+mi}{2}\PYG{o}{*}\PYG{n}{pi}\PYG{p}{,} \PYG{l+m+mi}{100}\PYG{p}{)}
\PYG{n}{plt}\PYG{o}{.}\PYG{n}{plot}\PYG{p}{(} \PYG{n}{u}\PYG{o}{+}\PYG{n}{a}\PYG{o}{*}\PYG{n}{np}\PYG{o}{.}\PYG{n}{cos}\PYG{p}{(}\PYG{n}{t}\PYG{p}{)} \PYG{p}{,} \PYG{n}{v}\PYG{o}{+}\PYG{n}{b}\PYG{o}{*}\PYG{n}{np}\PYG{o}{.}\PYG{n}{sin}\PYG{p}{(}\PYG{n}{t}\PYG{p}{)} \PYG{p}{)}
\PYG{n}{plt}\PYG{o}{.}\PYG{n}{grid}\PYG{p}{(}\PYG{n}{color}\PYG{o}{=}\PYG{l+s+s1}{\PYGZsq{}}\PYG{l+s+s1}{lightgray}\PYG{l+s+s1}{\PYGZsq{}}\PYG{p}{,}\PYG{n}{linestyle}\PYG{o}{=}\PYG{l+s+s1}{\PYGZsq{}}\PYG{l+s+s1}{\PYGZhy{}\PYGZhy{}}\PYG{l+s+s1}{\PYGZsq{}}\PYG{p}{)}
\PYG{n}{plt}\PYG{o}{.}\PYG{n}{show}\PYG{p}{(}\PYG{p}{)}
\end{sphinxVerbatim}


\section{Effective or Centrifugal Potential}
\label{\detokenize{chapter1:effective-or-centrifugal-potential}}
The total energy of a particle is
\begin{equation*}
\begin{split}
\begin{eqnarray}
E&=&U(r)+\frac{1}{2}mv_\theta^2+\frac{1}{2}m\dot{r}^2\\
\nonumber
&=&U(r)+\frac{1}{2}mr^2\dot{\theta}^2+\frac{1}{2}m\dot{r}^2\\
\nonumber
&=&U(r)+\frac{L^2}{2mr^2}+\frac{1}{2}m\dot{r}^2.
\end{eqnarray}
\end{split}
\end{equation*}
The second term then contributes to the energy like an additional
repulsive potential. The term is sometimes referred to as the
“centrifugal” potential, even though it is actually the kinetic energy
of the angular motion. Combined with \(U(r)\), it is sometimes referred
to as the “effective” potential,
\begin{equation*}
\begin{split}
\begin{eqnarray}
U_{\rm eff}(r)&=&U(r)+\frac{L^2}{2mr^2}.
\end{eqnarray}
\end{split}
\end{equation*}
Note that if one treats the effective potential like a real potential, one would expect to be able to generate an effective force,
\begin{equation*}
\begin{split}
\begin{eqnarray}
F_{\rm eff}&=&-\frac{d}{dr}U(r) -\frac{d}{dr}\frac{L^2}{2mr^2}\\
\nonumber
&=&F(r)+\frac{L^2}{mr^3}=F(r)+m\frac{v_\perp^2}{r},
\end{eqnarray}
\end{split}
\end{equation*}
which is indeed matches the form for \(m\ddot{r}\) in Eq. ({\hyperref[\detokenize{chapter1:eq:radialeqofmotion2}]{\emph{100}}}), which included the \sphinxstylestrong{centrifugal} force.

The following code plots this effective potential for a simple choice of parameters, with a standard gravitational potential \(-\alpha/r\). Here we have chosen \(L=m=\alpha=1\).

\begin{sphinxVerbatim}[commandchars=\\\{\}]
\PYG{c+c1}{\PYGZsh{} Common imports}
\PYG{k+kn}{import} \PYG{n+nn}{numpy} \PYG{k}{as} \PYG{n+nn}{np}
\PYG{k+kn}{from} \PYG{n+nn}{math} \PYG{k+kn}{import} \PYG{o}{*}
\PYG{k+kn}{import} \PYG{n+nn}{matplotlib}\PYG{n+nn}{.}\PYG{n+nn}{pyplot} \PYG{k}{as} \PYG{n+nn}{plt}

\PYG{n}{Deltax} \PYG{o}{=} \PYG{l+m+mf}{0.01}
\PYG{c+c1}{\PYGZsh{}set up arrays}
\PYG{n}{xinitial} \PYG{o}{=} \PYG{l+m+mf}{0.3}
\PYG{n}{xfinal} \PYG{o}{=} \PYG{l+m+mf}{5.0}
\PYG{n}{alpha} \PYG{o}{=} \PYG{l+m+mf}{1.0}   \PYG{c+c1}{\PYGZsh{} spring constant}
\PYG{n}{m} \PYG{o}{=} \PYG{l+m+mf}{1.0}   \PYG{c+c1}{\PYGZsh{} mass, you can change these}
\PYG{n}{AngMom} \PYG{o}{=} \PYG{l+m+mf}{1.0}  \PYG{c+c1}{\PYGZsh{}  The angular momentum}
\PYG{n}{n} \PYG{o}{=} \PYG{n}{ceil}\PYG{p}{(}\PYG{p}{(}\PYG{n}{xfinal}\PYG{o}{\PYGZhy{}}\PYG{n}{xinitial}\PYG{p}{)}\PYG{o}{/}\PYG{n}{Deltax}\PYG{p}{)}
\PYG{n}{x} \PYG{o}{=} \PYG{n}{np}\PYG{o}{.}\PYG{n}{zeros}\PYG{p}{(}\PYG{n}{n}\PYG{p}{)}
\PYG{k}{for} \PYG{n}{i} \PYG{o+ow}{in} \PYG{n+nb}{range}\PYG{p}{(}\PYG{n}{n}\PYG{p}{)}\PYG{p}{:}
    \PYG{n}{x}\PYG{p}{[}\PYG{n}{i}\PYG{p}{]} \PYG{o}{=} \PYG{n}{xinitial}\PYG{o}{+}\PYG{n}{i}\PYG{o}{*}\PYG{n}{Deltax}
\PYG{n}{V} \PYG{o}{=} \PYG{n}{np}\PYG{o}{.}\PYG{n}{zeros}\PYG{p}{(}\PYG{n}{n}\PYG{p}{)}
\PYG{n}{V} \PYG{o}{=} \PYG{o}{\PYGZhy{}}\PYG{n}{alpha}\PYG{o}{/}\PYG{n}{x}\PYG{o}{+}\PYG{l+m+mf}{0.5}\PYG{o}{*}\PYG{n}{AngMom}\PYG{o}{*}\PYG{n}{AngMom}\PYG{o}{/}\PYG{p}{(}\PYG{n}{m}\PYG{o}{*}\PYG{n}{x}\PYG{o}{*}\PYG{n}{x}\PYG{p}{)}
\PYG{c+c1}{\PYGZsh{} Plot potential}
\PYG{n}{fig}\PYG{p}{,} \PYG{n}{ax} \PYG{o}{=} \PYG{n}{plt}\PYG{o}{.}\PYG{n}{subplots}\PYG{p}{(}\PYG{p}{)}
\PYG{n}{ax}\PYG{o}{.}\PYG{n}{set\PYGZus{}xlabel}\PYG{p}{(}\PYG{l+s+s1}{\PYGZsq{}}\PYG{l+s+s1}{r[m]}\PYG{l+s+s1}{\PYGZsq{}}\PYG{p}{)}
\PYG{n}{ax}\PYG{o}{.}\PYG{n}{set\PYGZus{}ylabel}\PYG{p}{(}\PYG{l+s+s1}{\PYGZsq{}}\PYG{l+s+s1}{V[J]}\PYG{l+s+s1}{\PYGZsq{}}\PYG{p}{)}
\PYG{n}{ax}\PYG{o}{.}\PYG{n}{plot}\PYG{p}{(}\PYG{n}{x}\PYG{p}{,} \PYG{n}{V}\PYG{p}{)}
\PYG{n}{fig}\PYG{o}{.}\PYG{n}{tight\PYGZus{}layout}\PYG{p}{(}\PYG{p}{)}
\PYG{n}{plt}\PYG{o}{.}\PYG{n}{show}\PYG{p}{(}\PYG{p}{)}
\end{sphinxVerbatim}


\subsection{Gravitational force example}
\label{\detokenize{chapter1:gravitational-force-example}}
Using the above parameters, we can now study the evolution of the system using for example the velocity Verlet method.
This is done in the code here for an initial radius equal to the minimum of the potential well.  We seen then that the radius is always the same and corresponds to a circle (the radius is always constant).

\begin{sphinxVerbatim}[commandchars=\\\{\}]
\PYG{c+c1}{\PYGZsh{} Common imports}
\PYG{k+kn}{import} \PYG{n+nn}{numpy} \PYG{k}{as} \PYG{n+nn}{np}
\PYG{k+kn}{import} \PYG{n+nn}{pandas} \PYG{k}{as} \PYG{n+nn}{pd}
\PYG{k+kn}{from} \PYG{n+nn}{math} \PYG{k+kn}{import} \PYG{o}{*}
\PYG{k+kn}{import} \PYG{n+nn}{matplotlib}\PYG{n+nn}{.}\PYG{n+nn}{pyplot} \PYG{k}{as} \PYG{n+nn}{plt}
\PYG{k+kn}{import} \PYG{n+nn}{os}

\PYG{c+c1}{\PYGZsh{} Where to save the figures and data files}
\PYG{n}{PROJECT\PYGZus{}ROOT\PYGZus{}DIR} \PYG{o}{=} \PYG{l+s+s2}{\PYGZdq{}}\PYG{l+s+s2}{Results}\PYG{l+s+s2}{\PYGZdq{}}
\PYG{n}{FIGURE\PYGZus{}ID} \PYG{o}{=} \PYG{l+s+s2}{\PYGZdq{}}\PYG{l+s+s2}{Results/FigureFiles}\PYG{l+s+s2}{\PYGZdq{}}
\PYG{n}{DATA\PYGZus{}ID} \PYG{o}{=} \PYG{l+s+s2}{\PYGZdq{}}\PYG{l+s+s2}{DataFiles/}\PYG{l+s+s2}{\PYGZdq{}}

\PYG{k}{if} \PYG{o+ow}{not} \PYG{n}{os}\PYG{o}{.}\PYG{n}{path}\PYG{o}{.}\PYG{n}{exists}\PYG{p}{(}\PYG{n}{PROJECT\PYGZus{}ROOT\PYGZus{}DIR}\PYG{p}{)}\PYG{p}{:}
    \PYG{n}{os}\PYG{o}{.}\PYG{n}{mkdir}\PYG{p}{(}\PYG{n}{PROJECT\PYGZus{}ROOT\PYGZus{}DIR}\PYG{p}{)}

\PYG{k}{if} \PYG{o+ow}{not} \PYG{n}{os}\PYG{o}{.}\PYG{n}{path}\PYG{o}{.}\PYG{n}{exists}\PYG{p}{(}\PYG{n}{FIGURE\PYGZus{}ID}\PYG{p}{)}\PYG{p}{:}
    \PYG{n}{os}\PYG{o}{.}\PYG{n}{makedirs}\PYG{p}{(}\PYG{n}{FIGURE\PYGZus{}ID}\PYG{p}{)}

\PYG{k}{if} \PYG{o+ow}{not} \PYG{n}{os}\PYG{o}{.}\PYG{n}{path}\PYG{o}{.}\PYG{n}{exists}\PYG{p}{(}\PYG{n}{DATA\PYGZus{}ID}\PYG{p}{)}\PYG{p}{:}
    \PYG{n}{os}\PYG{o}{.}\PYG{n}{makedirs}\PYG{p}{(}\PYG{n}{DATA\PYGZus{}ID}\PYG{p}{)}

\PYG{k}{def} \PYG{n+nf}{image\PYGZus{}path}\PYG{p}{(}\PYG{n}{fig\PYGZus{}id}\PYG{p}{)}\PYG{p}{:}
    \PYG{k}{return} \PYG{n}{os}\PYG{o}{.}\PYG{n}{path}\PYG{o}{.}\PYG{n}{join}\PYG{p}{(}\PYG{n}{FIGURE\PYGZus{}ID}\PYG{p}{,} \PYG{n}{fig\PYGZus{}id}\PYG{p}{)}

\PYG{k}{def} \PYG{n+nf}{data\PYGZus{}path}\PYG{p}{(}\PYG{n}{dat\PYGZus{}id}\PYG{p}{)}\PYG{p}{:}
    \PYG{k}{return} \PYG{n}{os}\PYG{o}{.}\PYG{n}{path}\PYG{o}{.}\PYG{n}{join}\PYG{p}{(}\PYG{n}{DATA\PYGZus{}ID}\PYG{p}{,} \PYG{n}{dat\PYGZus{}id}\PYG{p}{)}

\PYG{k}{def} \PYG{n+nf}{save\PYGZus{}fig}\PYG{p}{(}\PYG{n}{fig\PYGZus{}id}\PYG{p}{)}\PYG{p}{:}
    \PYG{n}{plt}\PYG{o}{.}\PYG{n}{savefig}\PYG{p}{(}\PYG{n}{image\PYGZus{}path}\PYG{p}{(}\PYG{n}{fig\PYGZus{}id}\PYG{p}{)} \PYG{o}{+} \PYG{l+s+s2}{\PYGZdq{}}\PYG{l+s+s2}{.png}\PYG{l+s+s2}{\PYGZdq{}}\PYG{p}{,} \PYG{n+nb}{format}\PYG{o}{=}\PYG{l+s+s1}{\PYGZsq{}}\PYG{l+s+s1}{png}\PYG{l+s+s1}{\PYGZsq{}}\PYG{p}{)}


\PYG{c+c1}{\PYGZsh{} Simple Gravitational Force   \PYGZhy{}alpha/r}
    
\PYG{n}{DeltaT} \PYG{o}{=} \PYG{l+m+mf}{0.01}
\PYG{c+c1}{\PYGZsh{}set up arrays }
\PYG{n}{tfinal} \PYG{o}{=} \PYG{l+m+mf}{100.0}
\PYG{n}{n} \PYG{o}{=} \PYG{n}{ceil}\PYG{p}{(}\PYG{n}{tfinal}\PYG{o}{/}\PYG{n}{DeltaT}\PYG{p}{)}
\PYG{c+c1}{\PYGZsh{} set up arrays for t, v and r}
\PYG{n}{t} \PYG{o}{=} \PYG{n}{np}\PYG{o}{.}\PYG{n}{zeros}\PYG{p}{(}\PYG{n}{n}\PYG{p}{)}
\PYG{n}{v} \PYG{o}{=} \PYG{n}{np}\PYG{o}{.}\PYG{n}{zeros}\PYG{p}{(}\PYG{n}{n}\PYG{p}{)}
\PYG{n}{r} \PYG{o}{=} \PYG{n}{np}\PYG{o}{.}\PYG{n}{zeros}\PYG{p}{(}\PYG{n}{n}\PYG{p}{)}
\PYG{c+c1}{\PYGZsh{} Constants of the model, setting all variables to one for simplicity}
\PYG{n}{alpha} \PYG{o}{=} \PYG{l+m+mf}{1.0}
\PYG{n}{AngMom} \PYG{o}{=} \PYG{l+m+mf}{1.0}  \PYG{c+c1}{\PYGZsh{}  The angular momentum}
\PYG{n}{m} \PYG{o}{=} \PYG{l+m+mf}{1.0}  \PYG{c+c1}{\PYGZsh{} scale mass to one}
\PYG{n}{c1} \PYG{o}{=} \PYG{n}{AngMom}\PYG{o}{*}\PYG{n}{AngMom}\PYG{o}{/}\PYG{p}{(}\PYG{n}{m}\PYG{o}{*}\PYG{n}{m}\PYG{p}{)}
\PYG{n}{c2} \PYG{o}{=} \PYG{n}{AngMom}\PYG{o}{*}\PYG{n}{AngMom}\PYG{o}{/}\PYG{n}{m}
\PYG{n}{rmin} \PYG{o}{=} \PYG{p}{(}\PYG{n}{AngMom}\PYG{o}{*}\PYG{n}{AngMom}\PYG{o}{/}\PYG{n}{m}\PYG{o}{/}\PYG{n}{alpha}\PYG{p}{)}
\PYG{c+c1}{\PYGZsh{} Initial conditions}
\PYG{n}{r0} \PYG{o}{=} \PYG{n}{rmin}
\PYG{n}{v0} \PYG{o}{=} \PYG{l+m+mf}{0.0}
\PYG{n}{r}\PYG{p}{[}\PYG{l+m+mi}{0}\PYG{p}{]} \PYG{o}{=} \PYG{n}{r0}
\PYG{n}{v}\PYG{p}{[}\PYG{l+m+mi}{0}\PYG{p}{]} \PYG{o}{=} \PYG{n}{v0}
\PYG{c+c1}{\PYGZsh{} Start integrating using the Velocity\PYGZhy{}Verlet  method}
\PYG{k}{for} \PYG{n}{i} \PYG{o+ow}{in} \PYG{n+nb}{range}\PYG{p}{(}\PYG{n}{n}\PYG{o}{\PYGZhy{}}\PYG{l+m+mi}{1}\PYG{p}{)}\PYG{p}{:}
    \PYG{c+c1}{\PYGZsh{} Set up acceleration}
    \PYG{n}{a} \PYG{o}{=} \PYG{o}{\PYGZhy{}}\PYG{n}{alpha}\PYG{o}{/}\PYG{p}{(}\PYG{n}{r}\PYG{p}{[}\PYG{n}{i}\PYG{p}{]}\PYG{o}{*}\PYG{o}{*}\PYG{l+m+mi}{2}\PYG{p}{)}\PYG{o}{+}\PYG{n}{c1}\PYG{o}{/}\PYG{p}{(}\PYG{n}{r}\PYG{p}{[}\PYG{n}{i}\PYG{p}{]}\PYG{o}{*}\PYG{o}{*}\PYG{l+m+mi}{3}\PYG{p}{)}
    \PYG{c+c1}{\PYGZsh{} update velocity, time and position using the Velocity\PYGZhy{}Verlet method}
    \PYG{n}{r}\PYG{p}{[}\PYG{n}{i}\PYG{o}{+}\PYG{l+m+mi}{1}\PYG{p}{]} \PYG{o}{=} \PYG{n}{r}\PYG{p}{[}\PYG{n}{i}\PYG{p}{]} \PYG{o}{+} \PYG{n}{DeltaT}\PYG{o}{*}\PYG{n}{v}\PYG{p}{[}\PYG{n}{i}\PYG{p}{]}\PYG{o}{+}\PYG{l+m+mf}{0.5}\PYG{o}{*}\PYG{p}{(}\PYG{n}{DeltaT}\PYG{o}{*}\PYG{o}{*}\PYG{l+m+mi}{2}\PYG{p}{)}\PYG{o}{*}\PYG{n}{a}
    \PYG{n}{anew} \PYG{o}{=} \PYG{o}{\PYGZhy{}}\PYG{n}{alpha}\PYG{o}{/}\PYG{p}{(}\PYG{n}{r}\PYG{p}{[}\PYG{n}{i}\PYG{o}{+}\PYG{l+m+mi}{1}\PYG{p}{]}\PYG{o}{*}\PYG{o}{*}\PYG{l+m+mi}{2}\PYG{p}{)}\PYG{o}{+}\PYG{n}{c1}\PYG{o}{/}\PYG{p}{(}\PYG{n}{r}\PYG{p}{[}\PYG{n}{i}\PYG{o}{+}\PYG{l+m+mi}{1}\PYG{p}{]}\PYG{o}{*}\PYG{o}{*}\PYG{l+m+mi}{3}\PYG{p}{)}
    \PYG{n}{v}\PYG{p}{[}\PYG{n}{i}\PYG{o}{+}\PYG{l+m+mi}{1}\PYG{p}{]} \PYG{o}{=} \PYG{n}{v}\PYG{p}{[}\PYG{n}{i}\PYG{p}{]} \PYG{o}{+} \PYG{l+m+mf}{0.5}\PYG{o}{*}\PYG{n}{DeltaT}\PYG{o}{*}\PYG{p}{(}\PYG{n}{a}\PYG{o}{+}\PYG{n}{anew}\PYG{p}{)}
    \PYG{n}{t}\PYG{p}{[}\PYG{n}{i}\PYG{o}{+}\PYG{l+m+mi}{1}\PYG{p}{]} \PYG{o}{=} \PYG{n}{t}\PYG{p}{[}\PYG{n}{i}\PYG{p}{]} \PYG{o}{+} \PYG{n}{DeltaT}
    \PYG{c+c1}{\PYGZsh{} Plot position as function of time}
\PYG{n}{fig}\PYG{p}{,} \PYG{n}{ax} \PYG{o}{=} \PYG{n}{plt}\PYG{o}{.}\PYG{n}{subplots}\PYG{p}{(}\PYG{l+m+mi}{2}\PYG{p}{,}\PYG{l+m+mi}{1}\PYG{p}{)}
\PYG{n}{ax}\PYG{p}{[}\PYG{l+m+mi}{0}\PYG{p}{]}\PYG{o}{.}\PYG{n}{set\PYGZus{}xlabel}\PYG{p}{(}\PYG{l+s+s1}{\PYGZsq{}}\PYG{l+s+s1}{time}\PYG{l+s+s1}{\PYGZsq{}}\PYG{p}{)}
\PYG{n}{ax}\PYG{p}{[}\PYG{l+m+mi}{0}\PYG{p}{]}\PYG{o}{.}\PYG{n}{set\PYGZus{}ylabel}\PYG{p}{(}\PYG{l+s+s1}{\PYGZsq{}}\PYG{l+s+s1}{radius}\PYG{l+s+s1}{\PYGZsq{}}\PYG{p}{)}
\PYG{n}{ax}\PYG{p}{[}\PYG{l+m+mi}{0}\PYG{p}{]}\PYG{o}{.}\PYG{n}{plot}\PYG{p}{(}\PYG{n}{t}\PYG{p}{,}\PYG{n}{r}\PYG{p}{)}
\PYG{n}{ax}\PYG{p}{[}\PYG{l+m+mi}{1}\PYG{p}{]}\PYG{o}{.}\PYG{n}{set\PYGZus{}xlabel}\PYG{p}{(}\PYG{l+s+s1}{\PYGZsq{}}\PYG{l+s+s1}{time}\PYG{l+s+s1}{\PYGZsq{}}\PYG{p}{)}
\PYG{n}{ax}\PYG{p}{[}\PYG{l+m+mi}{1}\PYG{p}{]}\PYG{o}{.}\PYG{n}{set\PYGZus{}ylabel}\PYG{p}{(}\PYG{l+s+s1}{\PYGZsq{}}\PYG{l+s+s1}{Velocity}\PYG{l+s+s1}{\PYGZsq{}}\PYG{p}{)}
\PYG{n}{ax}\PYG{p}{[}\PYG{l+m+mi}{1}\PYG{p}{]}\PYG{o}{.}\PYG{n}{plot}\PYG{p}{(}\PYG{n}{t}\PYG{p}{,}\PYG{n}{v}\PYG{p}{)}
\PYG{n}{save\PYGZus{}fig}\PYG{p}{(}\PYG{l+s+s2}{\PYGZdq{}}\PYG{l+s+s2}{RadialGVV}\PYG{l+s+s2}{\PYGZdq{}}\PYG{p}{)}
\PYG{n}{plt}\PYG{o}{.}\PYG{n}{show}\PYG{p}{(}\PYG{p}{)}
\end{sphinxVerbatim}

Changing the value of the initial position to a value where the energy is positive, leads to an increasing radius with time, a so\sphinxhyphen{}called unbound orbit. Choosing on the other hand an initial radius that corresponds to a negative energy and different from the minimum value leads to a radius that oscillates back and forth between two values.


\subsection{Harmonic Oscillator in two dimensions}
\label{\detokenize{chapter1:harmonic-oscillator-in-two-dimensions}}
Consider a particle of mass \(m\) in a 2\sphinxhyphen{}dimensional harmonic oscillator with potential
\begin{equation*}
\begin{split}
U=\frac{1}{2}kr^2=\frac{1}{2}k(x^2+y^2).
\end{split}
\end{equation*}
If the orbit has angular momentum \(L\), we can find the radius and angular velocity of the circular orbit as well as the b) the angular frequency of small radial perturbations.

We consider the effective potential. The radius of a circular orbit is at the minimum of the potential (where the effective force is zero).
The potential is plotted here with the parameters \(k=m=0.1\) and \(L=1.0\).

\begin{sphinxVerbatim}[commandchars=\\\{\}]
\PYG{c+c1}{\PYGZsh{} Common imports}
\PYG{k+kn}{import} \PYG{n+nn}{numpy} \PYG{k}{as} \PYG{n+nn}{np}
\PYG{k+kn}{from} \PYG{n+nn}{math} \PYG{k+kn}{import} \PYG{o}{*}
\PYG{k+kn}{import} \PYG{n+nn}{matplotlib}\PYG{n+nn}{.}\PYG{n+nn}{pyplot} \PYG{k}{as} \PYG{n+nn}{plt}

\PYG{n}{Deltax} \PYG{o}{=} \PYG{l+m+mf}{0.01}
\PYG{c+c1}{\PYGZsh{}set up arrays}
\PYG{n}{xinitial} \PYG{o}{=} \PYG{l+m+mf}{1.0}
\PYG{n}{xfinal} \PYG{o}{=} \PYG{l+m+mf}{5.0}
\PYG{n}{k} \PYG{o}{=} \PYG{l+m+mf}{0.1}   \PYG{c+c1}{\PYGZsh{} spring constant}
\PYG{n}{m} \PYG{o}{=} \PYG{l+m+mf}{0.1}   \PYG{c+c1}{\PYGZsh{} mass, you can change these}
\PYG{n}{AngMom} \PYG{o}{=} \PYG{l+m+mf}{1.0}  \PYG{c+c1}{\PYGZsh{}  The angular momentum}
\PYG{n}{n} \PYG{o}{=} \PYG{n}{ceil}\PYG{p}{(}\PYG{p}{(}\PYG{n}{xfinal}\PYG{o}{\PYGZhy{}}\PYG{n}{xinitial}\PYG{p}{)}\PYG{o}{/}\PYG{n}{Deltax}\PYG{p}{)}
\PYG{n}{x} \PYG{o}{=} \PYG{n}{np}\PYG{o}{.}\PYG{n}{zeros}\PYG{p}{(}\PYG{n}{n}\PYG{p}{)}
\PYG{k}{for} \PYG{n}{i} \PYG{o+ow}{in} \PYG{n+nb}{range}\PYG{p}{(}\PYG{n}{n}\PYG{p}{)}\PYG{p}{:}
    \PYG{n}{x}\PYG{p}{[}\PYG{n}{i}\PYG{p}{]} \PYG{o}{=} \PYG{n}{xinitial}\PYG{o}{+}\PYG{n}{i}\PYG{o}{*}\PYG{n}{Deltax}
\PYG{n}{V} \PYG{o}{=} \PYG{n}{np}\PYG{o}{.}\PYG{n}{zeros}\PYG{p}{(}\PYG{n}{n}\PYG{p}{)}
\PYG{n}{V} \PYG{o}{=} \PYG{l+m+mf}{0.5}\PYG{o}{*}\PYG{n}{k}\PYG{o}{*}\PYG{n}{x}\PYG{o}{*}\PYG{n}{x}\PYG{o}{+}\PYG{l+m+mf}{0.5}\PYG{o}{*}\PYG{n}{AngMom}\PYG{o}{*}\PYG{n}{AngMom}\PYG{o}{/}\PYG{p}{(}\PYG{n}{m}\PYG{o}{*}\PYG{n}{x}\PYG{o}{*}\PYG{n}{x}\PYG{p}{)}
\PYG{c+c1}{\PYGZsh{} Plot potential}
\PYG{n}{fig}\PYG{p}{,} \PYG{n}{ax} \PYG{o}{=} \PYG{n}{plt}\PYG{o}{.}\PYG{n}{subplots}\PYG{p}{(}\PYG{p}{)}
\PYG{n}{ax}\PYG{o}{.}\PYG{n}{set\PYGZus{}xlabel}\PYG{p}{(}\PYG{l+s+s1}{\PYGZsq{}}\PYG{l+s+s1}{r[m]}\PYG{l+s+s1}{\PYGZsq{}}\PYG{p}{)}
\PYG{n}{ax}\PYG{o}{.}\PYG{n}{set\PYGZus{}ylabel}\PYG{p}{(}\PYG{l+s+s1}{\PYGZsq{}}\PYG{l+s+s1}{V[J]}\PYG{l+s+s1}{\PYGZsq{}}\PYG{p}{)}
\PYG{n}{ax}\PYG{o}{.}\PYG{n}{plot}\PYG{p}{(}\PYG{n}{x}\PYG{p}{,} \PYG{n}{V}\PYG{p}{)}
\PYG{n}{fig}\PYG{o}{.}\PYG{n}{tight\PYGZus{}layout}\PYG{p}{(}\PYG{p}{)}
\PYG{n}{plt}\PYG{o}{.}\PYG{n}{show}\PYG{p}{(}\PYG{p}{)}
\end{sphinxVerbatim}
\begin{equation*}
\begin{split}
\begin{eqnarray*}
U_{\rm eff}&=&\frac{1}{2}kr^2+\frac{L^2}{2mr^2}
\end{eqnarray*}
\end{split}
\end{equation*}
The effective potential looks like that of a harmonic oscillator for
large \(r\), but for small \(r\), the centrifugal potential repels the
particle from the origin. The combination of the two potentials has a
minimum for at some radius \(r_{\rm min}\).
\begin{equation*}
\begin{split}
\begin{eqnarray*}
0&=&kr_{\rm min}-\frac{L^2}{mr_{\rm min}^3},\\
r_{\rm min}&=&\left(\frac{L^2}{mk}\right)^{1/4},\\
\dot{\theta}&=&\frac{L}{mr_{\rm min}^2}=\sqrt{k/m}.
\end{eqnarray*}
\end{split}
\end{equation*}
For particles at \(r_{\rm min}\) with \(\dot{r}=0\), the particle does not
accelerate and \(r\) stays constant, i.e. a circular orbit. The radius
of the circular orbit can be adjusted by changing the angular momentum
\(L\).

For the above parameters this minimum is at \(r_{\rm min}=1\).

Now consider small vibrations about \(r_{\rm min}\). The effective spring constant is the curvature of the effective potential.
\begin{equation*}
\begin{split}
\begin{eqnarray*}
k_{\rm eff}&=&\left.\frac{d^2}{dr^2}U_{\rm eff}(r)\right|_{r=r_{\rm min}}=k+\frac{3L^2}{mr_{\rm min}^4}\\
&=&4k,\\
\omega&=&\sqrt{k_{\rm eff}/m}=2\sqrt{k/m}=2\dot{\theta}.
\end{eqnarray*}
\end{split}
\end{equation*}
Here, the second step used the result of the last step from part
(a). Because the radius oscillates with twice the angular frequency,
the orbit has two places where \(r\) reaches a minimum in one
cycle. This differs from the inverse\sphinxhyphen{}square force where there is one
minimum in an orbit. One can show that the orbit for the harmonic
oscillator is also elliptical, but in this case the center of the
potential is at the center of the ellipse, not at one of the foci.

The solution is also simple to write down exactly in Cartesian coordinates. The \(x\) and \(y\) equations of motion separate,
\begin{equation*}
\begin{split}
\begin{eqnarray*}
\ddot{x}&=&-kx,\\
\ddot{y}&=&-ky.
\end{eqnarray*}
\end{split}
\end{equation*}
So the general solution can be expressed as
\begin{equation*}
\begin{split}
\begin{eqnarray*}
x&=&A\cos\omega_0 t+B\sin\omega_0 t,\\
y&=&C\cos\omega_0 t+D\sin\omega_0 t.
\end{eqnarray*}
\end{split}
\end{equation*}
The code here finds the solution for \(x\) and \(y\) using the code we developed in homework 4.

\begin{sphinxVerbatim}[commandchars=\\\{\}]
\PYG{n}{DeltaT} \PYG{o}{=} \PYG{l+m+mf}{0.01}
\PYG{c+c1}{\PYGZsh{}set up arrays }
\PYG{n}{tfinal} \PYG{o}{=} \PYG{l+m+mf}{10.0}
\PYG{n}{n} \PYG{o}{=} \PYG{n}{ceil}\PYG{p}{(}\PYG{n}{tfinal}\PYG{o}{/}\PYG{n}{DeltaT}\PYG{p}{)}
\PYG{c+c1}{\PYGZsh{} set up arrays}
\PYG{n}{t} \PYG{o}{=} \PYG{n}{np}\PYG{o}{.}\PYG{n}{zeros}\PYG{p}{(}\PYG{n}{n}\PYG{p}{)}
\PYG{n}{v} \PYG{o}{=} \PYG{n}{np}\PYG{o}{.}\PYG{n}{zeros}\PYG{p}{(}\PYG{p}{(}\PYG{n}{n}\PYG{p}{,}\PYG{l+m+mi}{2}\PYG{p}{)}\PYG{p}{)}
\PYG{n}{r} \PYG{o}{=} \PYG{n}{np}\PYG{o}{.}\PYG{n}{zeros}\PYG{p}{(}\PYG{p}{(}\PYG{n}{n}\PYG{p}{,}\PYG{l+m+mi}{2}\PYG{p}{)}\PYG{p}{)}
\PYG{n}{radius} \PYG{o}{=} \PYG{n}{np}\PYG{o}{.}\PYG{n}{zeros}\PYG{p}{(}\PYG{n}{n}\PYG{p}{)}
\PYG{c+c1}{\PYGZsh{} Constants of the model}
\PYG{n}{k} \PYG{o}{=} \PYG{l+m+mf}{0.1}   \PYG{c+c1}{\PYGZsh{} spring constant}
\PYG{n}{m} \PYG{o}{=} \PYG{l+m+mf}{0.1}   \PYG{c+c1}{\PYGZsh{} mass, you can change these}
\PYG{n}{omega02} \PYG{o}{=} \PYG{n}{sqrt}\PYG{p}{(}\PYG{n}{k}\PYG{o}{/}\PYG{n}{m}\PYG{p}{)}  \PYG{c+c1}{\PYGZsh{} Frequency}
\PYG{n}{AngMom} \PYG{o}{=} \PYG{l+m+mf}{1.0}  \PYG{c+c1}{\PYGZsh{}  The angular momentum}
\PYG{n}{rmin} \PYG{o}{=} \PYG{p}{(}\PYG{n}{AngMom}\PYG{o}{*}\PYG{n}{AngMom}\PYG{o}{/}\PYG{n}{k}\PYG{o}{/}\PYG{n}{m}\PYG{p}{)}\PYG{o}{*}\PYG{o}{*}\PYG{l+m+mf}{0.25}
\PYG{c+c1}{\PYGZsh{} Initial conditions as compact 2\PYGZhy{}dimensional arrays}
\PYG{c+c1}{\PYGZsh{}x0 =rmin*0.5; y0 = sqrt(rmin*rmin\PYGZhy{}x0*x0)}
\PYG{n}{x0} \PYG{o}{=} \PYG{l+m+mf}{1.0}\PYG{p}{;} \PYG{n}{y0}\PYG{o}{=} \PYG{l+m+mf}{1.0}
\PYG{n}{r0} \PYG{o}{=} \PYG{n}{np}\PYG{o}{.}\PYG{n}{array}\PYG{p}{(}\PYG{p}{[}\PYG{n}{x0}\PYG{p}{,}\PYG{n}{y0}\PYG{p}{]}\PYG{p}{)} 
\PYG{n}{v0} \PYG{o}{=} \PYG{n}{np}\PYG{o}{.}\PYG{n}{array}\PYG{p}{(}\PYG{p}{[}\PYG{l+m+mf}{0.0}\PYG{p}{,}\PYG{l+m+mf}{0.0}\PYG{p}{]}\PYG{p}{)}
\PYG{n}{r}\PYG{p}{[}\PYG{l+m+mi}{0}\PYG{p}{]} \PYG{o}{=} \PYG{n}{r0}
\PYG{n}{v}\PYG{p}{[}\PYG{l+m+mi}{0}\PYG{p}{]} \PYG{o}{=} \PYG{n}{v0}
\PYG{c+c1}{\PYGZsh{} Start integrating using the Velocity\PYGZhy{}Verlet  method}
\PYG{k}{for} \PYG{n}{i} \PYG{o+ow}{in} \PYG{n+nb}{range}\PYG{p}{(}\PYG{n}{n}\PYG{o}{\PYGZhy{}}\PYG{l+m+mi}{1}\PYG{p}{)}\PYG{p}{:}
    \PYG{c+c1}{\PYGZsh{} Set up the acceleration}
    \PYG{n}{a} \PYG{o}{=}  \PYG{o}{\PYGZhy{}}\PYG{n}{r}\PYG{p}{[}\PYG{n}{i}\PYG{p}{]}\PYG{o}{*}\PYG{n}{omega02}  
    \PYG{c+c1}{\PYGZsh{} update velocity, time and position using the Velocity\PYGZhy{}Verlet method}
    \PYG{n}{r}\PYG{p}{[}\PYG{n}{i}\PYG{o}{+}\PYG{l+m+mi}{1}\PYG{p}{]} \PYG{o}{=} \PYG{n}{r}\PYG{p}{[}\PYG{n}{i}\PYG{p}{]} \PYG{o}{+} \PYG{n}{DeltaT}\PYG{o}{*}\PYG{n}{v}\PYG{p}{[}\PYG{n}{i}\PYG{p}{]}\PYG{o}{+}\PYG{l+m+mf}{0.5}\PYG{o}{*}\PYG{p}{(}\PYG{n}{DeltaT}\PYG{o}{*}\PYG{o}{*}\PYG{l+m+mi}{2}\PYG{p}{)}\PYG{o}{*}\PYG{n}{a}
    \PYG{n}{anew} \PYG{o}{=} \PYG{o}{\PYGZhy{}}\PYG{n}{r}\PYG{p}{[}\PYG{n}{i}\PYG{o}{+}\PYG{l+m+mi}{1}\PYG{p}{]}\PYG{o}{*}\PYG{n}{omega02}  
    \PYG{n}{v}\PYG{p}{[}\PYG{n}{i}\PYG{o}{+}\PYG{l+m+mi}{1}\PYG{p}{]} \PYG{o}{=} \PYG{n}{v}\PYG{p}{[}\PYG{n}{i}\PYG{p}{]} \PYG{o}{+} \PYG{l+m+mf}{0.5}\PYG{o}{*}\PYG{n}{DeltaT}\PYG{o}{*}\PYG{p}{(}\PYG{n}{a}\PYG{o}{+}\PYG{n}{anew}\PYG{p}{)}
    \PYG{n}{t}\PYG{p}{[}\PYG{n}{i}\PYG{o}{+}\PYG{l+m+mi}{1}\PYG{p}{]} \PYG{o}{=} \PYG{n}{t}\PYG{p}{[}\PYG{n}{i}\PYG{p}{]} \PYG{o}{+} \PYG{n}{DeltaT}
\PYG{c+c1}{\PYGZsh{} Plot position as function of time}
\PYG{n}{radius} \PYG{o}{=} \PYG{n}{np}\PYG{o}{.}\PYG{n}{sqrt}\PYG{p}{(}\PYG{n}{r}\PYG{p}{[}\PYG{p}{:}\PYG{p}{,}\PYG{l+m+mi}{0}\PYG{p}{]}\PYG{o}{*}\PYG{o}{*}\PYG{l+m+mi}{2}\PYG{o}{+}\PYG{n}{r}\PYG{p}{[}\PYG{p}{:}\PYG{p}{,}\PYG{l+m+mi}{1}\PYG{p}{]}\PYG{o}{*}\PYG{o}{*}\PYG{l+m+mi}{2}\PYG{p}{)}
\PYG{n}{fig}\PYG{p}{,} \PYG{n}{ax} \PYG{o}{=} \PYG{n}{plt}\PYG{o}{.}\PYG{n}{subplots}\PYG{p}{(}\PYG{l+m+mi}{3}\PYG{p}{,}\PYG{l+m+mi}{1}\PYG{p}{)}
\PYG{n}{ax}\PYG{p}{[}\PYG{l+m+mi}{0}\PYG{p}{]}\PYG{o}{.}\PYG{n}{set\PYGZus{}xlabel}\PYG{p}{(}\PYG{l+s+s1}{\PYGZsq{}}\PYG{l+s+s1}{time}\PYG{l+s+s1}{\PYGZsq{}}\PYG{p}{)}
\PYG{n}{ax}\PYG{p}{[}\PYG{l+m+mi}{0}\PYG{p}{]}\PYG{o}{.}\PYG{n}{set\PYGZus{}ylabel}\PYG{p}{(}\PYG{l+s+s1}{\PYGZsq{}}\PYG{l+s+s1}{radius squared}\PYG{l+s+s1}{\PYGZsq{}}\PYG{p}{)}
\PYG{n}{ax}\PYG{p}{[}\PYG{l+m+mi}{0}\PYG{p}{]}\PYG{o}{.}\PYG{n}{plot}\PYG{p}{(}\PYG{n}{t}\PYG{p}{,}\PYG{n}{r}\PYG{p}{[}\PYG{p}{:}\PYG{p}{,}\PYG{l+m+mi}{0}\PYG{p}{]}\PYG{o}{*}\PYG{o}{*}\PYG{l+m+mi}{2}\PYG{o}{+}\PYG{n}{r}\PYG{p}{[}\PYG{p}{:}\PYG{p}{,}\PYG{l+m+mi}{1}\PYG{p}{]}\PYG{o}{*}\PYG{o}{*}\PYG{l+m+mi}{2}\PYG{p}{)}
\PYG{n}{ax}\PYG{p}{[}\PYG{l+m+mi}{1}\PYG{p}{]}\PYG{o}{.}\PYG{n}{set\PYGZus{}xlabel}\PYG{p}{(}\PYG{l+s+s1}{\PYGZsq{}}\PYG{l+s+s1}{time}\PYG{l+s+s1}{\PYGZsq{}}\PYG{p}{)}
\PYG{n}{ax}\PYG{p}{[}\PYG{l+m+mi}{1}\PYG{p}{]}\PYG{o}{.}\PYG{n}{set\PYGZus{}ylabel}\PYG{p}{(}\PYG{l+s+s1}{\PYGZsq{}}\PYG{l+s+s1}{x position}\PYG{l+s+s1}{\PYGZsq{}}\PYG{p}{)}
\PYG{n}{ax}\PYG{p}{[}\PYG{l+m+mi}{1}\PYG{p}{]}\PYG{o}{.}\PYG{n}{plot}\PYG{p}{(}\PYG{n}{t}\PYG{p}{,}\PYG{n}{r}\PYG{p}{[}\PYG{p}{:}\PYG{p}{,}\PYG{l+m+mi}{0}\PYG{p}{]}\PYG{p}{)}
\PYG{n}{ax}\PYG{p}{[}\PYG{l+m+mi}{2}\PYG{p}{]}\PYG{o}{.}\PYG{n}{set\PYGZus{}xlabel}\PYG{p}{(}\PYG{l+s+s1}{\PYGZsq{}}\PYG{l+s+s1}{time}\PYG{l+s+s1}{\PYGZsq{}}\PYG{p}{)}
\PYG{n}{ax}\PYG{p}{[}\PYG{l+m+mi}{2}\PYG{p}{]}\PYG{o}{.}\PYG{n}{set\PYGZus{}ylabel}\PYG{p}{(}\PYG{l+s+s1}{\PYGZsq{}}\PYG{l+s+s1}{y position}\PYG{l+s+s1}{\PYGZsq{}}\PYG{p}{)}
\PYG{n}{ax}\PYG{p}{[}\PYG{l+m+mi}{2}\PYG{p}{]}\PYG{o}{.}\PYG{n}{plot}\PYG{p}{(}\PYG{n}{t}\PYG{p}{,}\PYG{n}{r}\PYG{p}{[}\PYG{p}{:}\PYG{p}{,}\PYG{l+m+mi}{1}\PYG{p}{]}\PYG{p}{)}

\PYG{n}{fig}\PYG{o}{.}\PYG{n}{tight\PYGZus{}layout}\PYG{p}{(}\PYG{p}{)}
\PYG{n}{save\PYGZus{}fig}\PYG{p}{(}\PYG{l+s+s2}{\PYGZdq{}}\PYG{l+s+s2}{2DimHOVV}\PYG{l+s+s2}{\PYGZdq{}}\PYG{p}{)}
\PYG{n}{plt}\PYG{o}{.}\PYG{n}{show}\PYG{p}{(}\PYG{p}{)}
\end{sphinxVerbatim}

With some work using double angle formulas, one can calculate
\begin{equation*}
\begin{split}
\begin{eqnarray*}
r^2&=&x^2+y^2\\
\nonumber
&=&(A^2+C^2)\cos^2(\omega_0t)+(B^2+D^2)\sin^2\omega_0t+(AB+CD)\cos(\omega_0t)\sin(\omega_0t)\\
\nonumber
&=&\alpha+\beta\cos 2\omega_0 t+\gamma\sin 2\omega_0 t,\\
\alpha&=&\frac{A^2+B^2+C^2+D^2}{2},~~\beta=\frac{A^2-B^2+C^2-D^2}{2},~~\gamma=AB+CD,\\
r^2&=&\alpha+(\beta^2+\gamma^2)^{1/2}\cos(2\omega_0 t-\delta),~~~\delta=\arctan(\gamma/\beta),
\end{eqnarray*}
\end{split}
\end{equation*}
and see that radius oscillates with frequency \(2\omega_0\). The
factor of two comes because the oscillation \(x=A\cos\omega_0t\) has two
maxima for \(x^2\), one at \(t=0\) and one a half period later.

The following code shows first how we can solve this problem using the radial degrees of freedom only.

\begin{sphinxVerbatim}[commandchars=\\\{\}]
\PYG{n}{DeltaT} \PYG{o}{=} \PYG{l+m+mf}{0.01}
\PYG{c+c1}{\PYGZsh{}set up arrays }
\PYG{n}{tfinal} \PYG{o}{=} \PYG{l+m+mf}{10.0}
\PYG{n}{n} \PYG{o}{=} \PYG{n}{ceil}\PYG{p}{(}\PYG{n}{tfinal}\PYG{o}{/}\PYG{n}{DeltaT}\PYG{p}{)}
\PYG{c+c1}{\PYGZsh{} set up arrays for t, v and r}
\PYG{n}{t} \PYG{o}{=} \PYG{n}{np}\PYG{o}{.}\PYG{n}{zeros}\PYG{p}{(}\PYG{n}{n}\PYG{p}{)}
\PYG{n}{v} \PYG{o}{=} \PYG{n}{np}\PYG{o}{.}\PYG{n}{zeros}\PYG{p}{(}\PYG{n}{n}\PYG{p}{)}
\PYG{n}{r} \PYG{o}{=} \PYG{n}{np}\PYG{o}{.}\PYG{n}{zeros}\PYG{p}{(}\PYG{n}{n}\PYG{p}{)}
\PYG{n}{E} \PYG{o}{=} \PYG{n}{np}\PYG{o}{.}\PYG{n}{zeros}\PYG{p}{(}\PYG{n}{n}\PYG{p}{)}
\PYG{c+c1}{\PYGZsh{} Constants of the model}
\PYG{n}{AngMom} \PYG{o}{=} \PYG{l+m+mf}{1.0}  \PYG{c+c1}{\PYGZsh{}  The angular momentum}
\PYG{n}{m} \PYG{o}{=} \PYG{l+m+mf}{0.1}
\PYG{n}{k} \PYG{o}{=} \PYG{l+m+mf}{0.1}
\PYG{n}{omega02} \PYG{o}{=} \PYG{n}{k}\PYG{o}{/}\PYG{n}{m}
\PYG{n}{c1} \PYG{o}{=} \PYG{n}{AngMom}\PYG{o}{*}\PYG{n}{AngMom}\PYG{o}{/}\PYG{p}{(}\PYG{n}{m}\PYG{o}{*}\PYG{n}{m}\PYG{p}{)}
\PYG{n}{c2} \PYG{o}{=} \PYG{n}{AngMom}\PYG{o}{*}\PYG{n}{AngMom}\PYG{o}{/}\PYG{n}{m}
\PYG{n}{rmin} \PYG{o}{=} \PYG{p}{(}\PYG{n}{AngMom}\PYG{o}{*}\PYG{n}{AngMom}\PYG{o}{/}\PYG{n}{k}\PYG{o}{/}\PYG{n}{m}\PYG{p}{)}\PYG{o}{*}\PYG{o}{*}\PYG{l+m+mf}{0.25}
\PYG{c+c1}{\PYGZsh{} Initial conditions}
\PYG{n}{r0} \PYG{o}{=} \PYG{n}{rmin}
\PYG{n}{v0} \PYG{o}{=} \PYG{l+m+mf}{0.0}
\PYG{n}{r}\PYG{p}{[}\PYG{l+m+mi}{0}\PYG{p}{]} \PYG{o}{=} \PYG{n}{r0}
\PYG{n}{v}\PYG{p}{[}\PYG{l+m+mi}{0}\PYG{p}{]} \PYG{o}{=} \PYG{n}{v0}
\PYG{n}{E}\PYG{p}{[}\PYG{l+m+mi}{0}\PYG{p}{]} \PYG{o}{=} \PYG{l+m+mf}{0.5}\PYG{o}{*}\PYG{n}{m}\PYG{o}{*}\PYG{n}{v0}\PYG{o}{*}\PYG{n}{v0}\PYG{o}{+}\PYG{l+m+mf}{0.5}\PYG{o}{*}\PYG{n}{k}\PYG{o}{*}\PYG{n}{r0}\PYG{o}{*}\PYG{n}{r0}\PYG{o}{+}\PYG{l+m+mf}{0.5}\PYG{o}{*}\PYG{n}{c2}\PYG{o}{/}\PYG{p}{(}\PYG{n}{r0}\PYG{o}{*}\PYG{n}{r0}\PYG{p}{)}
\PYG{c+c1}{\PYGZsh{} Start integrating using the Velocity\PYGZhy{}Verlet  method}
\PYG{k}{for} \PYG{n}{i} \PYG{o+ow}{in} \PYG{n+nb}{range}\PYG{p}{(}\PYG{n}{n}\PYG{o}{\PYGZhy{}}\PYG{l+m+mi}{1}\PYG{p}{)}\PYG{p}{:}
    \PYG{c+c1}{\PYGZsh{} Set up acceleration}
    \PYG{n}{a} \PYG{o}{=} \PYG{o}{\PYGZhy{}}\PYG{n}{r}\PYG{p}{[}\PYG{n}{i}\PYG{p}{]}\PYG{o}{*}\PYG{n}{omega02}\PYG{o}{+}\PYG{n}{c1}\PYG{o}{/}\PYG{p}{(}\PYG{n}{r}\PYG{p}{[}\PYG{n}{i}\PYG{p}{]}\PYG{o}{*}\PYG{o}{*}\PYG{l+m+mi}{3}\PYG{p}{)}    
    \PYG{c+c1}{\PYGZsh{} update velocity, time and position using the Velocity\PYGZhy{}Verlet method}
    \PYG{n}{r}\PYG{p}{[}\PYG{n}{i}\PYG{o}{+}\PYG{l+m+mi}{1}\PYG{p}{]} \PYG{o}{=} \PYG{n}{r}\PYG{p}{[}\PYG{n}{i}\PYG{p}{]} \PYG{o}{+} \PYG{n}{DeltaT}\PYG{o}{*}\PYG{n}{v}\PYG{p}{[}\PYG{n}{i}\PYG{p}{]}\PYG{o}{+}\PYG{l+m+mf}{0.5}\PYG{o}{*}\PYG{p}{(}\PYG{n}{DeltaT}\PYG{o}{*}\PYG{o}{*}\PYG{l+m+mi}{2}\PYG{p}{)}\PYG{o}{*}\PYG{n}{a}
    \PYG{n}{anew} \PYG{o}{=} \PYG{o}{\PYGZhy{}}\PYG{n}{r}\PYG{p}{[}\PYG{n}{i}\PYG{o}{+}\PYG{l+m+mi}{1}\PYG{p}{]}\PYG{o}{*}\PYG{n}{omega02}\PYG{o}{+}\PYG{n}{c1}\PYG{o}{/}\PYG{p}{(}\PYG{n}{r}\PYG{p}{[}\PYG{n}{i}\PYG{o}{+}\PYG{l+m+mi}{1}\PYG{p}{]}\PYG{o}{*}\PYG{o}{*}\PYG{l+m+mi}{3}\PYG{p}{)}
    \PYG{n}{v}\PYG{p}{[}\PYG{n}{i}\PYG{o}{+}\PYG{l+m+mi}{1}\PYG{p}{]} \PYG{o}{=} \PYG{n}{v}\PYG{p}{[}\PYG{n}{i}\PYG{p}{]} \PYG{o}{+} \PYG{l+m+mf}{0.5}\PYG{o}{*}\PYG{n}{DeltaT}\PYG{o}{*}\PYG{p}{(}\PYG{n}{a}\PYG{o}{+}\PYG{n}{anew}\PYG{p}{)}
    \PYG{n}{t}\PYG{p}{[}\PYG{n}{i}\PYG{o}{+}\PYG{l+m+mi}{1}\PYG{p}{]} \PYG{o}{=} \PYG{n}{t}\PYG{p}{[}\PYG{n}{i}\PYG{p}{]} \PYG{o}{+} \PYG{n}{DeltaT}
    \PYG{n}{E}\PYG{p}{[}\PYG{n}{i}\PYG{o}{+}\PYG{l+m+mi}{1}\PYG{p}{]} \PYG{o}{=} \PYG{l+m+mf}{0.5}\PYG{o}{*}\PYG{n}{m}\PYG{o}{*}\PYG{n}{v}\PYG{p}{[}\PYG{n}{i}\PYG{o}{+}\PYG{l+m+mi}{1}\PYG{p}{]}\PYG{o}{*}\PYG{n}{v}\PYG{p}{[}\PYG{n}{i}\PYG{o}{+}\PYG{l+m+mi}{1}\PYG{p}{]}\PYG{o}{+}\PYG{l+m+mf}{0.5}\PYG{o}{*}\PYG{n}{k}\PYG{o}{*}\PYG{n}{r}\PYG{p}{[}\PYG{n}{i}\PYG{o}{+}\PYG{l+m+mi}{1}\PYG{p}{]}\PYG{o}{*}\PYG{n}{r}\PYG{p}{[}\PYG{n}{i}\PYG{o}{+}\PYG{l+m+mi}{1}\PYG{p}{]}\PYG{o}{+}\PYG{l+m+mf}{0.5}\PYG{o}{*}\PYG{n}{c2}\PYG{o}{/}\PYG{p}{(}\PYG{n}{r}\PYG{p}{[}\PYG{n}{i}\PYG{o}{+}\PYG{l+m+mi}{1}\PYG{p}{]}\PYG{o}{*}\PYG{n}{r}\PYG{p}{[}\PYG{n}{i}\PYG{o}{+}\PYG{l+m+mi}{1}\PYG{p}{]}\PYG{p}{)}
    \PYG{c+c1}{\PYGZsh{} Plot position as function of time}
\PYG{n}{fig}\PYG{p}{,} \PYG{n}{ax} \PYG{o}{=} \PYG{n}{plt}\PYG{o}{.}\PYG{n}{subplots}\PYG{p}{(}\PYG{l+m+mi}{2}\PYG{p}{,}\PYG{l+m+mi}{1}\PYG{p}{)}
\PYG{n}{ax}\PYG{p}{[}\PYG{l+m+mi}{0}\PYG{p}{]}\PYG{o}{.}\PYG{n}{set\PYGZus{}xlabel}\PYG{p}{(}\PYG{l+s+s1}{\PYGZsq{}}\PYG{l+s+s1}{time}\PYG{l+s+s1}{\PYGZsq{}}\PYG{p}{)}
\PYG{n}{ax}\PYG{p}{[}\PYG{l+m+mi}{0}\PYG{p}{]}\PYG{o}{.}\PYG{n}{set\PYGZus{}ylabel}\PYG{p}{(}\PYG{l+s+s1}{\PYGZsq{}}\PYG{l+s+s1}{radius}\PYG{l+s+s1}{\PYGZsq{}}\PYG{p}{)}
\PYG{n}{ax}\PYG{p}{[}\PYG{l+m+mi}{0}\PYG{p}{]}\PYG{o}{.}\PYG{n}{plot}\PYG{p}{(}\PYG{n}{t}\PYG{p}{,}\PYG{n}{r}\PYG{p}{)}
\PYG{n}{ax}\PYG{p}{[}\PYG{l+m+mi}{1}\PYG{p}{]}\PYG{o}{.}\PYG{n}{set\PYGZus{}xlabel}\PYG{p}{(}\PYG{l+s+s1}{\PYGZsq{}}\PYG{l+s+s1}{time}\PYG{l+s+s1}{\PYGZsq{}}\PYG{p}{)}
\PYG{n}{ax}\PYG{p}{[}\PYG{l+m+mi}{1}\PYG{p}{]}\PYG{o}{.}\PYG{n}{set\PYGZus{}ylabel}\PYG{p}{(}\PYG{l+s+s1}{\PYGZsq{}}\PYG{l+s+s1}{Energy}\PYG{l+s+s1}{\PYGZsq{}}\PYG{p}{)}
\PYG{n}{ax}\PYG{p}{[}\PYG{l+m+mi}{1}\PYG{p}{]}\PYG{o}{.}\PYG{n}{plot}\PYG{p}{(}\PYG{n}{t}\PYG{p}{,}\PYG{n}{E}\PYG{p}{)}
\PYG{n}{save\PYGZus{}fig}\PYG{p}{(}\PYG{l+s+s2}{\PYGZdq{}}\PYG{l+s+s2}{RadialHOVV}\PYG{l+s+s2}{\PYGZdq{}}\PYG{p}{)}
\PYG{n}{plt}\PYG{o}{.}\PYG{n}{show}\PYG{p}{(}\PYG{p}{)}
\end{sphinxVerbatim}


\section{Stability of Orbits}
\label{\detokenize{chapter1:stability-of-orbits}}
The effective force can be extracted from the effective potential, \(U_{\rm eff}\). Beginning from the equations of motion, Eq. ({\hyperref[\detokenize{chapter1:eq:radialeqofmotion}]{\emph{98}}}), for \(r\),
\begin{equation*}
\begin{split}
\begin{eqnarray}
m\ddot{r}&=&F+\frac{L^2}{mr^3}\\
\nonumber
&=&F_{\rm eff}\\
\nonumber
&=&-\partial_rU_{\rm eff},\\
\nonumber
F_{\rm eff}&=&-\partial_r\left[U(r)+(L^2/2mr^2)\right].
\end{eqnarray}
\end{split}
\end{equation*}
For a circular orbit, the radius must be fixed as a function of time,
so one must be at a maximum or a minimum of the effective
potential. However, if one is at a maximum of the effective potential
the radius will be unstable. For the attractive Coulomb force the
effective potential will be dominated by the \(-\alpha/r\) term for
large \(r\) because the centrifugal part falls off more quickly, \(\sim
1/r^2\). At low \(r\) the centrifugal piece wins and the effective
potential is repulsive. Thus, the potential must have a minimum
somewhere with negative potential. The circular orbits are then stable
to perturbation.

The effective potential is sketched for two cases, a \(1/r\) attractive
potential and a \(1/r^3\) attractive potential. The \(1/r\) case has a
stable minimum, whereas the circular orbit in the \(1/r^3\) case is
unstable.

If one considers a potential that falls as \(1/r^3\), the situation is
reversed and the point where \(\partial_rU\) disappears will be a local
maximum rather than a local minimum. \sphinxstylestrong{Fig to come here with code}

The repulsive centrifugal piece dominates at large \(r\) and the attractive
Coulomb piece wins out at small \(r\). The circular orbit is then at a
maximum of the effective potential and the orbits are unstable. It is
the clear that for potentials that fall as \(r^n\), that one must have
\(n>-2\) for the orbits to be stable.

Consider a potential \(U(r)=\beta r\). For a particle of mass \(m\) with
angular momentum \(L\), find the angular frequency of a circular
orbit. Then find the angular frequency for small radial perturbations.

For the circular orbit you search for the position \(r_{\rm min}\) where the effective potential is minimized,
\begin{equation*}
\begin{split}
\begin{eqnarray*}
\partial_r\left\{\beta r+\frac{L^2}{2mr^2}\right\}&=&0,\\
\beta&=&\frac{L^2}{mr_{\rm min}^3},\\
r_{\rm min}&=&\left(\frac{L^2}{\beta m}\right)^{1/3},\\
\dot{\theta}&=&\frac{L}{mr_{\rm min}^2}=\frac{\beta^{2/3}}{(mL)^{1/3}}
\end{eqnarray*}
\end{split}
\end{equation*}
Now, we can find the angular frequency of small perturbations about the circular orbit. To do this we find the effective spring constant for the effective potential,
\begin{equation*}
\begin{split}
\begin{eqnarray*}
k_{\rm eff}&=&\partial_r^2 \left.U_{\rm eff}\right|_{r_{\rm min}}\\
&=&\frac{3L^2}{mr_{\rm min}^4},\\
\omega&=&\sqrt{\frac{k_{\rm eff}}{m}}\\
&=&\frac{\beta^{2/3}}{(mL)^{1/3}}\sqrt{3}.
\end{eqnarray*}
\end{split}
\end{equation*}
If the two frequencies, \(\dot{\theta}\) and \(\omega\), differ by an
integer factor, the orbit’s trajectory will repeat itself each time
around. This is the case for the inverse\sphinxhyphen{}square force,
\(\omega=\dot{\theta}\), and for the harmonic oscillator,
\(\omega=2\dot{\theta}\). In this case, \(\omega=\sqrt{3}\dot{\theta}\),
and the angles at which the maxima and minima occur change with each
orbit.


\subsection{Code example with gravitional force}
\label{\detokenize{chapter1:code-example-with-gravitional-force}}
The code example here is meant to illustrate how we can make a plot of the final orbit. We solve the equations in polar coordinates (the example here uses the minimum of the potential as initial value) and then we transform back to cartesian coordinates and plot \(x\) versus \(y\). We see that we get a perfect circle when we place ourselves at the minimum of the potential energy, as expected.

\begin{sphinxVerbatim}[commandchars=\\\{\}]
\PYG{c+c1}{\PYGZsh{} Simple Gravitational Force   \PYGZhy{}alpha/r}
    
\PYG{n}{DeltaT} \PYG{o}{=} \PYG{l+m+mf}{0.01}
\PYG{c+c1}{\PYGZsh{}set up arrays }
\PYG{n}{tfinal} \PYG{o}{=} \PYG{l+m+mf}{8.0}
\PYG{n}{n} \PYG{o}{=} \PYG{n}{ceil}\PYG{p}{(}\PYG{n}{tfinal}\PYG{o}{/}\PYG{n}{DeltaT}\PYG{p}{)}
\PYG{c+c1}{\PYGZsh{} set up arrays for t, v and r}
\PYG{n}{t} \PYG{o}{=} \PYG{n}{np}\PYG{o}{.}\PYG{n}{zeros}\PYG{p}{(}\PYG{n}{n}\PYG{p}{)}
\PYG{n}{v} \PYG{o}{=} \PYG{n}{np}\PYG{o}{.}\PYG{n}{zeros}\PYG{p}{(}\PYG{n}{n}\PYG{p}{)}
\PYG{n}{r} \PYG{o}{=} \PYG{n}{np}\PYG{o}{.}\PYG{n}{zeros}\PYG{p}{(}\PYG{n}{n}\PYG{p}{)}
\PYG{n}{phi} \PYG{o}{=} \PYG{n}{np}\PYG{o}{.}\PYG{n}{zeros}\PYG{p}{(}\PYG{n}{n}\PYG{p}{)}
\PYG{n}{x} \PYG{o}{=} \PYG{n}{np}\PYG{o}{.}\PYG{n}{zeros}\PYG{p}{(}\PYG{n}{n}\PYG{p}{)}
\PYG{n}{y} \PYG{o}{=} \PYG{n}{np}\PYG{o}{.}\PYG{n}{zeros}\PYG{p}{(}\PYG{n}{n}\PYG{p}{)}
\PYG{c+c1}{\PYGZsh{} Constants of the model, setting all variables to one for simplicity}
\PYG{n}{alpha} \PYG{o}{=} \PYG{l+m+mf}{1.0}
\PYG{n}{AngMom} \PYG{o}{=} \PYG{l+m+mf}{1.0}  \PYG{c+c1}{\PYGZsh{}  The angular momentum}
\PYG{n}{m} \PYG{o}{=} \PYG{l+m+mf}{1.0}  \PYG{c+c1}{\PYGZsh{} scale mass to one}
\PYG{n}{c1} \PYG{o}{=} \PYG{n}{AngMom}\PYG{o}{*}\PYG{n}{AngMom}\PYG{o}{/}\PYG{p}{(}\PYG{n}{m}\PYG{o}{*}\PYG{n}{m}\PYG{p}{)}
\PYG{n}{c2} \PYG{o}{=} \PYG{n}{AngMom}\PYG{o}{*}\PYG{n}{AngMom}\PYG{o}{/}\PYG{n}{m}
\PYG{n}{rmin} \PYG{o}{=} \PYG{p}{(}\PYG{n}{AngMom}\PYG{o}{*}\PYG{n}{AngMom}\PYG{o}{/}\PYG{n}{m}\PYG{o}{/}\PYG{n}{alpha}\PYG{p}{)}
\PYG{c+c1}{\PYGZsh{} Initial conditions, place yourself at the potential min}
\PYG{n}{r0} \PYG{o}{=} \PYG{n}{rmin}
\PYG{n}{v0} \PYG{o}{=} \PYG{l+m+mf}{0.0}  \PYG{c+c1}{\PYGZsh{} starts at rest}
\PYG{n}{r}\PYG{p}{[}\PYG{l+m+mi}{0}\PYG{p}{]} \PYG{o}{=} \PYG{n}{r0}
\PYG{n}{v}\PYG{p}{[}\PYG{l+m+mi}{0}\PYG{p}{]} \PYG{o}{=} \PYG{n}{v0}
\PYG{n}{phi}\PYG{p}{[}\PYG{l+m+mi}{0}\PYG{p}{]} \PYG{o}{=} \PYG{l+m+mf}{0.0}
\PYG{c+c1}{\PYGZsh{} Start integrating using the Velocity\PYGZhy{}Verlet  method}
\PYG{k}{for} \PYG{n}{i} \PYG{o+ow}{in} \PYG{n+nb}{range}\PYG{p}{(}\PYG{n}{n}\PYG{o}{\PYGZhy{}}\PYG{l+m+mi}{1}\PYG{p}{)}\PYG{p}{:}
    \PYG{c+c1}{\PYGZsh{} Set up acceleration}
    \PYG{n}{a} \PYG{o}{=} \PYG{o}{\PYGZhy{}}\PYG{n}{alpha}\PYG{o}{/}\PYG{p}{(}\PYG{n}{r}\PYG{p}{[}\PYG{n}{i}\PYG{p}{]}\PYG{o}{*}\PYG{o}{*}\PYG{l+m+mi}{2}\PYG{p}{)}\PYG{o}{+}\PYG{n}{c1}\PYG{o}{/}\PYG{p}{(}\PYG{n}{r}\PYG{p}{[}\PYG{n}{i}\PYG{p}{]}\PYG{o}{*}\PYG{o}{*}\PYG{l+m+mi}{3}\PYG{p}{)}
    \PYG{c+c1}{\PYGZsh{} update velocity, time and position using the Velocity\PYGZhy{}Verlet method}
    \PYG{n}{r}\PYG{p}{[}\PYG{n}{i}\PYG{o}{+}\PYG{l+m+mi}{1}\PYG{p}{]} \PYG{o}{=} \PYG{n}{r}\PYG{p}{[}\PYG{n}{i}\PYG{p}{]} \PYG{o}{+} \PYG{n}{DeltaT}\PYG{o}{*}\PYG{n}{v}\PYG{p}{[}\PYG{n}{i}\PYG{p}{]}\PYG{o}{+}\PYG{l+m+mf}{0.5}\PYG{o}{*}\PYG{p}{(}\PYG{n}{DeltaT}\PYG{o}{*}\PYG{o}{*}\PYG{l+m+mi}{2}\PYG{p}{)}\PYG{o}{*}\PYG{n}{a}
    \PYG{n}{anew} \PYG{o}{=} \PYG{o}{\PYGZhy{}}\PYG{n}{alpha}\PYG{o}{/}\PYG{p}{(}\PYG{n}{r}\PYG{p}{[}\PYG{n}{i}\PYG{o}{+}\PYG{l+m+mi}{1}\PYG{p}{]}\PYG{o}{*}\PYG{o}{*}\PYG{l+m+mi}{2}\PYG{p}{)}\PYG{o}{+}\PYG{n}{c1}\PYG{o}{/}\PYG{p}{(}\PYG{n}{r}\PYG{p}{[}\PYG{n}{i}\PYG{o}{+}\PYG{l+m+mi}{1}\PYG{p}{]}\PYG{o}{*}\PYG{o}{*}\PYG{l+m+mi}{3}\PYG{p}{)}
    \PYG{n}{v}\PYG{p}{[}\PYG{n}{i}\PYG{o}{+}\PYG{l+m+mi}{1}\PYG{p}{]} \PYG{o}{=} \PYG{n}{v}\PYG{p}{[}\PYG{n}{i}\PYG{p}{]} \PYG{o}{+} \PYG{l+m+mf}{0.5}\PYG{o}{*}\PYG{n}{DeltaT}\PYG{o}{*}\PYG{p}{(}\PYG{n}{a}\PYG{o}{+}\PYG{n}{anew}\PYG{p}{)}
    \PYG{n}{t}\PYG{p}{[}\PYG{n}{i}\PYG{o}{+}\PYG{l+m+mi}{1}\PYG{p}{]} \PYG{o}{=} \PYG{n}{t}\PYG{p}{[}\PYG{n}{i}\PYG{p}{]} \PYG{o}{+} \PYG{n}{DeltaT}
    \PYG{n}{phi}\PYG{p}{[}\PYG{n}{i}\PYG{o}{+}\PYG{l+m+mi}{1}\PYG{p}{]} \PYG{o}{=} \PYG{n}{t}\PYG{p}{[}\PYG{n}{i}\PYG{o}{+}\PYG{l+m+mi}{1}\PYG{p}{]}\PYG{o}{*}\PYG{n}{c2}\PYG{o}{/}\PYG{p}{(}\PYG{n}{r0}\PYG{o}{*}\PYG{o}{*}\PYG{l+m+mi}{2}\PYG{p}{)}
\PYG{c+c1}{\PYGZsh{} Find cartesian coordinates for easy plot    }
\PYG{n}{x} \PYG{o}{=} \PYG{n}{r}\PYG{o}{*}\PYG{n}{np}\PYG{o}{.}\PYG{n}{cos}\PYG{p}{(}\PYG{n}{phi}\PYG{p}{)}
\PYG{n}{y} \PYG{o}{=} \PYG{n}{r}\PYG{o}{*}\PYG{n}{np}\PYG{o}{.}\PYG{n}{sin}\PYG{p}{(}\PYG{n}{phi}\PYG{p}{)}
\PYG{n}{fig}\PYG{p}{,} \PYG{n}{ax} \PYG{o}{=} \PYG{n}{plt}\PYG{o}{.}\PYG{n}{subplots}\PYG{p}{(}\PYG{l+m+mi}{3}\PYG{p}{,}\PYG{l+m+mi}{1}\PYG{p}{)}
\PYG{n}{ax}\PYG{p}{[}\PYG{l+m+mi}{0}\PYG{p}{]}\PYG{o}{.}\PYG{n}{set\PYGZus{}xlabel}\PYG{p}{(}\PYG{l+s+s1}{\PYGZsq{}}\PYG{l+s+s1}{time}\PYG{l+s+s1}{\PYGZsq{}}\PYG{p}{)}
\PYG{n}{ax}\PYG{p}{[}\PYG{l+m+mi}{0}\PYG{p}{]}\PYG{o}{.}\PYG{n}{set\PYGZus{}ylabel}\PYG{p}{(}\PYG{l+s+s1}{\PYGZsq{}}\PYG{l+s+s1}{radius}\PYG{l+s+s1}{\PYGZsq{}}\PYG{p}{)}
\PYG{n}{ax}\PYG{p}{[}\PYG{l+m+mi}{0}\PYG{p}{]}\PYG{o}{.}\PYG{n}{plot}\PYG{p}{(}\PYG{n}{t}\PYG{p}{,}\PYG{n}{r}\PYG{p}{)}
\PYG{n}{ax}\PYG{p}{[}\PYG{l+m+mi}{1}\PYG{p}{]}\PYG{o}{.}\PYG{n}{set\PYGZus{}xlabel}\PYG{p}{(}\PYG{l+s+s1}{\PYGZsq{}}\PYG{l+s+s1}{time}\PYG{l+s+s1}{\PYGZsq{}}\PYG{p}{)}
\PYG{n}{ax}\PYG{p}{[}\PYG{l+m+mi}{1}\PYG{p}{]}\PYG{o}{.}\PYG{n}{set\PYGZus{}ylabel}\PYG{p}{(}\PYG{l+s+s1}{\PYGZsq{}}\PYG{l+s+s1}{Angle \PYGZdl{}}\PYG{l+s+s1}{\PYGZbs{}}\PYG{l+s+s1}{cos}\PYG{l+s+s1}{\PYGZob{}}\PYG{l+s+s1}{\PYGZbs{}}\PYG{l+s+s1}{phi\PYGZcb{}\PYGZdl{}}\PYG{l+s+s1}{\PYGZsq{}}\PYG{p}{)}
\PYG{n}{ax}\PYG{p}{[}\PYG{l+m+mi}{1}\PYG{p}{]}\PYG{o}{.}\PYG{n}{plot}\PYG{p}{(}\PYG{n}{t}\PYG{p}{,}\PYG{n}{np}\PYG{o}{.}\PYG{n}{cos}\PYG{p}{(}\PYG{n}{phi}\PYG{p}{)}\PYG{p}{)}
\PYG{n}{ax}\PYG{p}{[}\PYG{l+m+mi}{2}\PYG{p}{]}\PYG{o}{.}\PYG{n}{set\PYGZus{}ylabel}\PYG{p}{(}\PYG{l+s+s1}{\PYGZsq{}}\PYG{l+s+s1}{y}\PYG{l+s+s1}{\PYGZsq{}}\PYG{p}{)}
\PYG{n}{ax}\PYG{p}{[}\PYG{l+m+mi}{2}\PYG{p}{]}\PYG{o}{.}\PYG{n}{set\PYGZus{}xlabel}\PYG{p}{(}\PYG{l+s+s1}{\PYGZsq{}}\PYG{l+s+s1}{x}\PYG{l+s+s1}{\PYGZsq{}}\PYG{p}{)}
\PYG{n}{ax}\PYG{p}{[}\PYG{l+m+mi}{2}\PYG{p}{]}\PYG{o}{.}\PYG{n}{plot}\PYG{p}{(}\PYG{n}{x}\PYG{p}{,}\PYG{n}{y}\PYG{p}{)}

\PYG{n}{save\PYGZus{}fig}\PYG{p}{(}\PYG{l+s+s2}{\PYGZdq{}}\PYG{l+s+s2}{Phasespace}\PYG{l+s+s2}{\PYGZdq{}}\PYG{p}{)}
\PYG{n}{plt}\PYG{o}{.}\PYG{n}{show}\PYG{p}{(}\PYG{p}{)}
\end{sphinxVerbatim}

Try to change the initial value for \(r\) and see what kind of orbits you get.
In order to test different energies, it can be useful to look at the plot of the effective potential discussed above.

However, for orbits different from a circle the above code would need modifications in order to allow us to display say an ellipse. For the latter, it is much easier to run our code in cartesian coordinates, as done here. In this code we test also energy conservation and see that it is conserved to numerical precision. The code here is a simple extension of the code we developed for homework 4.

\begin{sphinxVerbatim}[commandchars=\\\{\}]
\PYG{c+c1}{\PYGZsh{} Common imports}
\PYG{k+kn}{import} \PYG{n+nn}{numpy} \PYG{k}{as} \PYG{n+nn}{np}
\PYG{k+kn}{import} \PYG{n+nn}{pandas} \PYG{k}{as} \PYG{n+nn}{pd}
\PYG{k+kn}{from} \PYG{n+nn}{math} \PYG{k+kn}{import} \PYG{o}{*}
\PYG{k+kn}{import} \PYG{n+nn}{matplotlib}\PYG{n+nn}{.}\PYG{n+nn}{pyplot} \PYG{k}{as} \PYG{n+nn}{plt}

\PYG{n}{DeltaT} \PYG{o}{=} \PYG{l+m+mf}{0.01}
\PYG{c+c1}{\PYGZsh{}set up arrays }
\PYG{n}{tfinal} \PYG{o}{=} \PYG{l+m+mf}{10.0}
\PYG{n}{n} \PYG{o}{=} \PYG{n}{ceil}\PYG{p}{(}\PYG{n}{tfinal}\PYG{o}{/}\PYG{n}{DeltaT}\PYG{p}{)}
\PYG{c+c1}{\PYGZsh{} set up arrays}
\PYG{n}{t} \PYG{o}{=} \PYG{n}{np}\PYG{o}{.}\PYG{n}{zeros}\PYG{p}{(}\PYG{n}{n}\PYG{p}{)}
\PYG{n}{v} \PYG{o}{=} \PYG{n}{np}\PYG{o}{.}\PYG{n}{zeros}\PYG{p}{(}\PYG{p}{(}\PYG{n}{n}\PYG{p}{,}\PYG{l+m+mi}{2}\PYG{p}{)}\PYG{p}{)}
\PYG{n}{r} \PYG{o}{=} \PYG{n}{np}\PYG{o}{.}\PYG{n}{zeros}\PYG{p}{(}\PYG{p}{(}\PYG{n}{n}\PYG{p}{,}\PYG{l+m+mi}{2}\PYG{p}{)}\PYG{p}{)}
\PYG{n}{E} \PYG{o}{=} \PYG{n}{np}\PYG{o}{.}\PYG{n}{zeros}\PYG{p}{(}\PYG{n}{n}\PYG{p}{)}
\PYG{c+c1}{\PYGZsh{} Constants of the model}
\PYG{n}{m} \PYG{o}{=} \PYG{l+m+mf}{1.0}   \PYG{c+c1}{\PYGZsh{} mass, you can change these}
\PYG{n}{alpha} \PYG{o}{=} \PYG{l+m+mf}{1.0}
\PYG{c+c1}{\PYGZsh{} Initial conditions as compact 2\PYGZhy{}dimensional arrays}
\PYG{n}{x0} \PYG{o}{=} \PYG{l+m+mf}{0.5}\PYG{p}{;} \PYG{n}{y0}\PYG{o}{=} \PYG{l+m+mf}{0.}
\PYG{n}{r0} \PYG{o}{=} \PYG{n}{np}\PYG{o}{.}\PYG{n}{array}\PYG{p}{(}\PYG{p}{[}\PYG{n}{x0}\PYG{p}{,}\PYG{n}{y0}\PYG{p}{]}\PYG{p}{)} 
\PYG{n}{v0} \PYG{o}{=} \PYG{n}{np}\PYG{o}{.}\PYG{n}{array}\PYG{p}{(}\PYG{p}{[}\PYG{l+m+mf}{0.0}\PYG{p}{,}\PYG{l+m+mf}{1.0}\PYG{p}{]}\PYG{p}{)}
\PYG{n}{r}\PYG{p}{[}\PYG{l+m+mi}{0}\PYG{p}{]} \PYG{o}{=} \PYG{n}{r0}
\PYG{n}{v}\PYG{p}{[}\PYG{l+m+mi}{0}\PYG{p}{]} \PYG{o}{=} \PYG{n}{v0}
\PYG{n}{rabs} \PYG{o}{=} \PYG{n}{sqrt}\PYG{p}{(}\PYG{n+nb}{sum}\PYG{p}{(}\PYG{n}{r}\PYG{p}{[}\PYG{l+m+mi}{0}\PYG{p}{]}\PYG{o}{*}\PYG{n}{r}\PYG{p}{[}\PYG{l+m+mi}{0}\PYG{p}{]}\PYG{p}{)}\PYG{p}{)}
\PYG{n}{E}\PYG{p}{[}\PYG{l+m+mi}{0}\PYG{p}{]} \PYG{o}{=} \PYG{l+m+mf}{0.5}\PYG{o}{*}\PYG{n}{m}\PYG{o}{*}\PYG{p}{(}\PYG{n}{v}\PYG{p}{[}\PYG{l+m+mi}{0}\PYG{p}{,}\PYG{l+m+mi}{0}\PYG{p}{]}\PYG{o}{*}\PYG{o}{*}\PYG{l+m+mi}{2}\PYG{o}{+}\PYG{n}{v}\PYG{p}{[}\PYG{l+m+mi}{0}\PYG{p}{,}\PYG{l+m+mi}{1}\PYG{p}{]}\PYG{o}{*}\PYG{o}{*}\PYG{l+m+mi}{2}\PYG{p}{)}\PYG{o}{\PYGZhy{}}\PYG{n}{alpha}\PYG{o}{/}\PYG{n}{rabs}
\PYG{c+c1}{\PYGZsh{} Start integrating using the Velocity\PYGZhy{}Verlet  method}
\PYG{k}{for} \PYG{n}{i} \PYG{o+ow}{in} \PYG{n+nb}{range}\PYG{p}{(}\PYG{n}{n}\PYG{o}{\PYGZhy{}}\PYG{l+m+mi}{1}\PYG{p}{)}\PYG{p}{:}
    \PYG{c+c1}{\PYGZsh{} Set up the acceleration}
    \PYG{n}{rabs} \PYG{o}{=} \PYG{n}{sqrt}\PYG{p}{(}\PYG{n+nb}{sum}\PYG{p}{(}\PYG{n}{r}\PYG{p}{[}\PYG{n}{i}\PYG{p}{]}\PYG{o}{*}\PYG{n}{r}\PYG{p}{[}\PYG{n}{i}\PYG{p}{]}\PYG{p}{)}\PYG{p}{)}
    \PYG{n}{a} \PYG{o}{=}  \PYG{o}{\PYGZhy{}}\PYG{n}{alpha}\PYG{o}{*}\PYG{n}{r}\PYG{p}{[}\PYG{n}{i}\PYG{p}{]}\PYG{o}{/}\PYG{p}{(}\PYG{n}{rabs}\PYG{o}{*}\PYG{o}{*}\PYG{l+m+mi}{3}\PYG{p}{)}
    \PYG{c+c1}{\PYGZsh{} update velocity, time and position using the Velocity\PYGZhy{}Verlet method}
    \PYG{n}{r}\PYG{p}{[}\PYG{n}{i}\PYG{o}{+}\PYG{l+m+mi}{1}\PYG{p}{]} \PYG{o}{=} \PYG{n}{r}\PYG{p}{[}\PYG{n}{i}\PYG{p}{]} \PYG{o}{+} \PYG{n}{DeltaT}\PYG{o}{*}\PYG{n}{v}\PYG{p}{[}\PYG{n}{i}\PYG{p}{]}\PYG{o}{+}\PYG{l+m+mf}{0.5}\PYG{o}{*}\PYG{p}{(}\PYG{n}{DeltaT}\PYG{o}{*}\PYG{o}{*}\PYG{l+m+mi}{2}\PYG{p}{)}\PYG{o}{*}\PYG{n}{a}
    \PYG{n}{rabs} \PYG{o}{=} \PYG{n}{sqrt}\PYG{p}{(}\PYG{n+nb}{sum}\PYG{p}{(}\PYG{n}{r}\PYG{p}{[}\PYG{n}{i}\PYG{o}{+}\PYG{l+m+mi}{1}\PYG{p}{]}\PYG{o}{*}\PYG{n}{r}\PYG{p}{[}\PYG{n}{i}\PYG{o}{+}\PYG{l+m+mi}{1}\PYG{p}{]}\PYG{p}{)}\PYG{p}{)}
    \PYG{n}{anew} \PYG{o}{=} \PYG{o}{\PYGZhy{}}\PYG{n}{alpha}\PYG{o}{*}\PYG{n}{r}\PYG{p}{[}\PYG{n}{i}\PYG{o}{+}\PYG{l+m+mi}{1}\PYG{p}{]}\PYG{o}{/}\PYG{p}{(}\PYG{n}{rabs}\PYG{o}{*}\PYG{o}{*}\PYG{l+m+mi}{3}\PYG{p}{)}
    \PYG{n}{v}\PYG{p}{[}\PYG{n}{i}\PYG{o}{+}\PYG{l+m+mi}{1}\PYG{p}{]} \PYG{o}{=} \PYG{n}{v}\PYG{p}{[}\PYG{n}{i}\PYG{p}{]} \PYG{o}{+} \PYG{l+m+mf}{0.5}\PYG{o}{*}\PYG{n}{DeltaT}\PYG{o}{*}\PYG{p}{(}\PYG{n}{a}\PYG{o}{+}\PYG{n}{anew}\PYG{p}{)}
    \PYG{n}{E}\PYG{p}{[}\PYG{n}{i}\PYG{o}{+}\PYG{l+m+mi}{1}\PYG{p}{]} \PYG{o}{=} \PYG{l+m+mf}{0.5}\PYG{o}{*}\PYG{n}{m}\PYG{o}{*}\PYG{p}{(}\PYG{n}{v}\PYG{p}{[}\PYG{n}{i}\PYG{o}{+}\PYG{l+m+mi}{1}\PYG{p}{,}\PYG{l+m+mi}{0}\PYG{p}{]}\PYG{o}{*}\PYG{o}{*}\PYG{l+m+mi}{2}\PYG{o}{+}\PYG{n}{v}\PYG{p}{[}\PYG{n}{i}\PYG{o}{+}\PYG{l+m+mi}{1}\PYG{p}{,}\PYG{l+m+mi}{1}\PYG{p}{]}\PYG{o}{*}\PYG{o}{*}\PYG{l+m+mi}{2}\PYG{p}{)}\PYG{o}{\PYGZhy{}}\PYG{n}{alpha}\PYG{o}{/}\PYG{n}{rabs}
    \PYG{n}{t}\PYG{p}{[}\PYG{n}{i}\PYG{o}{+}\PYG{l+m+mi}{1}\PYG{p}{]} \PYG{o}{=} \PYG{n}{t}\PYG{p}{[}\PYG{n}{i}\PYG{p}{]} \PYG{o}{+} \PYG{n}{DeltaT}
\PYG{c+c1}{\PYGZsh{} Plot position as function of time}
\PYG{n}{fig}\PYG{p}{,} \PYG{n}{ax} \PYG{o}{=} \PYG{n}{plt}\PYG{o}{.}\PYG{n}{subplots}\PYG{p}{(}\PYG{l+m+mi}{3}\PYG{p}{,}\PYG{l+m+mi}{1}\PYG{p}{)}
\PYG{n}{ax}\PYG{p}{[}\PYG{l+m+mi}{0}\PYG{p}{]}\PYG{o}{.}\PYG{n}{set\PYGZus{}ylabel}\PYG{p}{(}\PYG{l+s+s1}{\PYGZsq{}}\PYG{l+s+s1}{y}\PYG{l+s+s1}{\PYGZsq{}}\PYG{p}{)}
\PYG{n}{ax}\PYG{p}{[}\PYG{l+m+mi}{0}\PYG{p}{]}\PYG{o}{.}\PYG{n}{set\PYGZus{}xlabel}\PYG{p}{(}\PYG{l+s+s1}{\PYGZsq{}}\PYG{l+s+s1}{x}\PYG{l+s+s1}{\PYGZsq{}}\PYG{p}{)}
\PYG{n}{ax}\PYG{p}{[}\PYG{l+m+mi}{0}\PYG{p}{]}\PYG{o}{.}\PYG{n}{plot}\PYG{p}{(}\PYG{n}{r}\PYG{p}{[}\PYG{p}{:}\PYG{p}{,}\PYG{l+m+mi}{0}\PYG{p}{]}\PYG{p}{,}\PYG{n}{r}\PYG{p}{[}\PYG{p}{:}\PYG{p}{,}\PYG{l+m+mi}{1}\PYG{p}{]}\PYG{p}{)}
\PYG{n}{ax}\PYG{p}{[}\PYG{l+m+mi}{1}\PYG{p}{]}\PYG{o}{.}\PYG{n}{set\PYGZus{}xlabel}\PYG{p}{(}\PYG{l+s+s1}{\PYGZsq{}}\PYG{l+s+s1}{time}\PYG{l+s+s1}{\PYGZsq{}}\PYG{p}{)}
\PYG{n}{ax}\PYG{p}{[}\PYG{l+m+mi}{1}\PYG{p}{]}\PYG{o}{.}\PYG{n}{set\PYGZus{}ylabel}\PYG{p}{(}\PYG{l+s+s1}{\PYGZsq{}}\PYG{l+s+s1}{y position}\PYG{l+s+s1}{\PYGZsq{}}\PYG{p}{)}
\PYG{n}{ax}\PYG{p}{[}\PYG{l+m+mi}{1}\PYG{p}{]}\PYG{o}{.}\PYG{n}{plot}\PYG{p}{(}\PYG{n}{t}\PYG{p}{,}\PYG{n}{r}\PYG{p}{[}\PYG{p}{:}\PYG{p}{,}\PYG{l+m+mi}{0}\PYG{p}{]}\PYG{p}{)}
\PYG{n}{ax}\PYG{p}{[}\PYG{l+m+mi}{2}\PYG{p}{]}\PYG{o}{.}\PYG{n}{set\PYGZus{}xlabel}\PYG{p}{(}\PYG{l+s+s1}{\PYGZsq{}}\PYG{l+s+s1}{time}\PYG{l+s+s1}{\PYGZsq{}}\PYG{p}{)}
\PYG{n}{ax}\PYG{p}{[}\PYG{l+m+mi}{2}\PYG{p}{]}\PYG{o}{.}\PYG{n}{set\PYGZus{}ylabel}\PYG{p}{(}\PYG{l+s+s1}{\PYGZsq{}}\PYG{l+s+s1}{y position}\PYG{l+s+s1}{\PYGZsq{}}\PYG{p}{)}
\PYG{n}{ax}\PYG{p}{[}\PYG{l+m+mi}{2}\PYG{p}{]}\PYG{o}{.}\PYG{n}{plot}\PYG{p}{(}\PYG{n}{t}\PYG{p}{,}\PYG{n}{r}\PYG{p}{[}\PYG{p}{:}\PYG{p}{,}\PYG{l+m+mi}{1}\PYG{p}{]}\PYG{p}{)}

\PYG{n}{fig}\PYG{o}{.}\PYG{n}{tight\PYGZus{}layout}\PYG{p}{(}\PYG{p}{)}
\PYG{n}{save\PYGZus{}fig}\PYG{p}{(}\PYG{l+s+s2}{\PYGZdq{}}\PYG{l+s+s2}{2DimGravity}\PYG{l+s+s2}{\PYGZdq{}}\PYG{p}{)}
\PYG{n}{plt}\PYG{o}{.}\PYG{n}{show}\PYG{p}{(}\PYG{p}{)}
\PYG{n+nb}{print}\PYG{p}{(}\PYG{n}{E}\PYG{p}{)}
\end{sphinxVerbatim}


\section{Scattering and Cross Sections}
\label{\detokenize{chapter1:scattering-and-cross-sections}}
Scattering experiments don’t measure entire trajectories. For elastic
collisions, they measure the distribution of final scattering angles
at best. Most experiments use targets thin enough so that the number
of scatterings is typically zero or one. The cross section, \(\sigma\),
describes the cross\sphinxhyphen{}sectional area for particles to scatter with an
individual target atom or nucleus. Cross section measurements form the
basis for MANY fields of physics. BThe cross section, and the
differential cross section, encapsulates everything measurable for a
collision where all that is measured is the final state, e.g. the
outgoing particle had momentum \(\boldsymbol{p}_f\). y studying cross sections,
one can infer information about the potential interaction between the
two particles. Inferring, or constraining, the potential from the
cross section is a classic \{\textbackslash{}it inverse\} problem. Collisions are
either elastic or inelastic. Elastic collisions are those for which
the two bodies are in the same internal state before and after the
collision. If the collision excites one of the participants into a
higher state, or transforms the particles into different species, or
creates additional particles, the collision is inelastic. Here, we
consider only elastic collisions.

For Coulomb forces, the cross section is infinite because the range of
the Coulomb force is infinite, but for interactions such as the strong
interaction in nuclear or particle physics, there is no long\sphinxhyphen{}range
force and cross\sphinxhyphen{}sections are finite. Even for Coulomb forces, the part
of the cross section that corresponds to a specific scattering angle,
\(d\sigma/d\Omega\), which is a function of the scattering angle
\(\theta_s\) is still finite.

If a particle travels through a thin target, the chance the particle
scatters is \(P_{\rm scatt}=\sigma dN/dA\), where \(dN/dA\) is the number
of scattering centers per area the particle encounters. If the density
of the target is \(\rho\) particles per volume, and if the thickness of
the target is \(t\), the areal density (number of target scatterers per
area) is \(dN/dA=\rho t\). Because one wishes to quantify the collisions
independently of the target, experimentalists measure scattering
probabilities, then divide by the areal density to obtain
cross\sphinxhyphen{}sections,
\begin{equation*}
\begin{split}
\begin{eqnarray}
\sigma=\frac{P_{\rm scatt}}{dN/dA}.
\end{eqnarray}
\end{split}
\end{equation*}
Instead of merely stating that a particle collided, one can measure
the probability the particle scattered by a given angle. The
scattering angle \(\theta_s\) is defined so that at zero the particle is
unscattered and at \(\theta_s=\pi\) the particle is scattered directly
backward. Scattering angles are often described in the center\sphinxhyphen{}of\sphinxhyphen{}mass
frame, but that is a detail we will neglect for this first discussion,
where we will consider the scattering of particles moving classically
under the influence of fixed potentials \(U(\boldsymbol{r})\). Because the
distribution of scattering angles can be measured, one expresses the
differential cross section,




\begin{equation*}
\begin{split}
\begin{equation}
\frac{d^2\sigma}{d\cos\theta_s~d\phi}.
\label{_auto77} \tag{106}
\end{equation}
\end{split}
\end{equation*}
Usually, the literature expresses differential cross sections as




\begin{equation*}
\begin{split}
\begin{equation}
d\sigma/d\Omega=\frac{d\sigma}{d\cos\theta d\phi}=\frac{1}{2\pi}\frac{d\sigma}{d\cos\theta},
\label{_auto78} \tag{107}
\end{equation}
\end{split}
\end{equation*}
where the last equivalency is true when the scattering does not depend
on the azimuthal angle \(\phi\), as is the case for spherically
symmetric potentials.

The differential solid angle \(d\Omega\) can be thought of as the area
subtended by a measurement, \(dA_d\), divided by \(r^2\), where \(r\) is the
distance to the detector,
\begin{equation*}
\begin{split}
\begin{eqnarray}
dA_d=r^2 d\Omega.
\end{eqnarray}
\end{split}
\end{equation*}
With this definition \(d\sigma/d\Omega\) is independent of the distance
from which one places the detector, or the size of the detector (as
long as it is small).

Differential scattering cross sections are calculated by assuming a
random distribution of impact parameters \(b\). These represent the
distance in the \(xy\) plane for particles moving in the \(z\) direction
relative to the scattering center. An impact parameter \(b=0\) refers to
being aimed directly at the target’s center. The impact parameter
describes the transverse distance from the \(z=0\) axis for the
trajectory when it is still far away from the scattering center and
has not yet passed it. The differential cross section can be expressed
in terms of the impact parameter,




\begin{equation*}
\begin{split}
\begin{equation}
d\sigma=2\pi bdb,
\label{_auto79} \tag{108}
\end{equation}
\end{split}
\end{equation*}
which is the area of a thin ring of radius \(b\) and thickness \(db\). In
classical physics, one can calculate the trajectory given the incoming
kinetic energy \(E\) and the impact parameter if one knows the mass and
potential. From the trajectory, one then finds the scattering angle
\(\theta_s(b)\). The differential cross section is then




\begin{equation*}
\begin{split}
\begin{equation}
\frac{d\sigma}{d\Omega}=\frac{1}{2\pi}\frac{d\sigma}{d\cos\theta_s}=b\frac{db}{d\cos\theta_s}=\frac{b}{(d/db)\cos\theta_s(b)}.
\label{_auto80} \tag{109}
\end{equation}
\end{split}
\end{equation*}
Typically, one would calculate \(\cos\theta_s\) and \((d/db)\cos\theta_s\)
as functions of \(b\). This is sufficient to plot the differential cross
section as a function of \(\theta_s\).

The total cross section is




\begin{equation*}
\begin{split}
\begin{equation}
\sigma_{\rm tot}=\int d\Omega\frac{d\sigma}{d\Omega}=2\pi\int d\cos\theta_s~\frac{d\sigma}{d\Omega}. 
\label{_auto81} \tag{110}
\end{equation}
\end{split}
\end{equation*}
Even if the total cross section is infinite, e.g. Coulomb forces, one
can still have a finite differential cross section as we will see
later on.

An asteroid of mass \(m\) and kinetic energy \(E\) approaches a planet of
radius \(R\) and mass \(M\). What is the cross section for the asteroid to
impact the planet?


\subsection{Solution}
\label{\detokenize{chapter1:solution}}
Calculate the maximum impact parameter, \(b_{\rm max}\), for which the asteroid will hit the planet. The total cross  section for impact is \(\sigma_{\rm impact}=\pi b_{\rm max}^2\). The maximum cross\sphinxhyphen{}section can be found with the help of angular momentum conservation. The asteroid’s incoming momentum is \(p_0=\sqrt{2mE}\) and the angular momentum is \(L=p_0b\). If the asteroid just grazes the planet, it is moving with zero radial kinetic energy at impact. Combining energy and angular momentum conservation and having \(p_f\) refer to the momentum of the asteroid at a distance \(R\),
\begin{equation*}
\begin{split}
\begin{eqnarray*}
\frac{p_f^2}{2m}-\frac{GMm}{R}&=&E,\\
p_fR&=&p_0b_{\rm max},
\end{eqnarray*}
\end{split}
\end{equation*}
allows one to solve for \(b_{\rm max}\),
\begin{equation*}
\begin{split}
\begin{eqnarray*}
b_{\rm max}&=&R\frac{p_f}{p_0}\\
&=&R\frac{\sqrt{2m(E+GMm/R)}}{\sqrt{2mE}}\\
\sigma_{\rm impact}&=&\pi R^2\frac{E+GMm/R}{E}.
\end{eqnarray*}
\end{split}
\end{equation*}

\section{Rutherford Scattering}
\label{\detokenize{chapter1:rutherford-scattering}}
This refers to the calculation of \(d\sigma/d\Omega\) due to an inverse
square force, \(F_{12}=\pm\alpha/r^2\) for repulsive/attractive
interaction. Rutherford compared the scattering of \(\alpha\) particles
(\(^4\)He nuclei) off of a nucleus and found the scattering angle at
which the formula began to fail. This corresponded to the impact
parameter for which the trajectories would strike the nucleus. This
provided the first measure of the size of the atomic nucleus. At the
time, the distribution of the positive charge (the protons) was
considered to be just as spread out amongst the atomic volume as the
electrons. After Rutherford’s experiment, it was clear that the radius
of the nucleus tended to be roughly 4 orders of magnitude smaller than
that of the atom, which is less than the size of a football relative
to Spartan Stadium.

The incoming and outgoing angles of the trajectory are at
\(\pm\theta'\). They are related to the scattering angle by
\(2\theta'=\pi+\theta_s\).

In order to calculate differential cross section, we must find how the
impact parameter is related to the scattering angle. This requires
analysis of the trajectory. We consider our previous expression for
the trajectory where we derived the elliptic form for the trajectory,
Eq. ({\hyperref[\detokenize{chapter1:eq:Ctrajectory}]{\emph{105}}}). For that case we considered an attractive
force with the particle’s energy being negative, i.e. it was
bound. However, the same form will work for positive energy, and
repulsive forces can be considered by simple flipping the sign of
\(\alpha\). For positive energies, the trajectories will be hyperbolas,
rather than ellipses, with the asymptotes of the trajectories
representing the directions of the incoming and outgoing
tracks. Rewriting Eq. ({\hyperref[\detokenize{chapter1:eq:Ctrajectory}]{\emph{105}}}),




\begin{equation*}
\begin{split}
\begin{equation}\label{eq:ruthtraj} \tag{111}
r=\frac{1}{\frac{m\alpha}{L^2}+A\cos\theta}.
\end{equation}
\end{split}
\end{equation*}
Once \(A\) is large enough, which will happen when the energy is
positive, the denominator will become negative for a range of
\(\theta\). This is because the scattered particle will never reach
certain angles. The asymptotic angles \(\theta'\) are those for which
the denominator goes to zero,




\begin{equation*}
\begin{split}
\begin{equation}
\cos\theta'=-\frac{m\alpha}{AL^2}.
\label{_auto82} \tag{112}
\end{equation}
\end{split}
\end{equation*}
The trajectory’s point of closest approach is at \(\theta=0\) and the
two angles \(\theta'\), which have this value of \(\cos\theta'\), are the
angles of the incoming and outgoing particles. From
Fig (\sphinxstylestrong{to come}), one can see that the scattering angle
\(\theta_s\) is given by,




\begin{equation*}
\begin{split}
\begin{eqnarray}
\label{eq:sthetover2} \tag{113}
2\theta'-\pi&=&\theta_s,~~~\theta'=\frac{\pi}{2}+\frac{\theta_s}{2},\\
\nonumber
\sin(\theta_s/2)&=&-\cos\theta'\\
\nonumber
&=&\frac{m\alpha}{AL^2}.
\end{eqnarray}
\end{split}
\end{equation*}
Now that we have \(\theta_s\) in terms of \(m,\alpha,L\) and \(A\), we wish
to re\sphinxhyphen{}express \(L\) and \(A\) in terms of the impact parameter \(b\) and the
energy \(E\). This will set us up to calculate the differential cross
section, which requires knowing \(db/d\theta_s\). It is easy to write
the angular momentum as




\begin{equation*}
\begin{split}
\begin{equation}
L^2=p_0^2b^2=2mEb^2.
\label{_auto83} \tag{114}
\end{equation}
\end{split}
\end{equation*}
Finding \(A\) is more complicated. To accomplish this we realize that
the point of closest approach occurs at \(\theta=0\), so from
Eq. ({\hyperref[\detokenize{chapter1:eq:ruthtraj}]{\emph{111}}})




\begin{equation*}
\begin{split}
\begin{eqnarray}
\label{eq:rminofA} \tag{115}
\frac{1}{r_{\rm min}}&=&\frac{m\alpha}{L^2}+A,\\
\nonumber
A&=&\frac{1}{r_{\rm min}}-\frac{m\alpha}{L^2}.
\end{eqnarray}
\end{split}
\end{equation*}
Next, \(r_{\rm min}\) can be found in terms of the energy because at the
point of closest approach the kinetic energy is due purely to the
motion perpendicular to \(\hat{r}\) and




\begin{equation*}
\begin{split}
\begin{equation}
E=-\frac{\alpha}{r_{\rm min}}+\frac{L^2}{2mr_{\rm min}^2}.
\label{_auto84} \tag{116}
\end{equation}
\end{split}
\end{equation*}
One can solve the quadratic equation for \(1/r_{\rm min}\),




\begin{equation*}
\begin{split}
\begin{equation}
\frac{1}{r_{\rm min}}=\frac{m\alpha}{L^2}+\sqrt{(m\alpha/L^2)^2+2mE/L^2}.
\label{_auto85} \tag{117}
\end{equation}
\end{split}
\end{equation*}
We can plug the expression for \(r_{\rm min}\) into the expression for \(A\), Eq. ({\hyperref[\detokenize{chapter1:eq:rminofA}]{\emph{115}}}),




\begin{equation*}
\begin{split}
\begin{equation}
A=\sqrt{(m\alpha/L^2)^2+2mE/L^2}=\sqrt{(\alpha^2/(4E^2b^4)+1/b^2}
\label{_auto86} \tag{118}
\end{equation}
\end{split}
\end{equation*}
Finally, we insert the expression for \(A\) into that for the scattering angle, Eq. ({\hyperref[\detokenize{chapter1:eq:sthetover2}]{\emph{113}}}),




\begin{equation*}
\begin{split}
\begin{eqnarray}
\label{eq:scattangle} \tag{119}
\sin(\theta_s/2)&=&\frac{m\alpha}{AL^2}\\
\nonumber
&=&\frac{a}{\sqrt{a^2+b^2}}, ~~a\equiv \frac{\alpha}{2E}
\end{eqnarray}
\end{split}
\end{equation*}
The differential cross section can now be found by differentiating the
expression for \(\theta_s\) with \(b\),




\begin{equation*}
\begin{split}
\begin{eqnarray}
\label{eq:rutherford} \tag{120}
\frac{1}{2}\cos(\theta_s/2)d\theta_s&=&\frac{ab~db}{(a^2+b^2)^{3/2}}=\frac{bdb}{a^2}\sin^3(\theta_s/2),\\
\nonumber
d\sigma&=&2\pi bdb=\frac{\pi a^2}{\sin^3(\theta_s/2)}\cos(\theta_s/2)d\theta_s\\
\nonumber
&=&\frac{\pi a^2}{2\sin^4(\theta_s/2)}\sin\theta_s d\theta_s\\
\nonumber
\frac{d\sigma}{d\cos\theta_s}&=&\frac{\pi a^2}{2\sin^4(\theta_s/2)},\\
\nonumber
\frac{d\sigma}{d\Omega}&=&\frac{a^2}{4\sin^4(\theta_s/2)}.
\end{eqnarray}
\end{split}
\end{equation*}
where \(a= \alpha/2E\). This the Rutherford formula for the differential
cross section. It diverges as \(\theta_s\rightarrow 0\) because
scatterings with arbitrarily large impact parameters still scatter to
arbitrarily small scattering angles. The expression for
\(d\sigma/d\Omega\) is the same whether the interaction is positive or
negative.

Consider a particle of mass \(m\) and charge \(z\) with kinetic energy \(E\)
(Let it be the center\sphinxhyphen{}of\sphinxhyphen{}mass energy) incident on a heavy nucleus of
mass \(M\) and charge \(Z\) and radius \(R\). Find the angle at which the
Rutherford scattering formula breaks down.


\subsection{Solution}
\label{\detokenize{chapter1:id4}}
Let \(\alpha=Zze^2/(4\pi\epsilon_0)\). The scattering angle in Eq. ({\hyperref[\detokenize{chapter1:eq:scattangle}]{\emph{119}}}) is
\begin{equation*}
\begin{split}
\sin(\theta_s/2)=\frac{a}{\sqrt{a^2+b^2}}, ~~a\equiv \frac{\alpha}{2E}.
\end{split}
\end{equation*}
The impact parameter \(b\) for which the point of closest approach
equals \(R\) can be found by using angular momentum conservation,
\begin{equation*}
\begin{split}
\begin{eqnarray*}
p_0b&=&b\sqrt{2mE}=Rp_f=R\sqrt{2m(E-\alpha/R)},\\
b&=&R\frac{\sqrt{2m(E-\alpha/R)}}{\sqrt{2mE}}\\
&=&R\sqrt{1-\frac{\alpha}{ER}}.
\end{eqnarray*}
\end{split}
\end{equation*}
Putting these together
\begin{equation*}
\begin{split}
\theta_s=2\sin^{-1}\left\{
\frac{a}{\sqrt{a^2+R^2(1-\alpha/(RE))}}
\right\},~~~a=\frac{\alpha}{2E}.
\end{split}
\end{equation*}
It was from this departure of the experimentally measured
\(d\sigma/d\Omega\) from the Rutherford formula that allowed Rutherford
to infer the radius of the gold nucleus, \(R\).

Just like electrodynamics, one can define “fields”, which for a small
additional mass \(m\) are the force per mass and the additional
potential energy per mass. The \{\textbackslash{}it gravitational field\} related to
the force has dimensions of force per mass, or acceleration, and can
be labeled \(\boldsymbol{g}(\boldsymbol{r})\). The potential energy per mass has
dimensions of energy per mass. This is analogous to the
electromagnetic potential, which is the potential energy per charge,
and the electric field which is the force per charge.

Because the field \(\boldsymbol{g}\) obeys the same inverse square law for a
point mass as the electric field does for a point charge, the
gravitational field also satisfies a version of Gauss’s law,




\begin{equation*}
\begin{split}
\begin{equation}
\label{eq:GravGauss} \tag{121}
\oint d\boldsymbol{A}\cdot\boldsymbol{g}=-4\pi GM_{\rm inside}.
\end{equation}
\end{split}
\end{equation*}
Here, \(M_{\rm inside}\) is the net mass inside a closed area.

Gauss’s law can be understood by considering a nozzle that sprays
paint in all directions uniformly from a point source. Let \(B\) be the
number of gallons per minute of paint leaving the nozzle. If the
nozzle is at the center of a sphere of radius \(r\), the paint per
square meter per minute that is deposited on some part of the sphere
is
\begin{equation*}
\begin{split}
\begin{eqnarray}
F(r)&=&\frac{B}{4\pi r^2}.
\end{eqnarray}
\end{split}
\end{equation*}
Now, let \(F\) also be assigned a direction, so that it becomes a vector
pointing along the direction of the flying paint. For any surface that
surrounds the nozzle, not necessarily a sphere, one can state that




\begin{equation*}
\begin{split}
\begin{eqnarray}
\label{eq:paint} \tag{122}
\oint \boldsymbol{dA}\cdot\boldsymbol{F}&=&B,
\end{eqnarray}
\end{split}
\end{equation*}
regardless of the shape of the surface. This follows because the rate
at which paint is deposited on the surface should equal the rate at
which it leaves the nozzle. The dot product ensures that only the
component of \(\boldsymbol{F}\) into the surface contributes to the deposition
of paint. Similarly, if \(\boldsymbol{F}\) is any radial inverse\sphinxhyphen{}square forces,
that falls as \(B/(4\pi r^2)\), then one can apply
Eq. ({\hyperref[\detokenize{chapter1:eq:paint}]{\emph{122}}}). For gravitational fields, \(B/(4\pi)\) is replaced
by \(GM\), and one quickly “derives” Gauss’s law for gravity,
Eq. ({\hyperref[\detokenize{chapter1:eq:GravGauss}]{\emph{121}}}).

Consider Earth to have its mass \(M\) uniformly distributed in a sphere
of radius \(R\). Find the magnitude of the gravitational acceleration as
a function of the radius \(r\) in terms of the acceleration of gravity
at the surface \(g(R)\). Assume \(r<R\), i.e. you are inside the surface.

\{\textbackslash{}bf Solution\}: Take the ratio of Eq. ({\hyperref[\detokenize{chapter1:eq:GravGauss}]{\emph{121}}}) for two radii, \(R\) and \(r<R\),
\begin{equation*}
\begin{split}
\begin{eqnarray*}
\frac{4\pi r^2 g(r)}{4\pi R^2 g(R)}&=&\frac{4\pi GM_{\rm inside~r}}{4\pi GM_{\rm inside~R}}\\
\nonumber
&=&\frac{r^3}{R^3}\\
\nonumber
g(r)&=&g(R)\frac{r}{R}~.
\end{eqnarray*}
\end{split}
\end{equation*}
The potential energy per mass is similar conceptually to the voltage, or electric potential energy per charge, that was studied in electromagnetism, if \(V\equiv U/m\), \(\boldsymbol{g}=-\nabla V\).


\section{Tidal Forces}
\label{\detokenize{chapter1:tidal-forces}}
Consider a spherical planet of radius \(r\) a distance \(D\) from another
body of mass \(M\). The magnitude of the force due to \(M\) on an small
object of mass \(\delta m\) on surface of the planet can be calculated
by performing a Taylor expansion about the center of the spherical
planet.




\begin{equation*}
\begin{split}
\begin{equation}
F=-\frac{GM\delta m}{D^2}+2\frac{GM\delta m}{D^3}\Delta D+\cdots
\label{_auto87} \tag{123}
\end{equation}
\end{split}
\end{equation*}
If the \(z\) direction points toward the large object, \(\Delta D\) can be
referred to as \(z\). In the accelerating frame of an observer at the
center of the planet,




\begin{equation*}
\begin{split}
\begin{equation}
\delta m\frac{d^2 z}{dt^2}=F-\delta ma'+{\rm other~forces~acting~on~} \delta m,
\label{_auto88} \tag{124}
\end{equation}
\end{split}
\end{equation*}
where \(a'\) is the acceleration of the observer. Because \(\delta ma'\)
equals the gravitational force on \(\delta m\) if it were located at the
planet’s center, one can write




\begin{equation*}
\begin{split}
\begin{equation}
m\frac{d^2z}{dt^2}=2\frac{GM\delta m}{D^3}z+{\rm other~forces~acting~on~}\delta m.
\label{_auto89} \tag{125}
\end{equation}
\end{split}
\end{equation*}
Here the other forces could represent the forces acting on \(\delta m\)
from the spherical planet such as the gravitational force or the
contact force with the surface. If \(\theta\) is the angle w.r.t. the
\(z\) axis, the effective force acting on \(\delta m\) is




\begin{equation*}
\begin{split}
\begin{equation}
F_{\rm eff}\approx 2\frac{GM\delta m}{D^3}r\cos\theta\hat{z}+{\rm other~forces~acting~on~}\delta m.
\label{_auto90} \tag{126}
\end{equation}
\end{split}
\end{equation*}
This first force is the “tidal” force. It pulls objects outward from the center of the object. If the object were covered with water, it would distort the objects shape so that the shape would be elliptical, stretched out along the axis pointing toward the large mass \(M\). The force is always along (either parallel or antiparallel to) the \(\hat{z}\) direction.

Consider the Earth to be a sphere of radius \(R\) covered with water,
with the gravitational acceleration at the surface noted by \(g\). Now
assume that a distant body provides an additional constant
gravitational acceleration \(\boldsymbol{a}\) pointed along the \(z\) axis. Find
the distortion of the radius as a function of \(\theta\). Ignore
planetary rotation and assume \(a<<g\).

\{\textbackslash{}bf Solution\}: Because Earth would then accelerate with \(a\), the
field \(a\) would seem invisible in the accelerating frame. A tidal
force would only appear if \(a\) depended on position, i.e. \(\nabla
\boldsymbol{a}\ne 0\).

Now consider that the field is no longer constant, but that instead \(a=-kz\) with \(|kR|<<g\).

\{\textbackslash{}bf Solution\}: The surface of the planet needs to be at constant
potential (if the planet is not accelerating). The force per mass,
\(-kz\) is like a spring, and the potential per mass is
\(kz^2/2\). Otherwise water would move to a point of lower
potential. Thus, the potential energy for a sample mass \(\delta m\) is
\begin{equation*}
\begin{split}
\begin{eqnarray*}
V(R)+\delta m gh(\theta)-\frac{\delta m}{2}kr^2\cos^2\theta={\rm Constant}\\
V(R)+\delta mgh(\theta)-\frac{\delta m}{2}kR^2\cos^2\theta-\delta m kRh(\theta)\cos^2\theta-\frac{\delta m}{2}kh^2(\theta)\cos^2\theta={\rm Constant}.
\end{eqnarray*}
\end{split}
\end{equation*}
Here, the potential due to the external field is \((1/2)kz^2\) so that \(-\nabla U=-kz\). One now needs to solve for \(h(\theta)\). Absorbing all the constant terms from both sides of the equation into one constant \(C\), and because both \(h\) and \(kR\) are small, we can through away terms of order \(h^2\) or \(kRh\). This gives
\begin{equation*}
\begin{split}
\begin{eqnarray*}
gh(\theta)-\frac{1}{2}kR^2\cos^2\theta&=&C,\\
h(\theta)&=&\frac{C}{g}+\frac{1}{2g}kR^2\cos^2\theta,\\
h(\theta)&=&\frac{1}{2g}kR^2(\cos^2\theta-1/3).
\end{eqnarray*}
\end{split}
\end{equation*}
The term with the factor of \(1/3\) replaced the constant and was chosen so that the average height of the water would be zero.

The Sun’s mass is \(27\times 10^6\) the Moon’s mass, but the Sun is 390 times further away from Earth as the Sun. What is ratio of the tidal force of the Sun to that of the Moon.

\{\textbackslash{}bf Solution\}: The gravitational force due to an object \(M\) a distance \(D\) away goes as \(M/D^2\), but the tidal force is only the difference of that force over a distance \(R\),
\begin{equation*}
\begin{split}
F_{\rm tidal}\propto \frac{M}{D^3}R.
\end{split}
\end{equation*}
Therefore the ratio of force is
\begin{equation*}
\begin{split}
\begin{eqnarray*}
\frac{F_{\rm Sun's~tidal~force}}{F_{\rm Moon's~tidal~force}}
&=&\frac{M_{\rm sun}/D_{\rm sun}^3}{M_{\rm moon}/D_{\rm moon}^3}\\
&=&\frac{27\times 10^6}{390^3}=0.46.
\end{eqnarray*}
\end{split}
\end{equation*}
The Moon more strongly affects tides than the Sun.


\chapter{Variational Calculus and Lagrangian Formalism}
\label{\detokenize{chapter1:variational-calculus-and-lagrangian-formalism}}
The calculus of variations involves
problems where the quantity to be minimized or maximized is an integral.

The usual minimization problem one faces involves taking a function
\({\cal L}(x)\), then finding the single value \(x\) for which \({\cal L}\)
is either a maximum or minimum. In multivariate calculus one also
learns to solve problems where you minimize for multiple variables,
\({\cal L}(x_1,x_2,\cdots x_n)\), and finding the points \((x_1\cdots
y_n)\) in an \(n\)\sphinxhyphen{}dimensional space that maximize or minimize the
function. Here, we consider what seems to be a much more ambitious
problem. Imagine you have a function \({\cal L}(x(t),\dot{x}(t),t)\),
and you wish to find the extrema for an infinite number of values of
\(x\), i.e. \(x\) at each point \(t\). The function \({\cal L}\) will not only
depend on \(x\) at each point \(t\), but also on the slope at each point,
plus an additional dependence on \(t\). Note we are NOT finding an
optimum value of \(t\), we are finding the set of optimum values of \(x\)
at each point \(t\), or equivalently, finding the function \(x(t)\).

One treats the function \(x(t)\) as being unknown while minimizing the action
\begin{equation*}
\begin{split}
S=\int_{t_1}^{t_2}dt~{\cal L}(x(t),\dot{x}(t),t).
\end{split}
\end{equation*}
Thus, we are minimizing \(S\) with respect to an infinite number of
values of \(x(t_i)\) at points \(t_i\). As an additional criteria, we will
assume that \(x(t_1)\) and \(x(t_2)\) are fixed, and that that we will
only consider variations of \(x\) between the boundaries. The dependence
on the derivative, \(\dot{x}=dx/dt\), is crucial because otherwise the
solution would involve simply finding the one value of \(x\) that
minimized \({\cal L}\), and \(x(t)\) would equal a constant if there were no
explicit \(t\) dependence. Furthermore, \(x\) wouldn’t need to be
continuous at the boundary.

In the general case we have an integral of the type
\begin{equation*}
\begin{split}
S[q]= \int_{t_1}^{t_2} {\cal L}(q(t),\dot{q}(t),t)dt,
\end{split}
\end{equation*}
where \(S\) is the quantity which is sought minimized or maximized.  The
problem is that although \({\cal L}\)  is a function of the general variables
\(q(t),\dot{q}(t),t\) (note our change of variables), the exact dependence of \(q\) on \(t\) is not known.
This means again that even though the integral has fixed limits \(t_1\)
and \(t_2\), the path of integration is not known. In our case the unknown
quantities are the positions and general velocities of a given number
of objects and we wish to choose an integration path which makes the
functional \(S[q]\) stationary. This means that we want to find minima,
or maxima or saddle points. In physics we search normally for minima.
Our task is therefore to find the minimum of \(S[q]\) so that its
variation \(\delta S\) is zero subject to specific constraints.  The
constraints can be treated via the technique of Lagrangian multipliers
as we will see below.

We assume the existence of an optimum path, that is a path for which
\(S[q]\) is stationary. There are infinitely many such paths.  The
difference between two paths \(\delta q\) is called the variation of
\(q\).

We call the variation \(\eta(t)\) and it is scaled by a factor \(\alpha\).
The function \(\eta(t)\) is arbitrary except for
\begin{equation*}
\begin{split}
\eta(t_1)=\eta(t_2)=0,
\end{split}
\end{equation*}
and we assume that we can model the change in \(q\) as
\begin{equation*}
\begin{split}
q(t,\alpha) = q(t)+\alpha\eta(t),
\end{split}
\end{equation*}
and
\begin{equation*}
\begin{split}
\delta q = q(t,\alpha) -q(t,0)=\alpha\eta(t).
\end{split}
\end{equation*}
We choose \(q(t,\alpha=0)\) as the unkonwn path  that will minimize \(S\).  The value
\(q(t,\alpha\ne 0)\)  describes a neighbouring path.

We have
\begin{equation*}
\begin{split}
S[q(\alpha)]= \int_{t_1}^{t_2} {\cal L}(q(t,\alpha),\dot{q}(t,\alpha),t)dt.
\end{split}
\end{equation*}
The condition for an extreme of
\begin{equation*}
\begin{split}
S[q(\alpha)]= \int_{t_1}^{t_2} {\cal L}(q(t,\alpha),\dot{q}(t,\alpha),t)dt,
\end{split}
\end{equation*}
is
\begin{equation*}
\begin{split}
\left[\frac{\partial  S[q(\alpha)]}{\partial t}\right]_{\alpha=0} =0.
\end{split}
\end{equation*}
The \(\alpha\) dependence is contained in \(q(t,\alpha)\) and \(\dot{q}(t,\alpha)\) meaning that
\begin{equation*}
\begin{split}
\left[\frac{\partial  E[q(\alpha)]}{\partial \alpha}\right]=\int_{t_1}^{t_2} \left( \frac{\partial {\cal l}}{\partial q}\frac{\partial q}{\partial \alpha}+\frac{\partial {\cal L}}{\partial \dot{q}}\frac{\partial \dot{q}}{\partial \alpha}\right)dt.
\end{split}
\end{equation*}
We have defined
\begin{equation*}
\begin{split}
\frac{\partial q(x,\alpha)}{\partial \alpha}=\eta(x)
\end{split}
\end{equation*}
and thereby
\begin{equation*}
\begin{split}
\frac{\partial \dot{q}(t,\alpha)}{\partial \alpha}=\frac{d(\eta(t))}{dt}.
\end{split}
\end{equation*}
Using
\begin{equation*}
\begin{split}
\frac{\partial q(t,\alpha)}{\partial \alpha}=\eta(t),
\end{split}
\end{equation*}
and
\begin{equation*}
\begin{split}
\frac{\partial \dot{q}(t,\alpha)}{\partial \alpha}=\frac{d(\eta(t))}{dt},
\end{split}
\end{equation*}
in the integral gives
\begin{equation*}
\begin{split}
\left[\frac{\partial  S[q(\alpha)]}{\partial \alpha}\right]=\int_{t_1}^{t_2} \left( \frac{\partial {\cal L}}{\partial q}\eta(t)+\frac{\partial {\cal L}}{\partial \dot{q}}\frac{d(\eta(t))}{dt}\right)dt.
\end{split}
\end{equation*}
Integrating the second term by parts
\begin{equation*}
\begin{split}
\int_{t_1}^{t_2} \frac{\partial {\cal L}}{\partial \dot{q}}\frac{d(\eta(t))}{dt}dt =\eta(t)\frac{\partial {\cal L}}{\partial \dot{q}}|_{t_1}^{t_2}-
\int_a^b \eta(t)\frac{d}{dx}\frac{\partial {\cal L}}{\partial \dot{q}}dt,
\end{split}
\end{equation*}
and since the first term dissappears due to \(\eta(a)=\eta(b)=0\), we obtain
\begin{equation*}
\begin{split}
\left[\frac{\partial  S[q(\alpha)]}{\partial \alpha}\right]=\int_{t_1}^{t_2} \left( \frac{\partial {\cal L}}{\partial q}-\frac{d}{dx}\frac{\partial {\cal L}}{\partial \dot{q}}
\right)\eta(t)dt=0,
\end{split}
\end{equation*}
which can also be written as
\begin{equation*}
\begin{split}
\left[\frac{\partial  S[q(\alpha)]}{\partial \alpha}\right]_{\alpha=0}=\int_{t_1}^{t_2} \left( \frac{\partial {\cal L}}{\partial q}-\frac{d}{\
dx}\frac{\partial {\cal L}}{\partial \dot{q}}\right)\delta q(t)dt=\delta S = 0.
\end{split}
\end{equation*}
The condition for a stationary value is thus a partial differential equation
\begin{equation*}
\begin{split}
\frac{\partial {\cal L}}{\partial q}-\frac{d}{dx}\frac{\partial {\cal L}}{\partial \dot{q}}=0,
\end{split}
\end{equation*}
known as the \sphinxstylestrong{Euler\sphinxhyphen{}Lagrange} equation.


\section{Constrained motion}
\label{\detokenize{chapter1:constrained-motion}}
Sometimes an auxiliary constraint is added to the problem (beyond
fixing the end poits \(y_1\) and \(y_2\)). Just ahead, we will work on the
example of a hanging chain. The shape of the curve minimizes the
potential energy, under the constraint of a fixed length of
chain. Before presenting such an example we first review the method of
Lagrange multipliers as a method for finding minima or maxima under
constraints.

Imagine a function \(f(x_1,x_2\cdots x_n)\) for which you wish to find
the minima. Additionally, you are given a constraint
\begin{equation*}
\begin{split}
\begin{eqnarray}
C(x_1\cdots x_n)=0
\end{eqnarray}
\end{split}
\end{equation*}
The usual condition for a a minimum is
\begin{equation*}
\begin{split}
\begin{eqnarray}
\frac{\partial f}{\partial x_i}=0{\rm ,~~or~}\nabla f=0.
\end{eqnarray}
\end{split}
\end{equation*}
which would be \(n\) equations for the \(n\) variables. The gradient of a
scalar is a vector, so you should think of \(\nabla\) as
\(\boldsymbol{\nabla}\). However, the solution will likely not satisfy the
constraint, i.e. the point at which \(f(x_1\cdots x_n)\) has an extrema,
may not be a point where \(C(x_1\cdots x_n)=0\).

A necessary condition for the solution is that




\begin{equation*}
\begin{split}
\begin{equation}
\nabla f\cdot\boldsymbol{\epsilon}=0,
\label{_auto91} \tag{127}
\end{equation}
\end{split}
\end{equation*}
for any infinitesimal vector \(\boldsymbol{\epsilon}\) if \(\boldsymbol{\epsilon}\)
satisfies the condition




\begin{equation*}
\begin{split}
\begin{equation}
\delta C=\nabla C\cdot\boldsymbol{\epsilon}=0.
\label{_auto92} \tag{128}
\end{equation}
\end{split}
\end{equation*}
That is to say if I take a small step in a direction that doesn’t
change the constraint, then \(f\) must not change if it is an
extrema. Not changing the constraint implies the step is orthogonal to
\(\nabla C\). As there are \(n\) dimensions of \(x\), the vector \(\nabla C\)
defines one direction, and \(\boldsymbol{\epsilon}\) can be in any of the \(n-1\)
directions orthogonal to \(\nabla C\). If \(\nabla f\cdot\boldsymbol{\epsilon}=0\)
for ANY of the \(n-1\) directions of \(\boldsymbol{\epsilon}\) orthogonal to
\(\nabla C\), then




\begin{equation*}
\begin{split}
\begin{equation}
\nabla f ~||~ \nabla C.
\label{_auto93} \tag{129}
\end{equation}
\end{split}
\end{equation*}
Because the two vectors are parallel you can say there must exist some
constant \(\lambda\) such that




\begin{equation*}
\begin{split}
\begin{equation}
\nabla(f-\lambda C)=0.
\label{_auto94} \tag{130}
\end{equation}
\end{split}
\end{equation*}
Here, \(\lambda\) is known as a Lagrange multiplier. Satisfying
this equation  is a necessary, but not a sufficient
condition. One could add a constant to the constraint and the gradient
would not change. One must find the correct value of \(\lambda\) that
satisfies the constraint \(C=0\), rather than \(C=\) some other
constant. The strategy is then to solve
thw above equation  then adjust \(\lambda\) until one
finds the \(x_1\cdots x_n\) that gives \(C(x_1\cdots x_n)=0\).

The method of Lagrange multipliers is counter\sphinxhyphen{}intuitive to one’s
intuition to use the constraint to reduce the dimensionality of the
problem. Normally, minimizing a function of \(n\) variables, leads to
\(n\) equations and \(n\) unknowns. A constraint could be used, by
substitution, to replace the \(n\) variables with \(n-1\)
variables. Instead, we add an unknown parameter, \(\lambda\), and change
the equation to \(n+1\) equations with \(n+1\) unknowns, with the extra
unknown being the Lagrange multiplier \(\lambda\). Often, it is rather
easy to solve for \(x_1\cdots x_n\). Then one is left with the usually
difficult problem of finding \(\lambda\), often requiring the solution
of a transcendental equation.

Let us try to formalize this. We consider a function of three independent variables \(f(x,y,z)\) . For
the function \(f\) to be an extreme we have
\begin{equation*}
\begin{split}
df=0.
\end{split}
\end{equation*}
A necessary and sufficient condition is
\begin{equation*}
\begin{split}
\frac{\partial f}{\partial x} =\frac{\partial f}{\partial y}=\frac{\partial f}{\partial z}=0,
\end{split}
\end{equation*}
due to
\begin{equation*}
\begin{split}
df = \frac{\partial f}{\partial x}dx+\frac{\partial f}{\partial y}dy+\frac{\partial f}{\partial z}dz.
\end{split}
\end{equation*}
In physical problems the variables \(x,y,z\) are often subject to constraints (in our case \(q\) and the orthogonality constraint)
so that they are no longer all independent. It is possible at least in principle to use each constraint to eliminate one variable
and to proceed with a new and smaller set of independent varables.

The use of so\sphinxhyphen{}called Lagrangian  multipliers is an alternative technique  when the elimination of
of variables is incovenient or undesirable.  Assume that we have an equation of constraint on the variables \(x,y,z\)
\begin{equation*}
\begin{split}
\phi(x,y,z) = 0,
\end{split}
\end{equation*}
resulting in
\begin{equation*}
\begin{split}
d\phi = \frac{\partial \phi}{\partial x}dx+\frac{\partial \phi}{\partial y}dy+\frac{\partial \phi}{\partial z}dz =0.
\end{split}
\end{equation*}
Now we cannot set anymore
\begin{equation*}
\begin{split}
\frac{\partial f}{\partial x} =\frac{\partial f}{\partial y}=\frac{\partial f}{\partial z}=0,
\end{split}
\end{equation*}
if \(df=0\) is wanted
because there are now only two independent variables!  Assume \(x\) and \(y\) are the independent variables.
Then \(dz\) is no longer arbitrary.

However, we can add to
\begin{equation*}
\begin{split}
df = \frac{\partial f}{\partial x}dx+\frac{\partial f}{\partial y}dy+\frac{\partial f}{\partial z}dz,
\end{split}
\end{equation*}
a multiplum of \(d\phi\), viz. \(\lambda d\phi\), resulting  in
\begin{equation*}
\begin{split}
df+\lambda d\phi = (\frac{\partial f}{\partial z}+\lambda\frac{\partial \phi}{\partial x})dx+(\frac{\partial f}{\partial y}+\lambda\frac{\partial \phi}{\partial y})dy+(\frac{\partial f}{\partial z}+\lambda\frac{\partial \phi}{\partial z})dz =0.
\end{split}
\end{equation*}
Our multiplier is chosen so that
\begin{equation*}
\begin{split}
\frac{\partial f}{\partial z}+\lambda\frac{\partial \phi}{\partial z} =0.
\end{split}
\end{equation*}
However, we took \(dx\) and \(dy\) as to be arbitrary and thus we must have
\begin{equation*}
\begin{split}
\frac{\partial f}{\partial x}+\lambda\frac{\partial \phi}{\partial x} =0,
\end{split}
\end{equation*}
and
\begin{equation*}
\begin{split}
\frac{\partial f}{\partial y}+\lambda\frac{\partial \phi}{\partial y} =0.
\end{split}
\end{equation*}
When all these equations are satisfied, \(df=0\).  We have four
unknowns, \(x,y,z\) and \(\lambda\). Actually we want only \(x,y,z\),
\(\lambda\) need not to be determined, it is therefore often called
Lagrange’s undetermined multiplier.  If we have a set of constraints
\(\phi_k\) we have the equations
\begin{equation*}
\begin{split}
\frac{\partial f}{\partial x_i}+\sum_k\lambda_k\frac{\partial \phi_k}{\partial x_i} =0.
\end{split}
\end{equation*}

\subsection{Example: brachiostone}
\label{\detokenize{chapter1:example-brachiostone}}
Consider a particle constrained to move along a path (like a bead
moving without friction on a wire) and you need to design a path from
\(x=y=0\) to some final point \(x_f,y_f\). Assume there is a constant
force in the \(x\) direction, \(F_x=mg\). Design the path so that the time
the bead travels is a minimum.

The net time is
\begin{equation*}
\begin{split}
T=\int \frac{d\ell}{v}=\int_0^{x_f} dx~\frac{\sqrt{1+y'^2}}{\sqrt{2gx}}={\rm minimum}.
\end{split}
\end{equation*}
Here we made use of the fact that \(d\ell=\sqrt{dx^2+dy^2}\) and that
the velocity is determined by \(KE=mv^2/2=mgx\). The Euler equations can
be applied if you first define the function as
\begin{equation*}
\begin{split}
\begin{eqnarray*}
f(y,y';x)&=&\frac{\sqrt{1+y'^2}}{\sqrt{x}}.
\end{eqnarray*}
\end{split}
\end{equation*}
The equations are then
\begin{equation*}
\begin{split}
\begin{eqnarray*}
\frac{d}{dx}\frac{\partial f}{\partial y'}&=&0.
\end{eqnarray*}
\end{split}
\end{equation*}
The simplification ensued from \(f\) not having any dependence on \(y\). This yields the differential equation
\begin{equation*}
\begin{split}
\begin{eqnarray}
\frac{y'}{x^{1/2}(1+y'^2)^{1/2}}&=&(2a)^{-1/2},
\end{eqnarray}
\end{split}
\end{equation*}
because \(\partial f/\partial y'\) must be a constant, which with some
foresight we label \((2a)^{-1/2}\). One can now solve for \(y'\),
\begin{equation*}
\begin{split}
\begin{eqnarray*}
(y')^2&=&2ax(1+y'^2)\\
\nonumber
y'&=&\sqrt{\frac{x}{2a-x}},\\
\nonumber
y(t)&=&\int_0^x dx'~\frac{\sqrt{x'}dx'}{\sqrt{2a-x'}}=\int_0^x dx'~\frac{x'dx'}{\sqrt{2ax'-x'^2}}\\
\nonumber
&=&\frac{1}{2}\int_0^x\frac{(2x'-2a)dx'}{(2ax'-x'^2)^{1/2}}+a\int_0^x\frac{dx'}{\sqrt{2ax'-x'^2}}\\
\nonumber
&=&\frac{-1}{2}\int_0^{2ax-x^2}\frac{du}{\sqrt{u}}+a\int_0^x\frac{dx'}{\sqrt{a^2-(x'-a)^2}}\\
&=&-\sqrt{2ax-x^2}+a\cos^{-1}(1-x/a).
\end{eqnarray*}
\end{split}
\end{equation*}
This turns out to be the equation for a \{\textbackslash{}it cycloid\} or a \{\textbackslash{}it
brachiostone\}. If you rolled a wheel of radius \(a\) down the \(y\) axis
and followed a point on the rim, it would trace out a cycloid. Here,
the constant \(a\) must be chosen to match the boundary condition,
\(y_2=y(x_2)\). You can see the textbook for more details, plus you get
a chance to work with cycloids in the exercises at the end of this
chapter.


\subsection{Maximizing a Function}
\label{\detokenize{chapter1:maximizing-a-function}}
As an example of using Lagrange multipliers for a standard
optimization formula we attempt to maximize the following function,
\begin{equation*}
\begin{split}
F(x_1\cdots x_n)=-\sum_{i=1}^n x_i\ln(x_i),
\end{split}
\end{equation*}
with respect to the \(n\) variables \(x_i\). With no constraints, each
\(x_i\) would maximize the function for
\begin{equation*}
\begin{split}
\begin{eqnarray*}
\frac{d}{dx_j}~\left[-\sum_i x_i\ln(x_i)\right]&=&0\\
-\ln(x_j)-1&=&0,~~~~x_j=e^{-1}.
\end{eqnarray*}
\end{split}
\end{equation*}
Now, we repeat the problem but with two constraints,
\begin{equation*}
\begin{split}
\sum_ix_i=1~,~~~~\sum_ix_i\epsilon_i=E.
\end{split}
\end{equation*}
Here, \(\epsilon_i\) and \(E\) are fixed constants. We go forward by
finding the extrema for
\begin{equation*}
\begin{split}
\begin{eqnarray*}
G(x_1\cdots x_n)&=&F-\alpha\sum_i x_i-\beta\sum_i\epsilon_ix_i
=\sum_i \left\{-x_i\ln(x_i)-\alpha x_i-\beta\epsilon_ix_i\right\}.
\end{eqnarray*}
\end{split}
\end{equation*}
There are two Lagrange multipliers, \(\alpha\) and \(\beta\),
corresponding to the two constraints. One then solves for the extrema
\begin{equation*}
\begin{split}
\begin{eqnarray*}
\frac{d}{dx_j}G&=&0\\
&=&-\ln(x_j)-1-\alpha-\beta\epsilon_j,\\
x_j&=&\exp\left\{-1-\alpha-\beta\epsilon_j\right\}.
\end{eqnarray*}
\end{split}
\end{equation*}
For any given \(\alpha\) and \(\beta\) this provides a solution for
constraining \(\sum_i x_i\) and \(\sum_i\epsilon_ix_i\) to some values,
just not the values of unity and \(E\) that you wish. One would then
have to search for the correct values by adjusting \(\alpha\) and
\(\beta\) until the constraint are actually matched by solving a
transcendental equation. Although this can be complicated, it is
certainly less expensive than searching over all \(N\) values of
\(x_i\). This particular example corresponds to maximizing the entropy
for a system, \(S=-\sum_i x_i\ln(x_i)\), where \(x_i\) is the probability
of the system being in a particular discrete level \(i\) that has energy
\(\epsilon_i\). One wishes to maximize the entropy subject to the
constraints that the probabilities sum to unity and the average energy
has some given value. The result that \(x_i\sim e^{-\beta\epsilon_i}\)
demonstrates the origin of the Boltzmann factor, with the inverse
temperature \(\beta=1/T\).

Lagrange multipliers also assist with the Euler\sphinxhyphen{}Lagrange equation. If
one breaks an interval \(x_1<x<x_2\) into a large number
\(n\rightarrow\infty\) points separated by \(dx\), the Euler\sphinxhyphen{}Lagrange
equation involves finding the \(n\) values \(y_i\) at each point so that
\(\sum_i dx f\left\{y_i,y'_i=(y_{i+1}-y_{i-1})/(2dx)\right\}\) is
maximized for some given function \(f\). If an additional auxiliary
constraint is added, also some function of the \(n\) values \(y_i\), one
can use the method of Lagrange multipliers. In the constraint can also
be written as some function of \(C(y_i,y'_i)\), then one simply adds a
term \(\lambda C(y,y')\) to the function \(f\) and uses the Euler\sphinxhyphen{}Lagrange
equation to find the extrema of.
\begin{equation*}
\begin{split}
\begin{eqnarray}
J&=&\int_{x_1}^{x_2}dx~f\left\{y(t),y'(t),x\right\}-\lambda C\left\{y(t),y'(t),x\right\},
\end{eqnarray}
\end{split}
\end{equation*}
the one difference being that




\begin{equation*}
\begin{split}
\begin{equation}
f\left\{y(t),y'(t),x\right\}\rightarrow f\left\{y(t),y'(t),x\right\}-\lambda C\left\{y(t),y'(t),x\right\}
\label{_auto95} \tag{131}
\end{equation}
\end{split}
\end{equation*}

\subsection{Example}
\label{\detokenize{chapter1:example}}
Consider a chain of length \(L\) and mass per unit length \(\kappa\) that
hangs from point \(x=0,y=0\) to point \(x_f,y_f\). The shape must minimize
the potential energy. Find general expressions for the shape in terms
of three constants which must be chosen to match \(y(0)=0, y(x_f)=y_f\)
and the fixed length. Equivalently, one finds the function \(y(t)\) that
provides an extrema for the integral,

One must minimize
\begin{equation*}
\begin{split}
\int d\ell~\kappa gy-\lambda\int d\ell=
\int_0^{x_f} dx~\sqrt{1+y'^2}\kappa gy-\lambda \int_0^{x_f} dx\sqrt{1+y'^2}.
\end{split}
\end{equation*}
Here \(\lambda\) is the Lagrange multiplier associated with constraining
the length of the chain. The constrained length \(L\) appears nowhere in
the expression. Instead, one solves for form of the answer, then
adjusts \(\lambda\) to give the correct length. For the purposes of the
Euler\sphinxhyphen{}Lagrange minimization one considers the function
\begin{equation*}
\begin{split}
\begin{eqnarray}
f(y,y';x)&=&\kappa gy\sqrt{1+y'^2}-\lambda\sqrt{1+y'^2}.
\end{eqnarray}
\end{split}
\end{equation*}
Because \(\lambda\) is an unknown constant and because minimizing a
function multiplied by a constant is the same as minimizing the
function, we can equivlently minimize the integral using the function
\begin{equation*}
\begin{split}
\begin{eqnarray}
\tilde{f}(y,y';x)&=&y\sqrt{1+y'^2}-\tilde{\lambda}\sqrt{1+y'^2},\\
\nonumber
\tilde{\lambda}&\equiv&\frac{\lambda}{\kappa g}.
\end{eqnarray}
\end{split}
\end{equation*}
The Euler\sphinxhyphen{}Lagrange equations then become
\begin{equation*}
\begin{split}
\begin{eqnarray*}
\frac{d}{dx}\left\{
\frac{y'}{\sqrt{1+y'^2}}y-\tilde{\lambda}\frac{y'}{\sqrt{1+y'^2}}
\right\}&=&\sqrt{1+y'^2}.
\end{eqnarray*}
\end{split}
\end{equation*}
Here, we will guess at the form of the solution,
\begin{equation*}
\begin{split}
\begin{eqnarray*}
y'&=&\sinh[(x-x_0)/a],~~y=a\cosh[(x-x_0)/a]+y_0.
\end{eqnarray*}
\end{split}
\end{equation*}
Plugging into the Euler\sphinxhyphen{}Lagange equations,
\begin{equation*}
\begin{split}
\begin{eqnarray*}
\frac{d}{dx}\left\{(a\cosh[(x-x_0)/a]+y_0)\frac{\sinh[(x-x_0)/a]}{\cosh[(x-x_0)/a]}-\tilde{\lambda}\frac{\sinh[(x-x_0)/a]}{\cosh[(x-x_0)/a]}\right\}&=&\cosh[(x-x_0)/a],\\
\nonumber
\frac{d}{dx}\left\{(y_0-\tilde{\lambda})\tanh[(x-x_0)/a]\right\}=0.
\end{eqnarray*}
\end{split}
\end{equation*}
This solution works if \(y_0=\tilde{\lambda}\). So the general form of
the solution is
\begin{equation*}
\begin{split}
y=\tilde{\lambda}+a\cosh[(x-x_0)/a].
\end{split}
\end{equation*}
One must find \(\tilde{\lambda}\), \(x_0\) and \(a\) to satisfy three
conditions, \(y(x=0)=0\), \(y(x=x_f)=y_f\) and that the length is \(L\). For
a hanging chain \(a\) is positive. A solution with negative \(a\) would
represent a maximum of the potential energy. A remarkable property of
the solution is that once you define the length and the end\sphinxhyphen{}point
positions \(y_1\) and \(y_2\), the solution does not depend on \(\kappa\) or
\(g\). Thus, the shape of the chain would be the same if you took it to
the moon. These solutions are known as \sphinxhref{http://en.wikipedia.org/wiki/Catenary\%7D\%7Bhttp://en.wikipedia.org/wiki/Catenary}{catenaries}.


\section{Lagrangians}
\label{\detokenize{chapter1:lagrangians}}
Lagrangians represent a powerful method for solving problems that
would be nearly impossible by direct application of Newton’s third
law, \(\boldsymbol{F}=m\boldsymbol{a}\). The method works well for problems where a
system is well described by a few \textbackslash{}textit\{generalized coordinates\}. A
generalized coordinate might be the angle describing the position of a
pendulum. This one angle takes the place of using \(x\) and \(y\) to
describe the position of the pendulum, then applying a clumsy
constraint.

The Lagrangian equations of motion can be derived from a principle of
least action, where the action \(S\) is defined as




\begin{equation*}
\begin{split}
\begin{equation}
S=\int dt~ L(q,\dot{q},t),
\label{_auto96} \tag{132}
\end{equation}
\end{split}
\end{equation*}
where \(q\) is some coordinate that describes the orientation of a
system and the Lagrangian \(L\) is defined as




\begin{equation*}
\begin{split}
\begin{equation}
L=T-U,
\label{_auto97} \tag{133}
\end{equation}
\end{split}
\end{equation*}
the difference of the kinetic and potential energies. Minimizing the
action through the Euler\sphinxhyphen{}Lagrange equations gives the Lagrangian
equations of motion,




\begin{equation*}
\begin{split}
\begin{equation}
\frac{d}{dt}\frac{\partial L}{\partial \dot{q}}=\frac{\partial L}{\partial q}.
\label{_auto98} \tag{134}
\end{equation}
\end{split}
\end{equation*}
We begin with two simple examples, neither of which gains from the Lagrangian approach.

Consider a particle of mass \(m\) connected to a spring with stiffness \(k\). Derive the Lagrangian equations of motion.
\begin{equation*}
\begin{split}
\begin{eqnarray*}
L&=&\frac{1}{2}m\dot{x}^2-\frac{1}{2}kx^2,\\
\frac{d}{dt}\frac{\partial L}{\partial \dot{x}}&=&\frac{\partial L}{\partial x},\\
m\ddot{x}&=&-kx.
\end{eqnarray*}
\end{split}
\end{equation*}
Derive the Lagrangian equations of motion for a pendulum of mass \(m\)
and length \(\ell\).
\begin{equation*}
\begin{split}
\begin{eqnarray*}
L&=&\frac{m}{2}\ell^2\dot{\theta}^2-mg\ell(1-\cos\theta),\\
\frac{d}{dt}\frac{\partial L}{\partial \dot{\theta}}&=&\frac{\partial L}{\partial \theta},\\
m\ell^2\ddot{\theta}&=&-mg\ell\sin\theta,\\
\ddot{\theta}&=&-\frac{g}{\ell}\sin\theta,\\
\ddot{\theta}&\approx&-\frac{g}{\ell}\theta.
\end{eqnarray*}
\end{split}
\end{equation*}

\subsection{Proving Lagrange’s Equations of Motion from Newton’s Laws}
\label{\detokenize{chapter1:proving-lagrange-s-equations-of-motion-from-newton-s-laws}}
Lagrange’s equations of motion can only be applied for the following conditions:
\begin{itemize}
\item {} 
The potential energy is a function of the generalized coordinates \(q_i\), but not of \(\dot{q}_i\).

\item {} 
The relation between the original coordinates \(x,y,z\cdots\) and the generalized coordinates does not depend on \(\dot{q}_i\), e.g. \(x(q,t)\) not \(x(q,\dot{q},t)\).

\item {} 
Any constraints used to reduce the number of degrees of freedom are functions of \(\boldsymbol{q}\), but not of \(\dot{\boldsymbol{q}}\).

\item {} 
The motion is not dissipative (no damping or friction).

\end{itemize}

Going forward with the proof, consider \(x_i(q_1,q_2\cdots,t)\) and look
at the l.h.s. of Lagrange’s equations of motion.
\begin{equation*}
\begin{split}
\begin{eqnarray}
\frac{\partial T}{\partial\dot{q}_j}&=&\sum_i\frac{\partial T}{\partial\dot{x}_i}\frac{\partial\dot{x}_i}{\partial\dot{q_j}}
+\sum_i\frac{\partial T}{\partial x_i}\frac{\partial x_i}{\partial\dot{q_j}}\\
\nonumber
&=&\sum_i m\dot{x}_i\frac{\partial \dot{x}_i}{\partial\dot{q_j}}\\
\nonumber
&=&\sum_i m\dot{x}_i\frac{(\delta x_i/\delta t)|_{{\rm fixed~}q_{j'\ne j}}}{\delta q_j/\delta t}\\
\nonumber
&=&\sum_im\dot{x}_i\frac{\delta{x}_i|_{{\rm fixed~}q_{j'\ne j}}}{\delta q_j}\\
\nonumber
&=&\sum_i m\dot{x}_i\frac{\partial x_i}{\partial q_j}.
\end{eqnarray}
\end{split}
\end{equation*}
In the first line we used the fact that \(T\) does not depend on
\(x\). Continuing with taking the derivative of \(U\),
\begin{equation*}
\begin{split}
\begin{eqnarray}
-\frac{\partial U}{\partial\dot{q}_j}&=&-\sum_i\frac{\partial U}{\partial x_i}\frac{\partial x_i}{\partial\dot{q}_j}=0.
\end{eqnarray}
\end{split}
\end{equation*}
In the first line above we used the fact that \(U\) does not depend on \(\dot{x}\) then we used the second condition that \(x\) does not depend on \(\dot{q}\). Adding the two pieces together, then taking the derivative w.r.t. time,
\begin{equation*}
\begin{split}
\begin{eqnarray}
\nonumber
\frac{d}{dt}\frac{\partial}{\partial\dot{q}}(T-U)&=&\sum_im\ddot{x}_i\frac{\partial x_i}{\partial q_j}
+\sum_i m\dot{x}_i\frac{\partial\dot{x}_i}{\partial q_j}.
\end{eqnarray}
\end{split}
\end{equation*}
Now, we consider the r.h.s. of Lagrange’s equations. Because the
kinetic energy depends only on \(\dot{x}\) and not \(x\), and because the
potential depends on \(x\) but not \(\dot{x}\),
\begin{equation*}
\begin{split}
\begin{eqnarray}
\frac{\partial}{\partial q_j}(T-U)&=&\sum_i\frac{\partial T}{\partial\dot{x}_i}\frac{\partial\dot{x_i}}{\partial q_j}
-\sum_i\frac{\partial U}{\partial x_i}\frac{\partial x_i}{\partial q_j}\\
\nonumber
&=&\sum_i m\dot{x}_i\frac{\partial\dot{x_i}}{\partial q_j}
-\sum_i\frac{\partial U}{\partial x_i}\frac{\partial x_i}{\partial q_j}
\end{eqnarray}
\end{split}
\end{equation*}
Using the fact that \(m\ddot{x}_i=-(\partial/\partial x_i)U\), one can
see that the bottom expressions above are identical,




\begin{equation*}
\begin{split}
\begin{equation}
\frac{d}{dt}\frac{\partial}{\partial\dot{q}_i}(T-U)=\frac{\partial}{\partial q_i}(T-U).
\label{_auto99} \tag{135}
\end{equation}
\end{split}
\end{equation*}

\subsection{Lagrangian Examples}
\label{\detokenize{chapter1:lagrangian-examples}}
Two examples are presented here. In the first, there are two
generalized coordinates, but the two equations of motion can be
reduced to one through conservation laws (angular momentum in this
case). In the second, there is a time\sphinxhyphen{}dependent constraint.

Consider a cone of half angle \(\alpha\) standing on its tip at the
origin. The surface of the cone is defined as
\begin{equation*}
\begin{split}
r=\sqrt{x^2+y^2}=z\tan \alpha.
\end{split}
\end{equation*}
Find the equations of motion for a particle of mass \(m\) moving along the surface under the influence of a constant gravitational force, \(-mg\hat{z}\). For generalized coordinates use the azimuthal angle \(\phi\) and \(r\).

The kinetic energy is
\begin{equation*}
\begin{split}
\begin{eqnarray*}
T&=&\frac{1}{2}mr^2\dot{\theta}^2+\frac{1}{2}m(\dot{r}^2+\dot{z}^2)\\
&=&\frac{1}{2}mr^2\dot{\theta}^2+\frac{1}{2}m\dot{r}^2\left(1+\cot^2\alpha\right)\\
&=&\frac{1}{2}mr^2\dot{\theta}^2+\frac{1}{2}m\dot{r}^2\csc^2\alpha.
\end{eqnarray*}
\end{split}
\end{equation*}
The potential energy is
\begin{equation*}
\begin{split}
U=mgr\cot\alpha,
\end{split}
\end{equation*}
so Lagrange’s equations give
\begin{equation*}
\begin{split}
\begin{eqnarray*}
\frac{d}{dt}\left(mr^2\dot{\theta}\right)&=&0,\\
\frac{d}{dt}\left(m\csc^2\alpha \dot{r}\right)&=&mr\dot{\theta}^2-mg\cot\alpha,\\
\ddot{r}&=&r\dot{\theta}^2\sin^2\alpha-g\cos\alpha\sin\alpha
\end{eqnarray*}
\end{split}
\end{equation*}
The first equation is a statement of the conservation of angular
momentum with \(L=mr^2\dot{\theta}\), so the second equation can also be
expressed as
\begin{equation*}
\begin{split}
\ddot{r}=\frac{L^2\sin^2\alpha}{m^2r^3}-g\sin\alpha\cos\alpha.
\end{split}
\end{equation*}
A bead slides along a wire bent in the shape of a parabola,
\begin{equation*}
\begin{split}
z=\frac{1}{2}kr^2,~~r^2=x^2+y^2.
\end{split}
\end{equation*}
Also, the parabolic wire is rotating about the \(z\) axis with angular
velocity \(\omega\). Derive the equations of motion. Are there any
stable configurations?

Using the fact that
\begin{equation*}
\begin{split}
\dot{z}=\dot{r}\frac{\partial z}{\partial r}=kr\dot{r},
\end{split}
\end{equation*}
the kinetic and potential energies are
\begin{equation*}
\begin{split}
\begin{eqnarray*}
T&=&\frac{1}{2}m\left(\dot{r}^2+\dot{z}^2+r^2\omega^2\right)\\
&=&\frac{1}{2}m\left(\dot{r}^2+(kr\dot{r})^2+r^2\omega^2\right),\\
U&=&mgkr^2/2.
\end{eqnarray*}
\end{split}
\end{equation*}
The equations of motion are then
\begin{equation*}
\begin{split}
\begin{eqnarray*}
\frac{d}{dt}\left\{m\dot{r}(1+k^2r^2)\right\}&=&-mgkr+mk^2\dot{r}^2r+m\omega^2r,\\
\ddot{r}&=&\frac{-gkr+\omega^2r-k^2\dot{r}^2r}{1+k^2r^2}
\end{eqnarray*}
\end{split}
\end{equation*}
For a stable configuration, there needs to be a solution with
\(\dot{r}=0\) and \(\ddot{r}=0\). This can only happen at \(r=0\), and then
for the acceleration to be inward for small deviations of \(r\) one
needs to have \(gk>\omega^2\). If \(\omega^2>gk\) the bead will move
outward indefinitely.


\section{Small Vibrations and Normal Modes}
\label{\detokenize{chapter1:small-vibrations-and-normal-modes}}
Two examples are provided for solving for normal modes. These are
solutions with multiple generalized coordinates, where the motion is
that of simple harmonic motion. However, the motion is only simple for
a particular set of coordinates \(q_1\) and \(q_2\),
\begin{equation*}
\begin{split}
\begin{eqnarray}
q_1&=&A\cos(\omega_1 t),\\
\nonumber
q_2&=&B\cos(\omega_2 t),
\end{eqnarray}
\end{split}
\end{equation*}
while it is not necessarily simple in other coordinates. For example
if \(x=q_1+q_2\), and \(y=q_1-q_2\), the \(x\) and \(y\) motions will contain
mixtures of multiple frequencies. For many problems, or in the limit
of small vibrations about a minimum, there is some coordinate system
where the motion is simple. These are normal modes. Characterizing the
normal modes involves finding the frequencies, \(\omega_i\), and the
coordinate system where the motion is simple for each coordinate. This
involves finding the direction, or the linear combination of \(x_i\)
that form the coordinates \(q_i\) in which the motion is that of a
single oscillator in each coordinate.

For a first example, we consider a system of springs, where we write
the Lagrangian, then find the normal modes. For the second example, a
double pendulum is considered. In this case, one must first make a
small angle expansion before finding the modes. In principle, problems
could have the same number of normal modes a degrees of freedom. For
example, a system of 7 particles moving in three dimensions has 21
degrees of freedom. However, some of the degrees of freedom do not
have oscillatory behavior. For example, for a rigid body in free
space, the angles describing the orientation evolve, but do not
oscillate. Also, the center\sphinxhyphen{}of\sphinxhyphen{}mass coordinates of a system of
particles isolated from outside particles moves at constant
velocity. One can also describe these as normal modes, but acknowledge
that their characteristic frequency is zero, as there are no restoring
forces.

Consider two springs, whose relaxed lengths are \(\ell\), connected to three masses as depicted in the figure here. Describe the two normal modes of the motion. We can write the Lagrangian as
\begin{equation*}
\begin{split}
\begin{eqnarray*}
\mathcal{L}&=&\frac{m}{2}\dot{x}_1^2+m\dot{x}_2^2+\frac{m}{2}\dot{x}_3^2
-\frac{k}{2}(x_2-x_1-\ell)^2-\frac{k}{2}(x_3-x_2-\ell)^2.
\end{eqnarray*}
\end{split}
\end{equation*}
There are three coordinates, thus there are three equations of motion,
\begin{equation*}
\begin{split}
\begin{eqnarray*}
m\ddot{x}_1&=&-k(x_1-x_2+\ell)\\
2m\ddot{x}_2&=&-k(x_2-x_1-\ell)-k(x_2-x_3+\ell)\\
&=&-k(2x_2-x_1-x_3)\\
m\ddot{x}_3&=&-k(x_3-x_2+\ell).
\end{eqnarray*}
\end{split}
\end{equation*}
This is a bit complicated because the center\sphinxhyphen{}of\sphinxhyphen{}mass motion does not easily separate from the three equations. Instead, choose the following coordinates,
\begin{equation*}
\begin{split}
\begin{eqnarray*}
X&=&\frac{x_1+2x_2+x_3}{4},\\
q_1&=&x_1-x_2+\ell,\\
q_3&=&x_3-x_2-\ell.
\end{eqnarray*}
\end{split}
\end{equation*}
In these coordinates the potential energy only involves two coordinates,
\begin{equation*}
\begin{split}
\begin{eqnarray*}
U&=&\frac{k}{2}(q_1^2+q_3^2).
\end{eqnarray*}
\end{split}
\end{equation*}
To express the kinetic energy express \(x_1, x_2\) and \(x_3\) in terms of
\(X\), \(q_1\) and \(q_3\),
\begin{equation*}
\begin{split}
\begin{eqnarray*}
x_1&=&(3q_1-q_3-4\ell+4X)/4,\\
x_2&=&(4X-q_1-q_3)/4,\\
x_3&=&(3q_3-q_1+4\ell+4X)/4.
\end{eqnarray*}
\end{split}
\end{equation*}
The kinetic energy and Lagrangian are them
\begin{equation*}
\begin{split}
\begin{eqnarray*}
T&=&\frac{m}{2}\frac{1}{16}(3\dot{q}_1-\dot{q}_3+4\dot{X})^2
+m\frac{1}{16}(4\dot{X}-\dot{q}_1-\dot{q}_3)^2
+\frac{m}{2}\frac{1}{16}(3\dot{q}_3-\dot{q}_1+4\dot{X})^2\\
&=&\frac{3m}{8}(\dot{q}_1^2+\dot{q}_3^2)-\frac{m}{4}\dot{q}_1\dot{q}_3
+2m\dot{X}^2,\\
\mathcal{L}&=&\frac{3m}{8}(\dot{q}_1^2+\dot{q}_3^2)-\frac{m}{4}\dot{q}_1\dot{q}_3
+2m\dot{X}^2-\frac{k}{2}q_1^2-\frac{k}{2}q_3^2.
\end{eqnarray*}
\end{split}
\end{equation*}
The three equations of motion are then,
\begin{equation*}
\begin{split}
\begin{eqnarray*}
\frac{3}{4}m\ddot{q}_1-\frac{1}{4}m\ddot{q}_3&=&-kq_1,\\
\frac{3}{4}m\ddot{q}_3-\frac{1}{4}m\ddot{q}_1&=&-kq_3,\\
4M\ddot{X}&=&0.
\end{eqnarray*}
\end{split}
\end{equation*}
The last equation simply states that the center\sphinxhyphen{}of\sphinxhyphen{}mass velocity is
fixed. One could obtain the same result by summing the equations of
motion for \(x_1\), \(2x_2\) and \(x_3\) above. The second two equations are
more complicated. To solve them, we assume a form
\begin{equation*}
\begin{split}
\begin{eqnarray*}
q_1&=&Ae^{i\omega t},\\
q_3&=&Be^{i\omega t},
\end{eqnarray*}
\end{split}
\end{equation*}
Because this is a linear equation, we can multiply the solution by a
constant and it will still be a solution. Thus, we can set \(B=1\), then
solve for \(A\), effectively solving for \(A/B\). Putting this guess into
the equations of motion,
\begin{equation*}
\begin{split}
\begin{eqnarray*}
-\frac{3}{4}\frac{A}{B}\omega^2+\frac{1}{4}\omega^2&=&-\omega_0^2\frac{A}{B},\\
-\frac{3}{4}\omega^2+\frac{1}{4}\frac{A}{B}\omega^2&=&-\omega_0^2.
\end{eqnarray*}
\end{split}
\end{equation*}
This is two equations and two unknowns, \(\omega^2\) and
\(A/B\). Substituting for \(A/B\) gives a quadratic equation,
\begin{equation*}
\begin{split}
\begin{eqnarray*}
\omega^4-3\omega_0^2\omega^2+2\omega_0^4&=&0,\\
\omega_0^2&\equiv&k/m.
\end{eqnarray*}
\end{split}
\end{equation*}
The two solutions are
\begin{equation*}
\begin{split}
\begin{eqnarray*}
(1)~~\omega&=&\omega_0,~~~A=-B,\\
(2)~~\omega&=&\omega_0\sqrt{2},~~~A=B.
\end{eqnarray*}
\end{split}
\end{equation*}
The first solution corresponds to the two outer masses moving in
opposite directions, in sync, with the middle mass fixed. The second
solution has both outer masses moving in the same direction, but with
the center mass moving opposite. These two solutions are referred to
as normal modes, and are characterized by their frequency and by the
linear combinations of coordinates that oscillate together. In
general, the solution is a linear combination of normal modes, which
usually results in a chaotic looking motion. However, once the
solution is expressed in terms of the normal modes, each of which
oscillates independently in a simple manner, one can better understand
the motion. Further, the frequencies of these modes represent the
natural resonant frequencies of the system. This is important in the
construction of many structures, such as bridges or vehicles.

Consider a double pendulum confined to the \(x-y\) plane, where \(y\) is
vertical. A mass \(m\) is connected to the ceiling with a massless
string of length \(\ell\). A second mass \(m\) hangs from the first mass
with an identical massless string of the same length. Using \(\theta_1\)
and \(\theta_2\) to describe the orientations of the strings relative to
the vertical axis, find the Lagrangian and derive the equations of
motion, both for arbitrary angles and in the small\sphinxhyphen{}angle
approximation. Finally, express the equations of motion in the limit
of small oscillations.

The kinetic and potential energies are:
\begin{equation*}
\begin{split}
\begin{eqnarray*}
T&=&\frac{1}{2}m\ell^2\dot{\theta}_1^2
+\frac{1}{2}m\left\{(\ell\dot{\theta}_1\cos\theta_1+\ell\dot{\theta}_2\cos\theta_2)^2
+(\ell\dot{\theta}_1\sin\theta_1+\ell\dot{\theta}_2\sin\theta_2)^2\right\}\\
&=&\frac{1}{2}m\ell^2\left\{2\dot{\theta}_1^2+\dot{\theta}_2^2+2\dot{\theta}_1\dot{\theta}_2\cos(\theta_1-\theta_2)
\right\},\\
U&=&mg\ell(1-\cos\theta_1)+mg\left[\ell(1-\cos\theta_1)+\ell(1-\cos\theta_2)\right]\\
&=&mg\ell(3-2\cos\theta_1-\cos\theta_2)
\end{eqnarray*}
\end{split}
\end{equation*}
Lagrange’s equations for \(\theta_1\) lead to
\begin{equation*}
\begin{split}
\begin{eqnarray*}
m\ell^2\frac{d}{dt}\left\{2\dot{\theta}_1+\dot{\theta}_2\cos(\theta_1-\theta_2)\right\}&=&
-m\ell^2\dot{\theta}_1\dot{\theta}_2\sin(\theta_1-\theta_2)
-2mg\ell\sin\theta_1,\\
2\ddot{\theta}_1+\ddot{\theta}_2\cos(\theta_1-\theta_2)+\dot{\theta}_2^2\sin(\theta_1-\theta_2)
&=&-2\omega_0^2\sin\theta_1,\\
\omega_0^2&\equiv& g/\ell,
\end{eqnarray*}
\end{split}
\end{equation*}
and the equations for \(\theta_2\) are
\begin{equation*}
\begin{split}
\begin{eqnarray*}
m\ell^2\frac{d}{dt}\left\{\dot{\theta}_2+\dot{\theta}_1\cos(\theta_1-\theta_2)\right\}&=&
m\ell^2\dot{\theta}_1\dot{\theta}_2\sin(\theta_1-\theta_2)-mg\ell\sin\theta_2,\\
\ddot{\theta}_2+\ddot{\theta_1}\cos(\theta_1-\theta_2)&=&
-\omega_0^2\sin\theta_2.
\end{eqnarray*}
\end{split}
\end{equation*}
For small oscillations, one can only consider terms linear in \(\theta_1\) and \(\theta_2\) or their derivatives,
\begin{equation*}
\begin{split}
\begin{eqnarray}
2\ddot{\theta}_1+\ddot{\theta}_2&=&-2\omega_0^2\theta_1,\\
\nonumber
\ddot{\theta}_1+\ddot{\theta}_2&=&-\omega_0^2\theta_2.
\end{eqnarray}
\end{split}
\end{equation*}
To find the solutions, assume they are of the form
\(\theta_1=Ae^{i\omega t}, \theta_2=Be^{i\omega t}\). Solve for \(\omega\)
and \(A/B\), noting that \(B\) is arbitrary.

Plug in the desired form and find
\begin{equation*}
\begin{split}
\begin{eqnarray*}
e^{i\omega t}(-2\omega^2A-\omega^2B)&=&e^{i\omega t}(-2\omega_0^2A),\\
e^{i\omega t}(-\omega^2A-\omega^2B)&=&e^{i\omega t}(-\omega_0^2B).
\end{eqnarray*}
\end{split}
\end{equation*}
We can treat \(B\) as arbitrary and set it to unity. When we find \(A\),
it is the same as \(A/B\) for arbitrary \(B\). This gives the equations
\begin{equation*}
\begin{split}
\begin{eqnarray*}
2\omega^2A+\omega^2&=&2\omega_0^2A,\\
\omega^2A+\omega^2&=&\omega_0^2.
\end{eqnarray*}
\end{split}
\end{equation*}
This is two equations and two unknowns. Solving them leads to a
quadratic equation with solutions
\begin{equation*}
\begin{split}
\begin{eqnarray*}
A/B&=&\pm\frac{1}{\sqrt{2}},\\
\omega^2&=&\frac{\omega_0^2}{1\pm 1/\sqrt{2}}.
\end{eqnarray*}
\end{split}
\end{equation*}
Again, these two solutions are the normal modes, and the general
solution is a sum of the two solutions, with two arbitrary
constants. For the angles \(\theta_1\) and \(\theta_2\) are:
\begin{equation*}
\begin{split}
\begin{eqnarray*}
\theta_1&=&\frac{A_+}{\sqrt{2}}e^{i\omega_+t}, ~\theta_2=A_+e^{i\omega_+t},\\
\theta_1&=&\frac{-A_-}{\sqrt{2}}e^{i\omega_-t}, ~\theta_2=A_-e^{i\omega_-t},\\
\omega_{\pm}&=&\omega_0\sqrt{\frac{1}{1\pm 1/\sqrt{2}}}.
\end{eqnarray*}
\end{split}
\end{equation*}
One can also express the solution in vector notation, with the vectors
having arbitrary amplitudes \(A_+\) and \(A_-\),
\begin{equation*}
\begin{split}
\begin{eqnarray*}
\theta_+&=&\left(\begin{array}{c}
\frac{1}{\sqrt{2}}\\ 1\end{array}\right)A_+e^{i\omega_+t},\\
\theta_-&=&\left(\begin{array}{c}
\frac{-1}{\sqrt{2}}\\ 1\end{array}\right)A_-e^{i\omega_-t}.
\end{eqnarray*}
\end{split}
\end{equation*}
Here, the upper/lower components of the vector describe
\(\theta_1/\theta_2\) respectively.

These problems can be treated as linear algebra exercises. Linear
algebra is not used in this course, but nonetheless we describe how
this works for the curious student. In the limit of small vibrations,
the equations of motion can be expressed in the form,
\begin{equation*}
\begin{split}
\begin{eqnarray*}
M\ddot{q}&=&-Kq,
\end{eqnarray*}
\end{split}
\end{equation*}
a form that looks like the spring equation. However, \(q\) is an
\(n-\)dimensional vector and \(M\) and \(k\) are \(n\times n\) matrices. In
the double pendulum example, the dimensionality is 2 and the \(q\)
refers to the \(\theta_1\) and \(\theta_2\), and the matrices for \(M\) and
\(K\) can be read off (add ref).
\begin{equation*}
\begin{split}
\begin{eqnarray*}
M&=&\left(\begin{array}{cc}
2&1\\
1&1\end{array}\right)~,\hspace*{40pt} K=\left(\begin{array}{cc}
2\omega_0^2&0\\
0&\omega_0^2\end{array}\right).
\end{eqnarray*}
\end{split}
\end{equation*}
Multiplying both sides of the equation by the inverse matrix \(M^{-1}\),
\begin{equation*}
\begin{split}
\begin{eqnarray*}
\ddot{q}&=&-\left(M^{-1}K\right)q.
\end{eqnarray*}
\end{split}
\end{equation*}
Here,
\begin{equation*}
\begin{split}
\begin{eqnarray*}
M^{-1}&=&\left(\begin{array}{cc}
1&-1\\
-1&2\end{array}\right),\\
M^{-1}K&=&\left(\begin{array}{cc}
2&-1\\
-2&2\end{array}\right)\omega_0^2.
\end{eqnarray*}
\end{split}
\end{equation*}
One can find a transformation, basically a rotation, that transforms
to a frame where \(M^{-1}K\) is diagonal. In this coordinate system the
diagonal components of \(M^{-1}K\) represent the squared frequencies of
the normal modes,
\begin{equation*}
\begin{split}
M^{-1}K\rightarrow -\left(\begin{array}{cc}
\omega_+^2&0\\
0&\omega_-^2\end{array}\right)~,
\end{split}
\end{equation*}
and are known as “eigen” frequencies. The corresponding unit vectors,
\begin{equation*}
\begin{split}
\left(\begin{array}{c}
1\\0\end{array}\right)~{\rm and}~\left(\begin{array}{c}
0\\1\end{array}\right)~{\rm in~the~new~coordinate~system},
\end{split}
\end{equation*}
can be rotated back into the original frame, and become the solutions
for the normal modes. These are then called “eigenvectors”, which
are the same as the normal modes. Finding the eigenfrequencies is
performed by realizing that the determinant of a matrix is unchanged
by the rotation between coordinate systems. Writing the equations of
motion as an eigenvalue problem,
\begin{equation*}
\begin{split}
\begin{eqnarray}
\left[A-\lambda_i\mathbb{1}\right]u_i&=&0,~~~A\equiv M^{-1}K,~\lambda_i\equiv \omega^2_i.
\end{eqnarray}
\end{split}
\end{equation*}
In the coordinate system where \(M^{-1}K\) is diagonal, and the forms
for \(u_i\) are simple this requires that in that system, the diagonal
elements of \(M^{-1}K\) are the eigenvalues, \(\omega_i^2\). For each
\(\omega^2_i\), the determinant \(|A-\lambda_i\mathbb{1}|\) must
vanish. This is then true in any coordinate system,
\begin{equation*}
\begin{split}
\begin{eqnarray}
{\rm det}\left[A-\lambda\mathbb{1}\right]&=&0,
\end{eqnarray}
\end{split}
\end{equation*}
which for a \(2\times 2\) matrix becomes
\begin{equation*}
\begin{split}
\begin{eqnarray}
\left|
\begin{array}{cc}
A_{11}-\lambda&A_{12}\\
A_{21}&A_{22}-\lambda
\end{array}
\right|&=&0,\\
A_{11}A_{22}-\lambda A_{11}-\lambda A_{22}+\lambda^2-A_{21}A_{12}&=&0.
\end{eqnarray}
\end{split}
\end{equation*}
One can solve a quadratic equation for \(\lambda\), which gives two
eigenvalues corresponding to \(\omega_+^2\) and \(\omega_-^2\) found
above. Choosing one of the eigenvalues, one can insert one of the
eigenvalues \(\lambda_i\) into the eigenvalue problem  and solve for \(u_i\),
then choose the other eigenvalue and solve for the other corresponding
vector.

If this were a 3\sphinxhyphen{}dimensional set of equations, the determinant would
include terms like \(\lambda^3\) and would become a cubic equation with
three eigenvalues. One would then solve for three eigenvectors. If one
has a system with dimensionality \(n>2\), one usually resorts to solving
the problem numerically due to the messiness of the algebra. The main
programming languages all have packages which readily diagonalize
matrices and find eigenvectors and eigenvalues.


\section{Conservation Laws}
\label{\detokenize{chapter1:conservation-laws}}
Energy is conserved only when the Lagrangian has no explicit
dependence on time, i.e. \(L(q,\dot{q})\), not \(L(q,\dot{q},t)\). To show
this, we first define the Hamiltonian,
\begin{equation*}
\begin{split}
\begin{eqnarray}
H&=&\sum_i\left(\dot{q}_i\frac{\partial L}{\partial\dot{q}_i}\right)-L.
\end{eqnarray}
\end{split}
\end{equation*}
After showing that \(H\) is conserved, i.e. \((d/dt)H=0\), we then show
that \(H\) can be identified with the total energy, \(H=T+V\).

One can see that \(H\) is conserved by applying first using the chain
rule for \((d/dt)H\), then applying Lagrange’s
equations,
\begin{equation*}
\begin{split}
\begin{eqnarray}
\frac{d}{dt}H&=&\sum_i\left\{\ddot{q}_i\frac{\partial L}{\partial\dot{q}_i}+\dot{q}_i\frac{d}{dt}\left(\frac{\partial L}{\partial\dot{q}_i}\right)-\frac{\partial L}{\partial\dot{q}_i}\ddot{q}_i-\frac{\partial L}{\partial q_i}\dot{q}_i\right\}\\
\nonumber
&=&\sum_i\left\{\ddot{q}_i\frac{\partial L}{\partial\dot{q}_i}+\dot{q}_i\frac{\partial L}{\partial q_i}-\frac{\partial L}{\partial\dot{q}_i}\ddot{q}_i-\frac{\partial L}{\partial q_i}\dot{q}_i\right\}\\
\nonumber
&=&0.
\end{eqnarray}
\end{split}
\end{equation*}
These steps assumed that \(L\) had no explicit time dependence, i.e. \(L\)
is a function of \(q\) and \(\dot{q}\), but not of \(t\).

Next, we show that \(L\) can be identified with the energy. Because \(V\) does not depend on \(\dot{q}\),




\begin{equation*}
\begin{split}
\begin{equation}
H=\sum_i\frac{\partial T}{\partial\dot{q}_i}\dot{q}_i-T+V.
\label{_auto100} \tag{136}
\end{equation}
\end{split}
\end{equation*}
If the kinetic energy has a purely quadratic form in terms of \(\dot{q}\),




\begin{equation*}
\begin{split}
\begin{equation}
T=\sum_{ij}A_{ij}(q)\dot{q}_i\dot{q}_j,
\label{_auto101} \tag{137}
\end{equation}
\end{split}
\end{equation*}
the Hamiltonian becomes
\begin{equation*}
\begin{split}
\begin{eqnarray}
H&=&\sum_{ij}2A_{ij}(q)\dot{q}_i\dot{q}_j-\sum_{ij}A_{ij}(q)\dot{q}_i\dot{q}_j+V\\
\nonumber
&=&T+V.
\end{eqnarray}
\end{split}
\end{equation*}
The proof that \(H\) equals the energy hinged on the fact that the
kinetic energy was quadratic in \(\dot{q}\). This can be attributed to
time\sphinxhyphen{}reversal symmetry. Because the Cartesian coordinates \(x_i\) do not
depend on \(\dot{q}_i\) or on time, \(\dot{x}_i=(\partial x_i/\partial
q_j)\dot{q}_j\). Thus, the kinetic energy, \(T=m\dot{x}_i^2/2\), should
be proportional to two powers of \(\dot{q}\), which validates the
assumption above.

Here, energy conservation is predicated on the Lagrangian not having
an explicit time dependence. Without an explicit time dependence the
equations of motion are unchanged if one translates a fixed amount in
time because the physics does not depend on when the clock starts. In
contrast, the absolute time becomes relevant if there is an explicit
time dependence. In fact, conservation laws can usually be associated
with symmetries. In this case the translation symmetry in time leads
to energy conservation.

For another example of how symmetry leads to conservation laws,
consider a Lagrangian for a particle of mass \(m\) moving in a
two\sphinxhyphen{}dimensional plane where the generalized coordinates are the radius
\(r\) and the angle \(\theta\). The kinetic energy would be




\begin{equation*}
\begin{split}
\begin{equation}
T=\frac{1}{2}m\left\{\dot{r}^2+r^2\dot{\theta}^2\right\},
\label{_auto102} \tag{138}
\end{equation}
\end{split}
\end{equation*}
and if the potential energy \(V(r)\) depends only on the radius \(r\) and
not on the angle, Lagrange’s equations become
\begin{equation*}
\begin{split}
\begin{eqnarray}
\frac{d}{dt}(m\dot{r})&=&-\frac{\partial V}{\partial r}+m\dot{\theta}^2r,\\
\nonumber
\frac{d}{dt}(mr^2\dot{\theta})&=&0.
\end{eqnarray}
\end{split}
\end{equation*}
The second equation implies that \(mr^2\dot{\theta}\) is a
constant. Indeed, it is the angular momentum which is conserved for a
radial force. Here, the conservation of angular momentum is associated
with the independence of the physics to changes in \(\theta\), or in
other words, rotational invariance. Once one knows the fact that
\(L=mr^2\dot{\theta}\) is conserved, it can be inserted into the
equations of motion for \(\dot{r}\),




\begin{equation*}
\begin{split}
\begin{equation}
m\ddot{r}=-\frac{\partial V}{\partial r}+\frac{L^2}{mr^3}.
\label{_auto103} \tag{139}
\end{equation}
\end{split}
\end{equation*}
This is related to \sphinxhref{http://en.wikipedia.org/wiki/Noether\%27s\_theorem}{Emmy Noether’s theorem}

Simply stated, if the Lagrangian \(L\) is independent of \(q_i\), one can
see that the quantity \(\partial L/\partial\dot{q}_i\) is conserved,




\begin{equation*}
\begin{split}
\begin{equation}
\frac{d}{dt}\frac{\partial L}{\partial\dot{q}_i}=0.
\label{_auto104} \tag{140}
\end{equation}
\end{split}
\end{equation*}
Another easy example is in Cartesian coordinates where the potential
depends only on \(x\) and \(y\) but not on \(z\). In that case, there is a
translational symmetry. From the last equation, this translates
into conservation of the momentum in the \(z\) direction.

Consider a pair of particles of mass \(m_1\) and \(m_2\) where the potential is of the form
\begin{equation*}
\begin{split}
U(\boldsymbol{r}_1,\boldsymbol{r}_2)=V_a(|m_1\boldsymbol{r}_1+m_2\boldsymbol{r}_2|/(m_1+m_2))+V_b(|\boldsymbol{r}_1-\boldsymbol{r}_2|).
\end{split}
\end{equation*}
Using symmetry arguments alone, are there any conserved components of
the momentum? or the angular momentum??

There is no translational invariance, hence there are no conserved
components of the momentum. However, there is rotational invariance
about any axis that goes through the origin. Hence, there is angular
momentum conservation in all three directions. Symmetry arguments are
great ways to recognize the existence of conserved quantities, but
actually expressing them in terms of coordinates can be tricky. For
instance, you may need to write the Lagrangian in terms of angles.

















\renewcommand{\indexname}{Index}
\printindex
\end{document}