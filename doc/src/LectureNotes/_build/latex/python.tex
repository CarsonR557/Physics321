%% Generated by Sphinx.
\def\sphinxdocclass{report}
\documentclass[letterpaper,10pt,english]{sphinxmanual}
\ifdefined\pdfpxdimen
   \let\sphinxpxdimen\pdfpxdimen\else\newdimen\sphinxpxdimen
\fi \sphinxpxdimen=.75bp\relax

\PassOptionsToPackage{warn}{textcomp}
\usepackage[utf8]{inputenc}
\ifdefined\DeclareUnicodeCharacter
% support both utf8 and utf8x syntaxes
  \ifdefined\DeclareUnicodeCharacterAsOptional
    \def\sphinxDUC#1{\DeclareUnicodeCharacter{"#1}}
  \else
    \let\sphinxDUC\DeclareUnicodeCharacter
  \fi
  \sphinxDUC{00A0}{\nobreakspace}
  \sphinxDUC{2500}{\sphinxunichar{2500}}
  \sphinxDUC{2502}{\sphinxunichar{2502}}
  \sphinxDUC{2514}{\sphinxunichar{2514}}
  \sphinxDUC{251C}{\sphinxunichar{251C}}
  \sphinxDUC{2572}{\textbackslash}
\fi
\usepackage{cmap}
\usepackage[T1]{fontenc}
\usepackage{amsmath,amssymb,amstext}
\usepackage{babel}



\usepackage{times}
\expandafter\ifx\csname T@LGR\endcsname\relax
\else
% LGR was declared as font encoding
  \substitutefont{LGR}{\rmdefault}{cmr}
  \substitutefont{LGR}{\sfdefault}{cmss}
  \substitutefont{LGR}{\ttdefault}{cmtt}
\fi
\expandafter\ifx\csname T@X2\endcsname\relax
  \expandafter\ifx\csname T@T2A\endcsname\relax
  \else
  % T2A was declared as font encoding
    \substitutefont{T2A}{\rmdefault}{cmr}
    \substitutefont{T2A}{\sfdefault}{cmss}
    \substitutefont{T2A}{\ttdefault}{cmtt}
  \fi
\else
% X2 was declared as font encoding
  \substitutefont{X2}{\rmdefault}{cmr}
  \substitutefont{X2}{\sfdefault}{cmss}
  \substitutefont{X2}{\ttdefault}{cmtt}
\fi


\usepackage[Bjarne]{fncychap}
\usepackage[,numfigreset=1,mathnumfig]{sphinx}

\fvset{fontsize=\small}
\usepackage{geometry}


% Include hyperref last.
\usepackage{hyperref}
% Fix anchor placement for figures with captions.
\usepackage{hypcap}% it must be loaded after hyperref.
% Set up styles of URL: it should be placed after hyperref.
\urlstyle{same}

\usepackage{sphinxmessages}




\title{Classical Mechanics}
\date{Dec 13, 2020}
\release{}
\author{Morten Hjorth\sphinxhyphen{}Jensen}
\newcommand{\sphinxlogo}{\vbox{}}
\renewcommand{\releasename}{}
\makeindex
\begin{document}

\pagestyle{empty}
\sphinxmaketitle
\pagestyle{plain}
\sphinxtableofcontents
\pagestyle{normal}
\phantomsection\label{\detokenize{chapter1::doc}}








\sphinxstylestrong{\sphinxhref{http://mhjgit.github.io/info/doc/web/}{Morten Hjorth\sphinxhyphen{}Jensen}}, Department of Physics and Astronomy and National Superconducting Cyclotron Laboratory, Michigan State University, USA and Department of Physics, University of Oslo, Norway









Date: \sphinxstylestrong{Jan 22, 2020}

Copyright 1999\sphinxhyphen{}2020, \sphinxhref{http://mhjgit.github.io/info/doc/web/}{Morten Hjorth\sphinxhyphen{}Jensen}. Released under CC Attribution\sphinxhyphen{}NonCommercial 4.0 license


\chapter{Introduction}
\label{\detokenize{chapter1:introduction}}
Classical mechanics is a topic which has been taught intensively over
several centuries. It is, with its many variants and ways of
presenting the educational material, normally the first \sphinxstylestrong{real} physics
course many of us meet and it lays the foundation for further physics
studies. Many of the equations and ways of reasoning about the
underlying laws of motion and pertinent forces, shape our approaches and understanding
of the scientific method and discourse, as well as the way we develop our insights
and deeper understanding about physical systems.

There is a wealth of
well\sphinxhyphen{}tested (from both a physics point of view and a pedagogical
standpoint) exercises and problems which can be solved
analytically. However, many of these problems represent idealized and
less realistic situations.  The large majority of these problems are
solved by paper and pencil and are traditionally aimed
at what we normally refer to as continuous models from which we may find an analytical solution.  As a consequence,
when teaching mechanics, it implies that we can seldomly venture beyond an idealized case
in order to develop our understandings and insights about the
underlying forces and laws of motion.


\chapter{Numerical Elements}
\label{\detokenize{chapter1:numerical-elements}}
On the other hand, numerical algorithms call for approximate discrete
models and much of the development of methods for continuous models
are nowadays being replaced by methods for discrete models in science and
industry, simply because \sphinxstylestrong{much larger classes of problems can be addressed} with discrete models, often  by simpler and more
generic methodologies.

As we will see below, when properly scaling the equations at hand,
discrete models open up for more advanced abstractions and the possibility to
study  real life systems, with the added bonus that we can explore and
deepen our basic understanding of various physical systems

Analytical solutions are as important as before. In addition, such
solutions provide us with invaluable benchmarks and tests for our
discrete models. Such benchmarks, as we will see below, allow us
to discuss possible sources of errors and their behaviors.  And
finally, since most of our models are based on various algorithms from
numerical mathematics, we have a unique oppotunity to gain a deeper
understanding of the mathematical approaches we are using.

With computing and data science as important elements in essentially
all aspects of a modern society, we could  then try to define Computing as
\sphinxstylestrong{solving scientific problems using all possible tools, including
symbolic computing, computers and numerical algorithms, and analytical
paper and pencil solutions}.
Computing provides us with the tools to develope our own understanding of the scientific method by enhancing algorithmic thinking.


\chapter{Computations and the Scientific Method}
\label{\detokenize{chapter1:computations-and-the-scientific-method}}
The way we will teach this course reflects
this definition of computing. The course contains both classical paper
and pencil exercises as well as  computational projects and exercises. The
hope is that this will allow you to explore the physics of systems
governed by the degrees of freedom of classical mechanics at a deeper
level, and that these insights about the scientific method will help
you to develop a better understanding of how the underlying forces and
equations of motion and how they impact a given system. Furthermore, by introducing various numerical methods
via computational projects and exercises, we aim at developing your competences and skills about these topics.

These competences will enable you to
\begin{itemize}
\item {} 
understand how algorithms are used to solve mathematical problems,

\item {} 
derive, verify, and implement algorithms,

\item {} 
understand what can go wrong with algorithms,

\item {} 
use these algorithms to construct reproducible scientific outcomes and to engage in science in ethical ways, and

\item {} 
think algorithmically for the purposes of gaining deeper insights about scientific problems.

\end{itemize}

All these elements are central for maturing and gaining a better understanding of the modern scientific process \sphinxstyleemphasis{per se}.

The power of the scientific method lies in identifying a given problem
as a special case of an abstract class of problems, identifying
general solution methods for this class of problems, and applying a
general method to the specific problem (applying means, in the case of
computing, calculations by pen and paper, symbolic computing, or
numerical computing by ready\sphinxhyphen{}made and/or self\sphinxhyphen{}written software). This
generic view on problems and methods is particularly important for
understanding how to apply available, generic software to solve a
particular problem.

\sphinxstyleemphasis{However, verification of algorithms and understanding their limitations requires much of the classical knowledge about continuous models.}


\chapter{A well\sphinxhyphen{}known examples to illustrate many of the above concepts}
\label{\detokenize{chapter1:a-well-known-examples-to-illustrate-many-of-the-above-concepts}}
Before we venture into a reminder on Python and mechanics relevant applications, let us briefly outline some of the
abovementioned topics using an example many of you may have seen before in for example CMSE201.
A simple algorithm for integration is the Trapezoidal rule.
Integration of a function \(f(x)\) by the Trapezoidal Rule is given by following algorithm for an interval \(x \in [a,b]\)
\begin{equation*}
\begin{split}
\int_a^b(f(x) dx = \frac{1}{2}\left [f(a)+2f(a+h)+\dots+2f(b-h)+f(b)\right] +O(h^2),
\end{split}
\end{equation*}
where \(h\) is the so\sphinxhyphen{}called stepsize defined by the number of integration points \(N\) as \(h=(b-a)/(n)\).
Python offers an  extremely versatile programming  environment, allowing for
the inclusion of analytical studies in a numerical program. Here we show an
example code with the \sphinxstylestrong{trapezoidal rule}. We use also \sphinxstylestrong{SymPy} to evaluate the exact value of the integral and compute the absolute error
with respect to the numerically evaluated one of the integral
\(\int_0^1 dx x^2 = 1/3\).
The following code for  the trapezoidal rule allows you  to plot the relative error by comparing with the exact result. By increasing to \(10^8\) points one arrives at a region where numerical errors start to accumulate.

\begin{sphinxVerbatim}[commandchars=\\\{\}]
\PYG{o}{\PYGZpc{}}\PYG{k}{matplotlib} inline

\PYG{k+kn}{from} \PYG{n+nn}{math} \PYG{k+kn}{import} \PYG{n}{log10}
\PYG{k+kn}{import} \PYG{n+nn}{numpy} \PYG{k}{as} \PYG{n+nn}{np}
\PYG{k+kn}{from} \PYG{n+nn}{sympy} \PYG{k+kn}{import} \PYG{n}{Symbol}\PYG{p}{,} \PYG{n}{integrate}
\PYG{k+kn}{import} \PYG{n+nn}{matplotlib}\PYG{n+nn}{.}\PYG{n+nn}{pyplot} \PYG{k}{as} \PYG{n+nn}{plt}
\PYG{c+c1}{\PYGZsh{} function for the trapezoidal rule}
\PYG{k}{def} \PYG{n+nf}{Trapez}\PYG{p}{(}\PYG{n}{a}\PYG{p}{,}\PYG{n}{b}\PYG{p}{,}\PYG{n}{f}\PYG{p}{,}\PYG{n}{n}\PYG{p}{)}\PYG{p}{:}
   \PYG{n}{h} \PYG{o}{=} \PYG{p}{(}\PYG{n}{b}\PYG{o}{\PYGZhy{}}\PYG{n}{a}\PYG{p}{)}\PYG{o}{/}\PYG{n+nb}{float}\PYG{p}{(}\PYG{n}{n}\PYG{p}{)}
   \PYG{n}{s} \PYG{o}{=} \PYG{l+m+mi}{0}
   \PYG{n}{x} \PYG{o}{=} \PYG{n}{a}
   \PYG{k}{for} \PYG{n}{i} \PYG{o+ow}{in} \PYG{n+nb}{range}\PYG{p}{(}\PYG{l+m+mi}{1}\PYG{p}{,}\PYG{n}{n}\PYG{p}{,}\PYG{l+m+mi}{1}\PYG{p}{)}\PYG{p}{:}
       \PYG{n}{x} \PYG{o}{=} \PYG{n}{x}\PYG{o}{+}\PYG{n}{h}
       \PYG{n}{s} \PYG{o}{=} \PYG{n}{s}\PYG{o}{+} \PYG{n}{f}\PYG{p}{(}\PYG{n}{x}\PYG{p}{)}
   \PYG{n}{s} \PYG{o}{=} \PYG{l+m+mf}{0.5}\PYG{o}{*}\PYG{p}{(}\PYG{n}{f}\PYG{p}{(}\PYG{n}{a}\PYG{p}{)}\PYG{o}{+}\PYG{n}{f}\PYG{p}{(}\PYG{n}{b}\PYG{p}{)}\PYG{p}{)} \PYG{o}{+}\PYG{n}{s}
   \PYG{k}{return} \PYG{n}{h}\PYG{o}{*}\PYG{n}{s}
\PYG{c+c1}{\PYGZsh{}  function to compute pi}
\PYG{k}{def} \PYG{n+nf}{function}\PYG{p}{(}\PYG{n}{x}\PYG{p}{)}\PYG{p}{:}
    \PYG{k}{return} \PYG{n}{x}\PYG{o}{*}\PYG{n}{x}
\PYG{c+c1}{\PYGZsh{} define integration limits}
\PYG{n}{a} \PYG{o}{=} \PYG{l+m+mf}{0.0}\PYG{p}{;}  \PYG{n}{b} \PYG{o}{=} \PYG{l+m+mf}{1.0}\PYG{p}{;}
\PYG{c+c1}{\PYGZsh{} find result from sympy}
\PYG{c+c1}{\PYGZsh{} define x as a symbol to be used by sympy}
\PYG{n}{x} \PYG{o}{=} \PYG{n}{Symbol}\PYG{p}{(}\PYG{l+s+s1}{\PYGZsq{}}\PYG{l+s+s1}{x}\PYG{l+s+s1}{\PYGZsq{}}\PYG{p}{)}
\PYG{n}{exact} \PYG{o}{=} \PYG{n}{integrate}\PYG{p}{(}\PYG{n}{function}\PYG{p}{(}\PYG{n}{x}\PYG{p}{)}\PYG{p}{,} \PYG{p}{(}\PYG{n}{x}\PYG{p}{,} \PYG{n}{a}\PYG{p}{,} \PYG{n}{b}\PYG{p}{)}\PYG{p}{)}
\PYG{c+c1}{\PYGZsh{} set up the arrays for plotting the relative error}
\PYG{n}{n} \PYG{o}{=} \PYG{n}{np}\PYG{o}{.}\PYG{n}{zeros}\PYG{p}{(}\PYG{l+m+mi}{9}\PYG{p}{)}\PYG{p}{;} \PYG{n}{y} \PYG{o}{=} \PYG{n}{np}\PYG{o}{.}\PYG{n}{zeros}\PYG{p}{(}\PYG{l+m+mi}{9}\PYG{p}{)}\PYG{p}{;}
\PYG{c+c1}{\PYGZsh{} find the relative error as function of integration points}
\PYG{k}{for} \PYG{n}{i} \PYG{o+ow}{in} \PYG{n+nb}{range}\PYG{p}{(}\PYG{l+m+mi}{1}\PYG{p}{,} \PYG{l+m+mi}{8}\PYG{p}{,} \PYG{l+m+mi}{1}\PYG{p}{)}\PYG{p}{:}
    \PYG{n}{npts} \PYG{o}{=} \PYG{l+m+mi}{10}\PYG{o}{*}\PYG{o}{*}\PYG{n}{i}
    \PYG{n}{result} \PYG{o}{=} \PYG{n}{Trapez}\PYG{p}{(}\PYG{n}{a}\PYG{p}{,}\PYG{n}{b}\PYG{p}{,}\PYG{n}{function}\PYG{p}{,}\PYG{n}{npts}\PYG{p}{)}
    \PYG{n}{RelativeError} \PYG{o}{=} \PYG{n+nb}{abs}\PYG{p}{(}\PYG{p}{(}\PYG{n}{exact}\PYG{o}{\PYGZhy{}}\PYG{n}{result}\PYG{p}{)}\PYG{o}{/}\PYG{n}{exact}\PYG{p}{)}
    \PYG{n}{n}\PYG{p}{[}\PYG{n}{i}\PYG{p}{]} \PYG{o}{=} \PYG{n}{log10}\PYG{p}{(}\PYG{n}{npts}\PYG{p}{)}\PYG{p}{;} \PYG{n}{y}\PYG{p}{[}\PYG{n}{i}\PYG{p}{]} \PYG{o}{=} \PYG{n}{log10}\PYG{p}{(}\PYG{n}{RelativeError}\PYG{p}{)}\PYG{p}{;}
\PYG{n}{plt}\PYG{o}{.}\PYG{n}{plot}\PYG{p}{(}\PYG{n}{n}\PYG{p}{,}\PYG{n}{y}\PYG{p}{,} \PYG{l+s+s1}{\PYGZsq{}}\PYG{l+s+s1}{ro}\PYG{l+s+s1}{\PYGZsq{}}\PYG{p}{)}
\PYG{n}{plt}\PYG{o}{.}\PYG{n}{xlabel}\PYG{p}{(}\PYG{l+s+s1}{\PYGZsq{}}\PYG{l+s+s1}{n}\PYG{l+s+s1}{\PYGZsq{}}\PYG{p}{)}
\PYG{n}{plt}\PYG{o}{.}\PYG{n}{ylabel}\PYG{p}{(}\PYG{l+s+s1}{\PYGZsq{}}\PYG{l+s+s1}{Relative error}\PYG{l+s+s1}{\PYGZsq{}}\PYG{p}{)}
\PYG{n}{plt}\PYG{o}{.}\PYG{n}{show}\PYG{p}{(}\PYG{p}{)}
\end{sphinxVerbatim}

\noindent\sphinxincludegraphics{{chapter1_3_0}.png}


\chapter{Analyzing the above example}
\label{\detokenize{chapter1:analyzing-the-above-example}}
This example shows the potential of combining numerical algorithms with symbolic calculations, allowing us to
\begin{itemize}
\item {} 
Validate and verify  their  algorithms.

\item {} 
Including concepts like unit testing, one has the possibility to test and test several or all parts of the code.

\item {} 
Validation and verification are then included \sphinxstyleemphasis{naturally} and one can develop a better attitude to what is meant with an ethically sound scientific approach.

\item {} 
The above example allows the student to also test the mathematical error of the algorithm for the trapezoidal rule by changing the number of integration points. The students get \sphinxstylestrong{trained from day one to think error analysis}.

\item {} 
With a Jupyter notebook you can keep exploring similar examples and turn them in as your own notebooks.

\end{itemize}

In this process we can easily bake in
\begin{enumerate}
\sphinxsetlistlabels{\arabic}{enumi}{enumii}{}{.}%
\item {} 
How to structure a code in terms of functions

\item {} 
How to make a module

\item {} 
How to read input data flexibly from the command line

\item {} 
How to create graphical/web user interfaces

\item {} 
How to write unit tests (test functions or doctests)

\item {} 
How to refactor code in terms of classes (instead of functions only)

\item {} 
How to conduct and automate large\sphinxhyphen{}scale numerical experiments

\item {} 
How to write scientific reports in various formats (LaTeX, HTML)

\end{enumerate}

The conventions and techniques outlined here will save you a lot of time when you incrementally extend software over time from simpler to more complicated problems. In particular, you will benefit from many good habits:
\begin{enumerate}
\sphinxsetlistlabels{\arabic}{enumi}{enumii}{}{.}%
\item {} 
New code is added in a modular fashion to a library (modules)

\item {} 
Programs are run through convenient user interfaces

\item {} 
It takes one quick command to let all your code undergo heavy testing

\item {} 
Tedious manual work with running programs is automated,

\item {} 
Your scientific investigations are reproducible, scientific reports with top quality typesetting are produced both for paper and electronic devices.

\end{enumerate}


\chapter{Teaching team, grading and other practicalities}
\label{\detokenize{chapter1:teaching-team-grading-and-other-practicalities}}





\chapter{Grading and dates}
\label{\detokenize{chapter1:grading-and-dates}}





\chapter{Possible textbooks and lecture notes}
\label{\detokenize{chapter1:possible-textbooks-and-lecture-notes}}
\sphinxstylestrong{Recommended textbook}:
\begin{itemize}
\item {} 
\sphinxhref{https://www.uscibooks.com/taylor2.htm}{John R. Taylor, Classical Mechanics (Univ. Sci. Books 2005)}, see also \sphinxhref{https://github.com/mhjensen/Physics321/tree/master/doc/Literature}{the GitHub link of the course}

\end{itemize}

\sphinxstylestrong{Additional textbooks}:
\begin{itemize}
\item {} 
\sphinxhref{https://www.springer.com/gp/book/9783319195957}{Anders Malthe\sphinxhyphen{}Sørenssen, Elementary Mechanics using Python (Springer 2015)} and \sphinxhref{https://github.com/mhjensen/Physics321/tree/master/doc/Literature}{the GitHub link of the course}

\item {} 
\sphinxhref{https://www.springer.com/gp/book/9783319292564}{Alessandro Bettini, A Course in Classical Physics 1, Mechanics (Springer 2017)} and the \sphinxhref{https://github.com/mhjensen/Physics321/tree/master/doc/Literature}{GitHub link of the course}.

\end{itemize}

The books from Springer can be downloaded for free (pdf or ebook format) from any MSU IP address.

\sphinxstylestrong{Lecture notes}:
Posted lecture notes are in the doc/pub folder here or at \sphinxurl{https://mhjensen.github.io/Physics321/doc/web/course.html} for easier viewing. They are not meant to be a replacement for textbook. These notes are updated on a weekly basis and a \sphinxstylestrong{git pull} should thus always give you the latest update.




\section{PHY321: Forces, Newton’s Laws and Motion Example}
\label{\detokenize{chapter2:phy321-forces-newton-s-laws-and-motion-example}}\label{\detokenize{chapter2::doc}}




\sphinxstylestrong{\sphinxhref{http://mhjgit.github.io/info/doc/web/}{Morten Hjorth\sphinxhyphen{}Jensen}}, Department of Physics and Astronomy and National Superconducting Cyclotron Laboratory, Michigan State University, USA and Department of Physics, University of Oslo, Norway









Date: \sphinxstylestrong{Feb 19, 2020}

Copyright 1999\sphinxhyphen{}2020, \sphinxhref{http://mhjgit.github.io/info/doc/web/}{Morten Hjorth\sphinxhyphen{}Jensen}. Released under CC Attribution\sphinxhyphen{}NonCommercial 4.0 license


\subsection{Basic Steps of Scientific Investigations}
\label{\detokenize{chapter2:basic-steps-of-scientific-investigations}}
An overarching aim in this course is to give you a deeper
understanding of the scientific method. The problems we study will all
involve cases where we can apply classical mechanics. In our previous
material we already assumed that we had a model for the motion of an
object.  Alternatively we could have data from experiment (like Usain
Bolt’s 100m world record run in 2008).  Or we could have performed
ourselves an experiment and we want to understand which forces are at
play and whether these forces can be understood in terms of
fundamental forces.

Our first step consists in identifying the problem. What we sketch
here may include a mix of experiment and theoretical simulations, or
just experiment or only theory.


\subsection{Identifying our System}
\label{\detokenize{chapter2:identifying-our-system}}
Here we can ask questions like
\begin{enumerate}
\sphinxsetlistlabels{\arabic}{enumi}{enumii}{}{.}%
\item {} 
What kind of object is moving

\item {} 
What kind of data do we have

\item {} 
How do we measure position, velocity, acceleration etc

\item {} 
Which initial conditions influence our system

\item {} 
Other aspects which allow us to identify the system

\end{enumerate}


\subsection{Defining a Model}
\label{\detokenize{chapter2:defining-a-model}}
With our eventual data and observations we would now like to develop a
model for the system. In the end we want obviously to be able to
understand which forces are at play and how they influence our
specific system. That is, can we extract some deeper insights about a
system?

We need then to
\begin{enumerate}
\sphinxsetlistlabels{\arabic}{enumi}{enumii}{}{.}%
\item {} 
Find the forces that act on our system

\item {} 
Introduce models for the forces

\item {} 
Identify the equations which can govern the system (Newton’s second law for example)

\item {} 
More elements we deem important for defining our model

\end{enumerate}


\subsection{Solving the Equations}
\label{\detokenize{chapter2:solving-the-equations}}
With the model at hand, we can then solve the equations. In classical mechanics we normally end up  with solving sets of coupled ordinary differential equations or partial differential equations.
\begin{enumerate}
\sphinxsetlistlabels{\arabic}{enumi}{enumii}{}{.}%
\item {} 
Using Newton’s second law we have equations of the type \(\boldsymbol{F}=m\boldsymbol{a}=md\boldsymbol{v}/dt\)

\item {} 
We need to  define the initial conditions (typically the initial velocity and position as functions of time) and/or initial conditions and boundary conditions

\item {} 
The solution of the equations give us then the position, the velocity and other time\sphinxhyphen{}dependent quantities which may specify the motion of a given object.

\end{enumerate}

We are not yet done. With our lovely solvers, we need to start thinking.


\subsection{Analyze}
\label{\detokenize{chapter2:analyze}}
Now it is time to ask the big questions. What do our results mean? Can we give a simple interpretation in terms of fundamental laws?  What do our results mean? Are they correct?
Thus, typical questions we may ask are
\begin{enumerate}
\sphinxsetlistlabels{\arabic}{enumi}{enumii}{}{.}%
\item {} 
Are our results for say \(\boldsymbol{r}(t)\) valid?  Do we trust what we did?  Can you validate and verify the correctness of your results?

\item {} 
Evaluate the answers and their implications

\item {} 
Compare with experimental data if possible. Does our model make sense?

\item {} 
and obviously many other questions.

\end{enumerate}

The analysis stage feeds back to the first stage. It may happen that
the data we had were not good enough, there could be large statistical
uncertainties. We may need to collect more data or perhaps we did a
sloppy job in identifying the degrees of freedom.

All these steps are essential elements in a scientific
enquiry. Hopefully, through a mix of numerical simulations, analytical
calculations and experiments we may gain a deeper insight about the
physics of a specific system.

Let us now remind ourselves of Newton’s laws, since these are the laws of motion we will study in this course.


\subsection{Newton’s Laws}
\label{\detokenize{chapter2:newton-s-laws}}
When analyzing a physical system we normally start with distinguishing between the object we are studying (we will label this in more general terms as our \sphinxstylestrong{system}) and how this system interacts with the environment (which often means everything else!)

In our investigations we will thus analyze a specific physics problem in terms of the system and the environment.
In doing so we need to identify the forces that act on the system and assume that the
forces acting on the system must have a source, an identifiable cause in
the environment.

A force acting on for example a falling object must be related to an interaction with something in the environment.
This also means that we do not consider internal forces. The latter are forces between
one part of the object and another part. In this course we will mainly focus on external forces.

Forces are either contact forces or long\sphinxhyphen{}range forces.

Contact forces, as evident from the name, are forces that occur at the contact between
the system and the environment. Well\sphinxhyphen{}known long\sphinxhyphen{}range forces are the gravitional force and the electromagnetic force.


\subsection{Setting up a model for forces acting on an object}
\label{\detokenize{chapter2:setting-up-a-model-for-forces-acting-on-an-object}}
In order to set up the forces which act on an object, the following steps may be useful
\begin{enumerate}
\sphinxsetlistlabels{\arabic}{enumi}{enumii}{}{.}%
\item {} 
Divide the problem into system and environment.

\item {} 
Draw a figure of the object and everything in contact with the object.

\item {} 
Draw a closed curve around the system.

\item {} 
Find contact points—these are the points where contact forces may act.

\item {} 
Give names and symbols to all the contact forces.

\item {} 
Identify the long\sphinxhyphen{}range forces.

\item {} 
Make a drawing of the object. Draw the forces as arrows, vectors, starting from where the force is acting. The direction of the vector(s) indicates the (positive) direction of the force. Try to make the length of the arrow indicate the relative magnitude of the forces.

\item {} 
Draw in the axes of the coordinate system. It is often convenient to make one axis parallel to the direction of motion. When you choose the direction of the axis you also choose the positive direction for the axis.

\end{enumerate}


\subsection{Newton’s Laws, the Second one first}
\label{\detokenize{chapter2:newton-s-laws-the-second-one-first}}
Newton’s second law of motion: The force \(\boldsymbol{F}\) on an object of inertial mass \(m\)
is related to the acceleration a of the object through
\begin{equation*}
\begin{split}
\boldsymbol{F} = m\boldsymbol{a},
\end{split}
\end{equation*}
where \(\boldsymbol{a}\) is the acceleration.

Newton’s laws of motion are laws of nature that have been found by experimental
investigations and have been shown to hold up to continued experimental investigations.
Newton’s laws are valid over a wide range of length\sphinxhyphen{} and time\sphinxhyphen{}scales. We
use Newton’s laws of motion to describe everything from the motion of atoms to the
motion of galaxies.

The second law is a vector equation with the acceleration having the same
direction as the force. The acceleration is proportional to the force via the mass \(m\) of the system under study.

Newton’s second law introduces a new property of an object, the so\sphinxhyphen{}called
inertial mass \(m\). We determine the inertial mass of an object by measuring the
acceleration for a given applied force.


\subsection{Then the First Law}
\label{\detokenize{chapter2:then-the-first-law}}
What happens if the net external force on a body is zero? Applying Newton’s second
law, we find:
\begin{equation*}
\begin{split}
\boldsymbol{F} = 0 = m\boldsymbol{a},
\end{split}
\end{equation*}
which gives using the definition of the acceleration
\begin{equation*}
\begin{split}
\boldsymbol{a} = \frac{d\boldsymbol{v}}{dt}=0.
\end{split}
\end{equation*}
The acceleration is zero, which means that the velocity of the object is constant. This
is often referred to as Newton’s first law. An object in a state of uniform motion tends to remain in
that state unless an external force changes its state of motion.
Why do we need a separate law for this? Is it not simply a special case of Newton’s
second law? Yes, Newton’s first law can be deduced from the second law as we have
illustrated. However, the first law is often used for a different purpose: Newton’s
First Law tells us about the limit of applicability of Newton’s Second law. Newton’s
Second law can only be used in reference systems where the First law is obeyed. But
is not the First law always valid? No! The First law is only valid in reference systems
that are not accelerated. If you observe the motion of a ball from an accelerating
car, the ball will appear to accelerate even if there are no forces acting on it. We call
systems that are not accelerating inertial systems, and Newton’s first law is often
called the law of inertia. Newton’s first and second laws of motion are only valid in
inertial systems.

A system is an inertial system if it is not accelerated. It means that the reference system
must not be accelerating linearly or rotating. Unfortunately, this means that most
systems we know are not really inertial systems. For example, the surface of the
Earth is clearly not an inertial system, because the Earth is rotating. The Earth is also
not an inertial system, because it ismoving in a curved path around the Sun. However,
even if the surface of the Earth is not strictly an inertial system, it may be considered
to be approximately an inertial system for many laboratory\sphinxhyphen{}size experiments.


\subsection{And finally the Third Law}
\label{\detokenize{chapter2:and-finally-the-third-law}}
If there is a force from object A on object B, there is also a force from object B on object A.
This fundamental principle of interactions is called Newton’s third law. We do not
know of any force that do not obey this law: All forces appear in pairs. Newton’s
third law is usually formulated as: For every action there is an equal and opposite
reaction.


\subsection{Motion of a Single Object}
\label{\detokenize{chapter2:motion-of-a-single-object}}
Here we consider the motion of a single particle moving under
the influence of some set of forces.  We will consider some problems where
the force does not depend on the position. In that case Newton’s law
\(m\dot{\boldsymbol{v}}=\boldsymbol{F}(\boldsymbol{v})\) is a first\sphinxhyphen{}order differential
equation and one solves for \(\boldsymbol{v}(t)\), then moves on to integrate
\(\boldsymbol{v}\) to get the position. In essentially all of these cases we cna find an analytical solution.


\subsection{Air Resistance in One Dimension}
\label{\detokenize{chapter2:air-resistance-in-one-dimension}}
Air resistance tends to scale as the square of the velocity. This is
in contrast to many problems chosen for textbooks, where it is linear
in the velocity. The choice of a linear dependence is motivated by
mathematical simplicity (it keeps the differential equation linear)
rather than by physics. One can see that the force should be quadratic
in velocity by considering the momentum imparted on the air
molecules. If an object sweeps through a volume \(dV\) of air in time
\(dt\), the momentum imparted on the air is




\begin{equation*}
\begin{split}
\begin{equation}
dP=\rho_m dV v,
\label{_auto1} \tag{1}
\end{equation}
\end{split}
\end{equation*}
where \(v\) is the velocity of the object and \(\rho_m\) is the mass
density of the air. If the molecules bounce back as opposed to stop
you would double the size of the term. The opposite value of the
momentum is imparted onto the object itself. Geometrically, the
differential volume is




\begin{equation*}
\begin{split}
\begin{equation}
dV=Avdt,
\label{_auto2} \tag{2}
\end{equation}
\end{split}
\end{equation*}
where \(A\) is the cross\sphinxhyphen{}sectional area and \(vdt\) is the distance the
object moved in time \(dt\).


\subsection{Resulting Acceleration}
\label{\detokenize{chapter2:resulting-acceleration}}
Plugging this into the expression above,




\begin{equation*}
\begin{split}
\begin{equation}
\frac{dP}{dt}=-\rho_m A v^2.
\label{_auto3} \tag{3}
\end{equation}
\end{split}
\end{equation*}
This is the force felt by the particle, and is opposite to its
direction of motion. Now, because air doesn’t stop when it hits an
object, but flows around the best it can, the actual force is reduced
by a dimensionless factor \(c_W\), called the drag coefficient.




\begin{equation*}
\begin{split}
\begin{equation}
F_{\rm drag}=-c_W\rho_m Av^2,
\label{_auto4} \tag{4}
\end{equation}
\end{split}
\end{equation*}
and the acceleration is
\begin{equation*}
\begin{split}
\begin{eqnarray}
\frac{dv}{dt}=-\frac{c_W\rho_mA}{m}v^2.
\end{eqnarray}
\end{split}
\end{equation*}
For a particle with initial velocity \(v_0\), one can separate the \(dt\)
to one side of the equation, and move everything with \(v\)s to the
other side. We did this in our discussion of simple motion and will not repeat it here.

On more general terms,
for many systems, e.g. an automobile, there are multiple sources of
resistance. In addition to wind resistance, where the force is
proportional to \(v^2\), there are dissipative effects of the tires on
the pavement, and in the axel and drive train. These other forces can
have components that scale proportional to \(v\), and components that
are independent of \(v\). Those independent of \(v\), e.g. the usual
\(f=\mu_K N\) frictional force you consider in your first Physics courses, only set in
once the object is actually moving. As speeds become higher, the \(v^2\)
components begin to dominate relative to the others. For automobiles
at freeway speeds, the \(v^2\) terms are largely responsible for the
loss of efficiency. To travel a distance \(L\) at fixed speed \(v\), the
energy/work required to overcome the dissipative forces are \(fL\),
which for a force of the form \(f=\alpha v^n\) becomes
\begin{equation*}
\begin{split}
\begin{eqnarray}
W=\int dx~f=\alpha v^n L.
\end{eqnarray}
\end{split}
\end{equation*}
For \(n=0\) the work is
independent of speed, but for the wind resistance, where \(n=2\),
slowing down is essential if one wishes to reduce fuel consumption. It
is also important to consider that engines are designed to be most
efficient at a chosen range of power output. Thus, some cars will get
better mileage at higher speeds (They perform better at 50 mph than at
5 mph) despite the considerations mentioned above.


\subsection{Going Ballistic, Projectile Motion or a Softer Approach, Falling Raindrops}
\label{\detokenize{chapter2:going-ballistic-projectile-motion-or-a-softer-approach-falling-raindrops}}
As an example of Newton’s Laws we consider projectile motion (or a
falling raindrop or a ball we throw up in the air) with a drag force. Even though air resistance is
largely proportional to the square of the velocity, we will consider
the drag force to be linear to the velocity, \(\boldsymbol{F}=-m\gamma\boldsymbol{v}\),
for the purposes of this exercise. The acceleration for a projectile moving upwards,
\(\boldsymbol{a}=\boldsymbol{F}/m\), becomes
\begin{equation*}
\begin{split}
\begin{eqnarray}
\frac{dv_x}{dt}=-\gamma v_x,\\
\nonumber
\frac{dv_y}{dt}=-\gamma v_y-g,
\end{eqnarray}
\end{split}
\end{equation*}
and \(\gamma\) has dimensions of inverse time.

If you on the other hand have a falling raindrop, how do these equations change? See for example Figure 2.1 in Taylor.
Let us stay with a ball which is thrown up in the air at \(t=0\).


\subsection{Ways of solving these equations}
\label{\detokenize{chapter2:ways-of-solving-these-equations}}
We will go over two different ways to solve this equation. The first
by direct integration, and the second as a differential equation. To
do this by direct integration, one simply multiplies both sides of the
equations above by \(dt\), then divide by the appropriate factors so
that the \(v\)s are all on one side of the equation and the \(dt\) is on
the other. For the \(x\) motion one finds an easily integrable equation,
\begin{equation*}
\begin{split}
\begin{eqnarray}
\frac{dv_x}{v_x}&=&-\gamma dt,\\
\nonumber
\int_{v_{0x}}^{v_{x}}\frac{dv_x}{v_x}&=&-\gamma\int_0^{t}dt,\\
\nonumber
\ln\left(\frac{v_{x}}{v_{0x}}\right)&=&-\gamma t,\\
\nonumber
v_{x}(t)&=&v_{0x}e^{-\gamma t}.
\end{eqnarray}
\end{split}
\end{equation*}
This is very much the result you would have written down
by inspection. For the \(y\)\sphinxhyphen{}component of the velocity,
\begin{equation*}
\begin{split}
\begin{eqnarray}
\frac{dv_y}{v_y+g/\gamma}&=&-\gamma dt\\
\nonumber
\ln\left(\frac{v_{y}+g/\gamma}{v_{0y}-g/\gamma}\right)&=&-\gamma t_f,\\
\nonumber
v_{fy}&=&-\frac{g}{\gamma}+\left(v_{0y}+\frac{g}{\gamma}\right)e^{-\gamma t}.
\end{eqnarray}
\end{split}
\end{equation*}
Whereas \(v_x\) starts at some value and decays
exponentially to zero, \(v_y\) decays exponentially to the terminal
velocity, \(v_t=-g/\gamma\).


\subsection{Solving as differential equations}
\label{\detokenize{chapter2:solving-as-differential-equations}}
Although this direct integration is simpler than the method we invoke
below, the method below will come in useful for some slightly more
difficult differential equations in the future. The differential
equation for \(v_x\) is straight\sphinxhyphen{}forward to solve. Because it is first
order there is one arbitrary constant, \(A\), and by inspection the
solution is




\begin{equation*}
\begin{split}
\begin{equation}
v_x=Ae^{-\gamma t}.
\label{_auto5} \tag{5}
\end{equation}
\end{split}
\end{equation*}
The arbitrary constants for equations of motion are usually determined
by the initial conditions, or more generally boundary conditions. By
inspection \(A=v_{0x}\), the initial \(x\) component of the velocity.


\subsection{Differential Equations, contn}
\label{\detokenize{chapter2:differential-equations-contn}}
The differential equation for \(v_y\) is a bit more complicated due to
the presence of \(g\). Differential equations where all the terms are
linearly proportional to a function, in this case \(v_y\), or to
derivatives of the function, e.g., \(v_y\), \(dv_y/dt\),
\(d^2v_y/dt^2\cdots\), are called linear differential equations. If
there are terms proportional to \(v^2\), as would happen if the drag
force were proportional to the square of the velocity, the
differential equation is not longer linear. Because this expression
has only one derivative in \(v\) it is a first\sphinxhyphen{}order linear differential
equation. If a term were added proportional to \(d^2v/dt^2\) it would be
a second\sphinxhyphen{}order differential equation.  In this case we have a term
completely independent of \(v\), the gravitational acceleration \(g\), and
the usual strategy is to first rewrite the equation with all the
linear terms on one side of the equal sign,




\begin{equation*}
\begin{split}
\begin{equation}
\frac{dv_y}{dt}+\gamma v_y=-g.
\label{_auto6} \tag{6}
\end{equation}
\end{split}
\end{equation*}

\subsection{Splitting into two parts}
\label{\detokenize{chapter2:splitting-into-two-parts}}
Now, the solution to the equation can be broken into two
parts. Because this is a first\sphinxhyphen{}order differential equation we know
that there will be one arbitrary constant. Physically, the arbitrary
constant will be determined by setting the initial velocity, though it
could be determined by setting the velocity at any given time. Like
most differential equations, solutions are not “solved”. Instead,
one guesses at a form, then shows the guess is correct. For these
types of equations, one first tries to find a single solution,
i.e. one with no arbitrary constants. This is called the \{\textbackslash{}it
particular\} solution, \(y_p(t)\), though it should really be called
“a” particular solution because there are an infinite number of such
solutions. One then finds a solution to the \{\textbackslash{}it homogenous\} equation,
which is the equation with zero on the right\sphinxhyphen{}hand side,




\begin{equation*}
\begin{split}
\begin{equation}
\frac{dv_{y,h}}{dt}+\gamma v_{y,h}=0.
\label{_auto7} \tag{7}
\end{equation}
\end{split}
\end{equation*}
Homogenous solutions will have arbitrary constants.

The particular solution will solve the same equation as the original
general equation




\begin{equation*}
\begin{split}
\begin{equation}
\frac{dv_{y,p}}{dt}+\gamma v_{y,p}=-g.
\label{_auto8} \tag{8}
\end{equation}
\end{split}
\end{equation*}
However, we don’t need find one with arbitrary constants. Hence, it is
called a \sphinxstylestrong{particular} solution.

The sum of the two,




\begin{equation*}
\begin{split}
\begin{equation}
v_y=v_{y,p}+v_{y,h},
\label{_auto9} \tag{9}
\end{equation}
\end{split}
\end{equation*}
is a solution of the total equation because of the linear nature of
the differential equation. One has now found a \sphinxstyleemphasis{general} solution
encompassing all solutions, because it both satisfies the general
equation (like the particular solution), and has an arbitrary constant
that can be adjusted to fit any initial condition (like the homogneous
solution). If the equation were not linear, e.g if there were a term
such as \(v_y^2\) or \(v_y\dot{v}_y\), this technique would not work.


\subsection{More details}
\label{\detokenize{chapter2:more-details}}
Returning to the example above, the homogenous solution is the same as
that for \(v_x\), because there was no gravitational acceleration in
that case,




\begin{equation*}
\begin{split}
\begin{equation}
v_{y,h}=Be^{-\gamma t}.
\label{_auto10} \tag{10}
\end{equation}
\end{split}
\end{equation*}
In this case a particular solution is one with constant velocity,




\begin{equation*}
\begin{split}
\begin{equation}
v_{y,p}=-g/\gamma.
\label{_auto11} \tag{11}
\end{equation}
\end{split}
\end{equation*}
Note that this is the terminal velocity of a particle falling from a
great height. The general solution is thus,




\begin{equation*}
\begin{split}
\begin{equation}
v_y=Be^{-\gamma t}-g/\gamma,
\label{_auto12} \tag{12}
\end{equation}
\end{split}
\end{equation*}
and one can find \(B\) from the initial velocity,




\begin{equation*}
\begin{split}
\begin{equation}
v_{0y}=B-g/\gamma,~~~B=v_{0y}+g/\gamma.
\label{_auto13} \tag{13}
\end{equation}
\end{split}
\end{equation*}
Plugging in the expression for \(B\) gives the \(y\) motion given the initial velocity,




\begin{equation*}
\begin{split}
\begin{equation}
v_y=(v_{0y}+g/\gamma)e^{-\gamma t}-g/\gamma.
\label{_auto14} \tag{14}
\end{equation}
\end{split}
\end{equation*}
It is easy to see that this solution has \(v_y=v_{0y}\) when \(t=0\) and
\(v_y=-g/\gamma\) when \(t\rightarrow\infty\).

One can also integrate the two equations to find the coordinates \(x\)
and \(y\) as functions of \(t\),
\begin{equation*}
\begin{split}
\begin{eqnarray}
x&=&\int_0^t dt'~v_{0x}(t')=\frac{v_{0x}}{\gamma}\left(1-e^{-\gamma t}\right),\\
\nonumber
y&=&\int_0^t dt'~v_{0y}(t')=-\frac{gt}{\gamma}+\frac{v_{0y}+g/\gamma}{\gamma}\left(1-e^{-\gamma t}\right).
\end{eqnarray}
\end{split}
\end{equation*}
If the question was to find the position at a time \(t\), we would be
finished. However, the more common goal in a projectile equation
problem is to find the range, i.e. the distance \(x\) at which \(y\)
returns to zero. For the case without a drag force this was much
simpler. The solution for the \(y\) coordinate would have been
\(y=v_{0y}t-gt^2/2\). One would solve for \(t\) to make \(y=0\), which would
be \(t=2v_{0y}/g\), then plug that value for \(t\) into \(x=v_{0x}t\) to
find \(x=2v_{0x}v_{0y}/g=v_0\sin(2\theta_0)/g\). One follows the same
steps here, except that the expression for \(y(t)\) is more
complicated. Searching for the time where \(y=0\), and we get




\begin{equation*}
\begin{split}
\begin{equation}
0=-\frac{gt}{\gamma}+\frac{v_{0y}+g/\gamma}{\gamma}\left(1-e^{-\gamma t}\right).
\label{_auto15} \tag{15}
\end{equation}
\end{split}
\end{equation*}
This cannot be inverted into a simple expression \(t=\cdots\). Such
expressions are known as “transcendental equations”, and are not the
rare instance, but are the norm. In the days before computers, one
might plot the right\sphinxhyphen{}hand side of the above graphically as
a function of time, then find the point where it crosses zero.

Now, the most common way to solve for an equation of the above type
would be to apply Newton’s method numerically. This involves the
following algorithm for finding solutions of some equation \(F(t)=0\).
\begin{enumerate}
\sphinxsetlistlabels{\arabic}{enumi}{enumii}{}{.}%
\item {} 
First guess a value for the time, \(t_{\rm guess}\).

\item {} 
Calculate \(F\) and its derivative, \(F(t_{\rm guess})\) and \(F'(t_{\rm guess})\).

\item {} 
Unless you guessed perfectly, \(F\ne 0\), and assuming that \(\Delta F\approx F'\Delta t\), one would choose

\item {} 
\(\Delta t=-F(t_{\rm guess})/F'(t_{\rm guess})\).

\item {} 
Now repeat step 1, but with \(t_{\rm guess}\rightarrow t_{\rm guess}+\Delta t\).

\end{enumerate}

If the \(F(t)\) were perfectly linear in \(t\), one would find \(t\) in one
step. Instead, one typically finds a value of \(t\) that is closer to
the final answer than \(t_{\rm guess}\). One breaks the loop once one
finds \(F\) within some acceptable tolerance of zero. A program to do
this will be added shortly.


\subsection{Motion in a Magnetic Field}
\label{\detokenize{chapter2:motion-in-a-magnetic-field}}
Another example of a velocity\sphinxhyphen{}dependent force is magnetism,
\begin{equation*}
\begin{split}
\begin{eqnarray}
\boldsymbol{F}&=&q\boldsymbol{v}\times\boldsymbol{B},\\
\nonumber
F_i&=&q\sum_{jk}\epsilon_{ijk}v_jB_k.
\end{eqnarray}
\end{split}
\end{equation*}
For a uniform field in the \(z\) direction \(\boldsymbol{B}=B\hat{z}\), the force can only have \(x\) and \(y\) components,
\begin{equation*}
\begin{split}
\begin{eqnarray}
F_x&=&qBv_y\\
\nonumber
F_y&=&-qBv_x.
\end{eqnarray}
\end{split}
\end{equation*}
The differential equations are
\begin{equation*}
\begin{split}
\begin{eqnarray}
\dot{v}_x&=&\omega_c v_y,\omega_c= qB/m\\
\nonumber
\dot{v}_y&=&-\omega_c v_x.
\end{eqnarray}
\end{split}
\end{equation*}
One can solve the equations by taking time derivatives of either equation, then substituting into the other equation,
\begin{equation*}
\begin{split}
\begin{eqnarray}
\ddot{v}_x=\omega_c\dot{v_y}=-\omega_c^2v_x,\\
\nonumber
\ddot{v}_y&=&-\omega_c\dot{v}_x=-\omega_cv_y.
\end{eqnarray}
\end{split}
\end{equation*}
The solution to these equations can be seen by inspection,
\begin{equation*}
\begin{split}
\begin{eqnarray}
v_x&=&A\sin(\omega_ct+\phi),\\
\nonumber
v_y&=&A\cos(\omega_ct+\phi).
\end{eqnarray}
\end{split}
\end{equation*}
One can integrate the equations to find the positions as a function of time,
\begin{equation*}
\begin{split}
\begin{eqnarray}
x-x_0&=&\int_{x_0}^x dx=\int_0^t dt v(t)\\
\nonumber
&=&\frac{-A}{\omega_c}\cos(\omega_ct+\phi),\\
\nonumber
y-y_0&=&\frac{A}{\omega_c}\sin(\omega_ct+\phi).
\end{eqnarray}
\end{split}
\end{equation*}
The trajectory is a circle centered at \(x_0,y_0\) with amplitude \(A\) rotating in the clockwise direction.

The equations of motion for the \(z\) motion are




\begin{equation*}
\begin{split}
\begin{equation}
\dot{v_z}=0,
\label{_auto16} \tag{16}
\end{equation}
\end{split}
\end{equation*}
which leads to




\begin{equation*}
\begin{split}
\begin{equation}
z-z_0=V_zt.
\label{_auto17} \tag{17}
\end{equation}
\end{split}
\end{equation*}
Added onto the circle, the motion is helical.

Note that the kinetic energy,




\begin{equation*}
\begin{split}
\begin{equation}
T=\frac{1}{2}m(v_x^2+v_y^2+v_z^2)=\frac{1}{2}m(\omega_c^2A^2+V_z^2),
\label{_auto18} \tag{18}
\end{equation}
\end{split}
\end{equation*}
is constant. This is because the force is perpendicular to the
velocity, so that in any differential time element \(dt\) the work done
on the particle \(\boldsymbol{F}\cdot{dr}=dt\boldsymbol{F}\cdot{v}=0\).

One should think about the implications of a velocity dependent
force. Suppose one had a constant magnetic field in deep space. If a
particle came through with velocity \(v_0\), it would undergo cyclotron
motion with radius \(R=v_0/\omega_c\). However, if it were still its
motion would remain fixed. Now, suppose an observer looked at the
particle in one reference frame where the particle was moving, then
changed their velocity so that the particle’s velocity appeared to be
zero. The motion would change from circular to fixed. Is this
possible?

The solution to the puzzle above relies on understanding
relativity. Imagine that the first observer believes \(\boldsymbol{B}\ne 0\) and
that the electric field \(\boldsymbol{E}=0\). If the observer then changes
reference frames by accelerating to a velocity \(\boldsymbol{v}\), in the new
frame \(\boldsymbol{B}\) and \(\boldsymbol{E}\) both change. If the observer moved to the
frame where the charge, originally moving with a small velocity \(v\),
is now at rest, the new electric field is indeed \(\boldsymbol{v}\times\boldsymbol{B}\),
which then leads to the same acceleration as one had before. If the
velocity is not small compared to the speed of light, additional
\(\gamma\) factors come into play,
\(\gamma=1/\sqrt{1-(v/c)^2}\). Relativistic motion will not be
considered in this course.


\subsection{Sliding Block tied to a Wall}
\label{\detokenize{chapter2:sliding-block-tied-to-a-wall}}
Another classical case is that of simple harmonic oscillations, here represented by a block sliding on a horizontal frictionless surface. The block is tied to a wall with a spring. If the spring is not compressed or stretched too far, the force on the block at a given position \(x\) is
\begin{equation*}
\begin{split}
F=-kx.
\end{split}
\end{equation*}
The negative sign means that the force acts to restore the object to an equilibrium position. Newton’s equation of motion for this idealized system is then
\begin{equation*}
\begin{split}
m\frac{d^2x}{dt^2}=-kx,
\end{split}
\end{equation*}
or we could rephrase it as




\begin{equation*}
\begin{split}
\frac{d^2x}{dt^2}=-\frac{k}{m}x=-\omega_0^2x,
\label{eq:newton1} \tag{19}
\end{split}
\end{equation*}
with the angular frequency \(\omega_0^2=k/m\).

The above differential equation has the advantage that it can be solved  analytically with solutions on the form
\begin{equation*}
\begin{split}
x(t)=Acos(\omega_0t+\nu),
\end{split}
\end{equation*}
where \(A\) is the amplitude and \(\nu\) the phase constant.   This provides in turn an important test for the numerical
solution and the development of a program for more complicated cases which cannot be solved analytically.


\subsection{Simple Example, Block tied to a Wall}
\label{\detokenize{chapter2:simple-example-block-tied-to-a-wall}}
With the position \(x(t)\) and the velocity  \(v(t)=dx/dt\) we can reformulate Newton’s equation in the following way
\begin{equation*}
\begin{split}
\frac{dx(t)}{dt}=v(t),
\end{split}
\end{equation*}
and
\begin{equation*}
\begin{split}
\frac{dv(t)}{dt}=-\omega_0^2x(t).
\end{split}
\end{equation*}
We are now going to solve these equations using first the standard forward Euler  method. Later we will try to improve upon this.


\subsection{Simple Example, Block tied to a Wall}
\label{\detokenize{chapter2:id1}}
Before proceeding however, it is important to note that in addition to the exact solution, we have at least two further tests which can be used to check our solution.

Since functions like \(cos\) are periodic with a period \(2\pi\), then the solution \(x(t)\) has also to be periodic. This means that
\begin{equation*}
\begin{split}
x(t+T)=x(t),
\end{split}
\end{equation*}
with \(T\) the period defined as
\begin{equation*}
\begin{split}
T=\frac{2\pi}{\omega_0}=\frac{2\pi}{\sqrt{k/m}}.
\end{split}
\end{equation*}
Observe that \(T\) depends only on \(k/m\) and not on the amplitude of the solution.


\subsection{Simple Example, Block tied to a Wall}
\label{\detokenize{chapter2:id2}}
In addition to the periodicity test, the total energy has also to be conserved.

Suppose we choose the initial conditions
\begin{equation*}
\begin{split}
x(t=0)=1\hspace{0.1cm} \mathrm{m}\hspace{1cm} v(t=0)=0\hspace{0.1cm}\mathrm{m/s},
\end{split}
\end{equation*}
meaning that block is at rest at \(t=0\) but with a potential energy
\begin{equation*}
\begin{split}
E_0=\frac{1}{2}kx(t=0)^2=\frac{1}{2}k.
\end{split}
\end{equation*}
The total energy at any time \(t\) has however to be conserved, meaning that our solution has to fulfil the condition
\begin{equation*}
\begin{split}
E_0=\frac{1}{2}kx(t)^2+\frac{1}{2}mv(t)^2.
\end{split}
\end{equation*}
We will derive this equation in our discussion on \sphinxhref{https://mhjensen.github.io/Physics321/doc/pub/energyconserv/html/energyconserv.html}{energy conservation}.


\subsection{Simple Example, Block tied to a Wall}
\label{\detokenize{chapter2:id3}}
An algorithm which implements these equations is included below.
\begin{itemize}
\item {} 
Choose the initial position and speed, with the most common choice \(v(t=0)=0\) and some fixed value for the position.

\item {} 
Choose the method you wish to employ in solving the problem.

\item {} 
Subdivide the time interval \([t_i,t_f] \) into a grid with step size

\end{itemize}
\begin{equation*}
\begin{split}
h=\frac{t_f-t_i}{N},
\end{split}
\end{equation*}
where \(N\) is the number of mesh points.
\begin{itemize}
\item {} 
Calculate now the total energy given by

\end{itemize}
\begin{equation*}
\begin{split}
E_0=\frac{1}{2}kx(t=0)^2=\frac{1}{2}k.
\end{split}
\end{equation*}\begin{itemize}
\item {} 
Choose ODE solver to obtain \(x_{i+1}\) and \(v_{i+1}\) starting from the previous values \(x_i\) and \(v_i\).

\item {} 
When we have computed \(x(v)_{i+1}\) we upgrade  \(t_{i+1}=t_i+h\).

\item {} 
This iterative  process continues till we reach the maximum time \(t_f\).

\item {} 
The results are checked against the exact solution. Furthermore, one has to check the stability of the numerical solution against the chosen number of mesh points \(N\).

\end{itemize}


\subsection{Simple Example, Block tied to a Wall, python code}
\label{\detokenize{chapter2:simple-example-block-tied-to-a-wall-python-code}}
The following python program ( code will be added shortly)

\begin{sphinxVerbatim}[commandchars=\\\{\}]
\PYG{c+c1}{\PYGZsh{}}
\PYG{c+c1}{\PYGZsh{} This program solves Newtons equation for a block sliding on}
\PYG{c+c1}{\PYGZsh{} an horizontal frictionless surface.}
\PYG{c+c1}{\PYGZsh{} The block is tied to the wall with a spring, so N\PYGZsq{}s eq takes the form:}
\PYG{c+c1}{\PYGZsh{}}
\PYG{c+c1}{\PYGZsh{}  m d\PYGZca{}2x/dt\PYGZca{}2 = \PYGZhy{} kx}
\PYG{c+c1}{\PYGZsh{}}
\PYG{c+c1}{\PYGZsh{} In order to make the solution dimless, we set k/m = 1.}
\PYG{c+c1}{\PYGZsh{} This results in two coupled diff. eq\PYGZsq{}s that may be written as:}
\PYG{c+c1}{\PYGZsh{}}
\PYG{c+c1}{\PYGZsh{}  dx/dt = v}
\PYG{c+c1}{\PYGZsh{}  dv/dt = \PYGZhy{}x}
\PYG{c+c1}{\PYGZsh{}}
\PYG{c+c1}{\PYGZsh{} The user has to specify the initial velocity and position,}
\PYG{c+c1}{\PYGZsh{} and the number of steps. The time interval is fixed to}
\PYG{c+c1}{\PYGZsh{} t \PYGZbs{}in [0, 4\PYGZbs{}pi) (two periods)}
\PYG{c+c1}{\PYGZsh{}}
\end{sphinxVerbatim}


\subsection{The classical pendulum and scaling the equations}
\label{\detokenize{chapter2:the-classical-pendulum-and-scaling-the-equations}}
The angular equation of motion of the pendulum is given by
Newton’s equation and with no external force it reads




\begin{equation*}
\begin{split}
\begin{equation}
  ml\frac{d^2\theta}{dt^2}+mgsin(\theta)=0,
\label{_auto19} \tag{20}
\end{equation}
\end{split}
\end{equation*}
with an angular velocity and acceleration given by




\begin{equation*}
\begin{split}
\begin{equation}
     v=l\frac{d\theta}{dt},
\label{_auto20} \tag{21}
\end{equation}
\end{split}
\end{equation*}
and




\begin{equation*}
\begin{split}
\begin{equation}
     a=l\frac{d^2\theta}{dt^2}.
\label{_auto21} \tag{22}
\end{equation}
\end{split}
\end{equation*}

\subsection{More on the Pendulum}
\label{\detokenize{chapter2:more-on-the-pendulum}}
We do however expect that the motion will gradually come to an end due a viscous drag torque acting on the pendulum.
In the presence of the drag, the above equation becomes




\begin{equation*}
\begin{split}
\begin{equation}
   ml\frac{d^2\theta}{dt^2}+\nu\frac{d\theta}{dt}  +mgsin(\theta)=0, \label{eq:pend1} \tag{23}
\end{equation}
\end{split}
\end{equation*}
where \(\nu\) is now a positive constant parameterizing the viscosity
of the medium in question. In order to maintain the motion against
viscosity, it is necessary to add some external driving force.
We choose here a periodic driving force. The last equation becomes then




\begin{equation*}
\begin{split}
\begin{equation}
   ml\frac{d^2\theta}{dt^2}+\nu\frac{d\theta}{dt}  +mgsin(\theta)=Asin(\omega t), \label{eq:pend2} \tag{24}
\end{equation}
\end{split}
\end{equation*}
with \(A\) and \(\omega\) two constants representing the amplitude and
the angular frequency respectively. The latter is called the driving frequency.


\subsection{More on the Pendulum}
\label{\detokenize{chapter2:id4}}
We define
\begin{equation*}
\begin{split}
\omega_0=\sqrt{g/l},
\end{split}
\end{equation*}
the so\sphinxhyphen{}called natural frequency and the new dimensionless quantities
\begin{equation*}
\begin{split}
\hat{t}=\omega_0t,
\end{split}
\end{equation*}
with the dimensionless driving frequency
\begin{equation*}
\begin{split}
\hat{\omega}=\frac{\omega}{\omega_0},
\end{split}
\end{equation*}
and introducing the quantity \(Q\), called the \sphinxstyleemphasis{quality factor},
\begin{equation*}
\begin{split}
Q=\frac{mg}{\omega_0\nu},
\end{split}
\end{equation*}
and the dimensionless amplitude
\begin{equation*}
\begin{split}
\hat{A}=\frac{A}{mg}
\end{split}
\end{equation*}

\subsection{More on the Pendulum}
\label{\detokenize{chapter2:id5}}
We have
\begin{equation*}
\begin{split}
\frac{d^2\theta}{d\hat{t}^2}+\frac{1}{Q}\frac{d\theta}{d\hat{t}}  
     +sin(\theta)=\hat{A}cos(\hat{\omega}\hat{t}).
\end{split}
\end{equation*}
This equation can in turn be recast in terms of two coupled first\sphinxhyphen{}order differential equations as follows
\begin{equation*}
\begin{split}
\frac{d\theta}{d\hat{t}}=\hat{v},
\end{split}
\end{equation*}
and
\begin{equation*}
\begin{split}
\frac{d\hat{v}}{d\hat{t}}=-\frac{\hat{v}}{Q}-sin(\theta)+\hat{A}cos(\hat{\omega}\hat{t}).
\end{split}
\end{equation*}
These are the equations to be solved.  The factor \(Q\) represents the number of oscillations of the undriven system that must occur before  its energy is significantly reduced due to the viscous drag. The  amplitude \(\hat{A}\) is measured in units of the maximum possible  gravitational torque while \(\hat{\omega}\) is the angular frequency of the external torque measured in units of the pendulum’s natural frequency.




\section{Energy, Momentum and Conservation Laws}
\label{\detokenize{chapter3:energy-momentum-and-conservation-laws}}\label{\detokenize{chapter3::doc}}




\sphinxstylestrong{\sphinxhref{http://mhjgit.github.io/info/doc/web/}{Morten Hjorth\sphinxhyphen{}Jensen}}, Department of Physics and Astronomy and National Superconducting Cyclotron Laboratory, Michigan State University, USA and Department of Physics, University of Oslo, Norway









Date: \sphinxstylestrong{Feb 26, 2020}

Copyright 1999\sphinxhyphen{}2020, \sphinxhref{http://mhjgit.github.io/info/doc/web/}{Morten Hjorth\sphinxhyphen{}Jensen}. Released under CC Attribution\sphinxhyphen{}NonCommercial 4.0 license


\subsection{Work, Energy, Momentum and Conservation laws}
\label{\detokenize{chapter3:work-energy-momentum-and-conservation-laws}}
Energy conservation is most convenient as a strategy for addressing
problems where time does not appear. For example, a particle goes
from position \(x_0\) with speed \(v_0\), to position \(x_f\); what is its
new speed? However, it can also be applied to problems where time
does appear, such as in solving for the trajectory \(x(t)\), or
equivalently \(t(x)\).


\subsection{Work and Energy}
\label{\detokenize{chapter3:work-and-energy}}
Material to be added here.


\subsection{Energy Conservation}
\label{\detokenize{chapter3:energy-conservation}}
Energy is conserved in the case where the potential energy, \(V(\boldsymbol{r})\), depends only on position, and not on time. The force is determined by \(V\),




\begin{equation*}
\begin{split}
\begin{equation}
\boldsymbol{F}(\boldsymbol{r})=-\nabla V(\boldsymbol{r}).
\label{_auto1} \tag{1}
\end{equation}
\end{split}
\end{equation*}
The net energy, \(E=V+K\) where \(K\) is the kinetic energy, is then conserved,
\begin{equation*}
\begin{split}
\begin{eqnarray}
\frac{d}{dt}(K+V)&=&\frac{d}{dt}\left(\frac{m}{2}(v_x^2+v_y^2+v_z^2)+V(\boldsymbol{r})\right)\\
\nonumber
&=&m\left(v_x\frac{dv_x}{dt}+v_y\frac{dv_y}{dt}+v_z\frac{dv_z}{dt}\right)
+\partial_xV\frac{dx}{dt}+\partial_yV\frac{dy}{dt}+\partial_zV\frac{dz}{dt}\\
\nonumber
&=&v_xF_x+v_yF_y+v_zF_z-F_xv_x-F_yv_y-F_zv_z=0.
\end{eqnarray}
\end{split}
\end{equation*}
The same proof can be written more compactly with vector notation,
\begin{equation*}
\begin{split}
\begin{eqnarray}
\frac{d}{dt}\left(\frac{m}{2}v^2+V(\boldsymbol{r})\right)
&=&m\boldsymbol{v}\cdot\dot{\boldsymbol{v}}+\nabla V(\boldsymbol{r})\cdot\dot{\boldsymbol{r}}\\
\nonumber
&=&\boldsymbol{v}\cdot\boldsymbol{F}-\boldsymbol{F}\cdot\boldsymbol{v}=0.
\end{eqnarray}
\end{split}
\end{equation*}
Inverting the expression for kinetic energy,




\begin{equation*}
\begin{split}
\begin{equation}
v=\sqrt{2K/m}=\sqrt{2(E-V)/m},
\label{_auto2} \tag{2}
\end{equation}
\end{split}
\end{equation*}
allows one to solve for the one\sphinxhyphen{}dimensional trajectory \(x(t)\), by finding \(t(x)\),




\begin{equation*}
\begin{split}
\begin{equation}
t=\int_{x_0}^x \frac{dx'}{v(x')}=\int_{x_0}^x\frac{dx'}{\sqrt{2(E-V(x'))/m}}.
\label{_auto3} \tag{3}
\end{equation}
\end{split}
\end{equation*}
Note this would be much more difficult in higher dimensions, because
you would have to determine which points, \(x,y,z\), the particles might
reach in the trajectory, whereas in one dimension you can typically
tell by simply seeing whether the kinetic energy is positive at every
point between the old position and the new position.

Consider a simple harmonic oscillator potential, \(V(x)=kx^2/2\), with a particle emitted from \(x=0\) with velocity \(v_0\). Solve for the trajectory \(t(x)\),
\begin{equation*}
\begin{split}
\begin{eqnarray}
t&=&\int_{0}^x \frac{dx'}{\sqrt{2(E-kx^2/2)/m}}\\
\nonumber
&=&\sqrt{m/k}\int_0^x~\frac{dx'}{\sqrt{x_{\rm max}^2-x^{\prime 2}}},~~~x_{\rm max}^2=2E/k.
\end{eqnarray}
\end{split}
\end{equation*}
Here \(E=mv_0^2/2\) and \(x_{\rm max}\) is defined as the maximum
displacement before the particle turns around. This integral is done
by the substitution \(\sin\theta=x/x_{\rm max}\).
\begin{equation*}
\begin{split}
\begin{eqnarray}
(k/m)^{1/2}t&=&\sin^{-1}(x/x_{\rm max}),\\
\nonumber
x&=&x_{\rm max}\sin\omega t,~~~\omega=\sqrt{k/m}.
\end{eqnarray}
\end{split}
\end{equation*}

\subsection{Conservation of Momentum}
\label{\detokenize{chapter3:conservation-of-momentum}}
Newton’s third law which we met earlier states that \sphinxstylestrong{For every action there is an equal and opposite reaction}, is more accurately stated as
\sphinxstylestrong{If two bodies exert forces on each other, these forces are equal in magnitude and opposite in direction}.

This means that for two bodies \(i\) and \(j\), if the force on \(i\) due to \(j\) is called \(\boldsymbol{F}_{ij}\), then




\begin{equation*}
\begin{split}
\begin{equation}
\boldsymbol{F}_{ij}=-\boldsymbol{F}_{ji}. 
\label{_auto4} \tag{4}
\end{equation}
\end{split}
\end{equation*}
Newton’s second law, \(\boldsymbol{F}=m\boldsymbol{a}\), can be written for a particle \(i\) as




\begin{equation*}
\begin{split}
\begin{equation}
\boldsymbol{F}_i=\sum_{j\ne i} \boldsymbol{F}_{ij}=m_i\boldsymbol{a}_i,
\label{_auto5} \tag{5}
\end{equation}
\end{split}
\end{equation*}
where \(\boldsymbol{F}_i\) (a single subscript) denotes the net force acting on \(i\). Because the mass of \(i\) is fixed, one can see that




\begin{equation*}
\begin{split}
\begin{equation}
\boldsymbol{F}_i=\frac{d}{dt}m_i\boldsymbol{v}_i=\sum_{j\ne i}\boldsymbol{F}_{ij}.
\label{_auto6} \tag{6}
\end{equation}
\end{split}
\end{equation*}
Now, one can sum over all the particles and obtain
\begin{equation*}
\begin{split}
\begin{eqnarray}
\frac{d}{dt}\sum_i m_iv_i&=&\sum_{ij, i\ne j}\boldsymbol{F}_{ij}\\
\nonumber
&=&0.
\end{eqnarray}
\end{split}
\end{equation*}
The last step made use of the fact that for every term \(ij\), there is
an equivalent term \(ji\) with opposite force. Because the momentum is
defined as \(m\boldsymbol{v}\), for a system of particles,




\begin{equation*}
\begin{split}
\begin{equation}
\frac{d}{dt}\sum_im_i\boldsymbol{v}_i=0,~~{\rm for~isolated~particles}.
\label{_auto7} \tag{7}
\end{equation}
\end{split}
\end{equation*}
By “isolated” one means that the only force acting on any particle \(i\)
are those originating from other particles in the sum, i.e. “no
external” forces. Thus, Newton’s third law leads to the conservation
of total momentum,
\begin{equation*}
\begin{split}
\begin{eqnarray}
\boldsymbol{P}&=&\sum_i m_i\boldsymbol{v}_i,\\
\nonumber
\frac{d}{dt}\boldsymbol{P}&=&0.
\end{eqnarray}
\end{split}
\end{equation*}
Consider the rocket of mass \(M\) moving with velocity \(v\). After a
brief instant, the velocity of the rocket is \(v+\Delta v\) and the mass
is \(M-\Delta M\). Momentum conservation gives
\begin{equation*}
\begin{split}
\begin{eqnarray*}
Mv&=&(M-\Delta M)(v+\Delta v)+\Delta M(v-v_e)\\
0&=&-\Delta Mv+M\Delta v+\Delta M(v-v_e),\\
0&=&M\Delta v-\Delta Mv_e.
\end{eqnarray*}
\end{split}
\end{equation*}
In the second step we ignored the term \(\Delta M\Delta v\) because it is doubly small. The last equation gives
\begin{equation*}
\begin{split}
\begin{eqnarray}
\Delta v&=&\frac{v_e}{M}\Delta M,\\
\nonumber
\frac{dv}{dt}&=&\frac{v_e}{M}\frac{dM}{dt}.
\end{eqnarray}
\end{split}
\end{equation*}
Integrating the expression with lower limits \(v_0=0\) and \(M_0\), one finds
\begin{equation*}
\begin{split}
\begin{eqnarray*}
v&=&v_e\int_{M_0}^M \frac{dM'}{M'}\\
v&=&-v_e\ln(M/M_0)\\
&=&-v_e\ln[(M_0-\alpha t)/M_0].
\end{eqnarray*}
\end{split}
\end{equation*}
Because the total momentum of an isolated system is constant, one can
also quickly see that the center of mass of an isolated system is also
constant. The center of mass is the average position of a set of
masses weighted by the mass,




\begin{equation*}
\begin{split}
\begin{equation}
\bar{x}=\frac{\sum_im_ix_i}{\sum_i m_i}.
\label{_auto8} \tag{8}
\end{equation}
\end{split}
\end{equation*}
The rate of change of \(\bar{x}\) is
\begin{equation*}
\begin{split}
\begin{eqnarray}
\dot{\bar{x}}&=&\frac{1}{M}\sum_i m_i\dot{x}_i=\frac{1}{M}P_x.
\end{eqnarray}
\end{split}
\end{equation*}
Thus if the total momentum is constant the center of mass moves at a
constant velocity, and if the total momentum is zero the center of
mass is fixed.


\subsection{Conservation of Angular Momentum}
\label{\detokenize{chapter3:conservation-of-angular-momentum}}
Consider a case where the force always points radially,




\begin{equation*}
\begin{split}
\begin{equation}
\boldsymbol{F}(\boldsymbol{r})=F(r)\hat{r},
\label{_auto9} \tag{9}
\end{equation}
\end{split}
\end{equation*}
where \(\hat{r}\) is a unit vector pointing outward from the origin. The angular momentum is defined as




\begin{equation*}
\begin{split}
\begin{equation}
\boldsymbol{L}=\boldsymbol{r}\times\boldsymbol{p}=m\boldsymbol{r}\times\boldsymbol{v}.
\label{_auto10} \tag{10}
\end{equation}
\end{split}
\end{equation*}
The rate of change of the angular momentum is
\begin{equation*}
\begin{split}
\begin{eqnarray}
\frac{d\boldsymbol{L}}{dt}&=&m\boldsymbol{v}\times\boldsymbol{v}+m\boldsymbol{r}\times\dot{\boldsymbol{v}}\\
\nonumber
&=&m\boldsymbol{v}\times\boldsymbol{v}+\boldsymbol{r}\times{\boldsymbol{F}}=0.
\end{eqnarray}
\end{split}
\end{equation*}
The first term is zero because \(\boldsymbol{v}\) is parallel to itself, and the
second term is zero because \(\boldsymbol{F}\) is parallel to \(\boldsymbol{r}\).

As an aside, one can see from the Levi\sphinxhyphen{}Civita symbol that the cross
product of a vector with itself is zero. Here, we consider a vector
\begin{equation*}
\begin{split}
\begin{eqnarray}
\boldsymbol{V}&=&\boldsymbol{A}\times\boldsymbol{A},\\
\nonumber
V_i&=&(\boldsymbol{A}\times\boldsymbol{A})_i=\sum_{jk}\epsilon_{ijk}A_jA_k.
\end{eqnarray}
\end{split}
\end{equation*}
For any term \(i\), there are two contributions. For example, for \(i\)
denoting the \(x\) direction, either \(j\) denotes the \(y\) direction and
\(k\) denotes the \(z\) direction, or vice versa, so




\begin{equation*}
\begin{split}
\begin{equation}
V_1=\epsilon_{123}A_2A_3+\epsilon_{132}A_3A_2.
\label{_auto11} \tag{11}
\end{equation}
\end{split}
\end{equation*}
This is zero by the antisymmetry of \(\epsilon\) under permutations.

If the force is not radial, \(\boldsymbol{r}\times\boldsymbol{F}\ne 0\) as above, and angular momentum is no longer conserved,




\begin{equation*}
\begin{split}
\begin{equation}
\frac{d\boldsymbol{L}}{dt}=\boldsymbol{r}\times\boldsymbol{F}\equiv\boldsymbol{\tau},
\label{_auto12} \tag{12}
\end{equation}
\end{split}
\end{equation*}
where \(\boldsymbol{\tau}\) is the torque.

For a system of isolated particles, one can write
\begin{equation*}
\begin{split}
\begin{eqnarray}
\frac{d}{dt}\sum_i\boldsymbol{L}_i&=&\sum_{i\ne j}\boldsymbol{r}_i\times \boldsymbol{F}_{ij}\\
\nonumber
&=&\frac{1}{2}\sum_{i\ne j} \boldsymbol{r}_i\times \boldsymbol{F}_{ij}+\boldsymbol{r}_j\times\boldsymbol{F}_{ji}\\
\nonumber
&=&\frac{1}{2}\sum_{i\ne j} (\boldsymbol{r}_i-\boldsymbol{r}_j)\times\boldsymbol{F}_{ij}=0,
\end{eqnarray}
\end{split}
\end{equation*}
where the last step used Newton’s third law,
\(\boldsymbol{F}_{ij}=-\boldsymbol{F}_{ji}\). If the forces between the particles are
radial, i.e. \(\boldsymbol{F}_{ij} ~||~ (\boldsymbol{r}_i-\boldsymbol{r}_j)\), then each term in
the sum is zero and the net angular momentum is fixed. Otherwise, you
could imagine an isolated system that would start spinning
spontaneously.

One can write the torque about a given axis, which we will denote as \(\hat{z}\), in polar coordinates, where
\begin{equation*}
\begin{split}
\begin{eqnarray}
x&=&r\sin\theta\cos\phi,~~y=r\sin\theta\cos\phi,~~z=r\cos\theta,
\end{eqnarray}
\end{split}
\end{equation*}
to find the \(z\) component of the torque,
\begin{equation*}
\begin{split}
\begin{eqnarray}
\tau_z&=&xF_y-yF_x\\
\nonumber
&=&-r\sin\theta\left\{\cos\phi \partial_y-\sin\phi \partial_x\right\}V(x,y,z).
\end{eqnarray}
\end{split}
\end{equation*}
One can use the chain rule to write the partial derivative w.r.t. \(\phi\) (keeping \(r\) and \(\theta\) fixed),
\begin{equation*}
\begin{split}
\begin{eqnarray}
\partial_\phi&=&\frac{\partial x}{\partial\phi}\partial_x+\frac{\partial_y}{\partial\phi}\partial_y
+\frac{\partial z}{\partial\phi}\partial_z\\
\nonumber
&=&-r\sin\theta\sin\phi\partial_x+\sin\theta\cos\phi\partial_y.
\end{eqnarray}
\end{split}
\end{equation*}
Combining the two equations,
\begin{equation*}
\begin{split}
\begin{eqnarray}
\tau_z&=&-\partial_\phi V(r,\theta,\phi).
\end{eqnarray}
\end{split}
\end{equation*}
Thus, if the potential is independent of the azimuthal angle \(\phi\),
there is no torque about the \(z\) axis and \(L_z\) is conserved.


\subsection{Symmetries and Conservation Laws}
\label{\detokenize{chapter3:symmetries-and-conservation-laws}}
When we derived the conservation of energy, we assumed that the
potential depended only on position, not on time. If it depended
explicitly on time, one can quickly see that the energy would have
changed at a rate \(\partial_tV(x,y,z,t)\). Note that if there is no
explicit dependence on time, i.e. \(V(x,y,z)\), the potential energy can
depend on time through the variations of \(x,y,z\) with time. However,
that variation does not lead to energy non\sphinxhyphen{}conservation. Further, we
just saw that if a potential does not depend on the azimuthal angle
about some axis, \(\phi\), that the angular momentum about that axis is
conserved.

Now, we relate momentum conservation to translational
invariance. Considering a system of particles with positions,
\(\boldsymbol{r}_i\), if one changed the coordinate system by a translation by a
differential distance \(\boldsymbol{\epsilon}\), the net potential would change
by
\begin{equation*}
\begin{split}
\begin{eqnarray}
\delta V(\boldsymbol{r}_1,\boldsymbol{r}_2\cdots)&=&\sum_i \boldsymbol{\epsilon}\cdot\nabla_i V(\boldsymbol{r}_1,\boldsymbol{r}_2,\cdots)\\
\nonumber
&=&-\sum_i \boldsymbol{\epsilon}\cdot\boldsymbol{F}_i\\
\nonumber
&=&-\frac{d}{dt}\sum_i \boldsymbol{\epsilon}\cdot\boldsymbol{p}_i.
\end{eqnarray}
\end{split}
\end{equation*}
Thus, if the potential is unchanged by a translation of the coordinate
system, the total momentum is conserved. If the potential is
translationally invariant in a given direction, defined by a unit
vector, \(\hat{\epsilon}\) in the \(\boldsymbol{\epsilon}\) direction, one can see
that
\begin{equation*}
\begin{split}
\begin{eqnarray}
\hat{\epsilon}\cdot\nabla_i V(\boldsymbol{r}_i)&=&0.
\end{eqnarray}
\end{split}
\end{equation*}
The component of the total momentum along that axis is conserved. This
is rather obvious for a single particle. If \(V(\boldsymbol{r})\) does not
depend on some coordinate \(x\), then the force in the \(x\) direction is
\(F_x=-\partial_xV=0\), and momentum along the \(x\) direction is
constant.

We showed how the total momentum of an isolated system of particle was conserved, even if the particles feel internal forces in all directions. In that case the potential energy could be written
\begin{equation*}
\begin{split}
\begin{eqnarray}
V=\sum_{i,j\le i}V_{ij}(\boldsymbol{r}_i-\boldsymbol{r}_j).
\end{eqnarray}
\end{split}
\end{equation*}
In this case, a translation leads to \(\boldsymbol{r}_i\rightarrow
\boldsymbol{r}_i+\boldsymbol{\epsilon}\), with the translation equally affecting the
coordinates of each particle. Because the potential depends only on
the relative coordinates, \(\delta V\) is manifestly zero. If one were
to go through the exercise of calculating \(\delta V\) for small
\(\boldsymbol{\epsilon}\), one would find that the term
\(\nabla_i V(\boldsymbol{r}_i-\boldsymbol{r}_j)\) would be canceled by the term
\(\nabla_jV(\boldsymbol{r}_i-\boldsymbol{r}_j)\).

The relation between symmetries of the potential and conserved
quantities (also called constants of motion) is one of the most
profound concepts one should gain from this course. It plays a
critical role in all fields of physics. This is especially true in
quantum mechanics, where a quantity \(A\) is conserved if its operator
commutes with the Hamiltonian. For example if the momentum operator
\(-i\hbar\partial_x\) commutes with the Hamiltonian, momentum is
conserved, and clearly this operator commutes if the Hamiltonian
(which represents the total energy, not just the potential) does not
depend on \(x\). Also in quantum mechanics the angular momentum operator
is \(L_z=-i\hbar\partial_\phi\). In fact, if the potential is unchanged
by rotations about some axis, angular momentum about that axis is
conserved. We return to this concept, from a more formal perspective,
later in the course when Lagrangian mechanics is presented.


\subsection{Bulding a code for the Earth\sphinxhyphen{}Sun system}
\label{\detokenize{chapter3:bulding-a-code-for-the-earth-sun-system}}
We will now venture into a study of a system which is energy
conserving. The aim is to see if we (since it is not possible to solve
the general equations analytically) we can develop stable numerical
algorithms whose results we can trust!

We solve the equations of motion numerically. We will also compute
quantities like the energy numerically.

We start with a simpler case first, the Earth\sphinxhyphen{}Sun system  in two dimensions only.  The gravitational force \(F_G\) on the earth from the sun is
\begin{equation*}
\begin{split}
\boldsymbol{F}_G=-\frac{GM_{\odot}M_E}{r^3}\boldsymbol{r},
\end{split}
\end{equation*}
where \(G\) is the gravitational constant,
\begin{equation*}
\begin{split}
M_E=6\times 10^{24}\mathrm{Kg},
\end{split}
\end{equation*}
the mass of Earth,
\begin{equation*}
\begin{split}
M_{\odot}=2\times 10^{30}\mathrm{Kg},
\end{split}
\end{equation*}
the mass of the Sun and
\begin{equation*}
\begin{split}
r=1.5\times 10^{11}\mathrm{m},
\end{split}
\end{equation*}
is the distance between Earth and the Sun. The latter defines what we call an astronomical unit \sphinxstylestrong{AU}.
From Newton’s second law we have then for the \(x\) direction
\begin{equation*}
\begin{split}
\frac{d^2x}{dt^2}=-\frac{F_{x}}{M_E},
\end{split}
\end{equation*}
and
\begin{equation*}
\begin{split}
\frac{d^2y}{dt^2}=-\frac{F_{y}}{M_E},
\end{split}
\end{equation*}
for the \(y\) direction.

Here we will use  that  \(x=r\cos{(\theta)}\), \(y=r\sin{(\theta)}\) and
\begin{equation*}
\begin{split}
r = \sqrt{x^2+y^2}.
\end{split}
\end{equation*}
We can rewrite
\begin{equation*}
\begin{split}
F_{x}=-\frac{GM_{\odot}M_E}{r^2}\cos{(\theta)}=-\frac{GM_{\odot}M_E}{r^3}x,
\end{split}
\end{equation*}
and
\begin{equation*}
\begin{split}
F_{y}=-\frac{GM_{\odot}M_E}{r^2}\sin{(\theta)}=-\frac{GM_{\odot}M_E}{r^3}y,
\end{split}
\end{equation*}
for the \(y\) direction.

We can rewrite these two equations
\begin{equation*}
\begin{split}
F_{x}=-\frac{GM_{\odot}M_E}{r^2}\cos{(\theta)}=-\frac{GM_{\odot}M_E}{r^3}x,
\end{split}
\end{equation*}
and
\begin{equation*}
\begin{split}
F_{y}=-\frac{GM_{\odot}M_E}{r^2}\sin{(\theta)}=-\frac{GM_{\odot}M_E}{r^3}y,
\end{split}
\end{equation*}
as four first\sphinxhyphen{}order coupled differential equations

4
3

\textless{}
\textless{}
\textless{}
!
!
M
A
T
H
\_
B
L
O
C
K

4
4

\textless{}
\textless{}
\textless{}
!
!
M
A
T
H
\_
B
L
O
C
K

4
5

\textless{}
\textless{}
\textless{}
!
!
M
A
T
H
\_
B
L
O
C
K
\begin{equation*}
\begin{split}
\frac{dy}{dt}=v_y.
\end{split}
\end{equation*}

\subsection{Building a code for the solar system, final coupled equations}
\label{\detokenize{chapter3:building-a-code-for-the-solar-system-final-coupled-equations}}
The four coupled differential equations

4
7

\textless{}
\textless{}
\textless{}
!
!
M
A
T
H
\_
B
L
O
C
K

4
8

\textless{}
\textless{}
\textless{}
!
!
M
A
T
H
\_
B
L
O
C
K

4
9

\textless{}
\textless{}
\textless{}
!
!
M
A
T
H
\_
B
L
O
C
K
\begin{equation*}
\begin{split}
\frac{dy}{dt}=v_y,
\end{split}
\end{equation*}
can be turned into dimensionless equations or we can introduce astronomical units with \(1\) AU = \(1.5\times 10^{11}\).

Using the equations from circular motion (with \(r =1\mathrm{AU}\))
\begin{equation*}
\begin{split}
\frac{M_E v^2}{r} = F = \frac{GM_{\odot}M_E}{r^2},
\end{split}
\end{equation*}
we have
\begin{equation*}
\begin{split}
GM_{\odot}=v^2r,
\end{split}
\end{equation*}
and using that the velocity of Earth (assuming circular motion) is
\(v = 2\pi r/\mathrm{yr}=2\pi\mathrm{AU}/\mathrm{yr}\), we have
\begin{equation*}
\begin{split}
GM_{\odot}= v^2r = 4\pi^2 \frac{(\mathrm{AU})^3}{\mathrm{yr}^2}.
\end{split}
\end{equation*}

\subsection{Building a code for the solar system, discretized equations}
\label{\detokenize{chapter3:building-a-code-for-the-solar-system-discretized-equations}}
The four coupled differential equations can then be discretized using Euler’s method as (with step length \(h\))

5
4

\textless{}
\textless{}
\textless{}
!
!
M
A
T
H
\_
B
L
O
C
K

5
5

\textless{}
\textless{}
\textless{}
!
!
M
A
T
H
\_
B
L
O
C
K

5
6

\textless{}
\textless{}
\textless{}
!
!
M
A
T
H
\_
B
L
O
C
K
\begin{equation*}
\begin{split}
y_{i+1}=y_i+hv_{y,i},
\end{split}
\end{equation*}

\subsection{Code Example with Euler’s Method}
\label{\detokenize{chapter3:code-example-with-euler-s-method}}
The code here implements Euler’s method for the Earth\sphinxhyphen{}Sun system using a more compact way of representing the vectors. Alternatively, you could have spelled out all the variables \(v_x\), \(v_y\), \(x\) and \(y\) as one\sphinxhyphen{}dimensional arrays.

\begin{sphinxVerbatim}[commandchars=\\\{\}]
\PYG{o}{\PYGZpc{}}\PYG{k}{matplotlib} inline

\PYG{c+c1}{\PYGZsh{} Common imports}
\PYG{k+kn}{import} \PYG{n+nn}{numpy} \PYG{k}{as} \PYG{n+nn}{np}
\PYG{k+kn}{import} \PYG{n+nn}{pandas} \PYG{k}{as} \PYG{n+nn}{pd}
\PYG{k+kn}{from} \PYG{n+nn}{math} \PYG{k+kn}{import} \PYG{o}{*}
\PYG{k+kn}{import} \PYG{n+nn}{matplotlib}\PYG{n+nn}{.}\PYG{n+nn}{pyplot} \PYG{k}{as} \PYG{n+nn}{plt}
\PYG{k+kn}{import} \PYG{n+nn}{os}

\PYG{c+c1}{\PYGZsh{} Where to save the figures and data files}
\PYG{n}{PROJECT\PYGZus{}ROOT\PYGZus{}DIR} \PYG{o}{=} \PYG{l+s+s2}{\PYGZdq{}}\PYG{l+s+s2}{Results}\PYG{l+s+s2}{\PYGZdq{}}
\PYG{n}{FIGURE\PYGZus{}ID} \PYG{o}{=} \PYG{l+s+s2}{\PYGZdq{}}\PYG{l+s+s2}{Results/FigureFiles}\PYG{l+s+s2}{\PYGZdq{}}
\PYG{n}{DATA\PYGZus{}ID} \PYG{o}{=} \PYG{l+s+s2}{\PYGZdq{}}\PYG{l+s+s2}{DataFiles/}\PYG{l+s+s2}{\PYGZdq{}}

\PYG{k}{if} \PYG{o+ow}{not} \PYG{n}{os}\PYG{o}{.}\PYG{n}{path}\PYG{o}{.}\PYG{n}{exists}\PYG{p}{(}\PYG{n}{PROJECT\PYGZus{}ROOT\PYGZus{}DIR}\PYG{p}{)}\PYG{p}{:}
    \PYG{n}{os}\PYG{o}{.}\PYG{n}{mkdir}\PYG{p}{(}\PYG{n}{PROJECT\PYGZus{}ROOT\PYGZus{}DIR}\PYG{p}{)}

\PYG{k}{if} \PYG{o+ow}{not} \PYG{n}{os}\PYG{o}{.}\PYG{n}{path}\PYG{o}{.}\PYG{n}{exists}\PYG{p}{(}\PYG{n}{FIGURE\PYGZus{}ID}\PYG{p}{)}\PYG{p}{:}
    \PYG{n}{os}\PYG{o}{.}\PYG{n}{makedirs}\PYG{p}{(}\PYG{n}{FIGURE\PYGZus{}ID}\PYG{p}{)}

\PYG{k}{if} \PYG{o+ow}{not} \PYG{n}{os}\PYG{o}{.}\PYG{n}{path}\PYG{o}{.}\PYG{n}{exists}\PYG{p}{(}\PYG{n}{DATA\PYGZus{}ID}\PYG{p}{)}\PYG{p}{:}
    \PYG{n}{os}\PYG{o}{.}\PYG{n}{makedirs}\PYG{p}{(}\PYG{n}{DATA\PYGZus{}ID}\PYG{p}{)}

\PYG{k}{def} \PYG{n+nf}{image\PYGZus{}path}\PYG{p}{(}\PYG{n}{fig\PYGZus{}id}\PYG{p}{)}\PYG{p}{:}
    \PYG{k}{return} \PYG{n}{os}\PYG{o}{.}\PYG{n}{path}\PYG{o}{.}\PYG{n}{join}\PYG{p}{(}\PYG{n}{FIGURE\PYGZus{}ID}\PYG{p}{,} \PYG{n}{fig\PYGZus{}id}\PYG{p}{)}

\PYG{k}{def} \PYG{n+nf}{data\PYGZus{}path}\PYG{p}{(}\PYG{n}{dat\PYGZus{}id}\PYG{p}{)}\PYG{p}{:}
    \PYG{k}{return} \PYG{n}{os}\PYG{o}{.}\PYG{n}{path}\PYG{o}{.}\PYG{n}{join}\PYG{p}{(}\PYG{n}{DATA\PYGZus{}ID}\PYG{p}{,} \PYG{n}{dat\PYGZus{}id}\PYG{p}{)}

\PYG{k}{def} \PYG{n+nf}{save\PYGZus{}fig}\PYG{p}{(}\PYG{n}{fig\PYGZus{}id}\PYG{p}{)}\PYG{p}{:}
    \PYG{n}{plt}\PYG{o}{.}\PYG{n}{savefig}\PYG{p}{(}\PYG{n}{image\PYGZus{}path}\PYG{p}{(}\PYG{n}{fig\PYGZus{}id}\PYG{p}{)} \PYG{o}{+} \PYG{l+s+s2}{\PYGZdq{}}\PYG{l+s+s2}{.png}\PYG{l+s+s2}{\PYGZdq{}}\PYG{p}{,} \PYG{n+nb}{format}\PYG{o}{=}\PYG{l+s+s1}{\PYGZsq{}}\PYG{l+s+s1}{png}\PYG{l+s+s1}{\PYGZsq{}}\PYG{p}{)}


\PYG{n}{DeltaT} \PYG{o}{=} \PYG{l+m+mf}{0.001}
\PYG{c+c1}{\PYGZsh{}set up arrays }
\PYG{n}{tfinal} \PYG{o}{=} \PYG{l+m+mi}{10} \PYG{c+c1}{\PYGZsh{} in years}
\PYG{n}{n} \PYG{o}{=} \PYG{n}{ceil}\PYG{p}{(}\PYG{n}{tfinal}\PYG{o}{/}\PYG{n}{DeltaT}\PYG{p}{)}
\PYG{c+c1}{\PYGZsh{} set up arrays for t, a, v, and x}
\PYG{n}{t} \PYG{o}{=} \PYG{n}{np}\PYG{o}{.}\PYG{n}{zeros}\PYG{p}{(}\PYG{n}{n}\PYG{p}{)}
\PYG{n}{v} \PYG{o}{=} \PYG{n}{np}\PYG{o}{.}\PYG{n}{zeros}\PYG{p}{(}\PYG{p}{(}\PYG{n}{n}\PYG{p}{,}\PYG{l+m+mi}{2}\PYG{p}{)}\PYG{p}{)}
\PYG{n}{r} \PYG{o}{=} \PYG{n}{np}\PYG{o}{.}\PYG{n}{zeros}\PYG{p}{(}\PYG{p}{(}\PYG{n}{n}\PYG{p}{,}\PYG{l+m+mi}{2}\PYG{p}{)}\PYG{p}{)}
\PYG{c+c1}{\PYGZsh{} Initial conditions as compact 2\PYGZhy{}dimensional arrays}
\PYG{n}{r0} \PYG{o}{=} \PYG{n}{np}\PYG{o}{.}\PYG{n}{array}\PYG{p}{(}\PYG{p}{[}\PYG{l+m+mf}{1.0}\PYG{p}{,}\PYG{l+m+mf}{0.0}\PYG{p}{]}\PYG{p}{)}
\PYG{n}{v0} \PYG{o}{=} \PYG{n}{np}\PYG{o}{.}\PYG{n}{array}\PYG{p}{(}\PYG{p}{[}\PYG{l+m+mf}{0.0}\PYG{p}{,}\PYG{l+m+mi}{2}\PYG{o}{*}\PYG{n}{pi}\PYG{p}{]}\PYG{p}{)}
\PYG{n}{r}\PYG{p}{[}\PYG{l+m+mi}{0}\PYG{p}{]} \PYG{o}{=} \PYG{n}{r0}
\PYG{n}{v}\PYG{p}{[}\PYG{l+m+mi}{0}\PYG{p}{]} \PYG{o}{=} \PYG{n}{v0}
\PYG{n}{Fourpi2} \PYG{o}{=} \PYG{l+m+mi}{4}\PYG{o}{*}\PYG{n}{pi}\PYG{o}{*}\PYG{n}{pi}
\PYG{c+c1}{\PYGZsh{} Start integrating using Euler\PYGZsq{}s method}
\PYG{k}{for} \PYG{n}{i} \PYG{o+ow}{in} \PYG{n+nb}{range}\PYG{p}{(}\PYG{n}{n}\PYG{o}{\PYGZhy{}}\PYG{l+m+mi}{1}\PYG{p}{)}\PYG{p}{:}
    \PYG{c+c1}{\PYGZsh{} Set up the acceleration}
    \PYG{c+c1}{\PYGZsh{} Here you could have defined your own function for this}
    \PYG{n}{rabs} \PYG{o}{=} \PYG{n}{sqrt}\PYG{p}{(}\PYG{n+nb}{sum}\PYG{p}{(}\PYG{n}{r}\PYG{p}{[}\PYG{n}{i}\PYG{p}{]}\PYG{o}{*}\PYG{n}{r}\PYG{p}{[}\PYG{n}{i}\PYG{p}{]}\PYG{p}{)}\PYG{p}{)}
    \PYG{n}{a} \PYG{o}{=}  \PYG{o}{\PYGZhy{}}\PYG{n}{Fourpi2}\PYG{o}{*}\PYG{n}{r}\PYG{p}{[}\PYG{n}{i}\PYG{p}{]}\PYG{o}{/}\PYG{p}{(}\PYG{n}{rabs}\PYG{o}{*}\PYG{o}{*}\PYG{l+m+mi}{3}\PYG{p}{)}
    \PYG{c+c1}{\PYGZsh{} update velocity, time and position using Euler\PYGZsq{}s forward method}
    \PYG{n}{v}\PYG{p}{[}\PYG{n}{i}\PYG{o}{+}\PYG{l+m+mi}{1}\PYG{p}{]} \PYG{o}{=} \PYG{n}{v}\PYG{p}{[}\PYG{n}{i}\PYG{p}{]} \PYG{o}{+} \PYG{n}{DeltaT}\PYG{o}{*}\PYG{n}{a}
    \PYG{n}{r}\PYG{p}{[}\PYG{n}{i}\PYG{o}{+}\PYG{l+m+mi}{1}\PYG{p}{]} \PYG{o}{=} \PYG{n}{r}\PYG{p}{[}\PYG{n}{i}\PYG{p}{]} \PYG{o}{+} \PYG{n}{DeltaT}\PYG{o}{*}\PYG{n}{v}\PYG{p}{[}\PYG{n}{i}\PYG{p}{]}
    \PYG{n}{t}\PYG{p}{[}\PYG{n}{i}\PYG{o}{+}\PYG{l+m+mi}{1}\PYG{p}{]} \PYG{o}{=} \PYG{n}{t}\PYG{p}{[}\PYG{n}{i}\PYG{p}{]} \PYG{o}{+} \PYG{n}{DeltaT}
\PYG{c+c1}{\PYGZsh{} Plot position as function of time    }
\PYG{n}{fig}\PYG{p}{,} \PYG{n}{ax} \PYG{o}{=} \PYG{n}{plt}\PYG{o}{.}\PYG{n}{subplots}\PYG{p}{(}\PYG{p}{)}
\PYG{c+c1}{\PYGZsh{}ax.set\PYGZus{}xlim(0, tfinal)}
\PYG{n}{ax}\PYG{o}{.}\PYG{n}{set\PYGZus{}ylabel}\PYG{p}{(}\PYG{l+s+s1}{\PYGZsq{}}\PYG{l+s+s1}{x[m]}\PYG{l+s+s1}{\PYGZsq{}}\PYG{p}{)}
\PYG{n}{ax}\PYG{o}{.}\PYG{n}{set\PYGZus{}xlabel}\PYG{p}{(}\PYG{l+s+s1}{\PYGZsq{}}\PYG{l+s+s1}{y[m]}\PYG{l+s+s1}{\PYGZsq{}}\PYG{p}{)}
\PYG{n}{ax}\PYG{o}{.}\PYG{n}{plot}\PYG{p}{(}\PYG{n}{r}\PYG{p}{[}\PYG{p}{:}\PYG{p}{,}\PYG{l+m+mi}{0}\PYG{p}{]}\PYG{p}{,} \PYG{n}{r}\PYG{p}{[}\PYG{p}{:}\PYG{p}{,}\PYG{l+m+mi}{1}\PYG{p}{]}\PYG{p}{)}
\PYG{n}{fig}\PYG{o}{.}\PYG{n}{tight\PYGZus{}layout}\PYG{p}{(}\PYG{p}{)}
\PYG{n}{save\PYGZus{}fig}\PYG{p}{(}\PYG{l+s+s2}{\PYGZdq{}}\PYG{l+s+s2}{EarthSunEuler}\PYG{l+s+s2}{\PYGZdq{}}\PYG{p}{)}
\PYG{n}{plt}\PYG{o}{.}\PYG{n}{show}\PYG{p}{(}\PYG{p}{)}
\end{sphinxVerbatim}

\noindent\sphinxincludegraphics{{chapter3_108_0}.png}


\subsection{Problems with Euler’s Method}
\label{\detokenize{chapter3:problems-with-euler-s-method}}
We notice here that Euler’s method doesn’t give a stable orbit. It
means that we cannot trust Euler’s method. In a deeper way, as we will
see in homework 5, Euler’s method does not conserve energy. It is an
example of an integrator which is not
\sphinxhref{https://en.wikipedia.org/wiki/Symplectic\_integrator}{symplectic}.

Here we present thus two methods, which with simple changes allow us to avoid these pitfalls. The simplest possible extension is the so\sphinxhyphen{}called Euler\sphinxhyphen{}Cromer method.
The changes we need to make to our code are indeed marginal here.
We need simply to replace

\begin{sphinxVerbatim}[commandchars=\\\{\}]
    \PYG{n}{r}\PYG{p}{[}\PYG{n}{i}\PYG{o}{+}\PYG{l+m+mi}{1}\PYG{p}{]} \PYG{o}{=} \PYG{n}{r}\PYG{p}{[}\PYG{n}{i}\PYG{p}{]} \PYG{o}{+} \PYG{n}{DeltaT}\PYG{o}{*}\PYG{n}{v}\PYG{p}{[}\PYG{n}{i}\PYG{p}{]}
\end{sphinxVerbatim}

in the above code with the velocity at the new time \(t_{i+1}\)

\begin{sphinxVerbatim}[commandchars=\\\{\}]
    \PYG{n}{r}\PYG{p}{[}\PYG{n}{i}\PYG{o}{+}\PYG{l+m+mi}{1}\PYG{p}{]} \PYG{o}{=} \PYG{n}{r}\PYG{p}{[}\PYG{n}{i}\PYG{p}{]} \PYG{o}{+} \PYG{n}{DeltaT}\PYG{o}{*}\PYG{n}{v}\PYG{p}{[}\PYG{n}{i}\PYG{o}{+}\PYG{l+m+mi}{1}\PYG{p}{]}
\end{sphinxVerbatim}

By this simple caveat we get stable orbits.
Below we derive the Euler\sphinxhyphen{}Cromer method as well as one of the most utlized algorithms for sovling the above type of problems, the so\sphinxhyphen{}called Velocity\sphinxhyphen{}Verlet method.


\subsection{Deriving the Euler\sphinxhyphen{}Cromer Method}
\label{\detokenize{chapter3:deriving-the-euler-cromer-method}}
Let us repeat Euler’s method.
We have a differential equation




\begin{equation*}
\begin{split}
\begin{equation}
y'(t_i)=f(t_i,y_i)   
\label{_auto13} \tag{13}
\end{equation}
\end{split}
\end{equation*}
and if we truncate at the first derivative, we have from the Taylor expansion




\begin{equation*}
\begin{split}
\begin{equation}
y_{i+1}=y(t_i) + (\Delta t) f(t_i,y_i) + O(\Delta t^2), \label{eq:euler} \tag{14}
\end{equation}
\end{split}
\end{equation*}
which when complemented with \(t_{i+1}=t_i+\Delta t\) forms
the algorithm for the well\sphinxhyphen{}known Euler method.
Note that at every step we make an approximation error
of the order of \(O(\Delta t^2)\), however the total error is the sum over all
steps \(N=(b-a)/(\Delta t)\) for \(t\in [a,b]\), yielding thus a global error which goes like
\(NO(\Delta t^2)\approx O(\Delta t)\).

To make Euler’s method more precise we can obviously
decrease \(\Delta t\) (increase \(N\)), but this can lead to loss of numerical precision.
Euler’s method is not recommended for precision calculation,
although it is handy to use in order to get a first
view on how a solution may look like.

Euler’s method is asymmetric in time, since it uses information about the derivative at the beginning
of the time interval. This means that we evaluate the position at \(y_1\) using the velocity
at \(v_0\). A simple variation is to determine \(x_{n+1}\) using the velocity at
\(v_{n+1}\), that is (in a slightly more generalized form)




\begin{equation*}
\begin{split}
\begin{equation} 
y_{n+1}=y_{n}+ v_{n+1}+O(\Delta t^2)
\label{_auto14} \tag{15}
\end{equation}
\end{split}
\end{equation*}
and




\begin{equation*}
\begin{split}
\begin{equation}
v_{n+1}=v_{n}+(\Delta t) a_{n}+O(\Delta t^2).
\label{_auto15} \tag{16}
\end{equation}
\end{split}
\end{equation*}
The acceleration \(a_n\) is a function of \(a_n(y_n, v_n, t_n)\) and needs to be evaluated
as well. This is the Euler\sphinxhyphen{}Cromer method.

\sphinxstylestrong{Exercise}: go back to the above code with Euler’s method and add the Euler\sphinxhyphen{}Cromer method.


\subsection{Deriving the Velocity\sphinxhyphen{}Verlet Method}
\label{\detokenize{chapter3:deriving-the-velocity-verlet-method}}
Let us stay with \(x\) (position) and \(v\) (velocity) as the quantities we are interested in.

We have the Taylor expansion for the position given by
\begin{equation*}
\begin{split}
x_{i+1} = x_i+(\Delta t)v_i+\frac{(\Delta t)^2}{2}a_i+O((\Delta t)^3).
\end{split}
\end{equation*}
The corresponding expansion for the velocity is
\begin{equation*}
\begin{split}
v_{i+1} = v_i+(\Delta t)a_i+\frac{(\Delta t)^2}{2}v^{(2)}_i+O((\Delta t)^3).
\end{split}
\end{equation*}
Via Newton’s second law we have normally an analytical expression for the derivative of the velocity, namely
\begin{equation*}
\begin{split}
a_i= \frac{d^2 x}{dt^2}\vert_{i}=\frac{d v}{dt}\vert_{i}= \frac{F(x_i,v_i,t_i)}{m}.
\end{split}
\end{equation*}
If we add to this the corresponding expansion for the derivative of the velocity
\begin{equation*}
\begin{split}
v^{(1)}_{i+1} = a_{i+1}= a_i+(\Delta t)v^{(2)}_i+O((\Delta t)^2)=a_i+(\Delta t)v^{(2)}_i+O((\Delta t)^2),
\end{split}
\end{equation*}
and retain only terms up to the second derivative of the velocity since our error goes as \(O(h^3)\), we have
\begin{equation*}
\begin{split}
(\Delta t)v^{(2)}_i\approx a_{i+1}-a_i.
\end{split}
\end{equation*}
We can then rewrite the Taylor expansion for the velocity as
\begin{equation*}
\begin{split}
v_{i+1} = v_i+\frac{(\Delta t)}{2}\left( a_{i+1}+a_{i}\right)+O((\Delta t)^3).
\end{split}
\end{equation*}

\subsection{The velocity Verlet method}
\label{\detokenize{chapter3:the-velocity-verlet-method}}
Our final equations for the position and the velocity become then
\begin{equation*}
\begin{split}
x_{i+1} = x_i+(\Delta t)v_i+\frac{(\Delta t)^2}{2}a_{i}+O((\Delta t)^3),
\end{split}
\end{equation*}
and
\begin{equation*}
\begin{split}
v_{i+1} = v_i+\frac{(\Delta t)}{2}\left(a_{i+1}+a_{i}\right)+O((\Delta t)^3).
\end{split}
\end{equation*}
Note well that the term \(a_{i+1}\) depends on the position at \(x_{i+1}\). This means that you need to calculate
the position at the updated time \(t_{i+1}\) before the computing the next velocity.  Note also that the derivative of the velocity at the time
\(t_i\) used in the updating of the position can be reused in the calculation of the velocity update as well.


\subsection{Adding the Velocity\sphinxhyphen{}Verlet Method}
\label{\detokenize{chapter3:adding-the-velocity-verlet-method}}
We can now easily add the Verlet method to our original code as

\begin{sphinxVerbatim}[commandchars=\\\{\}]
\PYG{n}{DeltaT} \PYG{o}{=} \PYG{l+m+mf}{0.01}
\PYG{c+c1}{\PYGZsh{}set up arrays }
\PYG{n}{tfinal} \PYG{o}{=} \PYG{l+m+mi}{10}
\PYG{n}{n} \PYG{o}{=} \PYG{n}{ceil}\PYG{p}{(}\PYG{n}{tfinal}\PYG{o}{/}\PYG{n}{DeltaT}\PYG{p}{)}
\PYG{c+c1}{\PYGZsh{} set up arrays for t, a, v, and x}
\PYG{n}{t} \PYG{o}{=} \PYG{n}{np}\PYG{o}{.}\PYG{n}{zeros}\PYG{p}{(}\PYG{n}{n}\PYG{p}{)}
\PYG{n}{v} \PYG{o}{=} \PYG{n}{np}\PYG{o}{.}\PYG{n}{zeros}\PYG{p}{(}\PYG{p}{(}\PYG{n}{n}\PYG{p}{,}\PYG{l+m+mi}{2}\PYG{p}{)}\PYG{p}{)}
\PYG{n}{r} \PYG{o}{=} \PYG{n}{np}\PYG{o}{.}\PYG{n}{zeros}\PYG{p}{(}\PYG{p}{(}\PYG{n}{n}\PYG{p}{,}\PYG{l+m+mi}{2}\PYG{p}{)}\PYG{p}{)}
\PYG{c+c1}{\PYGZsh{} Initial conditions as compact 2\PYGZhy{}dimensional arrays}
\PYG{n}{r0} \PYG{o}{=} \PYG{n}{np}\PYG{o}{.}\PYG{n}{array}\PYG{p}{(}\PYG{p}{[}\PYG{l+m+mf}{1.0}\PYG{p}{,}\PYG{l+m+mf}{0.0}\PYG{p}{]}\PYG{p}{)}
\PYG{n}{v0} \PYG{o}{=} \PYG{n}{np}\PYG{o}{.}\PYG{n}{array}\PYG{p}{(}\PYG{p}{[}\PYG{l+m+mf}{0.0}\PYG{p}{,}\PYG{l+m+mi}{2}\PYG{o}{*}\PYG{n}{pi}\PYG{p}{]}\PYG{p}{)}
\PYG{n}{r}\PYG{p}{[}\PYG{l+m+mi}{0}\PYG{p}{]} \PYG{o}{=} \PYG{n}{r0}
\PYG{n}{v}\PYG{p}{[}\PYG{l+m+mi}{0}\PYG{p}{]} \PYG{o}{=} \PYG{n}{v0}
\PYG{n}{Fourpi2} \PYG{o}{=} \PYG{l+m+mi}{4}\PYG{o}{*}\PYG{n}{pi}\PYG{o}{*}\PYG{n}{pi}
\PYG{c+c1}{\PYGZsh{} Start integrating using the Velocity\PYGZhy{}Verlet  method}
\PYG{k}{for} \PYG{n}{i} \PYG{o+ow}{in} \PYG{n+nb}{range}\PYG{p}{(}\PYG{n}{n}\PYG{o}{\PYGZhy{}}\PYG{l+m+mi}{1}\PYG{p}{)}\PYG{p}{:}
    \PYG{c+c1}{\PYGZsh{} Set up forces, air resistance FD, note now that we need the norm of the vecto}
    \PYG{c+c1}{\PYGZsh{} Here you could have defined your own function for this}
    \PYG{n}{rabs} \PYG{o}{=} \PYG{n}{sqrt}\PYG{p}{(}\PYG{n+nb}{sum}\PYG{p}{(}\PYG{n}{r}\PYG{p}{[}\PYG{n}{i}\PYG{p}{]}\PYG{o}{*}\PYG{n}{r}\PYG{p}{[}\PYG{n}{i}\PYG{p}{]}\PYG{p}{)}\PYG{p}{)}
    \PYG{n}{a} \PYG{o}{=}  \PYG{o}{\PYGZhy{}}\PYG{n}{Fourpi2}\PYG{o}{*}\PYG{n}{r}\PYG{p}{[}\PYG{n}{i}\PYG{p}{]}\PYG{o}{/}\PYG{p}{(}\PYG{n}{rabs}\PYG{o}{*}\PYG{o}{*}\PYG{l+m+mi}{3}\PYG{p}{)}
    \PYG{c+c1}{\PYGZsh{} update velocity, time and position using the Velocity\PYGZhy{}Verlet method}
    \PYG{n}{r}\PYG{p}{[}\PYG{n}{i}\PYG{o}{+}\PYG{l+m+mi}{1}\PYG{p}{]} \PYG{o}{=} \PYG{n}{r}\PYG{p}{[}\PYG{n}{i}\PYG{p}{]} \PYG{o}{+} \PYG{n}{DeltaT}\PYG{o}{*}\PYG{n}{v}\PYG{p}{[}\PYG{n}{i}\PYG{p}{]}\PYG{o}{+}\PYG{l+m+mf}{0.5}\PYG{o}{*}\PYG{p}{(}\PYG{n}{DeltaT}\PYG{o}{*}\PYG{o}{*}\PYG{l+m+mi}{2}\PYG{p}{)}\PYG{o}{*}\PYG{n}{a}
    \PYG{n}{rabs} \PYG{o}{=} \PYG{n}{sqrt}\PYG{p}{(}\PYG{n+nb}{sum}\PYG{p}{(}\PYG{n}{r}\PYG{p}{[}\PYG{n}{i}\PYG{o}{+}\PYG{l+m+mi}{1}\PYG{p}{]}\PYG{o}{*}\PYG{n}{r}\PYG{p}{[}\PYG{n}{i}\PYG{o}{+}\PYG{l+m+mi}{1}\PYG{p}{]}\PYG{p}{)}\PYG{p}{)}
    \PYG{n}{anew} \PYG{o}{=} \PYG{o}{\PYGZhy{}}\PYG{l+m+mi}{4}\PYG{o}{*}\PYG{p}{(}\PYG{n}{pi}\PYG{o}{*}\PYG{o}{*}\PYG{l+m+mi}{2}\PYG{p}{)}\PYG{o}{*}\PYG{n}{r}\PYG{p}{[}\PYG{n}{i}\PYG{o}{+}\PYG{l+m+mi}{1}\PYG{p}{]}\PYG{o}{/}\PYG{p}{(}\PYG{n}{rabs}\PYG{o}{*}\PYG{o}{*}\PYG{l+m+mi}{3}\PYG{p}{)}
    \PYG{n}{v}\PYG{p}{[}\PYG{n}{i}\PYG{o}{+}\PYG{l+m+mi}{1}\PYG{p}{]} \PYG{o}{=} \PYG{n}{v}\PYG{p}{[}\PYG{n}{i}\PYG{p}{]} \PYG{o}{+} \PYG{l+m+mf}{0.5}\PYG{o}{*}\PYG{n}{DeltaT}\PYG{o}{*}\PYG{p}{(}\PYG{n}{a}\PYG{o}{+}\PYG{n}{anew}\PYG{p}{)}
    \PYG{n}{t}\PYG{p}{[}\PYG{n}{i}\PYG{o}{+}\PYG{l+m+mi}{1}\PYG{p}{]} \PYG{o}{=} \PYG{n}{t}\PYG{p}{[}\PYG{n}{i}\PYG{p}{]} \PYG{o}{+} \PYG{n}{DeltaT}
\PYG{c+c1}{\PYGZsh{} Plot position as function of time    }
\PYG{n}{fig}\PYG{p}{,} \PYG{n}{ax} \PYG{o}{=} \PYG{n}{plt}\PYG{o}{.}\PYG{n}{subplots}\PYG{p}{(}\PYG{p}{)}
\PYG{n}{ax}\PYG{o}{.}\PYG{n}{set\PYGZus{}ylabel}\PYG{p}{(}\PYG{l+s+s1}{\PYGZsq{}}\PYG{l+s+s1}{x[m]}\PYG{l+s+s1}{\PYGZsq{}}\PYG{p}{)}
\PYG{n}{ax}\PYG{o}{.}\PYG{n}{set\PYGZus{}xlabel}\PYG{p}{(}\PYG{l+s+s1}{\PYGZsq{}}\PYG{l+s+s1}{y[m]}\PYG{l+s+s1}{\PYGZsq{}}\PYG{p}{)}
\PYG{n}{ax}\PYG{o}{.}\PYG{n}{plot}\PYG{p}{(}\PYG{n}{r}\PYG{p}{[}\PYG{p}{:}\PYG{p}{,}\PYG{l+m+mi}{0}\PYG{p}{]}\PYG{p}{,} \PYG{n}{r}\PYG{p}{[}\PYG{p}{:}\PYG{p}{,}\PYG{l+m+mi}{1}\PYG{p}{]}\PYG{p}{)}
\PYG{n}{fig}\PYG{o}{.}\PYG{n}{tight\PYGZus{}layout}\PYG{p}{(}\PYG{p}{)}
\PYG{n}{save\PYGZus{}fig}\PYG{p}{(}\PYG{l+s+s2}{\PYGZdq{}}\PYG{l+s+s2}{EarthSunVV}\PYG{l+s+s2}{\PYGZdq{}}\PYG{p}{)}
\PYG{n}{plt}\PYG{o}{.}\PYG{n}{show}\PYG{p}{(}\PYG{p}{)}
\end{sphinxVerbatim}

\noindent\sphinxincludegraphics{{chapter3_138_0}.png}

You can easily generalize the calculation of the forces by defining a function
which takes in as input the various variables. We leave this as a challenge to you.


\subsection{Studying Energy Conservation}
\label{\detokenize{chapter3:studying-energy-conservation}}
In order to study the conservation of energy, we will need to perform
a numerical integration, unless we can integrate analytically. Here we
present the Trapezoidal rule as a the simplest possible approximation.


\subsection{Numerical Integration}
\label{\detokenize{chapter3:numerical-integration}}
It is also useful to consider methods to integrate numerically.
Let us consider the following case.
We have  classical electron which moves in the \(x\)\sphinxhyphen{}direction along a surface. The force from the surface is
\begin{equation*}
\begin{split}
\boldsymbol{F}(x)=-F_0\sin{(\frac{2\pi x}{b})}\boldsymbol{e}_x.
\end{split}
\end{equation*}
The constant \(b\) represents the distance between atoms at the surface of the material, \(F_0\) is a constant and \(x\) is the position of the electron.
Using the work\sphinxhyphen{}energy theorem we can find the work \(W\) done when moving an electron from a position \(x_0\) to a final position \(x\) through the
integral
\begin{equation*}
\begin{split}
W=-\int_{x_0}^x \boldsymbol{F}(x')dx' =  \int_{x_0}^x F_0\sin{(\frac{2\pi x'}{b})} dx',
\end{split}
\end{equation*}
which results in
\begin{equation*}
\begin{split}
W=\frac{F_0b}{2\pi}\left[\cos{(\frac{2\pi x}{b})}-\cos{(\frac{2\pi x_0}{b})}\right].
\end{split}
\end{equation*}

\subsection{Numerical Integration}
\label{\detokenize{chapter3:id1}}
There are several numerical algorithms for finding an integral
numerically. The more familiar ones like the rectangular rule or the
trapezoidal rule have simple geometric interpretations.

Let us look at the mathematical details of what are called equal\sphinxhyphen{}step methods, also known as Newton\sphinxhyphen{}Cotes quadrature.


\subsection{Newton\sphinxhyphen{}Cotes Quadrature or equal\sphinxhyphen{}step methods}
\label{\detokenize{chapter3:newton-cotes-quadrature-or-equal-step-methods}}
The integral




\begin{equation*}
\begin{split}
\begin{equation}
   I=\int_a^bf(x) dx
\label{eq:integraldef} \tag{17}
\end{equation}
\end{split}
\end{equation*}
has a very simple meaning. The integral is the
area enscribed by the function \(f(x)\) starting from \(x=a\) to  \(x=b\). It is subdivided in several smaller areas whose evaluation is to  be approximated by different techniques. The areas under the curve can for example  be approximated by rectangular boxes or trapezoids.




\subsection{Basic philosophy of equal\sphinxhyphen{}step methods}
\label{\detokenize{chapter3:basic-philosophy-of-equal-step-methods}}
In considering equal step  methods, our basic approach is that of approximating
a function \(f(x)\) with a polynomial of at most
degree \(N-1\), given \(N\) integration points. If our polynomial is of degree \(1\),
the function will be approximated with \(f(x)\approx a_0+a_1x\).




\subsection{Simple algorithm for equal step methods}
\label{\detokenize{chapter3:simple-algorithm-for-equal-step-methods}}
The algorithm for these integration methods  is rather simple, and the number of approximations perhaps  unlimited!
\begin{itemize}
\item {} 
Choose a step size \(h=(b-a)/N\)  where \(N\) is the number of steps and \(a\) and \(b\) the lower and upper limits of integration.

\item {} 
With a given step length we rewrite the integral as

\end{itemize}
\begin{equation*}
\begin{split}
\int_a^bf(x) dx= \int_a^{a+h}f(x)dx + \int_{a+h}^{a+2h}f(x)dx+\dots \int_{b-h}^{b}f(x)dx.
\end{split}
\end{equation*}\begin{itemize}
\item {} 
The strategy then is to find a reliable polynomial approximation   for \(f(x)\) in the various intervals.  Choosing a given approximation for  \(f(x)\), we obtain a specific approximation to the  integral.

\item {} 
With this approximation to \(f(x)\) we perform the integration by computing the integrals over all subintervals.

\end{itemize}




\subsection{Simple algorithm for equal step methods}
\label{\detokenize{chapter3:id2}}
One possible strategy then is to find a reliable polynomial expansion for \(f(x)\) in the smaller
subintervals. Consider for example evaluating
\begin{equation*}
\begin{split}
\int_a^{a+2h}f(x)dx,
\end{split}
\end{equation*}
which we rewrite as




\begin{equation*}
\begin{split}
\begin{equation}
\int_a^{a+2h}f(x)dx=\int_{x_0-h}^{x_0+h}f(x)dx.
\label{eq:hhint} \tag{18}
\end{equation}
\end{split}
\end{equation*}
We have chosen a midpoint \(x_0\) and have defined \(x_0=a+h\).




\subsection{The rectangle method}
\label{\detokenize{chapter3:the-rectangle-method}}
A very simple approach is the so\sphinxhyphen{}called midpoint or rectangle method.
In this case the integration area is split in a given number of rectangles with length \(h\) and height given by the mid\sphinxhyphen{}point value of the function.  This gives the following simple rule for approximating an integral




\begin{equation*}
\begin{split}
\begin{equation}
I=\int_a^bf(x) dx \approx  h\sum_{i=1}^N f(x_{i-1/2}), 
\label{eq:rectangle} \tag{19}
\end{equation}
\end{split}
\end{equation*}
where \(f(x_{i-1/2})\) is the midpoint value of \(f\) for a given rectangle. We will discuss its truncation
error below.  It is easy to implement this algorithm,  as shown below




\subsection{Truncation error for the rectangular rule}
\label{\detokenize{chapter3:truncation-error-for-the-rectangular-rule}}
The correct mathematical expression for the local error for the rectangular rule \(R_i(h)\) for element \(i\) is
\begin{equation*}
\begin{split}
\int_{-h}^hf(x)dx - R_i(h)=-\frac{h^3}{24}f^{(2)}(\xi),
\end{split}
\end{equation*}
and the global error reads
\begin{equation*}
\begin{split}
\int_a^bf(x)dx -R_h(f)=-\frac{b-a}{24}h^2f^{(2)}(\xi),
\end{split}
\end{equation*}
where \(R_h\) is the result obtained with rectangular rule and \(\xi \in [a,b]\).


\subsection{Codes for the Rectangular rule}
\label{\detokenize{chapter3:codes-for-the-rectangular-rule}}
We go back to our simple example above and set \(F_0=b=1\) and choose \(x_0=0\) and \(x=1/2\), and have
\begin{equation*}
\begin{split}
W=\frac{1}{\pi}.
\end{split}
\end{equation*}
The code here computes the integral using the rectangle rule and \(n=100\) integration points we have a relative error of
\(10^{-5}\).

\begin{sphinxVerbatim}[commandchars=\\\{\}]
\PYG{k+kn}{from} \PYG{n+nn}{math} \PYG{k+kn}{import} \PYG{n}{sin}\PYG{p}{,} \PYG{n}{pi}
\PYG{k+kn}{import} \PYG{n+nn}{numpy} \PYG{k}{as} \PYG{n+nn}{np}
\PYG{k+kn}{from} \PYG{n+nn}{sympy} \PYG{k+kn}{import} \PYG{n}{Symbol}\PYG{p}{,} \PYG{n}{integrate}
\PYG{c+c1}{\PYGZsh{} function for the Rectangular rule                                                                                        }
\PYG{k}{def} \PYG{n+nf}{Rectangular}\PYG{p}{(}\PYG{n}{a}\PYG{p}{,}\PYG{n}{b}\PYG{p}{,}\PYG{n}{f}\PYG{p}{,}\PYG{n}{n}\PYG{p}{)}\PYG{p}{:}
   \PYG{n}{h} \PYG{o}{=} \PYG{p}{(}\PYG{n}{b}\PYG{o}{\PYGZhy{}}\PYG{n}{a}\PYG{p}{)}\PYG{o}{/}\PYG{n+nb}{float}\PYG{p}{(}\PYG{n}{n}\PYG{p}{)}
   \PYG{n}{s} \PYG{o}{=} \PYG{l+m+mi}{0}
   \PYG{k}{for} \PYG{n}{i} \PYG{o+ow}{in} \PYG{n+nb}{range}\PYG{p}{(}\PYG{l+m+mi}{0}\PYG{p}{,}\PYG{n}{n}\PYG{p}{,}\PYG{l+m+mi}{1}\PYG{p}{)}\PYG{p}{:}
       \PYG{n}{x} \PYG{o}{=} \PYG{p}{(}\PYG{n}{i}\PYG{o}{+}\PYG{l+m+mf}{0.5}\PYG{p}{)}\PYG{o}{*}\PYG{n}{h}
       \PYG{n}{s} \PYG{o}{=} \PYG{n}{s}\PYG{o}{+} \PYG{n}{f}\PYG{p}{(}\PYG{n}{x}\PYG{p}{)}
   \PYG{k}{return} \PYG{n}{h}\PYG{o}{*}\PYG{n}{s}
\PYG{c+c1}{\PYGZsh{} function to integrate}
\PYG{k}{def} \PYG{n+nf}{function}\PYG{p}{(}\PYG{n}{x}\PYG{p}{)}\PYG{p}{:}
    \PYG{k}{return} \PYG{n}{sin}\PYG{p}{(}\PYG{l+m+mi}{2}\PYG{o}{*}\PYG{n}{pi}\PYG{o}{*}\PYG{n}{x}\PYG{p}{)}
\PYG{c+c1}{\PYGZsh{} define integration limits and integration points                                                                         }
\PYG{n}{a} \PYG{o}{=} \PYG{l+m+mf}{0.0}\PYG{p}{;} \PYG{n}{b} \PYG{o}{=} \PYG{l+m+mf}{0.5}\PYG{p}{;}
\PYG{n}{n} \PYG{o}{=} \PYG{l+m+mi}{100}
\PYG{n}{Exact} \PYG{o}{=} \PYG{l+m+mf}{1.}\PYG{o}{/}\PYG{n}{pi}
\PYG{n+nb}{print}\PYG{p}{(}\PYG{l+s+s2}{\PYGZdq{}}\PYG{l+s+s2}{Relative error= }\PYG{l+s+s2}{\PYGZdq{}}\PYG{p}{,} \PYG{n+nb}{abs}\PYG{p}{(} \PYG{p}{(}\PYG{n}{Rectangular}\PYG{p}{(}\PYG{n}{a}\PYG{p}{,}\PYG{n}{b}\PYG{p}{,}\PYG{n}{function}\PYG{p}{,}\PYG{n}{n}\PYG{p}{)}\PYG{o}{\PYGZhy{}}\PYG{n}{Exact}\PYG{p}{)}\PYG{o}{/}\PYG{n}{Exact}\PYG{p}{)}\PYG{p}{)}
\end{sphinxVerbatim}

\begin{sphinxVerbatim}[commandchars=\\\{\}]
Relative error=  4.112453549290521e\PYGZhy{}05
\end{sphinxVerbatim}




\subsection{The trapezoidal rule}
\label{\detokenize{chapter3:the-trapezoidal-rule}}
The other integral gives
\begin{equation*}
\begin{split}
\int_{x_0-h}^{x_0}f(x)dx=\frac{h}{2}\left(f(x_0) + f(x_0-h)\right)+O(h^3),
\end{split}
\end{equation*}
and adding up we obtain




\begin{equation*}
\begin{split}
\begin{equation}
   \int_{x_0-h}^{x_0+h}f(x)dx=\frac{h}{2}\left(f(x_0+h) + 2f(x_0) + f(x_0-h)\right)+O(h^3),
\label{eq:trapez} \tag{20}
\end{equation}
\end{split}
\end{equation*}
which is the well\sphinxhyphen{}known trapezoidal rule.  Concerning the error in the approximation made,
\(O(h^3)=O((b-a)^3/N^3)\), you should  note
that this is the local error.  Since we are splitting the integral from
\(a\) to \(b\) in \(N\) pieces, we will have to perform approximately \(N\)
such operations.

This means that the \sphinxstyleemphasis{global error} goes like \(\approx O(h^2)\).
The trapezoidal reads then




\begin{equation*}
\begin{split}
\begin{equation}
   I=\int_a^bf(x) dx=h\left(f(a)/2 + f(a+h) +f(a+2h)+
                          \dots +f(b-h)+ f_{b}/2\right),
\label{eq:trapez1} \tag{21}
\end{equation}
\end{split}
\end{equation*}
with a global error which goes like \(O(h^2)\).

Hereafter we use the shorthand notations \(f_{-h}=f(x_0-h)\), \(f_{0}=f(x_0)\)
and \(f_{h}=f(x_0+h)\).




\subsection{Error in the trapezoidal rule}
\label{\detokenize{chapter3:error-in-the-trapezoidal-rule}}
The correct mathematical expression for the local error for the trapezoidal rule is
\begin{equation*}
\begin{split}
\int_a^bf(x)dx -\frac{b-a}{2}\left[f(a)+f(b)\right]=-\frac{h^3}{12}f^{(2)}(\xi),
\end{split}
\end{equation*}
and the global error reads
\begin{equation*}
\begin{split}
\int_a^bf(x)dx -T_h(f)=-\frac{b-a}{12}h^2f^{(2)}(\xi),
\end{split}
\end{equation*}
where \(T_h\) is the trapezoidal result and \(\xi \in [a,b]\).




\subsection{Algorithm for the trapezoidal rule}
\label{\detokenize{chapter3:algorithm-for-the-trapezoidal-rule}}
The trapezoidal rule is easy to  implement numerically
through the following simple algorithm
\begin{itemize}
\item {} 
Choose the number of mesh points and fix the step length.

\item {} 
calculate \(f(a)\) and \(f(b)\) and multiply with \(h/2\).

\item {} 
Perform a loop over \(n=1\) to \(n-1\) (\(f(a)\) and \(f(b)\) are known) and sum up  the terms \(f(a+h) +f(a+2h)+f(a+3h)+\dots +f(b-h)\). Each step in the loop  corresponds to a given value \(a+nh\).

\item {} 
Multiply the final result by \(h\) and add \(hf(a)/2\) and \(hf(b)/2\).

\end{itemize}


\subsection{Trapezoidal Rule}
\label{\detokenize{chapter3:trapezoidal-rule}}
We use the same function and integrate now using the trapoezoidal rule.

\begin{sphinxVerbatim}[commandchars=\\\{\}]
\PYG{k+kn}{import} \PYG{n+nn}{numpy} \PYG{k}{as} \PYG{n+nn}{np}
\PYG{k+kn}{from} \PYG{n+nn}{sympy} \PYG{k+kn}{import} \PYG{n}{Symbol}\PYG{p}{,} \PYG{n}{integrate}
\PYG{c+c1}{\PYGZsh{} function for the trapezoidal rule}
\PYG{k}{def} \PYG{n+nf}{Trapez}\PYG{p}{(}\PYG{n}{a}\PYG{p}{,}\PYG{n}{b}\PYG{p}{,}\PYG{n}{f}\PYG{p}{,}\PYG{n}{n}\PYG{p}{)}\PYG{p}{:}
   \PYG{n}{h} \PYG{o}{=} \PYG{p}{(}\PYG{n}{b}\PYG{o}{\PYGZhy{}}\PYG{n}{a}\PYG{p}{)}\PYG{o}{/}\PYG{n+nb}{float}\PYG{p}{(}\PYG{n}{n}\PYG{p}{)}
   \PYG{n}{s} \PYG{o}{=} \PYG{l+m+mi}{0}
   \PYG{n}{x} \PYG{o}{=} \PYG{n}{a}
   \PYG{k}{for} \PYG{n}{i} \PYG{o+ow}{in} \PYG{n+nb}{range}\PYG{p}{(}\PYG{l+m+mi}{1}\PYG{p}{,}\PYG{n}{n}\PYG{p}{,}\PYG{l+m+mi}{1}\PYG{p}{)}\PYG{p}{:}
       \PYG{n}{x} \PYG{o}{=} \PYG{n}{x}\PYG{o}{+}\PYG{n}{h}
       \PYG{n}{s} \PYG{o}{=} \PYG{n}{s}\PYG{o}{+} \PYG{n}{f}\PYG{p}{(}\PYG{n}{x}\PYG{p}{)}
   \PYG{n}{s} \PYG{o}{=} \PYG{l+m+mf}{0.5}\PYG{o}{*}\PYG{p}{(}\PYG{n}{f}\PYG{p}{(}\PYG{n}{a}\PYG{p}{)}\PYG{o}{+}\PYG{n}{f}\PYG{p}{(}\PYG{n}{b}\PYG{p}{)}\PYG{p}{)} \PYG{o}{+}\PYG{n}{s}
   \PYG{k}{return} \PYG{n}{h}\PYG{o}{*}\PYG{n}{s}
\PYG{c+c1}{\PYGZsh{} function to integrate}
\PYG{k}{def} \PYG{n+nf}{function}\PYG{p}{(}\PYG{n}{x}\PYG{p}{)}\PYG{p}{:}
    \PYG{k}{return} \PYG{n}{sin}\PYG{p}{(}\PYG{l+m+mi}{2}\PYG{o}{*}\PYG{n}{pi}\PYG{o}{*}\PYG{n}{x}\PYG{p}{)}
\PYG{c+c1}{\PYGZsh{} define integration limits and integration points                                                                         }
\PYG{n}{a} \PYG{o}{=} \PYG{l+m+mf}{0.0}\PYG{p}{;} \PYG{n}{b} \PYG{o}{=} \PYG{l+m+mf}{0.5}\PYG{p}{;}
\PYG{n}{n} \PYG{o}{=} \PYG{l+m+mi}{100}
\PYG{n}{Exact} \PYG{o}{=} \PYG{l+m+mf}{1.}\PYG{o}{/}\PYG{n}{pi}
\PYG{n+nb}{print}\PYG{p}{(}\PYG{l+s+s2}{\PYGZdq{}}\PYG{l+s+s2}{Relative error= }\PYG{l+s+s2}{\PYGZdq{}}\PYG{p}{,} \PYG{n+nb}{abs}\PYG{p}{(} \PYG{p}{(}\PYG{n}{Trapez}\PYG{p}{(}\PYG{n}{a}\PYG{p}{,}\PYG{n}{b}\PYG{p}{,}\PYG{n}{function}\PYG{p}{,}\PYG{n}{n}\PYG{p}{)}\PYG{o}{\PYGZhy{}}\PYG{n}{Exact}\PYG{p}{)}\PYG{o}{/}\PYG{n}{Exact}\PYG{p}{)}\PYG{p}{)}
\end{sphinxVerbatim}

\begin{sphinxVerbatim}[commandchars=\\\{\}]
Relative error=  8.224805627923717e\PYGZhy{}05
\end{sphinxVerbatim}


\subsection{Simpsons’ rule}
\label{\detokenize{chapter3:simpsons-rule}}
Instead of using the above first\sphinxhyphen{}order polynomials
approximations for \(f\), we attempt at using a second\sphinxhyphen{}order polynomials.
In this case we need three points in order to define a second\sphinxhyphen{}order
polynomial approximation
\begin{equation*}
\begin{split}
f(x) \approx P_2(x)=a_0+a_1x+a_2x^2.
\end{split}
\end{equation*}
Using again Lagrange’s interpolation formula we have
\begin{equation*}
\begin{split}
P_2(x)=\frac{(x-x_0)(x-x_1)}{(x_2-x_0)(x_2-x_1)}y_2+
            \frac{(x-x_0)(x-x_2)}{(x_1-x_0)(x_1-x_2)}y_1+
            \frac{(x-x_1)(x-x_2)}{(x_0-x_1)(x_0-x_2)}y_0.
\end{split}
\end{equation*}
Inserting this formula in the integral of Eq.  ({\hyperref[\detokenize{chapter3:eq:hhint}]{\emph{18}}}) we obtain
\begin{equation*}
\begin{split}
\int_{-h}^{+h}f(x)dx=\frac{h}{3}\left(f_h + 4f_0 + f_{-h}\right)+O(h^5),
\end{split}
\end{equation*}
which is Simpson’s rule.




\subsection{Simpson’s rule}
\label{\detokenize{chapter3:simpson-s-rule}}
Note that the improved accuracy in the evaluation of
the derivatives gives a better error approximation, \(O(h^5)\) vs.\textbackslash{} \(O(h^3)\) .
But this is again the \sphinxstyleemphasis{local error approximation}.
Using Simpson’s rule we can easily compute
the integral     of Eq.  ({\hyperref[\detokenize{chapter3:eq:integraldef}]{\emph{17}}}) to be




\begin{equation*}
\begin{split}
\begin{equation}
   I=\int_a^bf(x) dx=\frac{h}{3}\left(f(a) + 4f(a+h) +2f(a+2h)+
                          \dots +4f(b-h)+ f_{b}\right),
\label{eq:simpson} \tag{22}
\end{equation}
\end{split}
\end{equation*}
with a global error which goes like \(O(h^4)\).




\subsection{Mathematical expressions for the truncation error}
\label{\detokenize{chapter3:mathematical-expressions-for-the-truncation-error}}
More formal expressions for the local and global errors are for the local error
\begin{equation*}
\begin{split}
\int_a^bf(x)dx -\frac{b-a}{6}\left[f(a)+4f((a+b)/2)+f(b)\right]=-\frac{h^5}{90}f^{(4)}(\xi),
\end{split}
\end{equation*}
and for the global error
\begin{equation*}
\begin{split}
\int_a^bf(x)dx -S_h(f)=-\frac{b-a}{180}h^4f^{(4)}(\xi).
\end{split}
\end{equation*}
with \(\xi\in[a,b]\) and \(S_h\) the results obtained with Simpson’s method.




\subsection{Algorithm for Simpson’s rule}
\label{\detokenize{chapter3:algorithm-for-simpson-s-rule}}
The method
can easily be implemented numerically through the following simple algorithm
\begin{itemize}
\item {} 
Choose the number of mesh points and fix the step.

\item {} 
calculate \(f(a)\) and \(f(b)\)

\item {} 
Perform a loop over \(n=1\) to \(n-1\) (\(f(a)\) and \(f(b)\) are known) and sum up   the terms \(4f(a+h) +2f(a+2h)+4f(a+3h)+\dots +4f(b-h)\). Each step in the loop  corresponds to a given value \(a+nh\). Odd values of \(n\) give \(4\) as factor  while even values yield \(2\) as factor.

\item {} 
Multiply the final result by \(\frac{h}{3}\).

\end{itemize}


\subsection{Code example}
\label{\detokenize{chapter3:code-example}}
\begin{sphinxVerbatim}[commandchars=\\\{\}]
\PYG{k+kn}{from} \PYG{n+nn}{math} \PYG{k+kn}{import} \PYG{n}{sin}\PYG{p}{,} \PYG{n}{pi}
\PYG{k+kn}{import} \PYG{n+nn}{numpy} \PYG{k}{as} \PYG{n+nn}{np}
\PYG{k+kn}{from} \PYG{n+nn}{sympy} \PYG{k+kn}{import} \PYG{n}{Symbol}\PYG{p}{,} \PYG{n}{integrate}
\PYG{c+c1}{\PYGZsh{} function for the trapezoidal rule                                                                                        }
\PYG{k}{def} \PYG{n+nf}{Simpson}\PYG{p}{(}\PYG{n}{a}\PYG{p}{,}\PYG{n}{b}\PYG{p}{,}\PYG{n}{f}\PYG{p}{,}\PYG{n}{n}\PYG{p}{)}\PYG{p}{:}
   \PYG{n}{h} \PYG{o}{=} \PYG{p}{(}\PYG{n}{b}\PYG{o}{\PYGZhy{}}\PYG{n}{a}\PYG{p}{)}\PYG{o}{/}\PYG{n+nb}{float}\PYG{p}{(}\PYG{n}{n}\PYG{p}{)}
   \PYG{n+nb}{sum} \PYG{o}{=} \PYG{n}{f}\PYG{p}{(}\PYG{n}{a}\PYG{p}{)}\PYG{o}{/}\PYG{n+nb}{float}\PYG{p}{(}\PYG{l+m+mi}{2}\PYG{p}{)}\PYG{p}{;}
   \PYG{k}{for} \PYG{n}{i} \PYG{o+ow}{in} \PYG{n+nb}{range}\PYG{p}{(}\PYG{l+m+mi}{1}\PYG{p}{,}\PYG{n}{n}\PYG{p}{)}\PYG{p}{:}
       \PYG{n+nb}{sum} \PYG{o}{=} \PYG{n+nb}{sum} \PYG{o}{+} \PYG{n}{f}\PYG{p}{(}\PYG{n}{a}\PYG{o}{+}\PYG{n}{i}\PYG{o}{*}\PYG{n}{h}\PYG{p}{)}\PYG{o}{*}\PYG{p}{(}\PYG{l+m+mi}{3}\PYG{o}{+}\PYG{p}{(}\PYG{o}{\PYGZhy{}}\PYG{l+m+mi}{1}\PYG{p}{)}\PYG{o}{*}\PYG{o}{*}\PYG{p}{(}\PYG{n}{i}\PYG{o}{+}\PYG{l+m+mi}{1}\PYG{p}{)}\PYG{p}{)}
   \PYG{n+nb}{sum} \PYG{o}{=} \PYG{n+nb}{sum} \PYG{o}{+} \PYG{n}{f}\PYG{p}{(}\PYG{n}{b}\PYG{p}{)}\PYG{o}{/}\PYG{n+nb}{float}\PYG{p}{(}\PYG{l+m+mi}{2}\PYG{p}{)}
   \PYG{k}{return} \PYG{n+nb}{sum}\PYG{o}{*}\PYG{n}{h}\PYG{o}{/}\PYG{l+m+mf}{3.0}
\PYG{c+c1}{\PYGZsh{} function to integrate                                                                                                    }
\PYG{k}{def} \PYG{n+nf}{function}\PYG{p}{(}\PYG{n}{x}\PYG{p}{)}\PYG{p}{:}
    \PYG{k}{return} \PYG{n}{sin}\PYG{p}{(}\PYG{l+m+mi}{2}\PYG{o}{*}\PYG{n}{pi}\PYG{o}{*}\PYG{n}{x}\PYG{p}{)}
\PYG{c+c1}{\PYGZsh{} define integration limits and integration points                                                                         }
\PYG{n}{a} \PYG{o}{=} \PYG{l+m+mf}{0.0}\PYG{p}{;} \PYG{n}{b} \PYG{o}{=} \PYG{l+m+mf}{0.5}\PYG{p}{;}
\PYG{n}{n} \PYG{o}{=} \PYG{l+m+mi}{100}
\PYG{n}{Exact} \PYG{o}{=} \PYG{l+m+mf}{1.}\PYG{o}{/}\PYG{n}{pi}
\PYG{n+nb}{print}\PYG{p}{(}\PYG{l+s+s2}{\PYGZdq{}}\PYG{l+s+s2}{Relative error= }\PYG{l+s+s2}{\PYGZdq{}}\PYG{p}{,} \PYG{n+nb}{abs}\PYG{p}{(} \PYG{p}{(}\PYG{n}{Simpson}\PYG{p}{(}\PYG{n}{a}\PYG{p}{,}\PYG{n}{b}\PYG{p}{,}\PYG{n}{function}\PYG{p}{,}\PYG{n}{n}\PYG{p}{)}\PYG{o}{\PYGZhy{}}\PYG{n}{Exact}\PYG{p}{)}\PYG{o}{/}\PYG{n}{Exact}\PYG{p}{)}\PYG{p}{)}
\end{sphinxVerbatim}

\begin{sphinxVerbatim}[commandchars=\\\{\}]
Relative error=  5.412252157986472e\PYGZhy{}09
\end{sphinxVerbatim}

We see that Simpson’s rule gives a much better estimation of the relative error with the same amount of points as we had for the Rectangle rule and the Trapezoidal rule.




\section{Oscillations}
\label{\detokenize{chapter4:oscillations}}\label{\detokenize{chapter4::doc}}




\sphinxstylestrong{\sphinxhref{http://mhjgit.github.io/info/doc/web/}{Morten Hjorth\sphinxhyphen{}Jensen}}, Department of Physics and Astronomy and National Superconducting Cyclotron Laboratory, Michigan State University, USA and Department of Physics, University of Oslo, Norway









Date: \sphinxstylestrong{Feb 22, 2020}

Copyright 1999\sphinxhyphen{}2020, \sphinxhref{http://mhjgit.github.io/info/doc/web/}{Morten Hjorth\sphinxhyphen{}Jensen}. Released under CC Attribution\sphinxhyphen{}NonCommercial 4.0 license


\subsection{Harmonic Oscillator}
\label{\detokenize{chapter4:harmonic-oscillator}}
The harmonic oscillator is omnipresent in physics. Although you may think
of this as being related to springs, it, or an equivalent
mathematical representation, appears in just about any problem where a
mode is sitting near its potential energy minimum. At that point,
\(\partial_x V(x)=0\), and the first non\sphinxhyphen{}zero term (aside from a
constant) in the potential energy is that of a harmonic oscillator. In
a solid, sound modes (phonons) are built on a picture of coupled
harmonic oscillators, and in relativistic field theory the fundamental
interactions are also built on coupled oscillators positioned
infinitesimally close to one another in space. The phenomena of a
resonance of an oscillator driven at a fixed frequency plays out
repeatedly in atomic, nuclear and high\sphinxhyphen{}energy physics, when quantum
mechanically the evolution of a state oscillates according to
\(e^{-iEt}\) and exciting discrete quantum states has very similar
mathematics as exciting discrete states of an oscillator.

The potential energy for a single particle as a function of its position \(x\) can be written as a Taylor expansion about some point \(x_0\)




\begin{equation*}
\begin{split}
\begin{equation}
V(x)=V(x_0)+(x-x_0)\left.\partial_xV(x)\right|_{x_0}+\frac{1}{2}(x-x_0)^2\left.\partial_x^2V(x)\right|_{x_0}
+\frac{1}{3!}\left.\partial_x^3V(x)\right|_{x_0}+\cdots
\label{_auto1} \tag{1}
\end{equation}
\end{split}
\end{equation*}
If the position \(x_0\) is at the minimum of the resonance, the first two non\sphinxhyphen{}zero terms of the potential are
\begin{equation*}
\begin{split}
\begin{eqnarray}
V(x)&\approx& V(x_0)+\frac{1}{2}(x-x_0)^2\left.\partial_x^2V(x)\right|_{x_0},\\
\nonumber
&=&V(x_0)+\frac{1}{2}k(x-x_0)^2,~~~~k\equiv \left.\partial_x^2V(x)\right|_{x_0},\\
\nonumber
F&=&-\partial_xV(x)=-k(x-x_0).
\end{eqnarray}
\end{split}
\end{equation*}
Put into Newton’s 2nd law (assuming \(x_0=0\)),
\begin{equation*}
\begin{split}
\begin{eqnarray}
m\ddot{x}&=&-kx,\\
x&=&A\cos(\omega_0 t-\phi),~~~\omega_0=\sqrt{k/m}.
\end{eqnarray}
\end{split}
\end{equation*}
Here \(A\) and \(\phi\) are arbitrary. Equivalently, one could have
written this as \(A\cos(\omega_0 t)+B\sin(\omega_0 t)\), or as the real
part of \(Ae^{i\omega_0 t}\). In this last case \(A\) could be an
arbitrary complex constant. Thus, there are 2 arbitrary constants
(either \(A\) and \(B\) or \(A\) and \(\phi\), or the real and imaginary part
of one complex constant. This is the expectation for a second order
differential equation, and also agrees with the physical expectation
that if you know a particle’s initial velocity and position you should
be able to define its future motion, and that those two arbitrary
conditions should translate to two arbitrary constants.

A key feature of harmonic motion is that the system repeats itself
after a time \(T=1/f\), where \(f\) is the frequency, and \(\omega=2\pi f\)
is the angular frequency. The period of the motion is independent of
the amplitude. However, this independence is only exact when one can
neglect higher terms of the potential, \(x^3, x^4\cdots\). Once can
neglect these terms for sufficiently small amplitudes, and for larger
amplitudes the motion is no longer purely sinusoidal, and even though
the motion repeats itself, the time for repeating the motion is no
longer independent of the amplitude.

One can also calculate the velocity and the kinetic energy as a function of time,
\begin{equation*}
\begin{split}
\begin{eqnarray}
\dot{x}&=&-\omega_0A\sin(\omega_0 t-\phi),\\
\nonumber
K&=&\frac{1}{2}m\dot{x}^2=\frac{m\omega_0^2A^2}{2}\sin^2(\omega_0t-\phi),\\
\nonumber
&=&\frac{k}{2}A^2\sin^2(\omega_0t-\phi).
\end{eqnarray}
\end{split}
\end{equation*}
The total energy is then




\begin{equation*}
\begin{split}
\begin{equation}
E=K+V=\frac{1}{2}m\dot{x}^2+\frac{1}{2}kx^2=\frac{1}{2}kA^2.
\label{_auto2} \tag{2}
\end{equation}
\end{split}
\end{equation*}
The total energy then goes as the square of the amplitude.

A pendulum is an example of a harmonic oscillator. By expanding the
kinetic and potential energies for small angles find the frequency for
a pendulum of length \(L\) with all the mass \(m\) centered at the end by
writing the eq.s of motion in the form of a harmonic oscillator.

The potential energy and kinetic energies are (for \(x\) being the displacement)
\begin{equation*}
\begin{split}
\begin{eqnarray*}
V&=&mgL(1-\cos\theta)\approx mgL\frac{x^2}{2L^2},\\
K&=&\frac{1}{2}mL^2\dot{\theta}^2\approx \frac{m}{2}\dot{x}^2.
\end{eqnarray*}
\end{split}
\end{equation*}
For small \(x\) Newton’s 2nd law becomes
\begin{equation*}
\begin{split}
m\ddot{x}=-\frac{mg}{L}x,
\end{split}
\end{equation*}
and the spring constant would appear to be \(k=mg/L\), which makes the
frequency equal to \(\omega_0=\sqrt{g/L}\). Note that the frequency is
independent of the mass.


\subsection{Damped Oscillators}
\label{\detokenize{chapter4:damped-oscillators}}
We consider only the case where the damping force is proportional to
the velocity. This is counter to dragging friction, where the force is
proportional in strength to the normal force and independent of
velocity, and is also inconsistent with wind resistance, where the
magnitude of the drag force is proportional the square of the
velocity. Rolling resistance does seem to be mainly proportional to
the velocity. However, the main motivation for considering damping
forces proportional to the velocity is that the math is more
friendly. This is because the differential equation is linear,
i.e. each term is of order \(x\), \(\dot{x}\), \(\ddot{x}\cdots\), or even
terms with no mention of \(x\), and there are no terms such as \(x^2\) or
\(x\ddot{x}\). The equations of motion for a spring with damping force
\(-b\dot{x}\) are




\begin{equation*}
\begin{split}
\begin{equation}
m\ddot{x}+b\dot{x}+kx=0.
\label{_auto3} \tag{3}
\end{equation}
\end{split}
\end{equation*}
Just to make the solution a bit less messy, we rewrite this equation as




\begin{equation*}
\begin{split}
\begin{equation}
\label{eq:dampeddiffyq} \tag{4}
\ddot{x}+2\beta\dot{x}+\omega_0^2x=0,~~~~\beta\equiv b/2m,~\omega_0\equiv\sqrt{k/m}.
\end{equation}
\end{split}
\end{equation*}
Both \(\beta\) and \(\omega\) have dimensions of inverse time. To find solutions (see appendix C in the text) you must make an educated guess at the form of the solution. To do this, first realize that the solution will need an arbitrary normalization \(A\) because the equation is linear. Secondly, realize that if the form is




\begin{equation*}
\begin{split}
\begin{equation}
x=Ae^{rt}
\label{_auto4} \tag{5}
\end{equation}
\end{split}
\end{equation*}
that each derivative simply brings out an extra power of \(r\). This
means that the \(Ae^{rt}\) factors out and one can simply solve for an
equation for \(r\). Plugging this form into Eq. ({\hyperref[\detokenize{chapter4:eq:dampeddiffyq}]{\emph{4}}}),




\begin{equation*}
\begin{split}
\begin{equation}
r^2+2\beta r+\omega_0^2=0.
\label{_auto5} \tag{6}
\end{equation}
\end{split}
\end{equation*}
Because this is a quadratic equation there will be two solutions,




\begin{equation*}
\begin{split}
\begin{equation}
r=-\beta\pm\sqrt{\beta^2-\omega_0^2}.
\label{_auto6} \tag{7}
\end{equation}
\end{split}
\end{equation*}
We refer to the two solutions as \(r_1\) and \(r_2\) corresponding to the
\(+\) and \(-\) roots. As expected, there should be two arbitrary
constants involved in the solution,




\begin{equation*}
\begin{split}
\begin{equation}
x=A_1e^{r_1t}+A_2e^{r_2t},
\label{_auto7} \tag{8}
\end{equation}
\end{split}
\end{equation*}
where the coefficients \(A_1\) and \(A_2\) are determined by initial
conditions.

The roots listed above, \(\sqrt{\omega_0^2-\beta_0^2}\), will be
imaginary if the damping is small and \(\beta<\omega_0\). In that case,
\(r\) is complex and the factor \(e{rt}\) will have some oscillatory
behavior. If the roots are real, there will only be exponentially
decaying solutions. There are three cases:


\subsubsection{Underdamped: \protect\(\beta<\omega_0\protect\)}
\label{\detokenize{chapter4:underdamped-beta-omega-0}}\begin{equation*}
\begin{split}
\begin{eqnarray}
x&=&A_1e^{-\beta t}e^{i\omega't}+A_2e^{-\beta t}e^{-i\omega't},~~\omega'\equiv\sqrt{\omega_0^2-\beta^2}\\
\nonumber
&=&(A_1+A_2)e^{-\beta t}\cos\omega't+i(A_1-A_2)e^{-\beta t}\sin\omega't.
\end{eqnarray}
\end{split}
\end{equation*}
Here we have made use of the identity
\(e^{i\omega't}=\cos\omega't+i\sin\omega't\). Because the constants are
arbitrary, and because the real and imaginary parts are both solutions
individually, we can simply consider the real part of the solution
alone:




\begin{equation*}
\begin{split}
\begin{eqnarray}
\label{eq:homogsolution} \tag{9}
x&=&B_1e^{-\beta t}\cos\omega't+B_2e^{-\beta t}\sin\omega't,\\
\nonumber 
\omega'&\equiv&\sqrt{\omega_0^2-\beta^2}.
\end{eqnarray}
\end{split}
\end{equation*}

\subsubsection{Critical dampling: \protect\(\beta=\omega_0\protect\)}
\label{\detokenize{chapter4:critical-dampling-beta-omega-0}}
In this case the two terms involving \(r_1\) and \(r_2\) are identical
because \(\omega'=0\). Because we need to arbitrary constants, there
needs to be another solution. This is found by simply guessing, or by
taking the limit of \(\omega'\rightarrow 0\) from the underdamped
solution. The solution is then




\begin{equation*}
\begin{split}
\begin{equation}
\label{eq:criticallydamped} \tag{10}
x=Ae^{-\beta t}+Bte^{-\beta t}.
\end{equation}
\end{split}
\end{equation*}
The critically damped solution is interesting because the solution
approaches zero quickly, but does not oscillate. For a problem with
zero initial velocity, the solution never crosses zero. This is a good
choice for designing shock absorbers or swinging doors.


\subsubsection{Overdamped: \protect\(\beta>\omega_0\protect\)}
\label{\detokenize{chapter4:overdamped-beta-omega-0}}\begin{equation*}
\begin{split}
\begin{eqnarray}
x&=&A_1\exp{-(\beta+\sqrt{\beta^2-\omega_0^2})t}+A_2\exp{-(\beta-\sqrt{\beta^2-\omega_0^2})t}
\end{eqnarray}
\end{split}
\end{equation*}
This solution will also never pass the origin more than once, and then
only if the initial velocity is strong and initially toward zero.

Given \(b\), \(m\) and \(\omega_0\), find \(x(t)\) for a particle whose
initial position is \(x=0\) and has initial velocity \(v_0\) (assuming an
underdamped solution).

The solution is of the form,
\begin{equation*}
\begin{split}
\begin{eqnarray*}
x&=&e^{-\beta t}\left[A_1\cos(\omega' t)+A_2\sin\omega't\right],\\
\dot{x}&=&-\beta x+\omega'e^{-\beta t}\left[-A_1\sin\omega't+A_2\cos\omega't\right].\\
\omega'&\equiv&\sqrt{\omega_0^2-\beta^2},~~~\beta\equiv b/2m.
\end{eqnarray*}
\end{split}
\end{equation*}
From the initial conditions, \(A_1=0\) because \(x(0)=0\) and \(\omega'A_2=v_0\). So
\begin{equation*}
\begin{split}
x=\frac{v_0}{\omega'}e^{-\beta t}\sin\omega't.
\end{split}
\end{equation*}

\subsection{Our Sliding Block Code}
\label{\detokenize{chapter4:our-sliding-block-code}}
Here we study first the case without additional friction term and scale our equation
in terms of a dimensionless time \(\tau\).

Let us remind ourselves about the differential equation we want to solve (the general case with damping due to friction)
\begin{equation*}
\begin{split}
m\frac{d^2x}{dt^2} + b\frac{dx}{dt}+kx(t) =0.
\end{split}
\end{equation*}
We divide by \(m\) and introduce \(\omega_0^2=\sqrt{k/m}\) and obtain
\begin{equation*}
\begin{split}
\frac{d^2x}{dt^2} + \frac{b}{m}\frac{dx}{dt}+\omega_0^2x(t) =0.
\end{split}
\end{equation*}
Thereafter we introduce a dimensionless time \(\tau = t\omega_0\) (check
that the dimensionality is correct) and rewrite our equation as
\begin{equation*}
\begin{split}
\frac{d^2x}{d\tau^2} + \frac{b}{m\omega_0}\frac{dx}{d\tau}+x(\tau) =0,
\end{split}
\end{equation*}
which gives us
\begin{equation*}
\begin{split}
\frac{d^2x}{d\tau^2} + \frac{b}{m\omega_0}\frac{dx}{d\tau}+x(\tau) =0.
\end{split}
\end{equation*}
We then define \(\gamma = b/(2m\omega_0)\) and rewrite our equations as
\begin{equation*}
\begin{split}
\frac{d^2x}{d\tau^2} + 2\gamma\frac{dx}{d\tau}+x(\tau) =0.
\end{split}
\end{equation*}
This is the equation we will code below. The first version employs the Euler\sphinxhyphen{}Cromer method.

\begin{sphinxVerbatim}[commandchars=\\\{\}]
\PYG{o}{\PYGZpc{}}\PYG{k}{matplotlib} inline

\PYG{c+c1}{\PYGZsh{} Common imports}
\PYG{k+kn}{import} \PYG{n+nn}{numpy} \PYG{k}{as} \PYG{n+nn}{np}
\PYG{k+kn}{import} \PYG{n+nn}{pandas} \PYG{k}{as} \PYG{n+nn}{pd}
\PYG{k+kn}{from} \PYG{n+nn}{math} \PYG{k+kn}{import} \PYG{o}{*}
\PYG{k+kn}{import} \PYG{n+nn}{matplotlib}\PYG{n+nn}{.}\PYG{n+nn}{pyplot} \PYG{k}{as} \PYG{n+nn}{plt}
\PYG{k+kn}{import} \PYG{n+nn}{os}

\PYG{c+c1}{\PYGZsh{} Where to save the figures and data files}
\PYG{n}{PROJECT\PYGZus{}ROOT\PYGZus{}DIR} \PYG{o}{=} \PYG{l+s+s2}{\PYGZdq{}}\PYG{l+s+s2}{Results}\PYG{l+s+s2}{\PYGZdq{}}
\PYG{n}{FIGURE\PYGZus{}ID} \PYG{o}{=} \PYG{l+s+s2}{\PYGZdq{}}\PYG{l+s+s2}{Results/FigureFiles}\PYG{l+s+s2}{\PYGZdq{}}
\PYG{n}{DATA\PYGZus{}ID} \PYG{o}{=} \PYG{l+s+s2}{\PYGZdq{}}\PYG{l+s+s2}{DataFiles/}\PYG{l+s+s2}{\PYGZdq{}}

\PYG{k}{if} \PYG{o+ow}{not} \PYG{n}{os}\PYG{o}{.}\PYG{n}{path}\PYG{o}{.}\PYG{n}{exists}\PYG{p}{(}\PYG{n}{PROJECT\PYGZus{}ROOT\PYGZus{}DIR}\PYG{p}{)}\PYG{p}{:}
    \PYG{n}{os}\PYG{o}{.}\PYG{n}{mkdir}\PYG{p}{(}\PYG{n}{PROJECT\PYGZus{}ROOT\PYGZus{}DIR}\PYG{p}{)}

\PYG{k}{if} \PYG{o+ow}{not} \PYG{n}{os}\PYG{o}{.}\PYG{n}{path}\PYG{o}{.}\PYG{n}{exists}\PYG{p}{(}\PYG{n}{FIGURE\PYGZus{}ID}\PYG{p}{)}\PYG{p}{:}
    \PYG{n}{os}\PYG{o}{.}\PYG{n}{makedirs}\PYG{p}{(}\PYG{n}{FIGURE\PYGZus{}ID}\PYG{p}{)}

\PYG{k}{if} \PYG{o+ow}{not} \PYG{n}{os}\PYG{o}{.}\PYG{n}{path}\PYG{o}{.}\PYG{n}{exists}\PYG{p}{(}\PYG{n}{DATA\PYGZus{}ID}\PYG{p}{)}\PYG{p}{:}
    \PYG{n}{os}\PYG{o}{.}\PYG{n}{makedirs}\PYG{p}{(}\PYG{n}{DATA\PYGZus{}ID}\PYG{p}{)}

\PYG{k}{def} \PYG{n+nf}{image\PYGZus{}path}\PYG{p}{(}\PYG{n}{fig\PYGZus{}id}\PYG{p}{)}\PYG{p}{:}
    \PYG{k}{return} \PYG{n}{os}\PYG{o}{.}\PYG{n}{path}\PYG{o}{.}\PYG{n}{join}\PYG{p}{(}\PYG{n}{FIGURE\PYGZus{}ID}\PYG{p}{,} \PYG{n}{fig\PYGZus{}id}\PYG{p}{)}

\PYG{k}{def} \PYG{n+nf}{data\PYGZus{}path}\PYG{p}{(}\PYG{n}{dat\PYGZus{}id}\PYG{p}{)}\PYG{p}{:}
    \PYG{k}{return} \PYG{n}{os}\PYG{o}{.}\PYG{n}{path}\PYG{o}{.}\PYG{n}{join}\PYG{p}{(}\PYG{n}{DATA\PYGZus{}ID}\PYG{p}{,} \PYG{n}{dat\PYGZus{}id}\PYG{p}{)}

\PYG{k}{def} \PYG{n+nf}{save\PYGZus{}fig}\PYG{p}{(}\PYG{n}{fig\PYGZus{}id}\PYG{p}{)}\PYG{p}{:}
    \PYG{n}{plt}\PYG{o}{.}\PYG{n}{savefig}\PYG{p}{(}\PYG{n}{image\PYGZus{}path}\PYG{p}{(}\PYG{n}{fig\PYGZus{}id}\PYG{p}{)} \PYG{o}{+} \PYG{l+s+s2}{\PYGZdq{}}\PYG{l+s+s2}{.png}\PYG{l+s+s2}{\PYGZdq{}}\PYG{p}{,} \PYG{n+nb}{format}\PYG{o}{=}\PYG{l+s+s1}{\PYGZsq{}}\PYG{l+s+s1}{png}\PYG{l+s+s1}{\PYGZsq{}}\PYG{p}{)}


\PYG{k+kn}{from} \PYG{n+nn}{pylab} \PYG{k+kn}{import} \PYG{n}{plt}\PYG{p}{,} \PYG{n}{mpl}
\PYG{n}{plt}\PYG{o}{.}\PYG{n}{style}\PYG{o}{.}\PYG{n}{use}\PYG{p}{(}\PYG{l+s+s1}{\PYGZsq{}}\PYG{l+s+s1}{seaborn}\PYG{l+s+s1}{\PYGZsq{}}\PYG{p}{)}
\PYG{n}{mpl}\PYG{o}{.}\PYG{n}{rcParams}\PYG{p}{[}\PYG{l+s+s1}{\PYGZsq{}}\PYG{l+s+s1}{font.family}\PYG{l+s+s1}{\PYGZsq{}}\PYG{p}{]} \PYG{o}{=} \PYG{l+s+s1}{\PYGZsq{}}\PYG{l+s+s1}{serif}\PYG{l+s+s1}{\PYGZsq{}}

\PYG{n}{DeltaT} \PYG{o}{=} \PYG{l+m+mf}{0.001}
\PYG{c+c1}{\PYGZsh{}set up arrays }
\PYG{n}{tfinal} \PYG{o}{=} \PYG{l+m+mi}{20} \PYG{c+c1}{\PYGZsh{} in years}
\PYG{n}{n} \PYG{o}{=} \PYG{n}{ceil}\PYG{p}{(}\PYG{n}{tfinal}\PYG{o}{/}\PYG{n}{DeltaT}\PYG{p}{)}
\PYG{c+c1}{\PYGZsh{} set up arrays for t, v, and x}
\PYG{n}{t} \PYG{o}{=} \PYG{n}{np}\PYG{o}{.}\PYG{n}{zeros}\PYG{p}{(}\PYG{n}{n}\PYG{p}{)}
\PYG{n}{v} \PYG{o}{=} \PYG{n}{np}\PYG{o}{.}\PYG{n}{zeros}\PYG{p}{(}\PYG{n}{n}\PYG{p}{)}
\PYG{n}{x} \PYG{o}{=} \PYG{n}{np}\PYG{o}{.}\PYG{n}{zeros}\PYG{p}{(}\PYG{n}{n}\PYG{p}{)}
\PYG{c+c1}{\PYGZsh{} Initial conditions as simple one\PYGZhy{}dimensional arrays of time}
\PYG{n}{x0} \PYG{o}{=}  \PYG{l+m+mf}{1.0} 
\PYG{n}{v0} \PYG{o}{=} \PYG{l+m+mf}{0.0}
\PYG{n}{x}\PYG{p}{[}\PYG{l+m+mi}{0}\PYG{p}{]} \PYG{o}{=} \PYG{n}{x0}
\PYG{n}{v}\PYG{p}{[}\PYG{l+m+mi}{0}\PYG{p}{]} \PYG{o}{=} \PYG{n}{v0}
\PYG{n}{gamma} \PYG{o}{=} \PYG{l+m+mf}{0.0}
\PYG{c+c1}{\PYGZsh{} Start integrating using Euler\PYGZhy{}Cromer\PYGZsq{}s method}
\PYG{k}{for} \PYG{n}{i} \PYG{o+ow}{in} \PYG{n+nb}{range}\PYG{p}{(}\PYG{n}{n}\PYG{o}{\PYGZhy{}}\PYG{l+m+mi}{1}\PYG{p}{)}\PYG{p}{:}
    \PYG{c+c1}{\PYGZsh{} Set up the acceleration}
    \PYG{c+c1}{\PYGZsh{} Here you could have defined your own function for this}
    \PYG{n}{a} \PYG{o}{=}  \PYG{o}{\PYGZhy{}}\PYG{l+m+mi}{2}\PYG{o}{*}\PYG{n}{gamma}\PYG{o}{*}\PYG{n}{v}\PYG{p}{[}\PYG{n}{i}\PYG{p}{]}\PYG{o}{\PYGZhy{}}\PYG{n}{x}\PYG{p}{[}\PYG{n}{i}\PYG{p}{]}
    \PYG{c+c1}{\PYGZsh{} update velocity, time and position}
    \PYG{n}{v}\PYG{p}{[}\PYG{n}{i}\PYG{o}{+}\PYG{l+m+mi}{1}\PYG{p}{]} \PYG{o}{=} \PYG{n}{v}\PYG{p}{[}\PYG{n}{i}\PYG{p}{]} \PYG{o}{+} \PYG{n}{DeltaT}\PYG{o}{*}\PYG{n}{a}
    \PYG{n}{x}\PYG{p}{[}\PYG{n}{i}\PYG{o}{+}\PYG{l+m+mi}{1}\PYG{p}{]} \PYG{o}{=} \PYG{n}{x}\PYG{p}{[}\PYG{n}{i}\PYG{p}{]} \PYG{o}{+} \PYG{n}{DeltaT}\PYG{o}{*}\PYG{n}{v}\PYG{p}{[}\PYG{n}{i}\PYG{o}{+}\PYG{l+m+mi}{1}\PYG{p}{]}
    \PYG{n}{t}\PYG{p}{[}\PYG{n}{i}\PYG{o}{+}\PYG{l+m+mi}{1}\PYG{p}{]} \PYG{o}{=} \PYG{n}{t}\PYG{p}{[}\PYG{n}{i}\PYG{p}{]} \PYG{o}{+} \PYG{n}{DeltaT}
\PYG{c+c1}{\PYGZsh{} Plot position as function of time    }
\PYG{n}{fig}\PYG{p}{,} \PYG{n}{ax} \PYG{o}{=} \PYG{n}{plt}\PYG{o}{.}\PYG{n}{subplots}\PYG{p}{(}\PYG{p}{)}
\PYG{c+c1}{\PYGZsh{}ax.set\PYGZus{}xlim(0, tfinal)}
\PYG{n}{ax}\PYG{o}{.}\PYG{n}{set\PYGZus{}ylabel}\PYG{p}{(}\PYG{l+s+s1}{\PYGZsq{}}\PYG{l+s+s1}{x[m]}\PYG{l+s+s1}{\PYGZsq{}}\PYG{p}{)}
\PYG{n}{ax}\PYG{o}{.}\PYG{n}{set\PYGZus{}xlabel}\PYG{p}{(}\PYG{l+s+s1}{\PYGZsq{}}\PYG{l+s+s1}{t[s]}\PYG{l+s+s1}{\PYGZsq{}}\PYG{p}{)}
\PYG{n}{ax}\PYG{o}{.}\PYG{n}{plot}\PYG{p}{(}\PYG{n}{t}\PYG{p}{,} \PYG{n}{x}\PYG{p}{)}
\PYG{n}{fig}\PYG{o}{.}\PYG{n}{tight\PYGZus{}layout}\PYG{p}{(}\PYG{p}{)}
\PYG{n}{save\PYGZus{}fig}\PYG{p}{(}\PYG{l+s+s2}{\PYGZdq{}}\PYG{l+s+s2}{BlockEulerCromer}\PYG{l+s+s2}{\PYGZdq{}}\PYG{p}{)}
\PYG{n}{plt}\PYG{o}{.}\PYG{n}{show}\PYG{p}{(}\PYG{p}{)}
\end{sphinxVerbatim}

\noindent\sphinxincludegraphics{{chapter4_49_0}.png}

When setting up the value of \(\gamma\) we see that for \(\gamma=0\) we get the simple oscillatory motion with no damping.
Choosing \(\gamma < 1\) leads to the classical underdamped case with oscillatory motion, but where the motion comes to an end.

Choosing \(\gamma =1\) leads to what normally is called critical damping and \(\gamma> 1\) leads to critical overdamping.
Try it out and try also to change the initial position and velocity. Setting \(\gamma=1\)
yields a situation, as discussed above, where the solution approaches quickly zero and does not oscillate. With zero initial velocity it will never cross zero.


\subsection{Sinusoidally Driven Oscillators}
\label{\detokenize{chapter4:sinusoidally-driven-oscillators}}
Here, we consider the force




\begin{equation*}
\begin{split}
\begin{equation}
F=-kx-b\dot{x}+F_0\cos\omega t,
\label{_auto8} \tag{11}
\end{equation}
\end{split}
\end{equation*}
which leads to the differential equation




\begin{equation*}
\begin{split}
\begin{equation}
\label{eq:drivenosc} \tag{12}
\ddot{x}+2\beta\dot{x}+\omega_0^2x=(F_0/m)\cos\omega t.
\end{equation}
\end{split}
\end{equation*}
Consider a single solution with no arbitrary constants, which we will
call a \{\textbackslash{}it particular solution\}, \(x_p(t)\). It should be emphasized
that this is \{\textbackslash{}bf A\} particular solution, because there exists an
infinite number of such solutions because the general solution should
have two arbitrary constants. Now consider solutions to the same
equation without the driving term, which include two arbitrary
constants. These are called either \{\textbackslash{}it homogenous solutions\} or \{\textbackslash{}it
complementary solutions\}, and were given in the previous section,
e.g. Eq. ({\hyperref[\detokenize{chapter4:eq:homogsolution}]{\emph{9}}}) for the underdamped case. The
homogenous solution already incorporates the two arbitrary constants,
so any sum of a homogenous solution and a particular solution will
represent the \{\textbackslash{}it general solution\} of the equation. The general
solution incorporates the two arbitrary constants \(A\) and \(B\) to
accommodate the two initial conditions. One could have picked a
different particular solution, i.e. the original particular solution
plus any homogenous solution with the arbitrary constants \(A_p\) and
\(B_p\) chosen at will. When one adds in the homogenous solution, which
has adjustable constants with arbitrary constants \(A'\) and \(B'\), to
the new particular solution, one can get the same general solution by
simply adjusting the new constants such that \(A'+A_p=A\) and
\(B'+B_p=B\). Thus, the choice of \(A_p\) and \(B_p\) are irrelevant, and
when choosing the particular solution it is best to make the simplest
choice possible.

To find a particular solution, one first guesses at the form,




\begin{equation*}
\begin{split}
\begin{equation}
\label{eq:partform} \tag{13}
x_p(t)=D\cos(\omega t-\delta),
\end{equation}
\end{split}
\end{equation*}
and rewrite the differential equation as




\begin{equation*}
\begin{split}
\begin{equation}
D\left\{-\omega^2\cos(\omega t-\delta)-2\beta\omega\sin(\omega t-\delta)+\omega_0^2\cos(\omega t-\delta)\right\}=\frac{F_0}{m}\cos(\omega t).
\label{_auto9} \tag{14}
\end{equation}
\end{split}
\end{equation*}
One can now use angle addition formulas to get
\begin{equation*}
\begin{split}
\begin{eqnarray}
D\left\{(-\omega^2\cos\delta+2\beta\omega\sin\delta+\omega_0^2\cos\delta)\cos(\omega t)\right.&&\\
\nonumber
\left.+(-\omega^2\sin\delta-2\beta\omega\cos\delta+\omega_0^2\sin\delta)\sin(\omega t)\right\}
&=&\frac{F_0}{m}\cos(\omega t).
\end{eqnarray}
\end{split}
\end{equation*}
Both the \(\cos\) and \(\sin\) terms need to equate if the expression is to hold at all times. Thus, this becomes two equations
\begin{equation*}
\begin{split}
\begin{eqnarray}
D\left\{-\omega^2\cos\delta+2\beta\omega\sin\delta+\omega_0^2\cos\delta\right\}&=&\frac{F_0}{m}\\
\nonumber
-\omega^2\sin\delta-2\beta\omega\cos\delta+\omega_0^2\sin\delta&=&0.
\end{eqnarray}
\end{split}
\end{equation*}
After dividing by \(\cos\delta\), the lower expression leads to




\begin{equation*}
\begin{split}
\begin{equation}
\tan\delta=\frac{2\beta\omega}{\omega_0^2-\omega^2}.
\label{_auto10} \tag{15}
\end{equation}
\end{split}
\end{equation*}
Using the identities \(\tan^2+1=\csc^2\) and \(\sin^2+\cos^2=1\), one can also express \(\sin\delta\) and \(\cos\delta\),
\begin{equation*}
\begin{split}
\begin{eqnarray}
\sin\delta&=&\frac{2\beta\omega}{\sqrt{(\omega_0^2-\omega^2)^2+4\omega^2\beta^2}},\\
\nonumber
\cos\delta&=&\frac{(\omega_0^2-\omega^2)}{\sqrt{(\omega_0^2-\omega^2)^2+4\omega^2\beta^2}}
\end{eqnarray}
\end{split}
\end{equation*}
Inserting the expressions for \(\cos\delta\) and \(\sin\delta\) into the expression for \(D\),




\begin{equation*}
\begin{split}
\begin{equation}
\label{eq:Ddrive} \tag{16}
D=\frac{F_0/m}{\sqrt{(\omega_0^2-\omega^2)^2+4\omega^2\beta^2}}.
\end{equation}
\end{split}
\end{equation*}
For a given initial condition, e.g. initial displacement and velocity,
one must add the homogenous solution then solve for the two arbitrary
constants. However, because the homogenous solutions decay with time
as \(e^{-\beta t}\), the particular solution is all that remains at
large times, and is therefore the steady state solution. Because the
arbitrary constants are all in the homogenous solution, all memory of
the initial conditions are lost at large times, \(t>>1/\beta\).

The amplitude of the motion, \(D\), is linearly proportional to the
driving force (\(F_0/m\)), but also depends on the driving frequency
\(\omega\). For small \(\beta\) the maximum will occur at
\(\omega=\omega_0\). This is referred to as a resonance. In the limit
\(\beta\rightarrow 0\) the amplitude at resonance approaches infinity.


\subsection{Alternative Derivation for Driven Oscillators}
\label{\detokenize{chapter4:alternative-derivation-for-driven-oscillators}}
Here, we derive the same expressions as in Equations ({\hyperref[\detokenize{chapter4:eq:partform}]{\emph{13}}}) and ({\hyperref[\detokenize{chapter4:eq:Ddrive}]{\emph{16}}}) but express the driving forces as
\begin{equation*}
\begin{split}
\begin{eqnarray}
F(t)&=&F_0e^{i\omega t},
\end{eqnarray}
\end{split}
\end{equation*}
rather than as \(F_0\cos\omega t\). The real part of \(F\) is the same as before. For the differential equation,




\begin{equation*}
\begin{split}
\begin{eqnarray}
\label{eq:compdrive} \tag{17}
\ddot{x}+2\beta\dot{x}+\omega_0^2x&=&\frac{F_0}{m}e^{i\omega t},
\end{eqnarray}
\end{split}
\end{equation*}
one can treat \(x(t)\) as an imaginary function. Because the operations
\(d^2/dt^2\) and \(d/dt\) are real and thus do not mix the real and
imaginary parts of \(x(t)\), Eq. ({\hyperref[\detokenize{chapter4:eq:compdrive}]{\emph{17}}}) is effectively 2
equations. Because \(e^{\omega t}=\cos\omega t+i\sin\omega t\), the real
part of the solution for \(x(t)\) gives the solution for a driving force
\(F_0\cos\omega t\), and the imaginary part of \(x\) corresponds to the
case where the driving force is \(F_0\sin\omega t\). It is rather easy
to solve for the complex \(x\) in this case, and by taking the real part
of the solution, one finds the answer for the \(\cos\omega t\) driving
force.

We assume a simple form for the particular solution




\begin{equation*}
\begin{split}
\begin{equation}
x_p=De^{i\omega t},
\label{_auto11} \tag{18}
\end{equation}
\end{split}
\end{equation*}
where \(D\) is a complex constant.

From Eq. ({\hyperref[\detokenize{chapter4:eq:compdrive}]{\emph{17}}}) one inserts the form for \(x_p\) above to get
\begin{equation*}
\begin{split}
\begin{eqnarray}
D\left\{-\omega^2+2i\beta\omega+\omega_0^2\right\}e^{i\omega t}=(F_0/m)e^{i\omega t},\\
\nonumber
D=\frac{F_0/m}{(\omega_0^2-\omega^2)+2i\beta\omega}.
\end{eqnarray}
\end{split}
\end{equation*}
The norm and phase for \(D=|D|e^{-i\delta}\) can be read by inspection,




\begin{equation*}
\begin{split}
\begin{equation}
|D|=\frac{F_0/m}{\sqrt{(\omega_0^2-\omega^2)^2+4\beta^2\omega^2}},~~~~\tan\delta=\frac{2\beta\omega}{\omega_0^2-\omega^2}.
\label{_auto12} \tag{19}
\end{equation}
\end{split}
\end{equation*}
This is the same expression for \(\delta\) as before. One then finds \(x_p(t)\),




\begin{equation*}
\begin{split}
\begin{eqnarray}
\label{eq:fastdriven1} \tag{20}
x_p(t)&=&\Re\frac{(F_0/m)e^{i\omega t-i\delta}}{\sqrt{(\omega_0^2-\omega^2)^2+4\beta^2\omega^2}}\\
\nonumber
&=&\frac{(F_0/m)\cos(\omega t-\delta)}{\sqrt{(\omega_0^2-\omega^2)^2+4\beta^2\omega^2}}.
\end{eqnarray}
\end{split}
\end{equation*}
This is the same answer as before.
If one wished to solve for the case where \(F(t)= F_0\sin\omega t\), the imaginary part of the solution would work




\begin{equation*}
\begin{split}
\begin{eqnarray}
\label{eq:fastdriven2} \tag{21}
x_p(t)&=&\Im\frac{(F_0/m)e^{i\omega t-i\delta}}{\sqrt{(\omega_0^2-\omega^2)^2+4\beta^2\omega^2}}\\
\nonumber
&=&\frac{(F_0/m)\sin(\omega t-\delta)}{\sqrt{(\omega_0^2-\omega^2)^2+4\beta^2\omega^2}}.
\end{eqnarray}
\end{split}
\end{equation*}
Consider the damped and driven harmonic oscillator worked out above. Given \(F_0, m,\beta\) and \(\omega_0\), solve for the complete solution \(x(t)\) for the case where \(F=F_0\sin\omega t\) with initial conditions \(x(t=0)=0\) and \(v(t=0)=0\). Assume the underdamped case.

The general solution including the arbitrary constants includes both the homogenous and particular solutions,
\begin{equation*}
\begin{split}
\begin{eqnarray*}
x(t)&=&\frac{F_0}{m}\frac{\sin(\omega t-\delta)}{\sqrt{(\omega_0^2-\omega^2)^2+4\beta^2\omega^2}}
+A\cos\omega't e^{-\beta t}+B\sin\omega't e^{-\beta t}.
\end{eqnarray*}
\end{split}
\end{equation*}
The quantities \(\delta\) and \(\omega'\) are given earlier in the
section, \(\omega'=\sqrt{\omega_0^2-\beta^2},
\delta=\tan^{-1}(2\beta\omega/(\omega_0^2-\omega^2)\). Here, solving
the problem means finding the arbitrary constants \(A\) and
\(B\). Satisfying the initial conditions for the initial position and
velocity:
\begin{equation*}
\begin{split}
\begin{eqnarray*}
x(t=0)=0&=&-\eta\sin\delta+A,\\
v(t=0)=0&=&\omega\eta\cos\delta-\beta A+\omega'B,\\
\eta&\equiv&\frac{F_0}{m}\frac{1}{\sqrt{(\omega_0^2-\omega^2)^2+4\beta^2\omega^2}}.
\end{eqnarray*}
\end{split}
\end{equation*}
The problem is now reduced to 2 equations and 2 unknowns, \(A\) and \(B\). The solution is
\begin{equation*}
\begin{split}
\begin{eqnarray}
A&=& \eta\sin\delta ,~~~B=\frac{-\omega\eta\cos\delta+\beta\eta\sin\delta}{\omega'}.
\end{eqnarray}
\end{split}
\end{equation*}

\subsection{Resonance Widths; the \protect\(Q\protect\) factor}
\label{\detokenize{chapter4:resonance-widths-the-q-factor}}
From the previous two sections, the particular solution for a driving force, \(F=F_0\cos\omega t\), is
\begin{equation*}
\begin{split}
\begin{eqnarray}
x_p(t)&=&\frac{F_0/m}{\sqrt{(\omega_0^2-\omega^2)^2+4\omega^2\beta^2}}\cos(\omega_t-\delta),\\
\nonumber
\delta&=&\tan^{-1}\left(\frac{2\beta\omega}{\omega_0^2-\omega^2}\right).
\end{eqnarray}
\end{split}
\end{equation*}
If one fixes the driving frequency \(\omega\) and adjusts the
fundamental frequency \(\omega_0=\sqrt{k/m}\), the maximum amplitude
occurs when \(\omega_0=\omega\) because that is when the term from the
denominator \((\omega_0^2-\omega^2)^2+4\omega^2\beta^2\) is at a
minimum. This is akin to dialing into a radio station. However, if one
fixes \(\omega_0\) and adjusts the driving frequency one minimize with
respect to \(\omega\), e.g. set




\begin{equation*}
\begin{split}
\begin{equation}
\frac{d}{d\omega}\left[(\omega_0^2-\omega^2)^2+4\omega^2\beta^2\right]=0,
\label{_auto13} \tag{22}
\end{equation}
\end{split}
\end{equation*}
and one finds that the maximum amplitude occurs when
\(\omega=\sqrt{\omega_0^2-2\beta^2}\). If \(\beta\) is small relative to
\(\omega_0\), one can simply state that the maximum amplitude is




\begin{equation*}
\begin{split}
\begin{equation}
x_{\rm max}\approx\frac{F_0}{2m\beta \omega_0}.
\label{_auto14} \tag{23}
\end{equation}
\end{split}
\end{equation*}\begin{equation*}
\begin{split}
\begin{eqnarray}
\frac{4\omega^2\beta^2}{(\omega_0^2-\omega^2)^2+4\omega^2\beta^2}=\frac{1}{2}.
\end{eqnarray}
\end{split}
\end{equation*}
For small damping this occurs when \(\omega=\omega_0\pm \beta\), so the \(FWHM\approx 2\beta\). For the purposes of tuning to a specific frequency, one wants the width to be as small as possible. The ratio of \(\omega_0\) to \(FWHM\) is known as the \{\textbackslash{}it quality\} factor, or \(Q\) factor,




\begin{equation*}
\begin{split}
\begin{equation}
Q\equiv \frac{\omega_0}{2\beta}.
\label{_auto15} \tag{24}
\end{equation}
\end{split}
\end{equation*}

\subsection{Numerical Studies of Driven Oscillations}
\label{\detokenize{chapter4:numerical-studies-of-driven-oscillations}}
Solving the problem of driven oscillations numerically gives us much
more flexibility to study different types of driving forces. We can
reuse our earlier code by simply adding a driving force. If we stay in
the \(x\)\sphinxhyphen{}direction only this can be easily done by adding a term
\(F_{\mathrm{ext}}(x,t)\). Note that we have kept it rather general
here, allowing for both a spatial and a temporal dependence.

Before we dive into the code, we need to briefly remind ourselves
about the equations we started with for the case with damping, namely
\begin{equation*}
\begin{split}
m\frac{d^2x}{dt^2} + b\frac{dx}{dt}+kx(t) =0,
\end{split}
\end{equation*}
with no external force applied to the system.

Let us now for simplicty assume that our external force is given by
\begin{equation*}
\begin{split}
F_{\mathrm{ext}}(t) = F_0\cos{(\omega t)},
\end{split}
\end{equation*}
where \(F_0\) is a constant (what is its dimension?) and \(\omega\) is the frequency of the applied external driving force.
\sphinxstylestrong{Small question:} would you expect energy to be conserved now?

Introducing the external force into our lovely differential equation
and dividing by \(m\) and introducing \(\omega_0^2=\sqrt{k/m}\) we have
\begin{equation*}
\begin{split}
\frac{d^2x}{dt^2} + \frac{b}{m}\frac{dx}{dt}+\omega_0^2x(t) =\frac{F_0}{m}\cos{(\omega t)},
\end{split}
\end{equation*}
Thereafter we introduce a dimensionless time \(\tau = t\omega_0\)
and a dimensionless frequency \(\tilde{\omega}=\omega/\omega_0\). We have then
\begin{equation*}
\begin{split}
\frac{d^2x}{d\tau^2} + \frac{b}{m\omega_0}\frac{dx}{d\tau}+x(\tau) =\frac{F_0}{m\omega_0^2}\cos{(\tilde{\omega}\tau)},
\end{split}
\end{equation*}
Introducing a new amplitude \(\tilde{F} =F_0/(m\omega_0^2)\) (check dimensionality again) we have
\begin{equation*}
\begin{split}
\frac{d^2x}{d\tau^2} + \frac{b}{m\omega_0}\frac{dx}{d\tau}+x(\tau) =\tilde{F}\cos{(\tilde{\omega}\tau)}.
\end{split}
\end{equation*}
Our final step, as we did in the case of various types of damping, is
to define \(\gamma = b/(2m\omega_0)\) and rewrite our equations as
\begin{equation*}
\begin{split}
\frac{d^2x}{d\tau^2} + 2\gamma\frac{dx}{d\tau}+x(\tau) =\tilde{F}\cos{(\tilde{\omega}\tau)}.
\end{split}
\end{equation*}
This is the equation we will code below using the Euler\sphinxhyphen{}Cromer method.

\begin{sphinxVerbatim}[commandchars=\\\{\}]
\PYG{n}{DeltaT} \PYG{o}{=} \PYG{l+m+mf}{0.001}
\PYG{c+c1}{\PYGZsh{}set up arrays }
\PYG{n}{tfinal} \PYG{o}{=} \PYG{l+m+mi}{20} \PYG{c+c1}{\PYGZsh{} in years}
\PYG{n}{n} \PYG{o}{=} \PYG{n}{ceil}\PYG{p}{(}\PYG{n}{tfinal}\PYG{o}{/}\PYG{n}{DeltaT}\PYG{p}{)}
\PYG{c+c1}{\PYGZsh{} set up arrays for t, v, and x}
\PYG{n}{t} \PYG{o}{=} \PYG{n}{np}\PYG{o}{.}\PYG{n}{zeros}\PYG{p}{(}\PYG{n}{n}\PYG{p}{)}
\PYG{n}{v} \PYG{o}{=} \PYG{n}{np}\PYG{o}{.}\PYG{n}{zeros}\PYG{p}{(}\PYG{n}{n}\PYG{p}{)}
\PYG{n}{x} \PYG{o}{=} \PYG{n}{np}\PYG{o}{.}\PYG{n}{zeros}\PYG{p}{(}\PYG{n}{n}\PYG{p}{)}
\PYG{c+c1}{\PYGZsh{} Initial conditions as one\PYGZhy{}dimensional arrays of time}
\PYG{n}{x0} \PYG{o}{=} \PYG{n}{sqrt}\PYG{p}{(}\PYG{l+m+mf}{1.}\PYG{o}{/}\PYG{l+m+mf}{3.}\PYG{p}{)}
\PYG{n}{v0} \PYG{o}{=} \PYG{l+m+mf}{0.0}
\PYG{n}{x}\PYG{p}{[}\PYG{l+m+mi}{0}\PYG{p}{]} \PYG{o}{=} \PYG{n}{x0}
\PYG{n}{v}\PYG{p}{[}\PYG{l+m+mi}{0}\PYG{p}{]} \PYG{o}{=} \PYG{n}{v0}
\PYG{n}{gamma} \PYG{o}{=} \PYG{l+m+mf}{0.0}
\PYG{n}{Omegatilde} \PYG{o}{=} \PYG{l+m+mf}{8.}\PYG{o}{/}\PYG{n}{sqrt}\PYG{p}{(}\PYG{l+m+mf}{2.}\PYG{p}{)}
\PYG{n}{Ftilde} \PYG{o}{=} \PYG{l+m+mf}{1.75}
\PYG{c+c1}{\PYGZsh{} Start integrating using Euler\PYGZhy{}Cromer\PYGZsq{}s method}
\PYG{k}{for} \PYG{n}{i} \PYG{o+ow}{in} \PYG{n+nb}{range}\PYG{p}{(}\PYG{n}{n}\PYG{o}{\PYGZhy{}}\PYG{l+m+mi}{1}\PYG{p}{)}\PYG{p}{:}
    \PYG{c+c1}{\PYGZsh{} Set up the acceleration}
    \PYG{c+c1}{\PYGZsh{} Here you could have defined your own function for this}
    \PYG{n}{a} \PYG{o}{=}  \PYG{o}{\PYGZhy{}}\PYG{l+m+mi}{2}\PYG{o}{*}\PYG{n}{gamma}\PYG{o}{*}\PYG{n}{v}\PYG{p}{[}\PYG{n}{i}\PYG{p}{]}\PYG{o}{\PYGZhy{}}\PYG{n}{x}\PYG{p}{[}\PYG{n}{i}\PYG{p}{]}\PYG{o}{+}\PYG{n}{Ftilde}\PYG{o}{*}\PYG{n}{cos}\PYG{p}{(}\PYG{n}{t}\PYG{p}{[}\PYG{n}{i}\PYG{p}{]}\PYG{o}{*}\PYG{n}{Omegatilde}\PYG{p}{)}
    \PYG{c+c1}{\PYGZsh{} update velocity, time and position}
    \PYG{n}{v}\PYG{p}{[}\PYG{n}{i}\PYG{o}{+}\PYG{l+m+mi}{1}\PYG{p}{]} \PYG{o}{=} \PYG{n}{v}\PYG{p}{[}\PYG{n}{i}\PYG{p}{]} \PYG{o}{+} \PYG{n}{DeltaT}\PYG{o}{*}\PYG{n}{a}
    \PYG{n}{x}\PYG{p}{[}\PYG{n}{i}\PYG{o}{+}\PYG{l+m+mi}{1}\PYG{p}{]} \PYG{o}{=} \PYG{n}{x}\PYG{p}{[}\PYG{n}{i}\PYG{p}{]} \PYG{o}{+} \PYG{n}{DeltaT}\PYG{o}{*}\PYG{n}{v}\PYG{p}{[}\PYG{n}{i}\PYG{o}{+}\PYG{l+m+mi}{1}\PYG{p}{]}
    \PYG{n}{t}\PYG{p}{[}\PYG{n}{i}\PYG{o}{+}\PYG{l+m+mi}{1}\PYG{p}{]} \PYG{o}{=} \PYG{n}{t}\PYG{p}{[}\PYG{n}{i}\PYG{p}{]} \PYG{o}{+} \PYG{n}{DeltaT}
\PYG{c+c1}{\PYGZsh{} Plot position as function of time    }
\PYG{n}{fig}\PYG{p}{,} \PYG{n}{ax} \PYG{o}{=} \PYG{n}{plt}\PYG{o}{.}\PYG{n}{subplots}\PYG{p}{(}\PYG{p}{)}
\PYG{n}{ax}\PYG{o}{.}\PYG{n}{set\PYGZus{}ylabel}\PYG{p}{(}\PYG{l+s+s1}{\PYGZsq{}}\PYG{l+s+s1}{x[m]}\PYG{l+s+s1}{\PYGZsq{}}\PYG{p}{)}
\PYG{n}{ax}\PYG{o}{.}\PYG{n}{set\PYGZus{}xlabel}\PYG{p}{(}\PYG{l+s+s1}{\PYGZsq{}}\PYG{l+s+s1}{t[s]}\PYG{l+s+s1}{\PYGZsq{}}\PYG{p}{)}
\PYG{n}{ax}\PYG{o}{.}\PYG{n}{plot}\PYG{p}{(}\PYG{n}{t}\PYG{p}{,} \PYG{n}{x}\PYG{p}{)}
\PYG{n}{fig}\PYG{o}{.}\PYG{n}{tight\PYGZus{}layout}\PYG{p}{(}\PYG{p}{)}
\PYG{n}{save\PYGZus{}fig}\PYG{p}{(}\PYG{l+s+s2}{\PYGZdq{}}\PYG{l+s+s2}{ForcedBlockEulerCromer}\PYG{l+s+s2}{\PYGZdq{}}\PYG{p}{)}
\PYG{n}{plt}\PYG{o}{.}\PYG{n}{show}\PYG{p}{(}\PYG{p}{)}
\end{sphinxVerbatim}

\noindent\sphinxincludegraphics{{chapter4_110_0}.png}

In the above example we have focused on the Euler\sphinxhyphen{}Cromer method. This
method has a local truncation error which is proportional to \(\Delta t^2\)
and thereby a global error which is proportional to \(\Delta t\).
We can improve this by using the Runge\sphinxhyphen{}Kutta family of
methods. The widely popular Runge\sphinxhyphen{}Kutta to fourth order or just \sphinxstylestrong{RK4}
has indeed a much better truncation error. The RK4 method has a global
error which is proportional to \(\Delta t\).

Let us revisit this method and see how we can implement it for the above example.


\subsection{Differential Equations, Runge\sphinxhyphen{}Kutta methods}
\label{\detokenize{chapter4:differential-equations-runge-kutta-methods}}
Runge\sphinxhyphen{}Kutta (RK) methods are based on Taylor expansion formulae, but yield
in general better algorithms for solutions of an ordinary differential equation.
The basic philosophy is that it provides an intermediate step in the computation of \(y_{i+1}\).

To see this, consider first the following definitions




\begin{equation*}
\begin{split}
\begin{equation}
\frac{dy}{dt}=f(t,y),  
\label{_auto16} \tag{25}
\end{equation}
\end{split}
\end{equation*}
and




\begin{equation*}
\begin{split}
\begin{equation}
y(t)=\int f(t,y) dt,  
\label{_auto17} \tag{26}
\end{equation}
\end{split}
\end{equation*}
and




\begin{equation*}
\begin{split}
\begin{equation}
y_{i+1}=y_i+ \int_{t_i}^{t_{i+1}} f(t,y) dt.
\label{_auto18} \tag{27}
\end{equation}
\end{split}
\end{equation*}
To demonstrate the philosophy behind RK methods, let us consider
the second\sphinxhyphen{}order RK method, RK2.
The first approximation consists in Taylor expanding \(f(t,y)\)
around the center of the integration interval \(t_i\) to \(t_{i+1}\),
that is, at \(t_i+h/2\), \(h\) being the step.
Using the midpoint formula for an integral,
defining \(y(t_i+h/2) = y_{i+1/2}\) and\(t_i+h/2 = t_{i+1/2}\), we obtain




\begin{equation*}
\begin{split}
\begin{equation}
\int_{t_i}^{t_{i+1}} f(t,y) dt \approx hf(t_{i+1/2},y_{i+1/2}) +O(h^3).
\label{_auto19} \tag{28}
\end{equation}
\end{split}
\end{equation*}
This means in turn that we have




\begin{equation*}
\begin{split}
\begin{equation}
y_{i+1}=y_i + hf(t_{i+1/2},y_{i+1/2}) +O(h^3).
\label{_auto20} \tag{29}
\end{equation}
\end{split}
\end{equation*}
However, we do not know the value of   \(y_{i+1/2}\). Here comes thus the next approximation, namely, we use Euler’s
method to approximate \(y_{i+1/2}\). We have then




\begin{equation*}
\begin{split}
\begin{equation}
y_{(i+1/2)}=y_i + \frac{h}{2}\frac{dy}{dt}=y(t_i) + \frac{h}{2}f(t_i,y_i).
\label{_auto21} \tag{30}
\end{equation}
\end{split}
\end{equation*}
This means that we can define the following algorithm for
the second\sphinxhyphen{}order Runge\sphinxhyphen{}Kutta method, RK2.

6
0

\textless{}
\textless{}
\textless{}
!
!
M
A
T
H
\_
B
L
O
C
K




\begin{equation*}
\begin{split}
\begin{equation}
k_2=hf(t_{i+1/2},y_i+k_1/2),
\label{_auto23} \tag{32}
\end{equation}
\end{split}
\end{equation*}
with the final value




\begin{equation*}
\begin{split}
\begin{equation} 
y_{i+i}\approx y_i + k_2 +O(h^3). 
\label{_auto24} \tag{33}
\end{equation}
\end{split}
\end{equation*}
The difference between the previous one\sphinxhyphen{}step methods
is that we now need an intermediate step in our evaluation,
namely \(t_i+h/2 = t_{(i+1/2)}\) where we evaluate the derivative \(f\).
This involves more operations, but the gain is a better stability
in the solution.

The fourth\sphinxhyphen{}order Runge\sphinxhyphen{}Kutta, RK4, has the following algorithm

6
3

\textless{}
\textless{}
\textless{}
!
!
M
A
T
H
\_
B
L
O
C
K
\begin{equation*}
\begin{split}
k_3=hf(t_i+h/2,y_i+k_2/2)\hspace{0.5cm}   k_4=hf(t_i+h,y_i+k_3)
\end{split}
\end{equation*}
with the final result
\begin{equation*}
\begin{split}
y_{i+1}=y_i +\frac{1}{6}\left( k_1 +2k_2+2k_3+k_4\right).
\end{split}
\end{equation*}
Thus, the algorithm consists in first calculating \(k_1\)
with \(t_i\), \(y_1\) and \(f\) as inputs. Thereafter, we increase the step
size by \(h/2\) and calculate \(k_2\), then \(k_3\) and finally \(k_4\). The global error goes as \(O(h^4)\).

However, at this stage, if we keep adding different methods in our
main program, the code will quickly become messy and ugly. Before we
proceed thus, we will now introduce functions that enbody the various
methods for solving differential equations. This means that we can
separate out these methods in own functions and files (and later as classes and more
generic functions) and simply call them when needed. Similarly, we
could easily encapsulate various forces or other quantities of
interest in terms of functions. To see this, let us bring up the code
we developed above for the simple sliding block, but now only with the simple forward Euler method. We introduce
two functions, one for the simple Euler method and one for the
force.

Note that here the forward Euler method does not know the specific force function to be called.
It receives just an input the name. We can easily change the force by adding another function.

\begin{sphinxVerbatim}[commandchars=\\\{\}]
\PYG{k}{def} \PYG{n+nf}{ForwardEuler}\PYG{p}{(}\PYG{n}{v}\PYG{p}{,}\PYG{n}{x}\PYG{p}{,}\PYG{n}{t}\PYG{p}{,}\PYG{n}{n}\PYG{p}{,}\PYG{n}{Force}\PYG{p}{)}\PYG{p}{:}
    \PYG{k}{for} \PYG{n}{i} \PYG{o+ow}{in} \PYG{n+nb}{range}\PYG{p}{(}\PYG{n}{n}\PYG{o}{\PYGZhy{}}\PYG{l+m+mi}{1}\PYG{p}{)}\PYG{p}{:}
        \PYG{n}{v}\PYG{p}{[}\PYG{n}{i}\PYG{o}{+}\PYG{l+m+mi}{1}\PYG{p}{]} \PYG{o}{=} \PYG{n}{v}\PYG{p}{[}\PYG{n}{i}\PYG{p}{]} \PYG{o}{+} \PYG{n}{DeltaT}\PYG{o}{*}\PYG{n}{Force}\PYG{p}{(}\PYG{n}{v}\PYG{p}{[}\PYG{n}{i}\PYG{p}{]}\PYG{p}{,}\PYG{n}{x}\PYG{p}{[}\PYG{n}{i}\PYG{p}{]}\PYG{p}{,}\PYG{n}{t}\PYG{p}{[}\PYG{n}{i}\PYG{p}{]}\PYG{p}{)}
        \PYG{n}{x}\PYG{p}{[}\PYG{n}{i}\PYG{o}{+}\PYG{l+m+mi}{1}\PYG{p}{]} \PYG{o}{=} \PYG{n}{x}\PYG{p}{[}\PYG{n}{i}\PYG{p}{]} \PYG{o}{+} \PYG{n}{DeltaT}\PYG{o}{*}\PYG{n}{v}\PYG{p}{[}\PYG{n}{i}\PYG{p}{]}
        \PYG{n}{t}\PYG{p}{[}\PYG{n}{i}\PYG{o}{+}\PYG{l+m+mi}{1}\PYG{p}{]} \PYG{o}{=} \PYG{n}{t}\PYG{p}{[}\PYG{n}{i}\PYG{p}{]} \PYG{o}{+} \PYG{n}{DeltaT}
\end{sphinxVerbatim}

\begin{sphinxVerbatim}[commandchars=\\\{\}]
\PYG{k}{def} \PYG{n+nf}{SpringForce}\PYG{p}{(}\PYG{n}{v}\PYG{p}{,}\PYG{n}{x}\PYG{p}{,}\PYG{n}{t}\PYG{p}{)}\PYG{p}{:}
\PYG{c+c1}{\PYGZsh{}   note here that we have divided by mass and we return the acceleration}
    \PYG{k}{return}  \PYG{o}{\PYGZhy{}}\PYG{l+m+mi}{2}\PYG{o}{*}\PYG{n}{gamma}\PYG{o}{*}\PYG{n}{v}\PYG{o}{\PYGZhy{}}\PYG{n}{x}\PYG{o}{+}\PYG{n}{Ftilde}\PYG{o}{*}\PYG{n}{cos}\PYG{p}{(}\PYG{n}{t}\PYG{o}{*}\PYG{n}{Omegatilde}\PYG{p}{)}
\end{sphinxVerbatim}

It is easy to add a new method like the Euler\sphinxhyphen{}Cromer

\begin{sphinxVerbatim}[commandchars=\\\{\}]
\PYG{k}{def} \PYG{n+nf}{ForwardEulerCromer}\PYG{p}{(}\PYG{n}{v}\PYG{p}{,}\PYG{n}{x}\PYG{p}{,}\PYG{n}{t}\PYG{p}{,}\PYG{n}{n}\PYG{p}{,}\PYG{n}{Force}\PYG{p}{)}\PYG{p}{:}
    \PYG{k}{for} \PYG{n}{i} \PYG{o+ow}{in} \PYG{n+nb}{range}\PYG{p}{(}\PYG{n}{n}\PYG{o}{\PYGZhy{}}\PYG{l+m+mi}{1}\PYG{p}{)}\PYG{p}{:}
        \PYG{n}{a} \PYG{o}{=} \PYG{n}{Force}\PYG{p}{(}\PYG{n}{v}\PYG{p}{[}\PYG{n}{i}\PYG{p}{]}\PYG{p}{,}\PYG{n}{x}\PYG{p}{[}\PYG{n}{i}\PYG{p}{]}\PYG{p}{,}\PYG{n}{t}\PYG{p}{[}\PYG{n}{i}\PYG{p}{]}\PYG{p}{)}
        \PYG{n}{v}\PYG{p}{[}\PYG{n}{i}\PYG{o}{+}\PYG{l+m+mi}{1}\PYG{p}{]} \PYG{o}{=} \PYG{n}{v}\PYG{p}{[}\PYG{n}{i}\PYG{p}{]} \PYG{o}{+} \PYG{n}{DeltaT}\PYG{o}{*}\PYG{n}{a}
        \PYG{n}{x}\PYG{p}{[}\PYG{n}{i}\PYG{o}{+}\PYG{l+m+mi}{1}\PYG{p}{]} \PYG{o}{=} \PYG{n}{x}\PYG{p}{[}\PYG{n}{i}\PYG{p}{]} \PYG{o}{+} \PYG{n}{DeltaT}\PYG{o}{*}\PYG{n}{v}\PYG{p}{[}\PYG{n}{i}\PYG{o}{+}\PYG{l+m+mi}{1}\PYG{p}{]}
        \PYG{n}{t}\PYG{p}{[}\PYG{n}{i}\PYG{o}{+}\PYG{l+m+mi}{1}\PYG{p}{]} \PYG{o}{=} \PYG{n}{t}\PYG{p}{[}\PYG{n}{i}\PYG{p}{]} \PYG{o}{+} \PYG{n}{DeltaT}
\end{sphinxVerbatim}

and the Velocity Verlet method (be careful with time\sphinxhyphen{}dependence here, it is not an ideal method for non\sphinxhyphen{}conservative forces))

\begin{sphinxVerbatim}[commandchars=\\\{\}]
\PYG{k}{def} \PYG{n+nf}{VelocityVerlet}\PYG{p}{(}\PYG{n}{v}\PYG{p}{,}\PYG{n}{x}\PYG{p}{,}\PYG{n}{t}\PYG{p}{,}\PYG{n}{n}\PYG{p}{,}\PYG{n}{Force}\PYG{p}{)}\PYG{p}{:}
    \PYG{k}{for} \PYG{n}{i} \PYG{o+ow}{in} \PYG{n+nb}{range}\PYG{p}{(}\PYG{n}{n}\PYG{o}{\PYGZhy{}}\PYG{l+m+mi}{1}\PYG{p}{)}\PYG{p}{:}
        \PYG{n}{a} \PYG{o}{=} \PYG{n}{Force}\PYG{p}{(}\PYG{n}{v}\PYG{p}{[}\PYG{n}{i}\PYG{p}{]}\PYG{p}{,}\PYG{n}{x}\PYG{p}{[}\PYG{n}{i}\PYG{p}{]}\PYG{p}{,}\PYG{n}{t}\PYG{p}{[}\PYG{n}{i}\PYG{p}{]}\PYG{p}{)}
        \PYG{n}{x}\PYG{p}{[}\PYG{n}{i}\PYG{o}{+}\PYG{l+m+mi}{1}\PYG{p}{]} \PYG{o}{=} \PYG{n}{x}\PYG{p}{[}\PYG{n}{i}\PYG{p}{]} \PYG{o}{+} \PYG{n}{DeltaT}\PYG{o}{*}\PYG{n}{v}\PYG{p}{[}\PYG{n}{i}\PYG{p}{]}\PYG{o}{+}\PYG{l+m+mf}{0.5}\PYG{o}{*}\PYG{n}{a}
        \PYG{n}{anew} \PYG{o}{=} \PYG{n}{Force}\PYG{p}{(}\PYG{n}{v}\PYG{p}{[}\PYG{n}{i}\PYG{p}{]}\PYG{p}{,}\PYG{n}{x}\PYG{p}{[}\PYG{n}{i}\PYG{o}{+}\PYG{l+m+mi}{1}\PYG{p}{]}\PYG{p}{,}\PYG{n}{t}\PYG{p}{[}\PYG{n}{i}\PYG{o}{+}\PYG{l+m+mi}{1}\PYG{p}{]}\PYG{p}{)}
        \PYG{n}{v}\PYG{p}{[}\PYG{n}{i}\PYG{o}{+}\PYG{l+m+mi}{1}\PYG{p}{]} \PYG{o}{=} \PYG{n}{v}\PYG{p}{[}\PYG{n}{i}\PYG{p}{]} \PYG{o}{+} \PYG{l+m+mf}{0.5}\PYG{o}{*}\PYG{n}{DeltaT}\PYG{o}{*}\PYG{p}{(}\PYG{n}{a}\PYG{o}{+}\PYG{n}{anew}\PYG{p}{)}
        \PYG{n}{t}\PYG{p}{[}\PYG{n}{i}\PYG{o}{+}\PYG{l+m+mi}{1}\PYG{p}{]} \PYG{o}{=} \PYG{n}{t}\PYG{p}{[}\PYG{n}{i}\PYG{p}{]} \PYG{o}{+} \PYG{n}{DeltaT}
\end{sphinxVerbatim}

Finally, we can now add the Runge\sphinxhyphen{}Kutta2 method via a new function

\begin{sphinxVerbatim}[commandchars=\\\{\}]
\PYG{k}{def} \PYG{n+nf}{RK2}\PYG{p}{(}\PYG{n}{v}\PYG{p}{,}\PYG{n}{x}\PYG{p}{,}\PYG{n}{t}\PYG{p}{,}\PYG{n}{n}\PYG{p}{,}\PYG{n}{Force}\PYG{p}{)}\PYG{p}{:}
    \PYG{k}{for} \PYG{n}{i} \PYG{o+ow}{in} \PYG{n+nb}{range}\PYG{p}{(}\PYG{n}{n}\PYG{o}{\PYGZhy{}}\PYG{l+m+mi}{1}\PYG{p}{)}\PYG{p}{:}
\PYG{c+c1}{\PYGZsh{} Setting up k1}
        \PYG{n}{k1x} \PYG{o}{=} \PYG{n}{DeltaT}\PYG{o}{*}\PYG{n}{v}\PYG{p}{[}\PYG{n}{i}\PYG{p}{]}
        \PYG{n}{k1v} \PYG{o}{=} \PYG{n}{DeltaT}\PYG{o}{*}\PYG{n}{Force}\PYG{p}{(}\PYG{n}{v}\PYG{p}{[}\PYG{n}{i}\PYG{p}{]}\PYG{p}{,}\PYG{n}{x}\PYG{p}{[}\PYG{n}{i}\PYG{p}{]}\PYG{p}{,}\PYG{n}{t}\PYG{p}{[}\PYG{n}{i}\PYG{p}{]}\PYG{p}{)}
\PYG{c+c1}{\PYGZsh{} Setting up k2}
        \PYG{n}{vv} \PYG{o}{=} \PYG{n}{v}\PYG{p}{[}\PYG{n}{i}\PYG{p}{]}\PYG{o}{+}\PYG{n}{k1v}\PYG{o}{*}\PYG{l+m+mf}{0.5}
        \PYG{n}{xx} \PYG{o}{=} \PYG{n}{x}\PYG{p}{[}\PYG{n}{i}\PYG{p}{]}\PYG{o}{+}\PYG{n}{k1x}\PYG{o}{*}\PYG{l+m+mf}{0.5}
        \PYG{n}{k2x} \PYG{o}{=} \PYG{n}{DeltaT}\PYG{o}{*}\PYG{n}{vv}
        \PYG{n}{k2v} \PYG{o}{=} \PYG{n}{DeltaT}\PYG{o}{*}\PYG{n}{Force}\PYG{p}{(}\PYG{n}{vv}\PYG{p}{,}\PYG{n}{xx}\PYG{p}{,}\PYG{n}{t}\PYG{p}{[}\PYG{n}{i}\PYG{p}{]}\PYG{o}{+}\PYG{n}{DeltaT}\PYG{o}{*}\PYG{l+m+mf}{0.5}\PYG{p}{)}
\PYG{c+c1}{\PYGZsh{} Final result}
        \PYG{n}{x}\PYG{p}{[}\PYG{n}{i}\PYG{o}{+}\PYG{l+m+mi}{1}\PYG{p}{]} \PYG{o}{=} \PYG{n}{x}\PYG{p}{[}\PYG{n}{i}\PYG{p}{]}\PYG{o}{+}\PYG{n}{k2x}
        \PYG{n}{v}\PYG{p}{[}\PYG{n}{i}\PYG{o}{+}\PYG{l+m+mi}{1}\PYG{p}{]} \PYG{o}{=} \PYG{n}{v}\PYG{p}{[}\PYG{n}{i}\PYG{p}{]}\PYG{o}{+}\PYG{n}{k2v}
	\PYG{n}{t}\PYG{p}{[}\PYG{n}{i}\PYG{o}{+}\PYG{l+m+mi}{1}\PYG{p}{]} \PYG{o}{=} \PYG{n}{t}\PYG{p}{[}\PYG{n}{i}\PYG{p}{]}\PYG{o}{+}\PYG{n}{DeltaT}
\end{sphinxVerbatim}

\begin{sphinxVerbatim}[commandchars=\\\{\}]
\PYG{g+gt}{  File}\PYG{n+nn}{ \PYGZdq{}\PYGZlt{}ipython\PYGZhy{}input\PYGZhy{}7\PYGZhy{}ffedbda27704\PYGZgt{}\PYGZdq{}}\PYG{g+gt}{, line }\PYG{l+m+mi}{14}
    \PYG{n}{t}\PYG{p}{[}\PYG{n}{i}\PYG{o}{+}\PYG{l+m+mi}{1}\PYG{p}{]} \PYG{o}{=} \PYG{n}{t}\PYG{p}{[}\PYG{n}{i}\PYG{p}{]}\PYG{o}{+}\PYG{n}{DeltaT}
                        \PYG{o}{\PYGZca{}}
\PYG{n+ne}{TabError}: inconsistent use of tabs and spaces in indentation
\end{sphinxVerbatim}

Finally, we can now add the Runge\sphinxhyphen{}Kutta2 method via a new function

\begin{sphinxVerbatim}[commandchars=\\\{\}]
\PYG{k}{def} \PYG{n+nf}{RK4}\PYG{p}{(}\PYG{n}{v}\PYG{p}{,}\PYG{n}{x}\PYG{p}{,}\PYG{n}{t}\PYG{p}{,}\PYG{n}{n}\PYG{p}{,}\PYG{n}{Force}\PYG{p}{)}\PYG{p}{:}
    \PYG{k}{for} \PYG{n}{i} \PYG{o+ow}{in} \PYG{n+nb}{range}\PYG{p}{(}\PYG{n}{n}\PYG{o}{\PYGZhy{}}\PYG{l+m+mi}{1}\PYG{p}{)}\PYG{p}{:}
\PYG{c+c1}{\PYGZsh{} Setting up k1}
        \PYG{n}{k1x} \PYG{o}{=} \PYG{n}{DeltaT}\PYG{o}{*}\PYG{n}{v}\PYG{p}{[}\PYG{n}{i}\PYG{p}{]}
        \PYG{n}{k1v} \PYG{o}{=} \PYG{n}{DeltaT}\PYG{o}{*}\PYG{n}{Force}\PYG{p}{(}\PYG{n}{v}\PYG{p}{[}\PYG{n}{i}\PYG{p}{]}\PYG{p}{,}\PYG{n}{x}\PYG{p}{[}\PYG{n}{i}\PYG{p}{]}\PYG{p}{,}\PYG{n}{t}\PYG{p}{[}\PYG{n}{i}\PYG{p}{]}\PYG{p}{)}
\PYG{c+c1}{\PYGZsh{} Setting up k2}
        \PYG{n}{vv} \PYG{o}{=} \PYG{n}{v}\PYG{p}{[}\PYG{n}{i}\PYG{p}{]}\PYG{o}{+}\PYG{n}{k1v}\PYG{o}{*}\PYG{l+m+mf}{0.5}
        \PYG{n}{xx} \PYG{o}{=} \PYG{n}{x}\PYG{p}{[}\PYG{n}{i}\PYG{p}{]}\PYG{o}{+}\PYG{n}{k1x}\PYG{o}{*}\PYG{l+m+mf}{0.5}
        \PYG{n}{k2x} \PYG{o}{=} \PYG{n}{DeltaT}\PYG{o}{*}\PYG{n}{vv}
        \PYG{n}{k2v} \PYG{o}{=} \PYG{n}{DeltaT}\PYG{o}{*}\PYG{n}{Force}\PYG{p}{(}\PYG{n}{vv}\PYG{p}{,}\PYG{n}{xx}\PYG{p}{,}\PYG{n}{t}\PYG{p}{[}\PYG{n}{i}\PYG{p}{]}\PYG{o}{+}\PYG{n}{DeltaT}\PYG{o}{*}\PYG{l+m+mf}{0.5}\PYG{p}{)}
\PYG{c+c1}{\PYGZsh{} Setting up k3}
        \PYG{n}{vv} \PYG{o}{=} \PYG{n}{v}\PYG{p}{[}\PYG{n}{i}\PYG{p}{]}\PYG{o}{+}\PYG{n}{k2v}\PYG{o}{*}\PYG{l+m+mf}{0.5}
        \PYG{n}{xx} \PYG{o}{=} \PYG{n}{x}\PYG{p}{[}\PYG{n}{i}\PYG{p}{]}\PYG{o}{+}\PYG{n}{k2x}\PYG{o}{*}\PYG{l+m+mf}{0.5}
        \PYG{n}{k3x} \PYG{o}{=} \PYG{n}{DeltaT}\PYG{o}{*}\PYG{n}{vv}
        \PYG{n}{k3v} \PYG{o}{=} \PYG{n}{DeltaT}\PYG{o}{*}\PYG{n}{Force}\PYG{p}{(}\PYG{n}{vv}\PYG{p}{,}\PYG{n}{xx}\PYG{p}{,}\PYG{n}{t}\PYG{p}{[}\PYG{n}{i}\PYG{p}{]}\PYG{o}{+}\PYG{n}{DeltaT}\PYG{o}{*}\PYG{l+m+mf}{0.5}\PYG{p}{)}
\PYG{c+c1}{\PYGZsh{} Setting up k4}
        \PYG{n}{vv} \PYG{o}{=} \PYG{n}{v}\PYG{p}{[}\PYG{n}{i}\PYG{p}{]}\PYG{o}{+}\PYG{n}{k3v}
        \PYG{n}{xx} \PYG{o}{=} \PYG{n}{x}\PYG{p}{[}\PYG{n}{i}\PYG{p}{]}\PYG{o}{+}\PYG{n}{k3x}
        \PYG{n}{k4x} \PYG{o}{=} \PYG{n}{DeltaT}\PYG{o}{*}\PYG{n}{vv}
        \PYG{n}{k4v} \PYG{o}{=} \PYG{n}{DeltaT}\PYG{o}{*}\PYG{n}{Force}\PYG{p}{(}\PYG{n}{vv}\PYG{p}{,}\PYG{n}{xx}\PYG{p}{,}\PYG{n}{t}\PYG{p}{[}\PYG{n}{i}\PYG{p}{]}\PYG{o}{+}\PYG{n}{DeltaT}\PYG{p}{)}
\PYG{c+c1}{\PYGZsh{} Final result}
        \PYG{n}{x}\PYG{p}{[}\PYG{n}{i}\PYG{o}{+}\PYG{l+m+mi}{1}\PYG{p}{]} \PYG{o}{=} \PYG{n}{x}\PYG{p}{[}\PYG{n}{i}\PYG{p}{]}\PYG{o}{+}\PYG{p}{(}\PYG{n}{k1x}\PYG{o}{+}\PYG{l+m+mi}{2}\PYG{o}{*}\PYG{n}{k2x}\PYG{o}{+}\PYG{l+m+mi}{2}\PYG{o}{*}\PYG{n}{k3x}\PYG{o}{+}\PYG{n}{k4x}\PYG{p}{)}\PYG{o}{/}\PYG{l+m+mf}{6.}
        \PYG{n}{v}\PYG{p}{[}\PYG{n}{i}\PYG{o}{+}\PYG{l+m+mi}{1}\PYG{p}{]} \PYG{o}{=} \PYG{n}{v}\PYG{p}{[}\PYG{n}{i}\PYG{p}{]}\PYG{o}{+}\PYG{p}{(}\PYG{n}{k1v}\PYG{o}{+}\PYG{l+m+mi}{2}\PYG{o}{*}\PYG{n}{k2v}\PYG{o}{+}\PYG{l+m+mi}{2}\PYG{o}{*}\PYG{n}{k3v}\PYG{o}{+}\PYG{n}{k4v}\PYG{p}{)}\PYG{o}{/}\PYG{l+m+mf}{6.}
        \PYG{n}{t}\PYG{p}{[}\PYG{n}{i}\PYG{o}{+}\PYG{l+m+mi}{1}\PYG{p}{]} \PYG{o}{=} \PYG{n}{t}\PYG{p}{[}\PYG{n}{i}\PYG{p}{]} \PYG{o}{+} \PYG{n}{DeltaT}
\end{sphinxVerbatim}

The Runge\sphinxhyphen{}Kutta family of methods are particularly useful when we have a time\sphinxhyphen{}dependent acceleration.
If we have forces which depend only the spatial degrees of freedom (no velocity and/or time\sphinxhyphen{}dependence), then energy conserving methods like the Velocity Verlet or the Euler\sphinxhyphen{}Cromer method are preferred. As soon as we introduce an explicit time\sphinxhyphen{}dependence and/or add dissipitave forces like friction or air resistance, then methods like the family of Runge\sphinxhyphen{}Kutta methods are well suited for this.
The code below uses the Runge\sphinxhyphen{}Kutta4 methods.

\begin{sphinxVerbatim}[commandchars=\\\{\}]
\PYG{n}{DeltaT} \PYG{o}{=} \PYG{l+m+mf}{0.001}
\PYG{c+c1}{\PYGZsh{}set up arrays }
\PYG{n}{tfinal} \PYG{o}{=} \PYG{l+m+mi}{10} \PYG{c+c1}{\PYGZsh{} in years}
\PYG{n}{n} \PYG{o}{=} \PYG{n}{ceil}\PYG{p}{(}\PYG{n}{tfinal}\PYG{o}{/}\PYG{n}{DeltaT}\PYG{p}{)}
\PYG{c+c1}{\PYGZsh{} set up arrays for t, v, and x}
\PYG{n}{t} \PYG{o}{=} \PYG{n}{np}\PYG{o}{.}\PYG{n}{zeros}\PYG{p}{(}\PYG{n}{n}\PYG{p}{)}
\PYG{n}{v} \PYG{o}{=} \PYG{n}{np}\PYG{o}{.}\PYG{n}{zeros}\PYG{p}{(}\PYG{n}{n}\PYG{p}{)}
\PYG{n}{x} \PYG{o}{=} \PYG{n}{np}\PYG{o}{.}\PYG{n}{zeros}\PYG{p}{(}\PYG{n}{n}\PYG{p}{)}
\PYG{c+c1}{\PYGZsh{} Initial conditions (can change to more than one dim)}
\PYG{n}{x0} \PYG{o}{=} \PYG{l+m+mf}{1.0}
\PYG{n}{v0} \PYG{o}{=} \PYG{l+m+mf}{0.0}
\PYG{n}{x}\PYG{p}{[}\PYG{l+m+mi}{0}\PYG{p}{]} \PYG{o}{=} \PYG{n}{x0}
\PYG{n}{v}\PYG{p}{[}\PYG{l+m+mi}{0}\PYG{p}{]} \PYG{o}{=} \PYG{n}{v0}
\PYG{n}{gamma} \PYG{o}{=} \PYG{l+m+mf}{0.0}
\PYG{n}{Omegatilde} \PYG{o}{=} \PYG{l+m+mf}{0.2}
\PYG{n}{Ftilde} \PYG{o}{=} \PYG{l+m+mf}{1.0}

\PYG{c+c1}{\PYGZsh{} Start integrating using Euler\PYGZsq{}s method}
\PYG{c+c1}{\PYGZsh{} Note that we define the force function as a SpringForce}
\PYG{n}{RK4}\PYG{p}{(}\PYG{n}{v}\PYG{p}{,}\PYG{n}{x}\PYG{p}{,}\PYG{n}{t}\PYG{p}{,}\PYG{n}{n}\PYG{p}{,}\PYG{n}{SpringForce}\PYG{p}{)}

\PYG{c+c1}{\PYGZsh{} Plot position as function of time    }
\PYG{n}{fig}\PYG{p}{,} \PYG{n}{ax} \PYG{o}{=} \PYG{n}{plt}\PYG{o}{.}\PYG{n}{subplots}\PYG{p}{(}\PYG{p}{)}
\PYG{n}{ax}\PYG{o}{.}\PYG{n}{set\PYGZus{}ylabel}\PYG{p}{(}\PYG{l+s+s1}{\PYGZsq{}}\PYG{l+s+s1}{x[m]}\PYG{l+s+s1}{\PYGZsq{}}\PYG{p}{)}
\PYG{n}{ax}\PYG{o}{.}\PYG{n}{set\PYGZus{}xlabel}\PYG{p}{(}\PYG{l+s+s1}{\PYGZsq{}}\PYG{l+s+s1}{t[s]}\PYG{l+s+s1}{\PYGZsq{}}\PYG{p}{)}
\PYG{n}{ax}\PYG{o}{.}\PYG{n}{plot}\PYG{p}{(}\PYG{n}{t}\PYG{p}{,} \PYG{n}{x}\PYG{p}{)}
\PYG{n}{fig}\PYG{o}{.}\PYG{n}{tight\PYGZus{}layout}\PYG{p}{(}\PYG{p}{)}
\PYG{n}{save\PYGZus{}fig}\PYG{p}{(}\PYG{l+s+s2}{\PYGZdq{}}\PYG{l+s+s2}{ForcedBlockRK4}\PYG{l+s+s2}{\PYGZdq{}}\PYG{p}{)}
\PYG{n}{plt}\PYG{o}{.}\PYG{n}{show}\PYG{p}{(}\PYG{p}{)}
\end{sphinxVerbatim}

\noindent\sphinxincludegraphics{{chapter4_145_0}.png}




\subsection{Principle of Superposition and Periodic Forces (Fourier Transforms)}
\label{\detokenize{chapter4:principle-of-superposition-and-periodic-forces-fourier-transforms}}
If one has several driving forces, \(F(t)=\sum_n F_n(t)\), one can find
the particular solution to each \(F_n\), \(x_{pn}(t)\), and the particular
solution for the entire driving force is




\begin{equation*}
\begin{split}
\begin{equation}
x_p(t)=\sum_nx_{pn}(t).
\label{_auto25} \tag{34}
\end{equation}
\end{split}
\end{equation*}
This is known as the principal of superposition. It only applies when
the homogenous equation is linear. If there were an anharmonic term
such as \(x^3\) in the homogenous equation, then when one summed various
solutions, \(x=(\sum_n x_n)^2\), one would get cross
terms. Superposition is especially useful when \(F(t)\) can be written
as a sum of sinusoidal terms, because the solutions for each
sinusoidal (sine or cosine)  term is analytic, as we saw above.

Driving forces are often periodic, even when they are not
sinusoidal. Periodicity implies that for some time \(\tau\)
\begin{equation*}
\begin{split}
\begin{eqnarray}
F(t+\tau)=F(t). 
\end{eqnarray}
\end{split}
\end{equation*}
One example of a non\sphinxhyphen{}sinusoidal periodic force is a square wave. Many
components in electric circuits are non\sphinxhyphen{}linear, e.g. diodes, which
makes many wave forms non\sphinxhyphen{}sinusoidal even when the circuits are being
driven by purely sinusoidal sources.

The code here shows a typical example of such a square wave generated using the functionality included in the \sphinxstylestrong{scipy} Python package. We have used a period of \(\tau=0.2\).

\begin{sphinxVerbatim}[commandchars=\\\{\}]
\PYG{k+kn}{import} \PYG{n+nn}{numpy} \PYG{k}{as} \PYG{n+nn}{np}
\PYG{k+kn}{import} \PYG{n+nn}{math}
\PYG{k+kn}{from} \PYG{n+nn}{scipy} \PYG{k+kn}{import} \PYG{n}{signal}
\PYG{k+kn}{import} \PYG{n+nn}{matplotlib}\PYG{n+nn}{.}\PYG{n+nn}{pyplot} \PYG{k}{as} \PYG{n+nn}{plt}

\PYG{c+c1}{\PYGZsh{} number of points                                                                                       }
\PYG{n}{n} \PYG{o}{=} \PYG{l+m+mi}{500}
\PYG{c+c1}{\PYGZsh{} start and final times                                                                                  }
\PYG{n}{t0} \PYG{o}{=} \PYG{l+m+mf}{0.0}
\PYG{n}{tn} \PYG{o}{=} \PYG{l+m+mf}{1.0}
\PYG{c+c1}{\PYGZsh{} Period                                                                                                 }
\PYG{n}{t} \PYG{o}{=} \PYG{n}{np}\PYG{o}{.}\PYG{n}{linspace}\PYG{p}{(}\PYG{n}{t0}\PYG{p}{,} \PYG{n}{tn}\PYG{p}{,} \PYG{n}{n}\PYG{p}{,} \PYG{n}{endpoint}\PYG{o}{=}\PYG{k+kc}{False}\PYG{p}{)}
\PYG{n}{SqrSignal} \PYG{o}{=} \PYG{n}{np}\PYG{o}{.}\PYG{n}{zeros}\PYG{p}{(}\PYG{n}{n}\PYG{p}{)}
\PYG{n}{SqrSignal} \PYG{o}{=} \PYG{l+m+mf}{1.0}\PYG{o}{+}\PYG{n}{signal}\PYG{o}{.}\PYG{n}{square}\PYG{p}{(}\PYG{l+m+mi}{2}\PYG{o}{*}\PYG{n}{np}\PYG{o}{.}\PYG{n}{pi}\PYG{o}{*}\PYG{l+m+mi}{5}\PYG{o}{*}\PYG{n}{t}\PYG{p}{)}
\PYG{n}{plt}\PYG{o}{.}\PYG{n}{plot}\PYG{p}{(}\PYG{n}{t}\PYG{p}{,} \PYG{n}{SqrSignal}\PYG{p}{)}
\PYG{n}{plt}\PYG{o}{.}\PYG{n}{ylim}\PYG{p}{(}\PYG{o}{\PYGZhy{}}\PYG{l+m+mf}{0.5}\PYG{p}{,} \PYG{l+m+mf}{2.5}\PYG{p}{)}
\PYG{n}{plt}\PYG{o}{.}\PYG{n}{show}\PYG{p}{(}\PYG{p}{)}
\end{sphinxVerbatim}

For the sinusoidal example studied in the previous subsections the
period is \(\tau=2\pi/\omega\). However, higher harmonics can also
satisfy the periodicity requirement. In general, any force that
satisfies the periodicity requirement can be expressed as a sum over
harmonics,




\begin{equation*}
\begin{split}
\begin{equation}
F(t)=\frac{f_0}{2}+\sum_{n>0} f_n\cos(2n\pi t/\tau)+g_n\sin(2n\pi t/\tau).
\label{_auto26} \tag{35}
\end{equation}
\end{split}
\end{equation*}
From the previous subsection, one can write down the answer for
\(x_{pn}(t)\), by substituting \(f_n/m\) or \(g_n/m\) for \(F_0/m\) into Eq.s
({\hyperref[\detokenize{chapter4:eq:fastdriven1}]{\emph{20}}}) or ({\hyperref[\detokenize{chapter4:eq:fastdriven2}]{\emph{21}}}) respectively. By
writing each factor \(2n\pi t/\tau\) as \(n\omega t\), with \(\omega\equiv
2\pi/\tau\),




\begin{equation*}
\begin{split}
\begin{equation}
\label{eq:fourierdef1} \tag{36}
F(t)=\frac{f_0}{2}+\sum_{n>0}f_n\cos(n\omega t)+g_n\sin(n\omega t).
\end{equation}
\end{split}
\end{equation*}
The solutions for \(x(t)\) then come from replacing \(\omega\) with
\(n\omega\) for each term in the particular solution in Equations
({\hyperref[\detokenize{chapter4:eq:partform}]{\emph{13}}}) and ({\hyperref[\detokenize{chapter4:eq:Ddrive}]{\emph{16}}}),
\begin{equation*}
\begin{split}
\begin{eqnarray}
x_p(t)&=&\frac{f_0}{2k}+\sum_{n>0} \alpha_n\cos(n\omega t-\delta_n)+\beta_n\sin(n\omega t-\delta_n),\\
\nonumber
\alpha_n&=&\frac{f_n/m}{\sqrt{((n\omega)^2-\omega_0^2)+4\beta^2n^2\omega^2}},\\
\nonumber
\beta_n&=&\frac{g_n/m}{\sqrt{((n\omega)^2-\omega_0^2)+4\beta^2n^2\omega^2}},\\
\nonumber
\delta_n&=&\tan^{-1}\left(\frac{2\beta n\omega}{\omega_0^2-n^2\omega^2}\right).
\end{eqnarray}
\end{split}
\end{equation*}
Because the forces have been applied for a long time, any non\sphinxhyphen{}zero
damping eliminates the homogenous parts of the solution, so one need
only consider the particular solution for each \(n\).

The problem will considered solved if one can find expressions for the
coefficients \(f_n\) and \(g_n\), even though the solutions are expressed
as an infinite sum. The coefficients can be extracted from the
function \(F(t)\) by




\begin{equation*}
\begin{split}
\begin{eqnarray}
\label{eq:fourierdef2} \tag{37}
f_n&=&\frac{2}{\tau}\int_{-\tau/2}^{\tau/2} dt~F(t)\cos(2n\pi t/\tau),\\
\nonumber
g_n&=&\frac{2}{\tau}\int_{-\tau/2}^{\tau/2} dt~F(t)\sin(2n\pi t/\tau).
\end{eqnarray}
\end{split}
\end{equation*}
To check the consistency of these expressions and to verify
Eq. ({\hyperref[\detokenize{chapter4:eq:fourierdef2}]{\emph{37}}}), one can insert the expansion of \(F(t)\) in
Eq. ({\hyperref[\detokenize{chapter4:eq:fourierdef1}]{\emph{36}}}) into the expression for the coefficients in
Eq. ({\hyperref[\detokenize{chapter4:eq:fourierdef2}]{\emph{37}}}) and see whether
\begin{equation*}
\begin{split}
\begin{eqnarray}
f_n&=?&\frac{2}{\tau}\int_{-\tau/2}^{\tau/2} dt~\left\{
\frac{f_0}{2}+\sum_{m>0}f_m\cos(m\omega t)+g_m\sin(m\omega t)
\right\}\cos(n\omega t).
\end{eqnarray}
\end{split}
\end{equation*}
Immediately, one can throw away all the terms with \(g_m\) because they
convolute an even and an odd function. The term with \(f_0/2\)
disappears because \(\cos(n\omega t)\) is equally positive and negative
over the interval and will integrate to zero. For all the terms
\(f_m\cos(m\omega t)\) appearing in the sum, one can use angle addition
formulas to see that \(\cos(m\omega t)\cos(n\omega
t)=(1/2)(\cos[(m+n)\omega t]+\cos[(m-n)\omega t]\). This will integrate
to zero unless \(m=n\). In that case the \(m=n\) term gives




\begin{equation*}
\begin{split}
\begin{equation}
\int_{-\tau/2}^{\tau/2}dt~\cos^2(m\omega t)=\frac{\tau}{2},
\label{_auto27} \tag{38}
\end{equation}
\end{split}
\end{equation*}
and
\begin{equation*}
\begin{split}
\begin{eqnarray}
f_n&=?&\frac{2}{\tau}\int_{-\tau/2}^{\tau/2} dt~f_n/2\\
\nonumber
&=&f_n~\checkmark.
\end{eqnarray}
\end{split}
\end{equation*}
The same method can be used to check for the consistency of \(g_n\).

Consider the driving force:




\begin{equation*}
\begin{split}
\begin{equation}
F(t)=At/\tau,~~-\tau/2<t<\tau/2,~~~F(t+\tau)=F(t).
\label{_auto28} \tag{39}
\end{equation}
\end{split}
\end{equation*}
Find the Fourier coefficients \(f_n\) and \(g_n\) for all \(n\) using Eq. ({\hyperref[\detokenize{chapter4:eq:fourierdef2}]{\emph{37}}}).

Only the odd coefficients enter by symmetry, i.e. \(f_n=0\). One can find \(g_n\) integrating by parts,




\begin{equation*}
\begin{split}
\begin{eqnarray}
\label{eq:fouriersolution} \tag{40}
g_n&=&\frac{2}{\tau}\int_{-\tau/2}^{\tau/2}dt~\sin(n\omega t) \frac{At}{\tau}\\
\nonumber
u&=&t,~dv=\sin(n\omega t)dt,~v=-\cos(n\omega t)/(n\omega),\\
\nonumber
g_n&=&\frac{-2A}{n\omega \tau^2}\int_{-\tau/2}^{\tau/2}dt~\cos(n\omega t)
+\left.2A\frac{-t\cos(n\omega t)}{n\omega\tau^2}\right|_{-\tau/2}^{\tau/2}.
\end{eqnarray}
\end{split}
\end{equation*}
The first term is zero because \(\cos(n\omega t)\) will be equally
positive and negative over the interval. Using the fact that
\(\omega\tau=2\pi\),
\begin{equation*}
\begin{split}
\begin{eqnarray}
g_n&=&-\frac{2A}{2n\pi}\cos(n\omega\tau/2)\\
\nonumber
&=&-\frac{A}{n\pi}\cos(n\pi)\\
\nonumber
&=&\frac{A}{n\pi}(-1)^{n+1}.
\end{eqnarray}
\end{split}
\end{equation*}

\subsection{Fourier Series}
\label{\detokenize{chapter4:fourier-series}}
More text will come here, chpater 5.7\sphinxhyphen{}5.8 of Taylor are discussed
during the lectures. The code here uses the Fourier series discussed
in chapter 5.7 for a square wave signal. The equations for the
coefficients are are discussed in Taylor section 5.7, see Example
5.4. The code here visualizes the various approximations given by
Fourier series compared with a square wave with period \(T=0.2\), witth
\(0.1\) and max value \(F=2\). We see that when we increase the number of
components in the Fourier series, the Fourier series approximation gets closes and closes to the square wave signal.

\begin{sphinxVerbatim}[commandchars=\\\{\}]
\PYG{k+kn}{import} \PYG{n+nn}{numpy} \PYG{k}{as} \PYG{n+nn}{np}
\PYG{k+kn}{import} \PYG{n+nn}{math}
\PYG{k+kn}{from} \PYG{n+nn}{scipy} \PYG{k+kn}{import} \PYG{n}{signal}
\PYG{k+kn}{import} \PYG{n+nn}{matplotlib}\PYG{n+nn}{.}\PYG{n+nn}{pyplot} \PYG{k}{as} \PYG{n+nn}{plt}

\PYG{c+c1}{\PYGZsh{} number of points                                                                                       }
\PYG{n}{n} \PYG{o}{=} \PYG{l+m+mi}{500}
\PYG{c+c1}{\PYGZsh{} start and final times                                                                                  }
\PYG{n}{t0} \PYG{o}{=} \PYG{l+m+mf}{0.0}
\PYG{n}{tn} \PYG{o}{=} \PYG{l+m+mf}{1.0}
\PYG{c+c1}{\PYGZsh{} Period                                                                                                 }
\PYG{n}{T} \PYG{o}{=}\PYG{l+m+mf}{0.2}
\PYG{c+c1}{\PYGZsh{} Max value of square signal                                                                             }
\PYG{n}{Fmax}\PYG{o}{=} \PYG{l+m+mf}{2.0}
\PYG{c+c1}{\PYGZsh{} Width of signal                                                                                        }
\PYG{n}{Width} \PYG{o}{=} \PYG{l+m+mf}{0.1}
\PYG{n}{t} \PYG{o}{=} \PYG{n}{np}\PYG{o}{.}\PYG{n}{linspace}\PYG{p}{(}\PYG{n}{t0}\PYG{p}{,} \PYG{n}{tn}\PYG{p}{,} \PYG{n}{n}\PYG{p}{,} \PYG{n}{endpoint}\PYG{o}{=}\PYG{k+kc}{False}\PYG{p}{)}
\PYG{n}{SqrSignal} \PYG{o}{=} \PYG{n}{np}\PYG{o}{.}\PYG{n}{zeros}\PYG{p}{(}\PYG{n}{n}\PYG{p}{)}
\PYG{n}{FourierSeriesSignal} \PYG{o}{=} \PYG{n}{np}\PYG{o}{.}\PYG{n}{zeros}\PYG{p}{(}\PYG{n}{n}\PYG{p}{)}
\PYG{n}{SqrSignal} \PYG{o}{=} \PYG{l+m+mf}{1.0}\PYG{o}{+}\PYG{n}{signal}\PYG{o}{.}\PYG{n}{square}\PYG{p}{(}\PYG{l+m+mi}{2}\PYG{o}{*}\PYG{n}{np}\PYG{o}{.}\PYG{n}{pi}\PYG{o}{*}\PYG{l+m+mi}{5}\PYG{o}{*}\PYG{n}{t}\PYG{o}{+}\PYG{n}{np}\PYG{o}{.}\PYG{n}{pi}\PYG{o}{*}\PYG{n}{Width}\PYG{o}{/}\PYG{n}{T}\PYG{p}{)}
\PYG{n}{a0} \PYG{o}{=} \PYG{n}{Fmax}\PYG{o}{*}\PYG{n}{Width}\PYG{o}{/}\PYG{n}{T}
\PYG{n}{FourierSeriesSignal} \PYG{o}{=} \PYG{n}{a0}
\PYG{n}{Factor} \PYG{o}{=} \PYG{l+m+mf}{2.0}\PYG{o}{*}\PYG{n}{Fmax}\PYG{o}{/}\PYG{n}{np}\PYG{o}{.}\PYG{n}{pi}
\PYG{k}{for} \PYG{n}{i} \PYG{o+ow}{in} \PYG{n+nb}{range}\PYG{p}{(}\PYG{l+m+mi}{1}\PYG{p}{,}\PYG{l+m+mi}{500}\PYG{p}{)}\PYG{p}{:}
    \PYG{n}{FourierSeriesSignal} \PYG{o}{+}\PYG{o}{=} \PYG{n}{Factor}\PYG{o}{/}\PYG{p}{(}\PYG{n}{i}\PYG{p}{)}\PYG{o}{*}\PYG{n}{np}\PYG{o}{.}\PYG{n}{sin}\PYG{p}{(}\PYG{n}{np}\PYG{o}{.}\PYG{n}{pi}\PYG{o}{*}\PYG{n}{i}\PYG{o}{*}\PYG{n}{Width}\PYG{o}{/}\PYG{n}{T}\PYG{p}{)}\PYG{o}{*}\PYG{n}{np}\PYG{o}{.}\PYG{n}{cos}\PYG{p}{(}\PYG{n}{i}\PYG{o}{*}\PYG{n}{t}\PYG{o}{*}\PYG{l+m+mi}{2}\PYG{o}{*}\PYG{n}{np}\PYG{o}{.}\PYG{n}{pi}\PYG{o}{/}\PYG{n}{T}\PYG{p}{)}
\PYG{n}{plt}\PYG{o}{.}\PYG{n}{plot}\PYG{p}{(}\PYG{n}{t}\PYG{p}{,} \PYG{n}{SqrSignal}\PYG{p}{)}
\PYG{n}{plt}\PYG{o}{.}\PYG{n}{plot}\PYG{p}{(}\PYG{n}{t}\PYG{p}{,} \PYG{n}{FourierSeriesSignal}\PYG{p}{)}
\PYG{n}{plt}\PYG{o}{.}\PYG{n}{ylim}\PYG{p}{(}\PYG{o}{\PYGZhy{}}\PYG{l+m+mf}{0.5}\PYG{p}{,} \PYG{l+m+mf}{2.5}\PYG{p}{)}
\PYG{n}{plt}\PYG{o}{.}\PYG{n}{show}\PYG{p}{(}\PYG{p}{)}
\end{sphinxVerbatim}


\subsection{Solving differential equations with Fouries series}
\label{\detokenize{chapter4:solving-differential-equations-with-fouries-series}}
The material here was discussed during the lecture of February 19 and 21.
It is also covered by Taylor in section 5.8.


\subsection{Response to Transient Force}
\label{\detokenize{chapter4:response-to-transient-force}}
Consider a particle at rest in the bottom of an underdamped harmonic
oscillator, that then feels a sudden impulse, or change in momentum,
\(I=F\Delta t\) at \(t=0\). This increases the velocity immediately by an
amount \(v_0=I/m\) while not changing the position. One can then solve
the trajectory by solving Eq. ({\hyperref[\detokenize{chapter4:eq:homogsolution}]{\emph{9}}}) with initial
conditions \(v_0=I/m\) and \(x_0=0\). This gives




\begin{equation*}
\begin{split}
\begin{equation}
x(t)=\frac{I}{m\omega'}e^{-\beta t}\sin\omega't, ~~t>0.
\label{_auto29} \tag{41}
\end{equation}
\end{split}
\end{equation*}
Here, \(\omega'=\sqrt{\omega_0^2-\beta^2}\). For an impulse \(I_i\) that
occurs at time \(t_i\) the trajectory would be




\begin{equation*}
\begin{split}
\begin{equation}
x(t)=\frac{I_i}{m\omega'}e^{-\beta (t-t_i)}\sin[\omega'(t-t_i)] \Theta(t-t_i),
\label{_auto30} \tag{42}
\end{equation}
\end{split}
\end{equation*}
where \(\Theta(t-t_i)\) is a step function, i.e. \(\Theta(x)\) is zero for
\(x<0\) and unity for \(x>0\). If there were several impulses linear
superposition tells us that we can sum over each contribution,




\begin{equation*}
\begin{split}
\begin{equation}
x(t)=\sum_i\frac{I_i}{m\omega'}e^{-\beta(t-t_i)}\sin[\omega'(t-t_i)]\Theta(t-t_i)
\label{_auto31} \tag{43}
\end{equation}
\end{split}
\end{equation*}
Now one can consider a series of impulses at times separated by
\(\Delta t\), where each impulse is given by \(F_i\Delta t\). The sum
above now becomes an integral,




\begin{equation*}
\begin{split}
\begin{eqnarray}\label{eq:Greeny} \tag{44}
x(t)&=&\int_{-\infty}^\infty dt'~F(t')\frac{e^{-\beta(t-t')}\sin[\omega'(t-t')]}{m\omega'}\Theta(t-t')\\
\nonumber
&=&\int_{-\infty}^\infty dt'~F(t')G(t-t'),\\
\nonumber
G(\Delta t)&=&\frac{e^{-\beta\Delta t}\sin[\omega' \Delta t]}{m\omega'}\Theta(\Delta t)
\end{eqnarray}
\end{split}
\end{equation*}
The quantity
\(e^{-\beta(t-t')}\sin[\omega'(t-t')]/m\omega'\Theta(t-t')\) is called a
Green’s function, \(G(t-t')\). It describes the response at \(t\) due to a
force applied at a time \(t'\), and is a function of \(t-t'\). The step
function ensures that the response does not occur before the force is
applied. One should remember that the form for \(G\) would change if the
oscillator were either critically\sphinxhyphen{} or over\sphinxhyphen{}damped.

When performing the integral in Eq. ({\hyperref[\detokenize{chapter4:eq:Greeny}]{\emph{44}}}) one can use
angle addition formulas to factor out the part with the \(t'\)
dependence in the integrand,




\begin{equation*}
\begin{split}
\begin{eqnarray}
\label{eq:Greeny2} \tag{45}
x(t)&=&\frac{1}{m\omega'}e^{-\beta t}\left[I_c(t)\sin(\omega't)-I_s(t)\cos(\omega't)\right],\\
\nonumber
I_c(t)&\equiv&\int_{-\infty}^t dt'~F(t')e^{\beta t'}\cos(\omega't'),\\
\nonumber
I_s(t)&\equiv&\int_{-\infty}^t dt'~F(t')e^{\beta t'}\sin(\omega't').
\end{eqnarray}
\end{split}
\end{equation*}
If the time \(t\) is beyond any time at which the force acts,
\(F(t'>t)=0\), the coefficients \(I_c\) and \(I_s\) become independent of
\(t\).

Consider an undamped oscillator (\(\beta\rightarrow 0\)), with
characteristic frequency \(\omega_0\) and mass \(m\), that is at rest
until it feels a force described by a Gaussian form,
\begin{equation*}
\begin{split}
\begin{eqnarray*}
F(t)&=&F_0 \exp\left\{\frac{-t^2}{2\tau^2}\right\}.
\end{eqnarray*}
\end{split}
\end{equation*}
For large times (\(t>>\tau\)), where the force has died off, find
\(x(t)\).\textbackslash{} Solve for the coefficients \(I_c\) and \(I_s\) in
Eq. ({\hyperref[\detokenize{chapter4:eq:Greeny2}]{\emph{45}}}). Because the Gaussian is an even function,
\(I_s=0\), and one need only solve for \(I_c\),
\begin{equation*}
\begin{split}
\begin{eqnarray*}
I_c&=&F_0\int_{-\infty}^\infty dt'~e^{-t^{\prime 2}/(2\tau^2)}\cos(\omega_0 t')\\
&=&\Re F_0 \int_{-\infty}^\infty dt'~e^{-t^{\prime 2}/(2\tau^2)}e^{i\omega_0 t'}\\
&=&\Re F_0 \int_{-\infty}^\infty dt'~e^{-(t'-i\omega_0\tau^2)^2/(2\tau^2)}e^{-\omega_0^2\tau^2/2}\\
&=&F_0\tau \sqrt{2\pi} e^{-\omega_0^2\tau^2/2}.
\end{eqnarray*}
\end{split}
\end{equation*}
The third step involved completing the square, and the final step used the fact that the integral
\begin{equation*}
\begin{split}
\begin{eqnarray*}
\int_{-\infty}^\infty dx~e^{-x^2/2}&=&\sqrt{2\pi}.
\end{eqnarray*}
\end{split}
\end{equation*}
To see that this integral is true, consider the square of the integral, which you can change to polar coordinates,
\begin{equation*}
\begin{split}
\begin{eqnarray*}
I&=&\int_{-\infty}^\infty dx~e^{-x^2/2}\\
I^2&=&\int_{-\infty}^\infty dxdy~e^{-(x^2+y^2)/2}\\
&=&2\pi\int_0^\infty rdr~e^{-r^2/2}\\
&=&2\pi.
\end{eqnarray*}
\end{split}
\end{equation*}
Finally, the expression for \(x\) from Eq. ({\hyperref[\detokenize{chapter4:eq:Greeny2}]{\emph{45}}}) is
\begin{equation*}
\begin{split}
\begin{eqnarray*}
x(t>>\tau)&=&\frac{F_0\tau}{m\omega_0} \sqrt{2\pi} e^{-\omega_0^2\tau^2/2}\sin(\omega_0t).
\end{eqnarray*}
\end{split}
\end{equation*}

\subsection{The classical pendulum and scaling the equations}
\label{\detokenize{chapter4:the-classical-pendulum-and-scaling-the-equations}}
Let us end our discussion of oscillations with another classical case, the pendulum.

The angular equation of motion of the pendulum is given by
Newton’s equation and with no external force it reads




\begin{equation*}
\begin{split}
\begin{equation}
  ml\frac{d^2\theta}{dt^2}+mgsin(\theta)=0,
\label{_auto32} \tag{46}
\end{equation}
\end{split}
\end{equation*}
with an angular velocity and acceleration given by




\begin{equation*}
\begin{split}
\begin{equation}
     v=l\frac{d\theta}{dt},
\label{_auto33} \tag{47}
\end{equation}
\end{split}
\end{equation*}
and




\begin{equation*}
\begin{split}
\begin{equation}
     a=l\frac{d^2\theta}{dt^2}.
\label{_auto34} \tag{48}
\end{equation}
\end{split}
\end{equation*}
We do however expect that the motion will gradually come to an end due a viscous drag torque acting on the pendulum.
In the presence of the drag, the above equation becomes




\begin{equation*}
\begin{split}
\begin{equation}
   ml\frac{d^2\theta}{dt^2}+\nu\frac{d\theta}{dt}  +mgsin(\theta)=0, \label{eq:pend1} \tag{49}
\end{equation}
\end{split}
\end{equation*}
where \(\nu\) is now a positive constant parameterizing the viscosity
of the medium in question. In order to maintain the motion against
viscosity, it is necessary to add some external driving force.
We choose here a periodic driving force. The last equation becomes then




\begin{equation*}
\begin{split}
\begin{equation}
   ml\frac{d^2\theta}{dt^2}+\nu\frac{d\theta}{dt}  +mgsin(\theta)=Asin(\omega t), \label{eq:pend2} \tag{50}
\end{equation}
\end{split}
\end{equation*}
with \(A\) and \(\omega\) two constants representing the amplitude and
the angular frequency respectively. The latter is called the driving frequency.

We define
\begin{equation*}
\begin{split}
\omega_0=\sqrt{g/l},
\end{split}
\end{equation*}
the so\sphinxhyphen{}called natural frequency and the new dimensionless quantities
\begin{equation*}
\begin{split}
\hat{t}=\omega_0t,
\end{split}
\end{equation*}
with the dimensionless driving frequency
\begin{equation*}
\begin{split}
\hat{\omega}=\frac{\omega}{\omega_0},
\end{split}
\end{equation*}
and introducing the quantity \(Q\), called the \sphinxstyleemphasis{quality factor},
\begin{equation*}
\begin{split}
Q=\frac{mg}{\omega_0\nu},
\end{split}
\end{equation*}
and the dimensionless amplitude
\begin{equation*}
\begin{split}
\hat{A}=\frac{A}{mg}
\end{split}
\end{equation*}

\subsection{More on the Pendulum}
\label{\detokenize{chapter4:more-on-the-pendulum}}
We have
\begin{equation*}
\begin{split}
\frac{d^2\theta}{d\hat{t}^2}+\frac{1}{Q}\frac{d\theta}{d\hat{t}}  
     +sin(\theta)=\hat{A}cos(\hat{\omega}\hat{t}).
\end{split}
\end{equation*}
This equation can in turn be recast in terms of two coupled first\sphinxhyphen{}order differential equations as follows
\begin{equation*}
\begin{split}
\frac{d\theta}{d\hat{t}}=\hat{v},
\end{split}
\end{equation*}
and
\begin{equation*}
\begin{split}
\frac{d\hat{v}}{d\hat{t}}=-\frac{\hat{v}}{Q}-sin(\theta)+\hat{A}cos(\hat{\omega}\hat{t}).
\end{split}
\end{equation*}
These are the equations to be solved.  The factor \(Q\) represents the
number of oscillations of the undriven system that must occur before
its energy is significantly reduced due to the viscous drag. The
amplitude \(\hat{A}\) is measured in units of the maximum possible
gravitational torque while \(\hat{\omega}\) is the angular frequency of
the external torque measured in units of the pendulum’s natural
frequency.







\renewcommand{\indexname}{Index}
\printindex
\end{document}