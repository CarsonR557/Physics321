%%
%% Automatically generated file from DocOnce source
%% (https://github.com/hplgit/doconce/)
%%
%%


%-------------------- begin preamble ----------------------

\documentclass[%
oneside,                 % oneside: electronic viewing, twoside: printing
final,                   % draft: marks overfull hboxes, figures with paths
10pt]{article}

\listfiles               %  print all files needed to compile this document

\usepackage{relsize,makeidx,color,setspace,amsmath,amsfonts,amssymb}
\usepackage[table]{xcolor}
\usepackage{bm,ltablex,microtype}

\usepackage[pdftex]{graphicx}

\usepackage[T1]{fontenc}
%\usepackage[latin1]{inputenc}
\usepackage{ucs}
\usepackage[utf8x]{inputenc}

\usepackage{lmodern}         % Latin Modern fonts derived from Computer Modern

% Hyperlinks in PDF:
\definecolor{linkcolor}{rgb}{0,0,0.4}
\usepackage{hyperref}
\hypersetup{
    breaklinks=true,
    colorlinks=true,
    linkcolor=linkcolor,
    urlcolor=linkcolor,
    citecolor=black,
    filecolor=black,
    %filecolor=blue,
    pdfmenubar=true,
    pdftoolbar=true,
    bookmarksdepth=3   % Uncomment (and tweak) for PDF bookmarks with more levels than the TOC
    }
%\hyperbaseurl{}   % hyperlinks are relative to this root

\setcounter{tocdepth}{2}  % levels in table of contents

% Tricks for having figures close to where they are defined:
% 1. define less restrictive rules for where to put figures
\setcounter{topnumber}{2}
\setcounter{bottomnumber}{2}
\setcounter{totalnumber}{4}
\renewcommand{\topfraction}{0.95}
\renewcommand{\bottomfraction}{0.95}
\renewcommand{\textfraction}{0}
\renewcommand{\floatpagefraction}{0.75}
% floatpagefraction must always be less than topfraction!
% 2. ensure all figures are flushed before next section
\usepackage[section]{placeins}
% 3. enable begin{figure}[H] (often leads to ugly pagebreaks)
%\usepackage{float}\restylefloat{figure}

% prevent orhpans and widows
\clubpenalty = 10000
\widowpenalty = 10000

% --- end of standard preamble for documents ---


% insert custom LaTeX commands...

\raggedbottom
\makeindex
\usepackage[totoc]{idxlayout}   % for index in the toc
\usepackage[nottoc]{tocbibind}  % for references/bibliography in the toc

%-------------------- end preamble ----------------------

\begin{document}

% matching end for #ifdef PREAMBLE

\newcommand{\exercisesection}[1]{\subsection*{#1}}


% ------------------- main content ----------------------



% ----------------- title -------------------------

\thispagestyle{empty}

\begin{center}
{\LARGE\bf
\begin{spacing}{1.25}
PHY321: Classical Mechanics 1
\end{spacing}
}
\end{center}

% ----------------- author(s) -------------------------

\begin{center}
{\bf Final  project, due Friday May 1${}^{}$} \\ [0mm]
\end{center}

\begin{center}
% List of all institutions:
\end{center}
    
% ----------------- end author(s) -------------------------

% --- begin date ---
\begin{center}
Apr 24, 2020
\end{center}
% --- end date ---

\vspace{1cm}


\paragraph{Practicalities about  homeworks and projects.}
\begin{enumerate}
\item You can work in groups (optimal groups are often 2-3 people) or by yourself. If you work as a group you can hand in one answer only if you wish. \textbf{Remember to write your name(s)}!

\item How do I(we)  hand in?  Due to the extraordinary situation we are in now, the midterm should be handed in fully via D2L. You can scan your handwritten notes and upload to D2L or you can hand in everyhting (if you are ok with typing mathematical formulae using say Latex) as a jupyter notebook at D2L. The numerical part should always be handed in as a jupyter notebook.
\end{enumerate}

\noindent
\paragraph{Introduction to the final project, total score 100 points.}
The relevant reading background is
\begin{enumerate}
\item chapters 2-9 and 14 of Taylor

\item lecture notes throughout the semester and previous homework and midterm projects.
\end{enumerate}

\noindent
The final project aims at covering most of the topics we have discussed during the semester. As a physical system to discuss what has been, we will use what in the literature is called the \textbf{mathematical pendulum} and variants thereof. 

\paragraph{Exercise 1: Mathematical pendulum.}
A mathematical pendulum consists of a point mass $m$ suspended by a massless thread/rod of length $l$ in a gravitational field, as shown in the figure here. The constraining force is labeled by $\bm{T}$
and the gravitational force is labeled $\bm{F}_g$.

% FIGURE: [figslides/simplependulum.png, width=600 frac=0.6]

We assume that the length $l$ is constant and we define the coordinates involved as

\[
\bm{r} = l(\sin(\phi)\bm{\hat{x}}+\cos(\phi)\bm{\hat{y}},
\]
where $\bm{\hat{x}}$ and $\bm{\hat{y}}$ are the unit vectors in the $x$ and $y$ directions, respectively.

\begin{itemize}
\item 1a (5pt): Set up the forces acting on the system and show that the equation of motion is $m\ddot{\bm{r}}=\bm{F}_g+\bm{T}$.

\item 1b (10pt): Show that you can rewrite the above equation of motion as two independent equations of motion, on for $\phi$ and one for the constraining force. Show that these equations are $\ddot{\phi}(t)=-\omega_0^2\sin{(\phi(t))}$ with $\omega_0^2=g/l$ and $-ml\dot{\phi}^2=mg\cos{(\phi)}-T$.
\end{itemize}

\noindent
The equation for $\phi$ is a second-order differential equation
\[
\ddot{\phi}(t)=-\omega_0^2\sin{(\phi(t))}.
\]

This equation can be solved analytically if we assume that the angle $\phi$ is very small. Then we can approximate our equation as

\[
\ddot{\phi}(t)=-\omega_0^2\phi(t).
\]

\begin{itemize}
\item 1c (10pt): Find the analytical solution for the last equation. Hint, look back at the solutions for the simple harmonic oscillator problem in one dimension in for example homework 6.
\end{itemize}

\noindent

% ------------------- end of main content ---------------

\end{document}

