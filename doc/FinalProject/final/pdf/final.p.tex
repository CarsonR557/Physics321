%%
%% Automatically generated file from DocOnce source
%% (https://github.com/hplgit/doconce/)
%%
%%
% #ifdef PTEX2TEX_EXPLANATION
%%
%% The file follows the ptex2tex extended LaTeX format, see
%% ptex2tex: http://code.google.com/p/ptex2tex/
%%
%% Run
%%      ptex2tex myfile
%% or
%%      doconce ptex2tex myfile
%%
%% to turn myfile.p.tex into an ordinary LaTeX file myfile.tex.
%% (The ptex2tex program: http://code.google.com/p/ptex2tex)
%% Many preprocess options can be added to ptex2tex or doconce ptex2tex
%%
%%      ptex2tex -DMINTED myfile
%%      doconce ptex2tex myfile envir=minted
%%
%% ptex2tex will typeset code environments according to a global or local
%% .ptex2tex.cfg configure file. doconce ptex2tex will typeset code
%% according to options on the command line (just type doconce ptex2tex to
%% see examples). If doconce ptex2tex has envir=minted, it enables the
%% minted style without needing -DMINTED.
% #endif

% #define PREAMBLE

% #ifdef PREAMBLE
%-------------------- begin preamble ----------------------

\documentclass[%
oneside,                 % oneside: electronic viewing, twoside: printing
final,                   % draft: marks overfull hboxes, figures with paths
10pt]{article}

\listfiles               %  print all files needed to compile this document

\usepackage{relsize,makeidx,color,setspace,amsmath,amsfonts,amssymb}
\usepackage[table]{xcolor}
\usepackage{bm,ltablex,microtype}

\usepackage[pdftex]{graphicx}

\usepackage[T1]{fontenc}
%\usepackage[latin1]{inputenc}
\usepackage{ucs}
\usepackage[utf8x]{inputenc}

\usepackage{lmodern}         % Latin Modern fonts derived from Computer Modern

% Hyperlinks in PDF:
\definecolor{linkcolor}{rgb}{0,0,0.4}
\usepackage{hyperref}
\hypersetup{
    breaklinks=true,
    colorlinks=true,
    linkcolor=linkcolor,
    urlcolor=linkcolor,
    citecolor=black,
    filecolor=black,
    %filecolor=blue,
    pdfmenubar=true,
    pdftoolbar=true,
    bookmarksdepth=3   % Uncomment (and tweak) for PDF bookmarks with more levels than the TOC
    }
%\hyperbaseurl{}   % hyperlinks are relative to this root

\setcounter{tocdepth}{2}  % levels in table of contents

% Tricks for having figures close to where they are defined:
% 1. define less restrictive rules for where to put figures
\setcounter{topnumber}{2}
\setcounter{bottomnumber}{2}
\setcounter{totalnumber}{4}
\renewcommand{\topfraction}{0.95}
\renewcommand{\bottomfraction}{0.95}
\renewcommand{\textfraction}{0}
\renewcommand{\floatpagefraction}{0.75}
% floatpagefraction must always be less than topfraction!
% 2. ensure all figures are flushed before next section
\usepackage[section]{placeins}
% 3. enable begin{figure}[H] (often leads to ugly pagebreaks)
%\usepackage{float}\restylefloat{figure}

% prevent orhpans and widows
\clubpenalty = 10000
\widowpenalty = 10000

% --- end of standard preamble for documents ---


% insert custom LaTeX commands...

\raggedbottom
\makeindex
\usepackage[totoc]{idxlayout}   % for index in the toc
\usepackage[nottoc]{tocbibind}  % for references/bibliography in the toc

%-------------------- end preamble ----------------------

\begin{document}

% matching end for #ifdef PREAMBLE
% #endif

\newcommand{\exercisesection}[1]{\subsection*{#1}}


% ------------------- main content ----------------------



% ----------------- title -------------------------

\thispagestyle{empty}

\begin{center}
{\LARGE\bf
\begin{spacing}{1.25}
PHY321: Classical Mechanics 1
\end{spacing}
}
\end{center}

% ----------------- author(s) -------------------------

\begin{center}
{\bf Final  project, due Friday May 1${}^{}$} \\ [0mm]
\end{center}

\begin{center}
% List of all institutions:
\end{center}
    
% ----------------- end author(s) -------------------------

% --- begin date ---
\begin{center}
Apr 24, 2020
\end{center}
% --- end date ---

\vspace{1cm}


\paragraph{Practicalities about  homeworks and projects.}
\begin{enumerate}
\item You can work in groups (optimal groups are often 2-3 people) or by yourself. If you work as a group you can hand in one answer only if you wish. \textbf{Remember to write your name(s)}!

\item How do I(we)  hand in?  Due to the extraordinary situation we are in now, the final projec should be handed in fully via D2L. You can scan your handwritten notes and upload to D2L or you can hand in everyhting (if you are ok with typing mathematical formulae using say Latex) as a jupyter notebook at D2L. The numerical part should always be handed in as a jupyter notebook.
\end{enumerate}

\noindent
\paragraph{Introduction to the final project, total score  points.}
The relevant reading background is
\begin{enumerate}
\item chapters 2-9 and 14 of Taylor

\item lecture notes throughout the semester and previous homework and midterm projects.
\end{enumerate}

\noindent
The final project aims at covering most of the topics we have discussed during the semester. As a physical system to discuss many of the principles we have discussed during the semester, we will use what in the literature is called the \textbf{mathematical pendulum} and variants thereof. 

\paragraph{Exercise 1: Mathematical pendulum.}
A mathematical pendulum consists of a point mass $m$ suspended by a massless thread/rod of length $l$ in a gravitational field, as shown in the figure here. The constraining force is labeled by $\bm{T}$
and the gravitational force is labeled $\bm{F}_g$.

% FIGURE: [figslides/simplependulum.png, width=600 frac=0.6]

We assume that the length $l$ is constant and we define the coordinates involved as

\[
\bm{r} = l(\sin(\phi)\bm{\hat{x}}+\cos(\phi)\bm{\hat{y}},
\]
where $\bm{\hat{x}}$ and $\bm{\hat{y}}$ are the unit vectors in the $x$ and $y$ directions, respectively.

\begin{itemize}
\item \textbf{1a (5pt):} Set up the forces acting on the system and show that the equation of motion is $m\ddot{\bm{r}}=\bm{F}_g+\bm{T}$.

\item \textbf{1b (10pt):} Show that you can rewrite the above equation of motion as two independent equations of motion, one for $\phi$ and one for the constraining force. Show that these equations are $\ddot{\phi}(t)=-\omega_0^2\sin{(\phi(t))}$ with $\omega_0^2=g/l$ and $T=ml\dot{\phi}^2+mg\cos{(\phi)}$.
\end{itemize}

\noindent
The equation for $\phi$ is a second-order differential equation
\[
\ddot{\phi}(t)=-\omega_0^2\sin{(\phi(t))}.
\]

This equation can be solved analytically if we assume that the angle $\phi$ is very small. Then we can approximate our equation as

\[
\ddot{\phi}(t)=-\omega_0^2\phi(t).
\]

\begin{itemize}
\item \textbf{1c (10pt):} Find the analytical solution for the last equation. Hint, look back at the solutions for the simple harmonic oscillator problem in one dimension in for example homework 6.
\end{itemize}

\noindent
For our numerical treatment of the full second-order differential  equation, we can proceed as we have done before and split the second-order differential in two first-order differential equations
as shown here

\[
\frac{d\dot{\phi}}{dt}=-\omega_0^2\sin{(\phi)}.
\]

and

\[
\frac{d\phi}{dt}=\dot{\phi}.
\]


\begin{itemize}
\item \textbf{1d (10pt):} Scale the equations in terms of a dimensionless time $\hat{t}=\omega_0t$. Choose between the Euler-Cromer, the Velocity-Verlet or the Runge-Kutta to fourth order and \textbf{write down} the algorithm for solving the last two equations numerically. Explain briefly your choice of numerical algorithm. Hint, look back at what you did in homework 6 and the two midterms.

\item \textbf{1e (10pt):} Choose initial conditions and compare your numerical solution with the analytical one. For which range of angles $\phi$ (determined by your initial conditions) are the analytical solutions comparable to your numerical results? Discuss the implications of your results.

\item \textbf{1f (10pt):} Find the expressions for the kinetic and potential energies in terms of the variables $r$ and $\phi$. Remember that $r=l$ and is a constant throughout the calculations. In your code, check then whether energy is conserved by calculating the total energy, the kinetic and potential energies ad functions of time. Discuss your results.

\item \textbf{1g (10pt):} With the potential $V$  and kinetic $T$ energies, define the Lagrangian for the mathematical pendulum discussed here. Add the constraint $r=l$ via a Lagrange multiplier $\lambda$ and derive the equations of motion. Show that these result in  $\ddot{\phi}(t)=-\omega_0^2\sin{(\phi(t))}$ with $\omega_0^2=g/l$ and $\lambda=ml\dot{\phi}^2+mg\cos{(\phi)}$.  How would you interpret $\lambda$? 
\end{itemize}

\noindent
\paragraph{Exercise 2: Rotating Pendulum in a Gravitational Field.}
Assume now that the same pendulum is rotating in the gravitational field with a constant angular velocity $\Omega$ as shown in the figure here, with a constant angle $\phi$.
% FIGURE: [figslides/rotatingpendulum.png, width=600 frac=0.6]

From our discussions on rotating frames, the acceleration for an object in the rotating fram $S$  is given by
\[
m\bm{a}_{S}=\bm{F}+m\dot{\bm{r}\times\bm{\Omega}}+2m\bm{v}_S\times\bm{\Omega}+m(\bm{\Omega}\times\bm{r}_s)\times\bm{\Omega}.
\]

The position the mass of the pendulum is $\bm{r}_S$. Recall also that the length $l$ is constant.

We have the Coriolis force
\[
\bm{F}_{\mathrm{Coriolis}}=2m\bm{v}_S\times\bm{\Omega},
\]
while the last term is the standard centrifugal force

\[
\bm{F}_{\mathrm{Centrifugal}}=m\left(\bm{\Omega}\times\bm{r}_S\right)\times\bm{\Omega}.
\]

In this exercise  we will assume that the angular acceleration of the rotating frame and the velocity $\bm{v}_S$ are  constant quantities, that is their time derivatives are zero.

\begin{itemize}
\item \textbf{2a (10pt):} Set up the forces acting on the mass and show that, since $\Omega$ and $\bm{v}_S$ are constants that you get 
\end{itemize}

\noindent
\[
m\bm{a}_{S}=0=\bm{T}+m\bm{g}+m(\bm{\Omega}\times\bm{r}_S)\times\bm{\Omega}.
\]

\begin{itemize}
\item \textbf{2b (10pt):} Use the last equation to show that, for $\Omega^2 \lg g/l$  the angle $\phi$ is given by
\end{itemize}

\noindent
\[
\phi=\mathrm{arccos}\left(\frac{g}{l\Omega^2}\right).
\]

\begin{itemize}
\item \textbf{2c (10pt):} Plot $\phi$ as function of $g/l\Omega^2$ and study the limits $\Omega\rightarrow \infty$ and $\Omega\rightarrow 0$. Discuss  the possibility that $\Omega^2 < g/l$. Discuss the implications of your results. 
\end{itemize}

\noindent
\paragraph{Exercise 3: Planar double  pendulum.}
The final exercise deals with a planar double pendulum as shown in the figure here
% FIGURE: [figslides/doublependulum.png,width=600 frac=0.6]

We will set the masses equal $m_1=m_2=m$ and the lengths $l_1=l_2=l$.

We define the variables (see the figure)

\[
\bm{r}_1=l(\sin(\phi_1)\bm{\hat{x}}+\cos(\phi_1)\bm{\hat{y}},
\]

and

\[
\bm{r}_2=l(\sin(\phi_1)+\sin(\phi_2))\bm{\hat{x}}+   l(\cos(\phi_1)+\cos(\phi_2)  )\bm{\hat{y}}.
\]


\begin{itemize}
\item \textbf{3a (10pt):} Set up the forces acting on the two pendula. Use the above equations to define $x_1$, $x_2$, $y_1$ and $y_2$.

\item \textbf{3b (10pt):} Use the expressions for $x_1$, $x_2$, $y_1$ and $y_2$  to find the equation for the kinetic energy in terms of the angles $\phi_1$ and $\phi_2$, the length $l$ and the mass $m$.

\item \textbf{3c (10pt):} Use the expressions for $x_1$, $x_2$, $y_1$ and $y_2$  to find the equation for the potential energy.

\item \textbf{3d (20pt):} Define the Lagrangian ${\cal L}=T-V$ and use the Euler-Lagrange equations to find the equations of motion for $\phi_1$ and $\phi_2$. 
\end{itemize}

\noindent

% ------------------- end of main content ---------------

% #ifdef PREAMBLE
\end{document}
% #endif

