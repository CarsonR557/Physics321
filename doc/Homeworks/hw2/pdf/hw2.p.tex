%%
%% Automatically generated file from DocOnce source
%% (https://github.com/hplgit/doconce/)
%%
%%
% #ifdef PTEX2TEX_EXPLANATION
%%
%% The file follows the ptex2tex extended LaTeX format, see
%% ptex2tex: http://code.google.com/p/ptex2tex/
%%
%% Run
%%      ptex2tex myfile
%% or
%%      doconce ptex2tex myfile
%%
%% to turn myfile.p.tex into an ordinary LaTeX file myfile.tex.
%% (The ptex2tex program: http://code.google.com/p/ptex2tex)
%% Many preprocess options can be added to ptex2tex or doconce ptex2tex
%%
%%      ptex2tex -DMINTED myfile
%%      doconce ptex2tex myfile envir=minted
%%
%% ptex2tex will typeset code environments according to a global or local
%% .ptex2tex.cfg configure file. doconce ptex2tex will typeset code
%% according to options on the command line (just type doconce ptex2tex to
%% see examples). If doconce ptex2tex has envir=minted, it enables the
%% minted style without needing -DMINTED.
% #endif

% #define PREAMBLE

% #ifdef PREAMBLE
%-------------------- begin preamble ----------------------

\documentclass[%
oneside,                 % oneside: electronic viewing, twoside: printing
final,                   % draft: marks overfull hboxes, figures with paths
10pt]{article}

\listfiles               %  print all files needed to compile this document

\usepackage{relsize,makeidx,color,setspace,amsmath,amsfonts,amssymb}
\usepackage[table]{xcolor}
\usepackage{bm,ltablex,microtype}

\usepackage[pdftex]{graphicx}

\usepackage{ptex2tex}
% #ifdef MINTED
\usepackage{minted}
\usemintedstyle{default}
% #endif

\usepackage[T1]{fontenc}
%\usepackage[latin1]{inputenc}
\usepackage{ucs}
\usepackage[utf8x]{inputenc}

\usepackage{lmodern}         % Latin Modern fonts derived from Computer Modern

% Hyperlinks in PDF:
\definecolor{linkcolor}{rgb}{0,0,0.4}
\usepackage{hyperref}
\hypersetup{
    breaklinks=true,
    colorlinks=true,
    linkcolor=linkcolor,
    urlcolor=linkcolor,
    citecolor=black,
    filecolor=black,
    %filecolor=blue,
    pdfmenubar=true,
    pdftoolbar=true,
    bookmarksdepth=3   % Uncomment (and tweak) for PDF bookmarks with more levels than the TOC
    }
%\hyperbaseurl{}   % hyperlinks are relative to this root

\setcounter{tocdepth}{2}  % levels in table of contents

% prevent orhpans and widows
\clubpenalty = 10000
\widowpenalty = 10000

% --- end of standard preamble for documents ---


% insert custom LaTeX commands...

\raggedbottom
\makeindex
\usepackage[totoc]{idxlayout}   % for index in the toc
\usepackage[nottoc]{tocbibind}  % for references/bibliography in the toc

%-------------------- end preamble ----------------------

\begin{document}

% matching end for #ifdef PREAMBLE
% #endif

\newcommand{\exercisesection}[1]{\subsection*{#1}}


% ------------------- main content ----------------------



% ----------------- title -------------------------

\thispagestyle{empty}

\begin{center}
{\LARGE\bf
\begin{spacing}{1.25}
PHY321: Classical Mechanics 1
\end{spacing}
}
\end{center}

% ----------------- author(s) -------------------------

\begin{center}
{\bf Homework 2, due Monday February 1 (Midnight)${}^{}$} \\ [0mm]
\end{center}

\begin{center}
% List of all institutions:
\end{center}
    
% ----------------- end author(s) -------------------------

% --- begin date ---
\begin{center}
Jan 6, 2021
\end{center}
% --- end date ---

\vspace{1cm}


\paragraph{Practicalities about  homeworks and projects.}
\begin{enumerate}
\item You can work in groups (optimal groups are often 2-3 people) or by yourself. If you work as a group you can hand in one answer only if you wish. \textbf{Remember to write your name(s)}!

\item Homeworks are available Wednesday/Thursday the week before the deadline. The deadline is at the Friday lecture.

\item How do I(we)  hand in?  You can hand in the paper and pencil exercises as a  scanned  document. For this homework this applies to exercises 1-5. The scanned document should be uploaded to D2L. Alternatively, you can hand in everyhting (if you are ok with typing mathematical formulae using say Latex) as a jupyter notebook at D2L. The numerical exercise(s) (exercise 6 here) should always be handed in as a jupyter notebook by the deadline at D2L. 
\end{enumerate}

\noindent
\paragraph{Exercise 1 (10 pt), Forces, discussion questions, test your intuition.}
\begin{itemize}
\item 1a (2pt) Single force. Can an object affected only by a single force have zero acceleration?

\item 1b (2pt) Zero velocity. If you throw a ball vertically it has zero velocity at its maximum point. Does it also have zero acceleration at this point?

\item 1c (3pt) Acceleration of gravity. You measure the acceleration of gravity in an elevator moving at a velocity of 9.8m/s downwards. What will you measure?

\item 1d (3pt) Air resistance. You throw a ball straight up and measure the velocity as it passes you on its way down. Will the velocity be larger, the same, or smaller if you did the same experiment in vacuum?
\end{itemize}

\noindent
\paragraph{Exercise 2 (10 pt), setting up forces, Newton's second law.}
Useful material here to read is
\begin{enumerate}
\item Taylor chapters 1.3 and 1.4 and

\item Malthe-Sørenssen chapters 5.1, 5.2 and 5.3
\end{enumerate}

\noindent
A person jumps from an airplane, falling freely for several seconds before she pulls the cord of her parachute and the parachute unfolds.
\begin{itemize}
\item 2a (3pt)  Identify the forces acting on the parachuter and draw a free-body diagram of the parachuter before she has pulled the cord.

\item 2b (3pt)  Identify the forces acting on the parachuter and draw a free-body diagram of the parachuter after she has pulled the cord.

\item 2c (4pt)  Sketch the net force acting on the parachuter as a function of time, F(t).
\end{itemize}

\noindent
\paragraph{Exercise 3 (10 pt), Space shuttle with air resistance.}
Useful material here to read is
\begin{enumerate}
\item Malthe-Sørenssen chapters 5.1, 5.2 and 5.3
\end{enumerate}

\noindent
During lift-off of the space shuttle the engines provide a force of $35\times 10^{6}$ N. The mass of the shuttle is approximately
$2\times 10^6$ kg.
\begin{itemize}
\item 3a (3pt) Draw a free-body diagram of the space shuttle immediately after lift-off.

\item 3b (3pt)  Find an expression for the acceleration of the space shuttle immediately after lift-off.
\end{itemize}

\noindent
Let us assume that the force from the engines is constant, and that the mass of the
space shuttle does not change significantly over the first 20 s.
\begin{itemize}
\item 3c (4pt) Find the velocity and position of the space shuttle after 20 s if you ignore air resistance.
\end{itemize}

\noindent
\paragraph{Exercise 4 (15 pt), now hitting a golf ball.}
Useful material here to read is
\begin{enumerate}
\item Taylor chapters 1.3-1.6 and

\item Malthe-Sørenssen chapter 6.3-6.4 and 7.1-7.3
\end{enumerate}

\noindent
\textbf{Taylor exercise 1.35}. The formulae you obtain here will be useful for the numerical exercises below (see exercise 6 below).

\paragraph{Exercise 5 (15 pt), hitting a puck instead.}
Taylor exercise 1.38. 

\paragraph{Exercise 6 (40pt), Numerical elements, moving to more than one dimension.}
\textbf{This exercise should be handed in as a jupyter-notebook} at D2L. Remember to write your name(s). 

Last week we:
\begin{enumerate}
\item Analytically mapped 1D motion over some time

\item Gained practice with functions

\item Reviewed vectors and matrices in Python
\end{enumerate}

\noindent
This week we will:
\begin{enumerate}
\item Practice using Python syntax and variable manipulation

\item Utilize analytical solutions to create more refined functions

\item Work in two, three or even higher dimensions
\end{enumerate}

\noindent
This material will then serve as background for the numerical part of homework 3. The first part is a simple warm-up, with hints and suggestions you can use for the code to write below. 

\bpycod
# As usual, here are some useful packages we will be using. Feel free to use more and experiment as you wish.

import numpy as np
import matplotlib.pyplot as plt
from mpl_toolkits import mplot3d
%matplotlib inline
\epycod

In class (the falling baseball example) we used an  analytical expression for the height of a falling ball.
In the first homework we used instead the position from experiment (Usain Bolt's 100m record run) and stored this
information with one-dimensional arrays in Python.

Let us get some practice with this. The cell below creates two arrays,
one containing the times to be analyzed and the other containing the $x$
and $y$ components of the position vector at each point in time.  This is a two-dimensional object. The
second array is initially empty. Then we define  the initial
position to be $x=2$ and $y=1$. Take a look at the code and comments
to get an understanding of what is happening. Feel free to play around with it.


\bpycod
tf = 4 #length of value to be analyzed
dt = .001 # step sizes
t = np.arange(0.0,tf,dt) # Creates an evenly spaced time array going from 0 to 3.999, with step sizes .001
p = np.zeros((len(t), 2)) # Creates an empty array of [x,y] arrays (our vectors). Array size is same as the one for time.
p[0] = [2.0,1.0] # This sets the inital position to be x = 2 and y = 1
\epycod

Below we are printing specific values of our array to see what is being
stored where. The first number in the array $r[]$ represents which array
iteration we are looking at, while the number after the  represents
which listed number in the array iteration we are getting back. 


\bpycod
print(p[0]) # Prints the first array
print(p[0,:]) # Same as above, these commands are interchangeable 
\epycod
\bpycod
print(p[3999]) # Prints the 4000th array
\epycod

\bpycod
print(p[0,0]) # Prints the first value of the first array
\epycod
\bpycod
print(p[0,1]) # Prints the second value of first array
print(p[:,0]) # Prints the first value of all the arrays
\epycod
Then try running this cell. Notice how it gives an error since we did not implement a third dimension into our arrays
\bpycod
print(p[:,2])
\epycod

In the cell below we want to manipulate the arrays.
In this example we make each vector's $x$ component valued the same as their respective vector's position in the iteration and the $y$ value will be twice that value, except for  the first vector, which we have already set. 
That is we have $p[0] = [2,1], p[1] = [1,2], p[2] = [2,4], p[3] = [3,6], ...$

Here we set up an array for $x$ and $y$ values. 
\bpycod
for i in range(1,3999):
    p[i] = [i,2*i]
# Checker cell to make sure your code is performing correctly
c = 0
for i in range(0,3999):
    if i == 0:
        if p[i,0] != 2.0:
            c += 1
        if p[i,1] != 1.0:
            c += 1
    else:
        if p[i,0] != 1.0*i:
            c += 1
        if p[i,1] != 2.0*i:
            c += 1

if c == 0:
    print("Success!")
else:
    print("There is an error in your code")
\epycod

You could also think of an alternative way of storing the above information. Feel free to explore how to store
multidimensional objects. 


Last week we studied Usain Bolt's 100m run and in class we studied a falling baseball. We made basic plots of the baseball
moving in one dimension. This week we will be working with a three-dimensional variant. This will be useful for our next homeworks and numerical projects. 

Assume we have a soccer ball moving in three dimensions with the following trajectory:

\begin{enumerate}
\item $x(t) = 10t\cos{45^{\circ}} $

\item $y(t) = 10t\sin{45^{\circ}} $

\item $z(t) = 10t - \dfrac{9.81}{2}t^2$
\end{enumerate}

\noindent
Now let us create a three-dimensional (3D) plot using these equations. In the cell below
we write the equations into their respective labels. We fix a final time in the code below.

Important Concept: Numpy comes with many mathematical packages, some
of them being the trigonometric functions sine, cosine, tangent. We
are going to utilize these this week. Additionally, these functions
work with radians, so we will also be using a function from Numpy that
converts degrees to radians.


\bpycod
tf = 2.04  # The final time to be evaluated
dt = 0.1  # The time step size
t = np.arange(0,tf,dt) # The time array
theta_deg = 45 # Degrees
theta_rad = np.radians(theta_deg) # Converts degrees to their radian counterparts
x = 10*t*np.cos(theta_rad) # Equation for our x component, utilizing np.cos() and our calculated radians
y = 10*t*np.sin(theta_rad) # Put the y equation here
z = 10*t-9.81/2*t**2# Put the z equation here
\epycod
Then we plot it
\bpycod
## Once you have entered the proper equations in the cell above, run this cell to plot in 3D
fig = plt.axes(projection='3d')
fig.set_xlabel('x')
fig.set_ylabel('y')
fig.set_zlabel('z')
fig.scatter(x,y,z)
\epycod

\begin{itemize}
\item 6a (8pt) How would you express $x(t)$, $y(t)$, $z(t)$ for this problem as a single vector, $\bm{r}(t)$?
\end{itemize}

\noindent
Then run the code and plot using the array $r$
\bpycod
## Run this code to plot using our r array 
fig = plt.axes(projection='3d')
fig.set_xlabel('x')
fig.set_ylabel('y')
fig.set_zlabel('z')
fig.scatter(r[0],r[1],r[2])
\epycod

\begin{itemize}
\item 6b (8pt) What do you think the benefits and/or disadvantages are from expressing our three equations as a single array/vector? This can be both from a computational and physics stand point. Use the \textbf{Numpy} package to also print the maximum $x$, $y$ and $z$ components from $\bm{r}$.
\end{itemize}

\noindent
Complete Exercise 4 above (Taylor exercise 1.35) before moving further. (Recall that the golf ball was hit due east at an angle $\theta$ with respect to the horizontal, and the coordinate directions are $x$ measured east, $y$ north, and $z$ vertically up.)

\begin{itemize}
\item 6c (8pt) What is the analytical solution for our theoretical golf ball's position $\bm{r}(t)$ over time from Exercise 4?  Also what is the formula for the time $t_f$ when the golf ball hits the ground? Use this to develop a program with a function called for example Golfball that utilizes our analytical solutions. This program should take in an initial velocity and the angle $\theta$ that the golfball was hit with in degrees. It should also produce  a 3D graph of the motion. You need also to find the maximum values for $x$, $y$ and $z$.

\item 6d (8pt) Given initial values of $v_i = 90 m/s$, $\theta = 30^{\circ}$, what would our maximum x, y and z components be? 

\item 6e (8pt) Given initial values of $v_i = 45 m/s$, $\theta = 45^{\circ}$, what would our maximum x, y and z components be? 
\end{itemize}

\noindent

% ------------------- end of main content ---------------

% #ifdef PREAMBLE
\end{document}
% #endif

