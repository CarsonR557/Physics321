%%
%% Automatically generated file from DocOnce source
%% (https://github.com/hplgit/doconce/)
%%
%%


%-------------------- begin preamble ----------------------

\documentclass[%
oneside,                 % oneside: electronic viewing, twoside: printing
final,                   % draft: marks overfull hboxes, figures with paths
10pt]{article}

\listfiles               %  print all files needed to compile this document

\usepackage{relsize,makeidx,color,setspace,amsmath,amsfonts,amssymb}
\usepackage[table]{xcolor}
\usepackage{bm,ltablex,microtype}

\usepackage[pdftex]{graphicx}

\usepackage{fancyvrb} % packages needed for verbatim environments
\usepackage{verbatim}
%\usemintedstyle{default}

\usepackage[T1]{fontenc}
%\usepackage[latin1]{inputenc}
\usepackage{ucs}
\usepackage[utf8x]{inputenc}

\usepackage{lmodern}         % Latin Modern fonts derived from Computer Modern

% Hyperlinks in PDF:
\definecolor{linkcolor}{rgb}{0,0,0.4}
\usepackage{hyperref}
\hypersetup{
    breaklinks=true,
    colorlinks=true,
    linkcolor=linkcolor,
    urlcolor=linkcolor,
    citecolor=black,
    filecolor=black,
    %filecolor=blue,
    pdfmenubar=true,
    pdftoolbar=true,
    bookmarksdepth=3   % Uncomment (and tweak) for PDF bookmarks with more levels than the TOC
    }
%\hyperbaseurl{}   % hyperlinks are relative to this root

\setcounter{tocdepth}{2}  % levels in table of contents

% prevent orhpans and widows
\clubpenalty = 10000
\widowpenalty = 10000

% --- end of standard preamble for documents ---


% insert custom LaTeX commands...

\raggedbottom
\makeindex
\usepackage[totoc]{idxlayout}   % for index in the toc
\usepackage[nottoc]{tocbibind}  % for references/bibliography in the toc

%-------------------- end preamble ----------------------

\begin{document}

% matching end for #ifdef PREAMBLE

\newcommand{\exercisesection}[1]{\subsection*{#1}}


% ------------------- main content ----------------------



% ----------------- title -------------------------

\thispagestyle{empty}

\begin{center}
{\LARGE\bf
\begin{spacing}{1.25}
PHY321: Classical Mechanics 1
\end{spacing}
}
\end{center}

% ----------------- author(s) -------------------------

\begin{center}
{\bf Homework 5, due Monday  February 17${}^{}$} \\ [0mm]
\end{center}

\begin{center}
% List of all institutions:
\end{center}
    
% ----------------- end author(s) -------------------------

% --- begin date ---
\begin{center}
Feb 12, 2020
\end{center}
% --- end date ---

\vspace{1cm}


\paragraph{Practicalities about  homeworks and projects.}
\begin{enumerate}
\item You can work in groups (optimal groups are often 2-3 people) or by yourself. If you work as a group you can hand in one answer only if you wish. \textbf{Remember to write your name(s)}!

\item Homeworks are available Wednesday/Thursday the week before the deadline. The deadline is at the Friday lecture.

\item How do I(we)  hand in?  You can hand in the paper and pencil exercises as a hand-written document. For this homework this applies to exercises 1-5. Alternatively, you can hand in everyhting (if you are ok with typing mathematical formulae using say Latex) as a jupyter notebook at D2L. The numerical exercise(s) (exercise 6 here) should always be handed in as a jupyter notebook by the deadline at D2L. 
\end{enumerate}

\noindent
\paragraph{Introduction to homework 5.}
This week's sets of classical pen and paper and computational
exercises are a continuation of the topics from the previous homework set. We keep dealing with simple motion problems and conservation laws; energy, momentum and angular momentum. These conservation laws are central in Physics and understanding them properly lays the foundation for understanding and analyzing more complicated physics problems.
The relevant reading background is
\begin{enumerate}
\item chapters 3 and 4 of Taylor (there are many good examples there) and

\item chapters 10-14 of Malthe-Sørenssen.
\end{enumerate}

\noindent
In both textbooks there are many nice worked out examples. Malthe-Sørenssen's text contains also several coding examples you may find useful. 

The numerical homework focuses on another motion problem where you can
use the code you developed in homework 4, almost entirely. Please take
a look at the posted solution (jupyter-notebook) for homework 4. You
need only to change the forces at play. The problem at hand is again a
classic one, a block fastened to a spring moving back and forth along the $x$-axis. It is a simpler problem compared to the previous homework. It allows us however to introduce damping (friction) and study more realistic situations. 



\paragraph{Exercise 1 (15 pt), Work-energy theorem and conservation laws.}
This exercise was partly discussed during the lectures. It has not yet been edited in the online notes.
We will study a classical electron which moves in the $x$-direction along a surface. The force from the surface is
\[
\bm{F}(x)=-F_0\sin{(\frac{2\pi x}{b})}\bm{e}_x.
\]
The constant $b$ represents the distance between atoms at the surface of the material, $F_0$ is a constant and $x$ is the position of the electron.

\begin{itemize}
\item 1a (2pt) Is this a conservative force? And if so, what does that imply?

\item 1b (4pt) Use the work-energy theorem to find the velocity $v(x)$. 

\item 1c (4pt) With the above expression for the force, find the potential energy.

\item 1d (5pt) Make a plot of the potential energy and discuss the equilibrium points where the force on the electron is zero. Discuss the physical interpretation of stable and unstable equilibrium points. Use energy conservation. Recommended read here  is Malthe-Sørenssen chapter 11.3.2.
\end{itemize}

\noindent
\paragraph{Exsercise 2 (15pt), Rocket, Momentum and mass.}
Taylor exercise 3.11.   It is an exercise meant to illustrate momentum conservation or not. 
This exercise was partly discussed during the lectures, see the notes on \href{{https://mhjensen.github.io/Physics321/doc/pub/energyconserv/html/energyconserv.html}}{Energy and Momentum etc, see the part on Momentum conservation}. Taylor's chapter 3.2 covers also this example.


\paragraph{Exercise 3 (10pt), More Rockets.}
Taylor exercises 3.13 (5pt) and 3.14 (5pt). This is a continuation of the previous exercise and most of the relevant background material can be found in Taylor chapter 3.2. 

\paragraph{Exercise 4 (10pt), Center of mass.}
Taylor exercise 3.20. Here Taylor's chapter 3.3 can be of use. This relation will turn out to be very useful when we discuss systems of many classical particles.


\paragraph{Exercise 5 (10pt), Sliding Block  and Spring (please not again), Scaling the Equations and getting started with numerical project.}
The relevant material with codes etc is covered by the \href{{https://mhjensen.github.io/Physics321/doc/pub/harmonic/html/harmonic.html}}{Lecture Notes on Oscillations}. Taylor's chapter 5, in particular sections 5.1 amd 5.2. 5.4 is relevant for the bonus exercise.
We start now with our next numerical application with simple rewrites of our equations. The system we will look at is that of a block fastened to a spring (which in turn is tied to a wall).
The force acting on the block from the spring in the $x$-drection only is
\[
m\frac{d^2x(t)}{dt^2}=F(x) = -k(x-b),
\]
where $k$ is a material specific constant and $b$ is the equilibrium position. The block has mass $m$ and $t$ is time. Define the initial time as $t_0$. We will for simplicity set the equilibrium position to zero, that is $b=0$.

\begin{itemize}
\item 6a (3pt) Does this force conserve energy? If so, with given initial position and velocity $x_0$ and $v_0$, respectively, find the expression for energy conservation in terms of the potential and kinetic energies.  

\item 6b (3pt) Define a constant $\omega_0=\sqrt{k/m}$ and show that you can write the acceleration as $a(t) = -\omega_0^2 x(t)$. What is the dimentionality of $\omega_0$? (it is normally called a natural frequency).  

\item 6c (4pt) Introduce now a dimensionless time $\tau = t\omega_0$. Show that you can rewrite the  equation for the acceleration in terms of two first-order differential equations
\end{itemize}

\noindent
\[
\frac{dv}{d\tau} = -x,
\]
and
\[
\frac{dx}{d\tau} = v.
\]
What are the dimensionalities of these two equations?

These are the equations which we will code in the next exercise.

\paragraph{Exercise 6 (40pt), Sliding Block  and Spring, now the code.}
\textbf{This exercise should be handed in as a jupyter-notebook} at D2L. Remember to write your name(s). 

Last week we:
\begin{enumerate}
\item Studied  the Euler-Cromer and Velocity Verlet  to find the position and the velocity of  Earth orbiting around the sun.

\item We added the Euler-Cromer and Velocity-Verlet methods to our toolbag for solving ordinary differential equations.

\item We compared and analyzed the stability of the  computational solutions.
\end{enumerate}

\noindent
This week we will reuse our code from homework 4 and study a block
which slides back and forth without friction in the $x$-direction.
The only thing we need to is to change the forces at play and since we
limit motion only to the $x$-direction we can simplify our code to
only that degree of freedom.
The advantage with the block sliding back and forth is that we have analytical solutions for the velocity and the position. 


\begin{itemize}
\item 7a (20pt) Find the analytical solutions for the velocity and position as function of the initial conditions. 

\item 7b (20 pt)  Write then a program which solves the above differential equations for the sliding blook using the Euler-Cromer  method and the velocity Verlet method. Study the results for the position, velocity and energy conservation as function of time and the step size in time. Compare the numerical results with the analytical ones. Check that your results for position and velocity obey periodicity. 
\end{itemize}

\noindent
Is energy properly conserved? Discuss your results. 

Here follows a simple code example which could be used to study the sliding block.
\begin{verbatim}[fontsize=\fontsize{9pt}{9pt},linenos=false,mathescape,baselinestretch=1.0,fontfamily=tt,xleftmargin=7mm]{python}
DeltaT = 0.001
#set up arrays 
tfinal = 10 # in seconds
n = ceil(tfinal/DeltaT)
# set up arrays for time t, velocity v, and position r
t = np.zeros(n)
v = np.zeros(n)
x = np.zeros(n)
# Example of Initial conditions
x0 = 1.0
v0 = 0.0
r[0] = r0
v[0] = v0
# Start integrating using Euler's method
for i in range(n-1):
    # Set up the acceleration
    # Here you could have defined your own function for this
    a = -x[i]
    # update velocity, time and position using the Euler-Cromer method
    v[i+1] = v[i] + DeltaT*a
    x[i+1] = x[i] + DeltaT*v[i+1]
    t[i+1] = t[i] + DeltaT
\end{verbatim}


\paragraph{Exercise 8 (30pt), Bonus exercise.}
You don't need to do this exercise, but it gives you a bonus score of 30 points. Taylor's section 5.4 and the lecture notes on damped oscillations cover most of the theoretical background.

In this exercise we will add to our program a first derivative of position (the velocity) which is meant to mimick the role of friction.
Our differential equation is now
\[
m\frac{d^2x(t)}{dt^2}+\beta\frac{dx}{dt}+kx=ma(t)+\beta v(t)+kx(t)=0,
\]

where $k$ is a material specific constant and $\beta$ is a constant
representing the resitance from the plane on which the block slides.
We have thus assumed that friction depends on velocity in a linear
way.

We introduce again the dimensionless time $\tau = t\omega_0$ with
$\omega_0=\sqrt{k/m}$.

\begin{itemize}
\item 8a (10pt) Divide by $m$ and show that you can rewrite your equation as two first-order coupled differential equations
\end{itemize}

\noindent
\[
\frac{dv}{d\tau} = -2v(t)\gamma-x(t),
\]
and
\[
\frac{dx}{d\tau} = v(t).
\]
What are the dimensionalities of these two equations? We have defined $\gamma = \beta/2m$.  


\begin{itemize}
\item 8b (20pt)  Add the role of friction to your code from exercise 7 and study the cases where (i) $\gamma < \omega_0$ (underdamping), (ii) $\gamma = omega_0$ ( critical damping) and (iii) $\gamma > omega_0$ (overdamping). Plot your resulting positions as functions of time and discuss the physical meaning of your results.
\end{itemize}

\noindent

% ------------------- end of main content ---------------

\end{document}

