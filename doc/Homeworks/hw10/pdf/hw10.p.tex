%%
%% Automatically generated file from DocOnce source
%% (https://github.com/hplgit/doconce/)
%%
%%
% #ifdef PTEX2TEX_EXPLANATION
%%
%% The file follows the ptex2tex extended LaTeX format, see
%% ptex2tex: http://code.google.com/p/ptex2tex/
%%
%% Run
%%      ptex2tex myfile
%% or
%%      doconce ptex2tex myfile
%%
%% to turn myfile.p.tex into an ordinary LaTeX file myfile.tex.
%% (The ptex2tex program: http://code.google.com/p/ptex2tex)
%% Many preprocess options can be added to ptex2tex or doconce ptex2tex
%%
%%      ptex2tex -DMINTED myfile
%%      doconce ptex2tex myfile envir=minted
%%
%% ptex2tex will typeset code environments according to a global or local
%% .ptex2tex.cfg configure file. doconce ptex2tex will typeset code
%% according to options on the command line (just type doconce ptex2tex to
%% see examples). If doconce ptex2tex has envir=minted, it enables the
%% minted style without needing -DMINTED.
% #endif

% #define PREAMBLE

% #ifdef PREAMBLE
%-------------------- begin preamble ----------------------

\documentclass[%
oneside,                 % oneside: electronic viewing, twoside: printing
final,                   % draft: marks overfull hboxes, figures with paths
10pt]{article}

\listfiles               %  print all files needed to compile this document

\usepackage{relsize,makeidx,color,setspace,amsmath,amsfonts,amssymb}
\usepackage[table]{xcolor}
\usepackage{bm,ltablex,microtype}

\usepackage[pdftex]{graphicx}

\usepackage[T1]{fontenc}
%\usepackage[latin1]{inputenc}
\usepackage{ucs}
\usepackage[utf8x]{inputenc}

\usepackage{lmodern}         % Latin Modern fonts derived from Computer Modern

% Hyperlinks in PDF:
\definecolor{linkcolor}{rgb}{0,0,0.4}
\usepackage{hyperref}
\hypersetup{
    breaklinks=true,
    colorlinks=true,
    linkcolor=linkcolor,
    urlcolor=linkcolor,
    citecolor=black,
    filecolor=black,
    %filecolor=blue,
    pdfmenubar=true,
    pdftoolbar=true,
    bookmarksdepth=3   % Uncomment (and tweak) for PDF bookmarks with more levels than the TOC
    }
%\hyperbaseurl{}   % hyperlinks are relative to this root

\setcounter{tocdepth}{2}  % levels in table of contents

% prevent orhpans and widows
\clubpenalty = 10000
\widowpenalty = 10000

% --- end of standard preamble for documents ---


% insert custom LaTeX commands...

\raggedbottom
\makeindex
\usepackage[totoc]{idxlayout}   % for index in the toc
\usepackage[nottoc]{tocbibind}  % for references/bibliography in the toc

%-------------------- end preamble ----------------------

\begin{document}

% matching end for #ifdef PREAMBLE
% #endif

\newcommand{\exercisesection}[1]{\subsection*{#1}}


% ------------------- main content ----------------------



% ----------------- title -------------------------

\thispagestyle{empty}

\begin{center}
{\LARGE\bf
\begin{spacing}{1.25}
PHY321: Classical Mechanics 1
\end{spacing}
}
\end{center}

% ----------------- author(s) -------------------------

\begin{center}
{\bf Homework 10, due Friday  April 24${}^{}$} \\ [0mm]
\end{center}

\begin{center}
% List of all institutions:
\end{center}
    
% ----------------- end author(s) -------------------------

% --- begin date ---
\begin{center}
Apr 20, 2020
\end{center}
% --- end date ---

\vspace{1cm}


\paragraph{Practicalities about  homeworks and projects.}
\begin{enumerate}
\item You can work in groups (optimal groups are often 2-3 people) or by yourself. If you work as a group you can hand in one answer only if you wish. \textbf{Remember to write your name(s)}!

\item Homeworks are available  the week before the deadline. 

\item How do I(we)  hand in?  Due to the corona virus and many of you not being on campus, we recommend that you scan your handwritten notes and upload them to D2L. If you are ok with typing mathematical formulae using say Latex, you can hand in everything as a single jupyter notebook at D2L. The numerical exercise(s) should always be handed in as a jupyter notebook by the deadline at D2L. 
\end{enumerate}

\noindent
\paragraph{Introduction to homework 10.}
This homework is optional but gives an extra score of 10\% on top of
all you have done throughout the semester. The exercises are
essentially good old fashioned paper and pencil exercises and each
covers different aspects of what has been done after spring break.

\paragraph{Exercise 1 (10pt), new reference frame.}
Show that if one transforms to a reference frame where the total
momentum is zero, $\bm{p}_1=-\bm{p}_2$, that the relative momentum
$\bm{q}$ corresponds to either $\bm{p}_1$ or $-\bm{p}_2$. This
means that in this frame the magnitude of $\bm{q}$ is one half the
magnitude of $\bm{p}_1-\bm{p}_2$.

\paragraph{Exercise 2 (10pt) Center of mass and relative coordinates.}
Given the center of mass and relative coordinates $\bm{R}$ and $\bm{r}$, respectively, for
particles of mass $m_1$ and $m_2$, find the coordinates $\bm{r}_1$
and $\bm{r}_2$ in terms of the masses, $\bm{R}$ and $\bm{r}$.


\paragraph{Exercise 3 (30pt),  Two-body problems.}
Consider a particle of mass $m$ moving in a potential
\[
V(r)=\alpha\ln(r/a),
\]
where $a$ is a constant.

\begin{itemize}
\item 3a (5pt)  If the particle is moving in a circular orbit of radius $R$, find the angular frequency $\dot{\theta}$. Solve this by setting $F=-m\dot{\theta}^2r$ (force and acceleration point inward).

\item 3b (5pt) Express the angular momentum $L$ in terms of $\alpha$, $m$ and $R$. Also express $R$ in terms of $L$, $\alpha$ and $m$.

\item 3c (5pt) Sketch the effective radial potential, $V_{\rm eff}(r)$, for a particle with angular momentum $L$. (No longer necessarily moving in a circular orbit.)

\item 3d (5pt)  Find the position of the minimum of $V_{\rm eff}$ in terms of $L$, $\alpha$ and $m$, then compare to the result of (3b).

\item 3e (5pt)  What is the effective spring constant for a particle at the minimum of $V_{\rm eff}$? Express your answer in terms of $L$, $m$ and $\alpha$. 

\item 3f (5pt)  What is the angular frequency, $\omega$, for small oscillations of $r$ about the $R_{\rm min}$?  Express your answer in terms of $\dot{\theta}$ from part (3a).
\end{itemize}

\noindent
\paragraph{Exercise 4 (10pt) Non-inertial frames.}
A high-speed cannon shoots a projectile with an initial velocity of
1000 m/s in the east direction. The cannon is situated in
Minneapolis (latitude of 45 degrees) The projectile velocity is
nearly horizontal and it hits the ground after a distance $x=3000$
  m. Find the alteration of the point of impact in the north-south
  ($y$) direction due to the Coriolis force. Assume the effect is
  small so that you can approximate the eastward ($x$) component of
  the velocity as being constant. Be sure to indicate whether the
  deflection is north or south.

As a hint, if you let $y$ be the north-south direction the equation of motions with the Coriolis force reads

\[
\frac{dv_y}{dt}= -2\Omega_zv_x + 2\Omega_xv_z,
\]
with the angular velocity $\Omega_z=\Omega_{\mathrm{Earth}}/\sqrt{2}$ and $\Omega_x=0$.
Integrate up and the find the velocity in the $y$-direction and the corresponding displacement. 



\paragraph{Exercise 5 (10pt), Lagrangian formalism.}
Consider a mass $m$ connected to a spring with spring constant
  $k$. Rather than being fixed, the other end of the spring oscillates
  with frequency $\omega$ and amplitude $A$. For a generalized
  coordinate, use the displacement of the mass from its relaxed
  position and call it $y=x-\ell-A\cos\omega t$. In this system the
  potential energy of the spring is $ky^2/2$.


\item 5a (5pt)  Write the kinetic energy in terms of the generalized coordinate.

\item 5b (5pt) Write down the Lagrangian and find the equations of motion for $y$.


\paragraph{Exercise 6 (30pt), Pendulum and Lagrangians.}
Consider a pendulum of length $\ell$ with all the mass $m$ at
  its end. The pendulum is allowed to swing freely in both
  directions. Using $\phi$ to describe the azimuthal angle about the
  $z$ axis and $\theta$ to measure the angular deviation of the
  pendulum from the downward direction, address the following
  questions:


\item 6a (5pt) If the pendulum is initially moving horizontally with velocity $v_0$ and angle $\theta_0=90^\circ$ (horizontal), use energy and angular momentum conservation to find the minimum angles of $\theta_{\rm min}$ subtended by the pendulum. (Note that the angle  will oscillate between $90^\circ$ and the minimum angle.

\item 6b (5pt) Write the Lagrangian using $\theta$ and $\phi$ as generalized coordinates.

\item 6c (5pt) Write the equations of motion for $\theta$ and $\phi$.

\item 6d (5pt) Rewrite the equations of motion for $\theta$ using angular momentum conservation to eliminate and reference to $\phi$.

\item 6e (5pt) Find the value of $L$ required for the stable orbit to be at $\theta=45^\circ$.

\item 6f (5pt)  For the steady orbit found in (e) consider small perturbations of the orbit. Find the frequency with which the pendulum oscillates around $\theta=45^\circ$.



% ------------------- end of main content ---------------

% #ifdef PREAMBLE
\end{document}
% #endif

