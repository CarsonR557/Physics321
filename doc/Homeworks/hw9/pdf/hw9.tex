%%
%% Automatically generated file from DocOnce source
%% (https://github.com/hplgit/doconce/)
%%
%%


%-------------------- begin preamble ----------------------

\documentclass[%
oneside,                 % oneside: electronic viewing, twoside: printing
final,                   % draft: marks overfull hboxes, figures with paths
10pt]{article}

\listfiles               %  print all files needed to compile this document

\usepackage{relsize,makeidx,color,setspace,amsmath,amsfonts,amssymb}
\usepackage[table]{xcolor}
\usepackage{bm,ltablex,microtype}

\usepackage[pdftex]{graphicx}

\usepackage[T1]{fontenc}
%\usepackage[latin1]{inputenc}
\usepackage{ucs}
\usepackage[utf8x]{inputenc}

\usepackage{lmodern}         % Latin Modern fonts derived from Computer Modern

% Hyperlinks in PDF:
\definecolor{linkcolor}{rgb}{0,0,0.4}
\usepackage{hyperref}
\hypersetup{
    breaklinks=true,
    colorlinks=true,
    linkcolor=linkcolor,
    urlcolor=linkcolor,
    citecolor=black,
    filecolor=black,
    %filecolor=blue,
    pdfmenubar=true,
    pdftoolbar=true,
    bookmarksdepth=3   % Uncomment (and tweak) for PDF bookmarks with more levels than the TOC
    }
%\hyperbaseurl{}   % hyperlinks are relative to this root

\setcounter{tocdepth}{2}  % levels in table of contents

% prevent orhpans and widows
\clubpenalty = 10000
\widowpenalty = 10000

% --- end of standard preamble for documents ---


% insert custom LaTeX commands...

\raggedbottom
\makeindex
\usepackage[totoc]{idxlayout}   % for index in the toc
\usepackage[nottoc]{tocbibind}  % for references/bibliography in the toc

%-------------------- end preamble ----------------------

\begin{document}

% matching end for #ifdef PREAMBLE

\newcommand{\exercisesection}[1]{\subsection*{#1}}


% ------------------- main content ----------------------



% ----------------- title -------------------------

\thispagestyle{empty}

\begin{center}
{\LARGE\bf
\begin{spacing}{1.25}
PHY321: Classical Mechanics 1
\end{spacing}
}
\end{center}

% ----------------- author(s) -------------------------

\begin{center}
{\bf Homework 9, due Monday  April 6${}^{}$} \\ [0mm]
\end{center}

\begin{center}
% List of all institutions:
\end{center}
    
% ----------------- end author(s) -------------------------

% --- begin date ---
\begin{center}
Mar 30, 2020
\end{center}
% --- end date ---

\vspace{1cm}


\paragraph{Practicalities about  homeworks and projects.}
\begin{enumerate}
\item You can work in groups (optimal groups are often 2-3 people) or by yourself. If you work as a group you can hand in one answer only if you wish. \textbf{Remember to write your name(s)}!

\item Homeworks are available  the week before the deadline. 

\item How do I(we)  hand in?  Due to the corona virus and many of you not being on campus, we recommend that you scan your handwritten notes and upload them to D2L. If you are ok with typing mathematical formulae using say Latex, you can hand in everything as a single jupyter notebook at D2L. The numerical exercise(s) should always be handed in as a jupyter notebook by the deadline at D2L. 
\end{enumerate}

\noindent
\paragraph{Introduction to homework 9.}
This week's exercises are purely numerical and focus on solving
gravitational problems. It is based on what you did in homework 4 as
well as in the previous two sets. We will, based on the code for the
Sun-Earth system in homework 4, add Jupyter and study a three-body
problem.  We will play around with different masses for Jupyter.

The bonus exercise in this homework is a study of the perihelion of
Mercury by adding Einstein's relatistic effect (we will not derive it
here). This correction to the equations of motion was the very first
confirmation of Einstein's general theory of relativity.  The effect
on the perihelion of Mercury can be measured. In your numerical
simulations you will find this to be a very small effect compared to
calculations without this relativistic correction.



In homework 4 we limited ourselves (in order to test the algorithm) to
a hypothetical solar system with the Earth only orbiting around the
Sun. We assumed that the only force in the problem is
gravity. Newton's law of gravitation is given by a force $F_G$ (we assume this is the force acting on Earth from the Sun)

\[
F_G=-\frac{GM_{\odot}M_{\mathrm{Earth}}}{r^2},
\]

where $M_{\odot}$ is the mass of the Sun and $M_{\mathrm{Earth}}$ is
the mass of the Earth. The gravitational constant is $G$ and $r$ is
the distance between the Earth and the Sun.  We assumed that the Sun
has a mass which is much larger than that of the Earth. We could
therefore safely neglect the motion of the Sun.

In homework 4 assumed that the orbit of the Earth around the Sun 
was co-planar, and we took this to be the $xy$-plane.
Using Newton's second law of motion we got the following equations

\[
\frac{d^2x}{dt^2}=-\frac{F_{G,x}}{M_{\mathrm{Earth}}},
\]
and

\[
\frac{d^2y}{dt^2}=-\frac{F_{G,y}}{M_{\mathrm{Earth}}},
\]

where $F_{G,x}$ and $F_{G,y}$ are the $x$ and $y$ components of the
gravitational force.

We will again use so-called astronomical units when rewriting our
equations.  Using astronomical units (AU as abbreviation)it means that
one astronomical unit of length, known as 1 AU, is the average
distance between the Sun and Earth, that is $1$ AU = $1.5\times
10^{11}$ m.  It can also be convenient to use years instead of seconds
since years match better the time evolution of the solar system. The
mass of the Sun is $M_{\mathrm{sun}}=M_{\odot}=2\times 10^{30}$
kg. The masses of all relevant planets and their distances from the
sun are listed in the table here in kg and AU.


\begin{quote}
\begin{tabular}{ccc}
\hline
\multicolumn{1}{c}{ Planet } & \multicolumn{1}{c}{ Mass in kg } & \multicolumn{1}{c}{ Distance to  sun in AU } \\
\hline
Earth   & $M_{\mathrm{Earth}}=6\times 10^{24}$ kg     & 1AU                    \\
Jupiter & $M_{\mathrm{Jupiter}}=1.9\times 10^{27}$ kg & 5.20 AU                \\
Mars    & $M_{\mathrm{Mars}}=6.6\times 10^{23}$ kg    & 1.52 AU                \\
Venus   & $M_{\mathrm{Venus}}=4.9\times 10^{24}$ kg   & 0.72 AU                \\
Saturn  & $M_{\mathrm{Saturn}}=5.5\times 10^{26}$ kg  & 9.54 AU                \\
Mercury & $M_{\mathrm{Mercury}}=3.3\times 10^{23}$ kg & 0.39 AU                \\
Uranus  & $M_{\mathrm{Uranus}}=8.8\times 10^{25}$ kg  & 19.19 AU               \\
Neptun  & $M_{\mathrm{Neptun}}=1.03\times 10^{26}$ kg & 30.06 AU               \\
Pluto   & $M_{\mathrm{Pluto}}=1.31\times 10^{22}$ kg  & 39.53 AU               \\
\hline
\end{tabular}
\end{quote}

\noindent
Pluto is no longer considered a planet, but we add it here for
historical reasons. It is optional in this project to include Pluto
and eventual moons.

In setting up the equations we can limit ourselves to a co-planar
motion and use only the $x$ and $y$ coordinates. But you should feel
free to extend your equations to three dimensions, it is not very
difficult and the data from NASA are all in three dimensions.

\href{{http://www.nasa.gov/index.html}}{NASA} has an excellent site at \href{{http://ssd.jpl.nasa.gov/horizons.cgi#top}}{\nolinkurl{http://ssd.jpl.nasa.gov/horizons.cgi\#top}}.
From there you can extract initial conditions in order to start your differential equation solver.
At the above website you need to change from \textbf{OBSERVER} to \textbf{VECTOR} and then write in the planet you are interested in.
The generated data contain the $x$, $y$ and $z$ values as well as their corresponding velocities. The velocities are in units of AU per day.
Alternatively they can be obtained in terms of km and km/s. 


Using our code from homework 4, we will now add Jupyter and play
around with different masses for this planet and study numerically a
three-body problem.  This is a well-studied problem in classical
mechanics, \href{{https://en.wikipedia.org/wiki/Three-body_problem}}{with many interesting results, from stable orbits to chaotic motion}.

\paragraph{Exercise 1 The three-body problem (100pt).}
We will now study the three-body problem, still with the Sun kept
fixed as the center of mass of the system but including Jupiter (the
most massive planet in the solar system, having a mass that is
approximately 1000 times smaller than that of the Sun) together with
the Earth. This leads to a three-body problem. Without Jupiter, the
Earth's motion is stable and unchanging with time. The aim here is to
find out how much Jupiter alters the Earth's motion.

The program you have developed in homework 4 can easily be modified by
simply adding the magnitude of the force betweem the Earth and
Jupiter.

This force is given again by

\[
F_{\mathrm{Earth-Jupiter}}=-\frac{GM_{\mathrm{Jupiter}}M_{\mathrm{Earth}}}{r_{\mathrm{Earth-Jupiter}}^2},
\]

where $M_{\mathrm{Jupiter}}$ is the mass of Jupyter and
$M_{\mathrm{Earth}}$ is the mass of the Earth.  The gravitational constant
is $G$ and $r_{\mathrm{Earth-Jupiter}}$ is the distance between Earth
and Jupiter.

We assume again that the orbits of the two planets are co-planar, and
we take this to be the $xy$-plane (you can easily extend the equations
to three dimensions).

\begin{itemize}
\item 1a (20pt) Modify your coupled first-order differential equations from homework 4 in order to accomodate both the motion of the Earth and Jupiter by taking into account the distance in $x$ and $y$ between the Earth and Jupiter. Write out the differential equations for  Earth and Jupyter, keeping the Sun at rest (mass center of the system).

\item 1b (10pt) Scale these equations in terms of Astronomical Units.

\item 1c (30pt) Set up the algorithm and plot the positions of the Earth and Jupiter using the Velocity Verlet algorithm. Discuss the stability of the solutions using your Velocity Verlet solver.

\item 1c (40pt)  Repeat the calculations by increasing the mass of Jupiter by a factor of 10, 100 and 1000 and plot the position of the Earth.  Discuss your results and study again the stability of the Velocity Verlet solver. Is energy conserved? 
\end{itemize}

\noindent
\paragraph{Exercise 2, the bonus part (50pt).  The perihelion precession of Mercury.}
This is the bonus exercise and gives an additional score of 50
points. It is fully optional. \textbf{I would grade this as a more difficult
exercise compared to previous ones}. \href{{https://en.wikipedia.org/wiki/Tests_of_general_relativity}}{It requires also that you read
some background literature about the perihelion of Mercury}. You don't
need to derive the relativistic correction here. This is something you
will meet in a graduate course on General Relativity. The bonus here
is that it allows you explore physics you could not have done without
a numerical code.

An important test of the general theory of relativity was to compare
its prediction for the perihelion precession of Mercury to the
observed value. The observed value of the perihelion precession, when
all classical effects (such as the perturbation of the orbit due to
gravitational attraction from the other planets) are subtracted, is
$43''$ ($43$ arc seconds) per century.

Closed elliptical orbits are a special feature of the Newtonian
$1/r^2$ force. In general, any correction to the pure $1/r^2$
behaviour will lead to an orbit which is not closed, i.e.~after one
complete orbit around the Sun, the planet will not be at exactly the
same position as it started. If the correction is small, then each
orbit around the Sun will be almost the same as the classical ellipse,
and the orbit can be thought of as an ellipse whose orientation in
space slowly rotates. In other words, the perihelion of the ellipse
slowly precesses around the Sun.

You will now study the orbit of Mercury around the Sun, adding a general relativistic correction to the Newtonian
gravitational force, so that the force becomes

\[
F = -\frac{GM_\mathrm{Sun}M_\mathrm{Mercury}}{r^2}\left[1 + \frac{3l^2}{r^2c^2}\right]
\]

where $M_\mathrm{Mercury}$ is the mass of Mercury, $r$ is the distance
between Mercury and the Sun, $l=|\vec{r}\times\vec{v}|$ is the
magnitude of Mercury's orbital angular momentum per unit mass, and $c$
is the speed of light in vacuum. Run a simulation over one century of
Mercury's orbit around the Sun with no other planets present, starting
with Mercury at the perihelion on the $x$ axis.  Check then the value
of the perihelion angle $\theta_\mathrm{p}$, using

\[
\tan \theta_\mathrm{p} = \frac{y_\mathrm{p}}{x_\mathrm{p}}
\]

where $x_\mathrm{p}$ ($y_\mathrm{p}$) is the $x$ ($y$) position of
Mercury at perihelion, i.e.~at the point where Mercury is at its
closest to the Sun. You may use that the speed of Mercury at
perihelion is $12.44\,\mathrm{AU}/\mathrm{yr}$, and that the distance
to the Sun at perihelion is $0.3075\,\mathrm{AU}$.

You need to make
sure that the time resolution used in your simulation is sufficient,
for example by checking that the perihelion precession you get with a
pure Newtonian force is at least a few orders of magnitude smaller
than the observed perihelion precession of Mercury. Can the observed
perihelion precession of Mercury be explained by the general theory of
relativity?


% ------------------- end of main content ---------------

\end{document}

