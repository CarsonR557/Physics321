%%
%% Automatically generated file from DocOnce source
%% (https://github.com/hplgit/doconce/)
%%
%%
% #ifdef PTEX2TEX_EXPLANATION
%%
%% The file follows the ptex2tex extended LaTeX format, see
%% ptex2tex: http://code.google.com/p/ptex2tex/
%%
%% Run
%%      ptex2tex myfile
%% or
%%      doconce ptex2tex myfile
%%
%% to turn myfile.p.tex into an ordinary LaTeX file myfile.tex.
%% (The ptex2tex program: http://code.google.com/p/ptex2tex)
%% Many preprocess options can be added to ptex2tex or doconce ptex2tex
%%
%%      ptex2tex -DMINTED myfile
%%      doconce ptex2tex myfile envir=minted
%%
%% ptex2tex will typeset code environments according to a global or local
%% .ptex2tex.cfg configure file. doconce ptex2tex will typeset code
%% according to options on the command line (just type doconce ptex2tex to
%% see examples). If doconce ptex2tex has envir=minted, it enables the
%% minted style without needing -DMINTED.
% #endif

% #define PREAMBLE

% #ifdef PREAMBLE
%-------------------- begin preamble ----------------------

\documentclass[%
oneside,                 % oneside: electronic viewing, twoside: printing
final,                   % draft: marks overfull hboxes, figures with paths
10pt]{article}

\listfiles               %  print all files needed to compile this document

\usepackage{relsize,makeidx,color,setspace,amsmath,amsfonts,amssymb}
\usepackage[table]{xcolor}
\usepackage{bm,ltablex,microtype}

\usepackage[pdftex]{graphicx}

\usepackage{ptex2tex}
% #ifdef MINTED
\usepackage{minted}
\usemintedstyle{default}
% #endif

\usepackage[T1]{fontenc}
%\usepackage[latin1]{inputenc}
\usepackage{ucs}
\usepackage[utf8x]{inputenc}

\usepackage{lmodern}         % Latin Modern fonts derived from Computer Modern

% Hyperlinks in PDF:
\definecolor{linkcolor}{rgb}{0,0,0.4}
\usepackage{hyperref}
\hypersetup{
    breaklinks=true,
    colorlinks=true,
    linkcolor=linkcolor,
    urlcolor=linkcolor,
    citecolor=black,
    filecolor=black,
    %filecolor=blue,
    pdfmenubar=true,
    pdftoolbar=true,
    bookmarksdepth=3   % Uncomment (and tweak) for PDF bookmarks with more levels than the TOC
    }
%\hyperbaseurl{}   % hyperlinks are relative to this root

\setcounter{tocdepth}{2}  % levels in table of contents

% prevent orhpans and widows
\clubpenalty = 10000
\widowpenalty = 10000

% --- end of standard preamble for documents ---


% insert custom LaTeX commands...

\raggedbottom
\makeindex
\usepackage[totoc]{idxlayout}   % for index in the toc
\usepackage[nottoc]{tocbibind}  % for references/bibliography in the toc

%-------------------- end preamble ----------------------

\begin{document}

% matching end for #ifdef PREAMBLE
% #endif

\newcommand{\exercisesection}[1]{\subsection*{#1}}


% ------------------- main content ----------------------



% ----------------- title -------------------------

\thispagestyle{empty}

\begin{center}
{\LARGE\bf
\begin{spacing}{1.25}
PHY321: Classical Mechanics 1
\end{spacing}
}
\end{center}

% ----------------- author(s) -------------------------

\begin{center}
{\bf Homework 4, due Monday  February 15${}^{}$} \\ [0mm]
\end{center}

\begin{center}
% List of all institutions:
\end{center}
    
% ----------------- end author(s) -------------------------

% --- begin date ---
\begin{center}
Jan 9, 2021
\end{center}
% --- end date ---

\vspace{1cm}


\paragraph{Practicalities about  homeworks and projects.}
\begin{enumerate}
\item You can work in groups (optimal groups are often 2-3 people) or by yourself. If you work as a group you can hand in one answer only if you wish. \textbf{Remember to write your name(s)}!

\item Homeworks are available ten days before the deadline.

\item How do I(we)  hand in?  You can hand in the paper and pencil exercises as a scanned document. For this homework this applies to exercises 1-5. Alternatively, you can hand in everyhting (if you are ok with typing mathematical formulae using say Latex) as a jupyter notebook at D2L. The numerical exercise(s) (exercise 6 here) should always be handed in as a jupyter notebook by the deadline at D2L. 
\end{enumerate}

\noindent
\paragraph{Introduction to homework 4.}
This week's sets of classical pen and paper and computational
exercises deal with simple motion problems and conservation laws; energy, momentum and angular momentum. These conservation laws are central in Physics and understanding them properly lays the foundation for understanding and analyzing more complicated physics problems.
The relevant reading background is
\begin{enumerate}
\item chapters 3, 4.1, 4.2 and 4.3 of Taylor (there are many good examples there) and

\item chapters 10-13 of Malthe-Sørenssen.
\end{enumerate}

\noindent
In both textbooks there are many nice worked out examples. Malthe-Sørenssen's text contains also several coding examples you may find useful. 

The numerical homework focuses on another motion problem where you can
use the code you developed in homework 3, almost entirely. Please take
a look at the posted solution (jupyter-notebook) for homework 3. You
need only to change the forces at play. The problem at hand is a
classic, the gravitational force acting between the Sun and the
Earth. Here you will notice also that the standard Euler-integration
algorithm is not the best choice and we will introduce the so-called
Euler-Cromer method and the Velocity-Verlet method. These methods will
give much more stable numerical results with only few additions to
your code.

The code you develop here will also be reused when we analyze energy
conservation in homework set 5. And for those of you doing the honors project, it serves as a starting point for the solar system variant.

\paragraph{Exercise 1 (10 pt), Conservation laws, Energy and momentum.}
\begin{itemize}
\item 1a (2pt) How do we define a conservative force?

\item 1b (4pt) Use the work-energy theorem to show that energy is conserved with a conservative force.

\item 1c (4pt) Assume that you have only internal two-body forces acting on $N$ objects in an isolated system. The force from object $i$ on object $j$ is $\bm{F}_{ij}$. Show that the linear momentum is conserved.
\end{itemize}

\noindent
\paragraph{Exercise 2 (10 pt), Conservation of angular momentum.}
\begin{itemize}
\item 2a (2pt) Define angular momentum and the torque for a single object with external forces only. 

\item 2b (4pt) Define angular momentum and the torque for a system with $N$ objects/particles  with external and internal forces. The force from object $i$ on object $j$ is $\bm{F}_{ij}$.

\item 2c (4pt) With internal forces only, what is the mathematical form of the forces that allows for angular momentum to be conserved? 
\end{itemize}

\noindent
\paragraph{Exsercise 3 (10pt), Example of potential.}
Consider a particle of mass $m$ moving according to the potential
\[
V(x,y,z)=A\exp\left\{-\frac{x^2+z^2}{2a^2}\right\}.
\]

\begin{itemize}
\item 3a (2pt) Is energy conserved? If so, why? 

\item 3b (4pt) Which of  the quantities, $p_x,p_y,p_z$ are conserved?

\item 3c (4pt) Which of  the quantities, $L_x,L_y,L_z$ are conserved?
\end{itemize}

\noindent
\paragraph{Exercise 4 (10pt), Angular momentum case.}
At $t=0$ we have a single object with position $\bm{r}_0=x_0\bm{e}_x+y_0\bm{e}_y$.We add also a force in the $x$-direction at $t=0$. We assume that the object is at rest at $t=0$. 
\[
\bm{F} = F\bm{e}_x.
\]

\begin{itemize}
\item 4a (3pt) Find the velocity and momentum at a given time $t$ by integrating over time with the above initial conditions.

\item 4b (3pt) Find also the position at a time $t$. 

\item 4c (4pt) Use the position and the momentum to find the angular momentum and the torque. Is angular momentum conserved?
\end{itemize}

\noindent
\paragraph{Exercise 5 (10pt), forces  and potentials.}
A particle of mass $m$ has velocity $v=\alpha/x$, where $x$ is its displacement.

\begin{itemize}
\item 5a (3pt) Find the force $F(x)$ responsible for the motion.
\end{itemize}

\noindent
A particle is thereafter under the influence of a force $F=-kx+kx^3/\alpha^2$, where $k$ and $\alpha$ are constants and $k$ is positive.

\begin{itemize}
\item 5b (3pt) Determine $U(x)$  and discuss the motion. It can be convenient here to make a sketch/plot of the potential as function of $x$.

\item 5c (4pt)  What happens when the energy of the particle is $E=(1/4)k\alpha^2$? Hint: what is the maximum value of the potential energy?
\end{itemize}

\noindent
\paragraph{Exercise 6 (10pt), Warming up for the Earth-Sun system, Scaling the Equations.}
The aim of this exercise (as well as the next) is to study the motion
of objects under the influence of the gravitational force.  We will
limit ourselves to the Earth-Sun system. Here we will scale the
equations and sketch our first algorithm for solving the equations,
namely using Euler's method again, as we did in homework 3.  This part
together with the numerical part forms also the entry point for the
solar system honors project. Furthermore, we will reuse parts of these
results when analyzing energy conservation in homework 5.

We will limit ourselves (in order to test the algorithm) to a
hypothetical solar system with the Earth only orbiting around the sun.
The only force in the problem is gravity. Newton's law of gravitation
is given by a force $F_G$

\[
F_G=\frac{GM_{\odot}M_{\mathrm{Earth}}}{r^2},
\]

where $M_{\odot}$ is the mass of the Sun and $M_{\mathrm{Earth}}$ is
the mass of the Earth. The gravitational constant is $G$ and $r$ is
the distance between the Earth and the Sun.  The Sun
has a mass which is much larger than that of the Earth. We can
therefore safely neglect the motion of the Sun in this problem.


We assume that the orbit of the Earth around the Sun 
is co-planar, and we take this to be the $xy$-plane.
Using Newton's second law of motion we get the following equations
\[
\frac{d^2x}{dt^2}=\frac{F_{G,x}}{M_{\mathrm{Earth}}},
\]
and
\[
\frac{d^2y}{dt^2}=\frac{F_{G,y}}{M_{\mathrm{Earth}}},
\]

where $F_{G,x}$ and $F_{G,y}$ are the $x$ and $y$ components of the
gravitational force.

We will use so-called astronomical units when rewriting our equations.
Using astronomical units (AU as abbreviation)it means that one
astronomical unit of length, known as 1 AU, is the average distance
between the Sun and Earth, that is $1$ AU = $1.5\times 10^{11}$ m.  It
can also be convenient to use years instead of seconds since years
match better the time evolution of the solar system. The mass of the
Sun is $M_{\mathrm{sun}}=M_{\odot}=2\times 10^{30}$ kg. The masses of
all relevant planets and their distances from the sun are listed in
the table here in kg and AU.


\begin{quote}
\begin{tabular}{ccc}
\hline
\multicolumn{1}{c}{ Planet } & \multicolumn{1}{c}{ Mass in kg } & \multicolumn{1}{c}{ Distance to  sun in AU } \\
\hline
Earth   & $M_{\mathrm{Earth}}=6\times 10^{24}$ kg     & 1AU                    \\
Jupiter & $M_{\mathrm{Jupiter}}=1.9\times 10^{27}$ kg & 5.20 AU                \\
Mars    & $M_{\mathrm{Mars}}=6.6\times 10^{23}$ kg    & 1.52 AU                \\
Venus   & $M_{\mathrm{Venus}}=4.9\times 10^{24}$ kg   & 0.72 AU                \\
Saturn  & $M_{\mathrm{Saturn}}=5.5\times 10^{26}$ kg  & 9.54 AU                \\
Mercury & $M_{\mathrm{Mercury}}=3.3\times 10^{23}$ kg & 0.39 AU                \\
Uranus  & $M_{\mathrm{Uranus}}=8.8\times 10^{25}$ kg  & 19.19 AU               \\
Neptun  & $M_{\mathrm{Neptun}}=1.03\times 10^{26}$ kg & 30.06 AU               \\
Pluto   & $M_{\mathrm{Pluto}}=1.31\times 10^{22}$ kg  & 39.53 AU               \\
\hline
\end{tabular}
\end{quote}

\noindent

In setting up the equations we limit ourselves to a co-planar
motion and use only the $x$ and $y$ coordinates. But you should feel
free to extend your equations to three dimensions, it is not very
difficult and the data from NASA are all in three dimensions.

\href{{http://www.nasa.gov/index.html}}{NASA} has an excellent site at \href{{http://ssd.jpl.nasa.gov/horizons.cgi#top}}{\nolinkurl{http://ssd.jpl.nasa.gov/horizons.cgi\#top}}.
From there you can extract initial conditions in order to start your differential equation solver.
At the above website you need to change from \textbf{OBSERVER} to \textbf{VECTOR} and then write in the planet you are interested in.
The generated data contain the $x$, $y$ and $z$ values as well as their corresponding velocities. The velocities are in units of AU per day.
Alternatively they can be obtained in terms of km and km/s. 

For the system below involving only the Earth and the Sun, you
could just initialize the position with say $x=1$ AU and $y=0$ AU.



We assume that mass units can be obtained by using the fact that Earth's orbit is almost circular around the Sun.

For circular motion we know that the force must obey the following relation
\[
F_G= \frac{M_{\mathrm{Earth}}v^2}{r}=\frac{GM_{\odot}M_{\mathrm{Earth}}}{r^2},
\]
where $v$ is the velocity of Earth. 
The latter equation can be used to show that
\[
v^2r=GM_{\odot}=4\pi^2\mathrm{AU}^3/\mathrm{yr}^2.
\]
\begin{itemize}
\item 6a (5pt) Show how to derive the last equation and use this to scale the differential equations, getting thus rid of the constant $G$ and the two masses. Split the differential equations for the motion in the $x$ and $y$ directions in terms of four coupled differential equations.

\item 6b (5pt)  Discretize the above differential equations and set up an algorithm for solving these equations using Euler's forward algorithm and the so-called velocity Verlet method \href{{https://mhjensen.github.io/Physics321/doc/pub/energyconserv/html/energyconserv.html}}{discussed in the lecture notes}. Here you can reuse what you did in homework 3, exercises 6 and 7. 
\end{itemize}

\noindent
\paragraph{Exercise 7 (40pt), Numerical elements, solving exercise 6 numerically.}
\textbf{This exercise should be handed in as a jupyter-notebook} at D2L. Remember to write your name(s). 

Last week we:
\begin{enumerate}
\item Studied  Euler's Method to find the position and the velocity of a falling object, including air resistance and gravity

\item Compared analytical and computational solutions 

\item Added additional forces to our model
\end{enumerate}

\noindent
This week we will reuse our code from homework 3 (exercises 6 and 7,
\href{{https://github.com/mhjensen/Physics321/tree/master/doc/src/motion/Programs}}{see also the code examples at the course
homepage})
and replace the air resistance and force from the ground with the
gravitational force. Then we will study the stability of the system as function of initial conditions and the time step length.

We start by importing some standard packages
\bpycod
# let's start by importing useful packages we are familiar with
import numpy as np
from math import *
import matplotlib.pyplot as plt
%matplotlib inline
\epycod

\begin{itemize}
\item 7a (5 pt)  Write then a program which solves the above differential equations for the Earth-Sun system using Euler's  method and the velocity Verlet method.  Find out which initial value for the velocity that gives a circular orbit and test the stability of your algorithm as function of different time steps $\Delta t$.  Make a plot of the results you obtain for the position of the Earth (plot the $x$ and $y$ values and/or if you prefer to use three dimensions the $z$-value as well) orbiting  the Sun. Discuss eventual differences between the Verlet algorithm and the Euler algorithm. 

\item 7b (5pt) Consider then a planet which begins at a distance of 1 AU from the sun. Find out by trial and error what the initial velocity must be in order for the planet to escape from the sun.  Can you find an exact answer?  How does that match your numerical results?
\end{itemize}

\noindent
Here we add a code-example which may aid in the above studies using Euler's forward method.
\bpycod
DeltaT = 0.001
#set up arrays 
tfinal = 10 # in years
n = ceil(tfinal/DeltaT)
# set up arrays for time t, velocity v, and position r
t = np.zeros(n)
v = np.zeros((n,2))
r = np.zeros((n,2))
# Initial conditions as compact 2-dimensional arrays
r0 = np.array([1.0,0.0])
v0 = np.array([0.0,2*pi])
r[0] = r0
v[0] = v0
Fourpi2 = 4*pi*pi
# Start integrating using Euler's method
for i in range(n-1):
    # Set up the acceleration
    # Here you could have defined your own function for this
    rabs = sqrt(sum(r[i]*r[i]))
    a =  -Fourpi2*r[i]/(rabs**3)
    # update velocity, time and position using Euler's forward method
    v[i+1] = v[i] + DeltaT*a
    r[i+1] = r[i] + DeltaT*v[i]
    t[i+1] = t[i] + DeltaT
\epycod


\paragraph{Exercise 8 (30pt), Bonus exercise.}
You don't need to do this exercise, but it gives you a bonus score of 30 points.

This time the bonus exercise is a simple survey. We are now moving
into our sixth week and we would very much have your feedback on how
things are functioning. In particular, since this is (together with fall
2019) the first time we have numerical exercises every week, we need
to see if the pace is ok, if there is a good link between lectures and
the homework exrcises etc etc.

\begin{itemize}
\item Is the weekly load with paper and pencil exercises and the numerical exercises reasonable?

\item Is there enough material (lectures and lecture material) to get you started with the exercises?  We are thinking of both the paper and pencil and the numerical exercises?

\item Is the pace during the lectures reasonable? And do the lectures link well with the exercises?

\item In the beginning there will always be some elements of repetition of material many of you have seen before. Do you find the choice of material the first weeks adequate? Too easy? Too difficult?

\item For those of you who have taken CMSE 201 Introduction to Computational Modeling, do you feel the material taught there links well with the exercises you have done in this course? Is there is something we are missing?

\item Any other topic you would like to comment on?
\end{itemize}

\noindent

% ------------------- end of main content ---------------

% #ifdef PREAMBLE
\end{document}
% #endif

