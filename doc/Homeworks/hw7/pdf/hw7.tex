%%
%% Automatically generated file from DocOnce source
%% (https://github.com/hplgit/doconce/)
%%
%%


%-------------------- begin preamble ----------------------

\documentclass[%
oneside,                 % oneside: electronic viewing, twoside: printing
final,                   % draft: marks overfull hboxes, figures with paths
10pt]{article}

\listfiles               %  print all files needed to compile this document

\usepackage{relsize,makeidx,color,setspace,amsmath,amsfonts,amssymb}
\usepackage[table]{xcolor}
\usepackage{bm,ltablex,microtype}

\usepackage[pdftex]{graphicx}

\usepackage{fancyvrb} % packages needed for verbatim environments

\usepackage[T1]{fontenc}
%\usepackage[latin1]{inputenc}
\usepackage{ucs}
\usepackage[utf8x]{inputenc}

\usepackage{lmodern}         % Latin Modern fonts derived from Computer Modern

% Hyperlinks in PDF:
\definecolor{linkcolor}{rgb}{0,0,0.4}
\usepackage{hyperref}
\hypersetup{
    breaklinks=true,
    colorlinks=true,
    linkcolor=linkcolor,
    urlcolor=linkcolor,
    citecolor=black,
    filecolor=black,
    %filecolor=blue,
    pdfmenubar=true,
    pdftoolbar=true,
    bookmarksdepth=3   % Uncomment (and tweak) for PDF bookmarks with more levels than the TOC
    }
%\hyperbaseurl{}   % hyperlinks are relative to this root

\setcounter{tocdepth}{2}  % levels in table of contents

% prevent orhpans and widows
\clubpenalty = 10000
\widowpenalty = 10000

% --- end of standard preamble for documents ---


% insert custom LaTeX commands...

\raggedbottom
\makeindex
\usepackage[totoc]{idxlayout}   % for index in the toc
\usepackage[nottoc]{tocbibind}  % for references/bibliography in the toc

%-------------------- end preamble ----------------------

\begin{document}

% matching end for #ifdef PREAMBLE

\newcommand{\exercisesection}[1]{\subsection*{#1}}


% ------------------- main content ----------------------



% ----------------- title -------------------------

\thispagestyle{empty}

\begin{center}
{\LARGE\bf
\begin{spacing}{1.25}
PHY321: Classical Mechanics 1
\end{spacing}
}
\end{center}

% ----------------- author(s) -------------------------

\begin{center}
{\bf Homework 7, due Monday  March 23${}^{}$} \\ [0mm]
\end{center}

\begin{center}
% List of all institutions:
\end{center}
    
% ----------------- end author(s) -------------------------

% --- begin date ---
\begin{center}
Mar 17, 2020
\end{center}
% --- end date ---

\vspace{1cm}


\paragraph{Practicalities about  homeworks and projects.}
\begin{enumerate}
\item You can work in groups (optimal groups are often 2-3 people) or by yourself. If you work as a group you can hand in one answer only if you wish. \textbf{Remember to write your name(s)}!

\item Homeworks are available  the week before the deadline. 

\item How do I(we)  hand in?  Due to the corona virus and many of you not being on campus, we recommend that you scan your handwritten notes and upload them to D2L. If you are ok with typing mathematical formulae using say Latex, you can hand in everything as a single jupyter notebook at D2L. The numerical exercise(s) should always be handed in as a jupyter notebook by the deadline at D2L. 
\end{enumerate}

\noindent
\paragraph{Introduction to homework 7.}
This week's sets of classical pen and paper and computational
exercises are tailored to the topic of two-body problems and central
forces, following what was discussed during the lectures during week
11 (March 9-13) and week 12 (March 16-20). Some of the exercise are
simple derivations of the equations we discussed during lectures while
other are applications to specific forces. Plotting the energy-diagram
is again a very useful exercise. And you will see that many of the
classical topics like conservation laws, conservative forces, central
forces, energy, momentum and angular momentum conservation pop up
again. These are indeed overarching features which allow us to develop
our intuition about a given physical system.

The relevant reading background is
\begin{enumerate}
\item Sections 8.1-8.7 of Taylor.
\end{enumerate}

\noindent
The numerical homework is based on what you did in homework 4 on the Earth-Sun system, but we will add to this the numerical solution of elliptical orbits. 


\paragraph{Exercise 1 (20pt), Center-of-Mass and Relative Coordinates and Reference Frames.}
We define the two-body center-of-mass coordinate and relative coordinate by expressing the trajectories for
$\bm{r}_1$ and $\bm{r}_2$ into the center-of-mass coordinate
$\bm{R}_{\rm cm}$ 
\[
\bm{R}_{\rm cm}\equiv\frac{m_1\bm{r}_1+m_2\bm{r}_2}{m_1+m_2},
\]
and the relative coordinate 
\[
\bm{r}\equiv\bm{r}_1-\bm{r_2}.
\]
Here, we assume the two particles interact only with one another, so $\bm{F}_{12}=-\bm{F}_{21}$ (where $\bm{F}_{ij}$ is the force on $i$ due to $j$.

\begin{itemize}
\item 1a (5pt) Show that the equations of motion then become $\ddot{\bm{R}}_{\rm cm}=0$ and $\ddot{\bm{r}}=\mu\bm{F}_{12}$, with the reduced mass $\mu=m_1m_2/(m_1+m_2)$.
\end{itemize}

\noindent
The first expression simply states that the center of mass coordinate $\bm{R}_{\rm cm}$ moves at a fixed velocity. The second expression can be rewritten in terms of the reduced mass $\mu$.

\begin{itemize}
\item 1b (5pt) Show that the linear momenta for the center-of-mass $\bm{P}$ motion and the relative motion $\bm{q}$ are given by $\bm{P}=M\dot{\bm{R}}_{\rm cm}$ with $M=m_1+m_2$ and $\bm{q}=\mu\dot{\bm{r}}$.

\item 1c (5pt) Show then the kinetic energy for two objects can then be written as
\end{itemize}

\noindent
\[
K=\frac{P^2}{2M}+\frac{q^2}{2\mu}.
\]

\begin{itemize}
\item 1d (5pt) Show that the total angular momentum for two-particles in the center-of-mass frame $\bm{R}=0$, is given by
\end{itemize}

\noindent
\[
\bm{L}=\bm{r}\times \mu\dot{\bm{r}}.
\]

\paragraph{Exercise 2 (10pt), Conservation of Energy.}
The equations of motion in the center-of-mass frame in two dimension with $x=r\cos{(\phi)}$ and $y=r\sin{(\phi)}$ and
$r\in [0,\infty)$, $\phi\in [0,2\pi]$ and $r=\sqrt{x^2+y^2}$ are given by
\[
\mu \ddot{r}=-\frac{dV(r)}{dr}+\mu\dot{\phi}^2,
\]
and
\[
\dot{\phi}=\frac{L}{\mu r^2}.
\]
Here $V(r)$ is any central force which depends only on the relative coordinate.
\begin{itemize}
\item 2a (5pt) Show that you can rewrite the radial equation in terms of an effective potential $V_{\mathrm{eff}}(r)=V(r)+L^2/(2\mu r^2)$. 

\item 2b (5pt) Write out the final differential equation for the radial degrees of freedom when we specify that $V(r)=-\alpha/r$.  Plot the effective potential (choose values for $\alpha$ and $L$ and discuss (see Taylor section 8.4 and example 8.2) the physics of the system for two energies, one larger than zero and one smaller than zero. This is similar to what you did in the first midterm, except that the potential is different.
\end{itemize}

\noindent
\paragraph{Exercise 3 (40pt), Harmonic oscillator again.}
See the lecture notes on central forces for a discussion of this problem. It is given as an example in the text.

Consider a particle of mass $m$ in a $2$-dimensional harmonic oscillator with potential
\[
V(r)=\frac{1}{2}kr^2=\frac{1}{2}k(x^2+y^2).
\]

We assume the orbit has a final non-zero angular momentum $L$.  The
effective potential looks like that of a harmonic oscillator for large
$r$, but for small $r$, the centrifugal potential repels the particle
from the origin. The combination of the two potentials has a minimum
for at some radius $r_{\rm min}$.


\begin{itemize}
\item 3a (10pt) Set up the effective potential and plot it. Find $r_{\rm min}$ and $\dot{\phi}$. Show that the latter is given by $\dot{\phi}=\sqrt{k/m}$.  At $r_{\rm min}$ the particle does not accelerate and $r$ stays constant and the motion is circular. With fixed $k$ and $m$, which parameter can we adjust to change the value of $r$ at $r_{\rm min}$?

\item 3b (10pt) Now consider small vibrations about $r_{\rm min}$. The effective spring constant is the curvature of the effective potential.  Use the curvature at $r_{\rm min}$ to find the effective spring constant (hint, look at  exercise 4 in homework 6) $k_{\mathrm{eff}}$. Show also that $\omega=\sqrt{k_{\mathrm{eff}}/m}=2\dot{\phi}$  

\item 3c (10pt) The solution to the equations of motion in Cartesian coordinates is simple. The $x$ and $y$ equations of motion separate, and we have $\ddot{x}=-kx$ and $\ddot{y}=-ky$. The harmonic oscillator is indeed a system where the degrees of freedom separate and we can find analytical solutions. Define a natural frequency $\omega_0=\sqrt{k/m}$ and show that (where $A$, $B$, $C$ and $D$ are arbitrary constants defined by the initial conditions)
\end{itemize}

\noindent
\begin{eqnarray*}
x&=&A\cos\omega_0 t+B\sin\omega_0 t,\\
y&=&C\cos\omega_0 t+D\sin\omega_0 t.
\end{eqnarray*}

\begin{itemize}
\item 3d (10pt) With the solutions for $x$ and $y$, and $r^2=x^2+y^2$ and the definitions $\alpha=\frac{A^2+B^2+C^2+D^2}{2}$, $\beta=\frac{A^2-B^2+C^2-D^2}{2}$ and $\gamma=AB+CD$, show that
\end{itemize}

\noindent
\[
r^2=\alpha+(\beta^2+\gamma^2)^{1/2}\cos(2\omega_0 t-\delta),
\]
with
\[
\delta=\arctan(\gamma/\beta),
\]


\paragraph{Exercise 4 (30pt), Numerical Solution of the Harmonic Oscillator.}
Using the code we developed in homework 4 for the Earth-Sun system, we can solve the above harmonic oscillator problem in two dimensions using our code from this homework. We need however to change the acceleration from the gravitational force to the one given by the harmonic oscillator potential.

The code is given here for the Velocity-Verlet algorithm, with obvious elements to fill in. 

% !split
\subsection*{Code Example with Euler's Method}

The code here implements Euler's method for the Earth-Sun system using a more compact way of representing the vectors. Alternatively, you could have spelled out all the variables $v_x$, $v_y$, $x$ and $y$ as one-dimensional arrays. Note that we have not specified the acceleration and the initial conditions. These need to be added by you. 

\begin{verbatim}
# Common imports
import numpy as np
import pandas as pd
from math import *
import matplotlib.pyplot as plt
import os

# Where to save the figures and data files
PROJECT_ROOT_DIR = "Results"
FIGURE_ID = "Results/FigureFiles"
DATA_ID = "DataFiles/"

if not os.path.exists(PROJECT_ROOT_DIR):
    os.mkdir(PROJECT_ROOT_DIR)

if not os.path.exists(FIGURE_ID):
    os.makedirs(FIGURE_ID)

if not os.path.exists(DATA_ID):
    os.makedirs(DATA_ID)

def image_path(fig_id):
    return os.path.join(FIGURE_ID, fig_id)

def data_path(dat_id):
    return os.path.join(DATA_ID, dat_id)

def save_fig(fig_id):
    plt.savefig(image_path(fig_id) + ".png", format='png')

DeltaT = 0.01
#set up arrays 
tfinal = 10.0
n = ceil(tfinal/DeltaT)
# set up arrays for t, a, v, and x
t = np.zeros(n)
v = np.zeros((n,2))
r = np.zeros((n,2))
# Initial conditions as compact 2-dimensional arrays
r0 = np.array([1.0,0.5])  # You must change these to fit rmin
v0 = np.array([0.0,0.0]) # You must change these to fit rmin
r[0] = r0
v[0] = v0
k = 0.1   # spring constant
m = 0.1   # mass, you can change these
omega02 = sqrt(k/m)
# Start integrating using the Velocity-Verlet  method
for i in range(n-1):
    # Set up forces, define acceleration first
    a =  -r[i]*omega02  # you may need to change this
    # update velocity, time and position using the Velocity-Verlet method
    r[i+1] = r[i] + DeltaT*v[i]+0.5*(DeltaT**2)*a
    # new accelerationfor the Verlet method
    anew = -r[i+1]*omega02  
    v[i+1] = v[i] + 0.5*DeltaT*(a+anew)
    t[i+1] = t[i] + DeltaT
# Plot position as function of time    
fig, ax = plt.subplots()
ax.set_xlabel('t[s]')
ax.set_ylabel('x[m] and y[m]')
ax.plot(t,r[:,0])
ax.plot(t,r[:,1])
fig.tight_layout()
save_fig("2DimHOVV")
plt.show()

\end{verbatim}

\begin{itemize}
\item 4a (15pt) Use for example the above code to set up the acceleration and use the inital conditions fixed by for example $r_{\rm min}$ from exercise 3. Which value does the initial velocity take if you place yourself at $r_{\rm min}$?  Check the solutions as function of  different initial conditions (one set suffices) and compare with the analytical solutions for $x$ and $y$. Check also that energy is conserved.
\end{itemize}

\noindent
Instead of solving the equations in the cartesian frame we will now rewrite the above code in terms of the radial degrees of freedom only. Our differential equation is now
\[
\mu \ddot{r}=-\frac{dV(r)}{dr}+\mu\dot{\phi}^2,
\]
and
\[
\dot{\phi}=\frac{L}{\mu r^2}.
\]

\begin{itemize}
\item 4b (15pt) We will use $r_{\rm min}$ to fix a value of $L$, as seen in exercise 3. This fixes also $\dot{\phi}$. Write a code which now implements the radial equation for $r$ using the same $r_{\rm min}$ as you did in 4a. Compare the results with those from 4a with the same initial conditions. Do they agree? Use only one set of initial conditions.
\end{itemize}

\noindent
\paragraph{Exercise 5, the Bonus Exercise (30pt).}
As in previous sets, this exercise is not compulsory but gives you a bonus of 30 points.
The aim here is to compare the numerical result from 4b with the analytical ones  listed in 3d. Do your numerical results agree with the analytical ones?  And is energy conserved?  For this you need to set up the expression for the energy in terms of the effective potential and the kinetic energy in terms of $r$. 

% ------------------- end of main content ---------------

\end{document}

