%%
%% Automatically generated file from DocOnce source
%% (https://github.com/hplgit/doconce/)
%%
%%
% #ifdef PTEX2TEX_EXPLANATION
%%
%% The file follows the ptex2tex extended LaTeX format, see
%% ptex2tex: http://code.google.com/p/ptex2tex/
%%
%% Run
%%      ptex2tex myfile
%% or
%%      doconce ptex2tex myfile
%%
%% to turn myfile.p.tex into an ordinary LaTeX file myfile.tex.
%% (The ptex2tex program: http://code.google.com/p/ptex2tex)
%% Many preprocess options can be added to ptex2tex or doconce ptex2tex
%%
%%      ptex2tex -DMINTED myfile
%%      doconce ptex2tex myfile envir=minted
%%
%% ptex2tex will typeset code environments according to a global or local
%% .ptex2tex.cfg configure file. doconce ptex2tex will typeset code
%% according to options on the command line (just type doconce ptex2tex to
%% see examples). If doconce ptex2tex has envir=minted, it enables the
%% minted style without needing -DMINTED.
% #endif

% #define PREAMBLE

% #ifdef PREAMBLE
%-------------------- begin preamble ----------------------

\documentclass[%
oneside,                 % oneside: electronic viewing, twoside: printing
final,                   % draft: marks overfull hboxes, figures with paths
10pt]{article}

\listfiles               %  print all files needed to compile this document

\usepackage{relsize,makeidx,color,setspace,amsmath,amsfonts,amssymb}
\usepackage[table]{xcolor}
\usepackage{bm,ltablex,microtype}

\usepackage[pdftex]{graphicx}

\usepackage[T1]{fontenc}
%\usepackage[latin1]{inputenc}
\usepackage{ucs}
\usepackage[utf8x]{inputenc}

\usepackage{lmodern}         % Latin Modern fonts derived from Computer Modern

% Hyperlinks in PDF:
\definecolor{linkcolor}{rgb}{0,0,0.4}
\usepackage{hyperref}
\hypersetup{
    breaklinks=true,
    colorlinks=true,
    linkcolor=linkcolor,
    urlcolor=linkcolor,
    citecolor=black,
    filecolor=black,
    %filecolor=blue,
    pdfmenubar=true,
    pdftoolbar=true,
    bookmarksdepth=3   % Uncomment (and tweak) for PDF bookmarks with more levels than the TOC
    }
%\hyperbaseurl{}   % hyperlinks are relative to this root

\setcounter{tocdepth}{2}  % levels in table of contents

% prevent orhpans and widows
\clubpenalty = 10000
\widowpenalty = 10000

% --- end of standard preamble for documents ---


% insert custom LaTeX commands...

\raggedbottom
\makeindex
\usepackage[totoc]{idxlayout}   % for index in the toc
\usepackage[nottoc]{tocbibind}  % for references/bibliography in the toc

%-------------------- end preamble ----------------------

\begin{document}

% matching end for #ifdef PREAMBLE
% #endif

\newcommand{\exercisesection}[1]{\subsection*{#1}}


% ------------------- main content ----------------------



% ----------------- title -------------------------

\thispagestyle{empty}

\begin{center}
{\LARGE\bf
\begin{spacing}{1.25}
PHY321: Classical Mechanics 1
\end{spacing}
}
\end{center}

% ----------------- author(s) -------------------------

\begin{center}
{\bf Second midterm project, due Friday April 17${}^{}$} \\ [0mm]
\end{center}

\begin{center}
% List of all institutions:
\end{center}
    
% ----------------- end author(s) -------------------------

% --- begin date ---
\begin{center}
Apr 6, 2020
\end{center}
% --- end date ---

\vspace{1cm}


\paragraph{Practicalities about  homeworks and projects.}
\begin{enumerate}
\item You can work in groups (optimal groups are often 2-3 people) or by yourself. If you work as a group you can hand in one answer only if you wish. \textbf{Remember to write your name(s)}!

\item How do I(we)  hand in?  Due to the extraordinary situation we are in now, the midterm should be handed in fully via D2L. You can scan your handwritten notes and upload to D2L or you can hand in everyhting (if you are ok with typing mathematical formulae using say Latex) as a jupyter notebook at D2L. The numerical part should always be handed in as a jupyter notebook.
\end{enumerate}

\noindent
\paragraph{Introduction to the second midterm project, total score 100 points.}
In this midterm we will attempt at writing a program that simulates
the solar system. We start with the Earth-Sun system we studied in
homework 4 and study elliptical orbits and their properties. We test
also elliptical orbits and their dependence on powers $\beta$ of
$r^{\beta}$. We will test other aspects of the Earth-Sun system and
link these to the theoretical discussion on two-body problems with
central forces. Here we will need the results from homeworks 7 and 8.

Thereafter, based on the three-body problem studied in homework 9, we
attempt at making a code which simulates the solar system.

The relevant reading background is
\begin{enumerate}
\item chapter 8 of  Taylor.

\item Lecture notes on central forces and two-body problems

\item Homeworks 4, 7, 8 and 9
\end{enumerate}

\noindent
\paragraph{Part 1, the inverse-square law and the stability of planetary orbits.}
In homework 8 (exercises 2 and 3) we studied an attractive potential
\[
V(r)=-\alpha/r,
\]

where the quantity $r$ is the absolute value of the relative position and $\alpha$ is a constant.

When we rewrote the equations of motion in polar coordinates, the analytical solution to the radial equation of motion was
\[
r(\phi) = \frac{c}{1+\epsilon\cos{(\phi)}},
\]
where $c=L^2/\mu\alpha$, with
the reduced mass $\mu$ and the angular momentum $L$, as
discussed during the lectures. With the transformation of a two-body
problem to the center-of-mass frame, the actual equations look like an
\emph{effective} one-body problem. The energy of the system is $E$ and the
minimum of the effective potential is $r_{\rm min}$.

The quantity $\epsilon$ is what we called the eccentricity. Since we will mainly study bounded orbits,
we have $0 \le \epsilon < 1$.

In this part we will limit ourselves to the Earth-Sun system we studied in homework 4. You can reuse your code with either the Velocity-Verlet or the Euler-Cromer algorithms from homework 4.

This means also that $\alpha=GM_{\odot}M_{\mathrm{Earth}}$. We will use $\alpha$ as a shorthand in the equations here. Keep in mind that in homework 4 you scaled $GM_{\odot}=4\pi^2$ in our ycode. When comparing your numerical results with your analytical results here, you need to keep track of this.

\begin{itemize}
\item 1a (10pt) Show that the semimajor axis of the ellipse is $a=L^2/(\mu\alpha(1-\epsilon^2)$. Make a drawing of an ellipse with the semi-major and semi-minor axes and the foci of the ellipse. We would place the sun in one of the foci.  

\item 1b (10pt) Show that the closest point to the Sun (the perihelion) is $r_{\mathrm{min}}=a(1-\epsilon)$ and the farthest point (aphelion) is $r_{\mathrm{min}}=a(1+\epsilon)$. Here $a$ is the semi-major axis.

\item 1c (10pt) Angular momentum is conserved for a central and symmetric force.  From the expression of the semi-major axis we have that $L=\sqrt{a\mu\alpha(1-\epsilon^2)}$. Use conservation of angular momentum with $\mu r_{\mathrm{min}}v_{\mathrm{max}}$ and $\mu r_{\mathrm{max}}v_{\mathrm{min}}$ to show that
\end{itemize}

\noindent
\[
v_{\mathrm{min}}=\sqrt{GM_{\odot}}\sqrt{\frac{(1+\epsilon)}{a(1-\epsilon)}(1+\frac{M_{\mathrm{Earth}}}{M_{\odot}})},
\]
and 
\[
v_{\mathrm{max}}=\sqrt{GM_{\odot}}\sqrt{\frac{(1-\epsilon)}{a(1+\epsilon)}(1+\frac{M_{\mathrm{Earth}}}{M_{\odot}})},
\]

\begin{itemize}
\item 1d (10pt) Use now your code from homework 4 (in cartesian coordinates) and play around with 2-3 different values of $\epsilon$ and $L$ and set the initial conditions for velocity and position using the above min and max positions and velocities. Plot $x$ versus $y$ as function of time and discuss the orbits you get. Try to extract the perihelion from your numerical results. You could start with a circular orbit setting $\epsilon=0$. For the Earth, the orbit is close to circular and at perihelion, the Earth's center is about 0.98329 astronomical units (AU) or 147,098,070 km from the Sun's center. The outer planets have more elliptical orbits. For example, Mars has its perihelion at 206,655,215 km and its apehelion at 249,232,432 km. For Earth, the orbital eccentricity is $\epsilon\approx 0.0167$. 

\item 1e (10pt) Till now we have assumed that we have an inverse-square force $F(r) = -\alpha/r^2$. Let us rewrite this force as $F(r) = -\alpha/r^{\beta}$ with $\beta=[2,2.01,2.10,2.5,3.0,3.5]$. Run your Sun-Earth code from homework 4 with these values of $\beta$ and plot $x$ versus $y$. Discuss your results. Could you use observations of planetary motion to determine by what amount Nature deviates from a perfect inverse-square law? Here you run with just one set of initial values. 
\end{itemize}

\noindent
\paragraph{Part 2, making a program for the solar system.}
We will use so-called astronomical units when rewriting our equations. 
Using astronomical units (AU as abbreviation)it means that 
one astronomical unit of length, known as 1 AU, is the average distance between the Sun and Earth, that is
$1$ AU = $1.5\times 10^{11}$ m.  It can also be convenient to use years instead of seconds since years match
better the time evolution of the solar system. The mass of the Sun is $M_{\mathrm{sun}}=M_{\odot}=2\times 10^{30}$ kg. The masses of all relevant planets and their distances from the sun are listed in the table here in kg and AU.


\begin{quote}
\begin{tabular}{ccc}
\hline
\multicolumn{1}{c}{ Planet } & \multicolumn{1}{c}{ Mass in kg } & \multicolumn{1}{c}{ Distance to  sun in AU } \\
\hline
Earth   & $M_{\mathrm{Earth}}=6\times 10^{24}$ kg     & 1AU                    \\
Jupiter & $M_{\mathrm{Jupiter}}=1.9\times 10^{27}$ kg & 5.20 AU                \\
Mars    & $M_{\mathrm{Mars}}=6.6\times 10^{23}$ kg    & 1.52 AU                \\
Venus   & $M_{\mathrm{Venus}}=4.9\times 10^{24}$ kg   & 0.72 AU                \\
Saturn  & $M_{\mathrm{Saturn}}=5.5\times 10^{26}$ kg  & 9.54 AU                \\
Mercury & $M_{\mathrm{Mercury}}=3.3\times 10^{23}$ kg & 0.39 AU                \\
Uranus  & $M_{\mathrm{Uranus}}=8.8\times 10^{25}$ kg  & 19.19 AU               \\
Neptun  & $M_{\mathrm{Neptun}}=1.03\times 10^{26}$ kg & 30.06 AU               \\
Pluto   & $M_{\mathrm{Pluto}}=1.31\times 10^{22}$ kg  & 39.53 AU               \\
\hline
\end{tabular}
\end{quote}

\noindent
Pluto is no longer considered  a planet, but we add it here for historical reasons. It is optional in this project to include Pluto and eventual moons. 

In setting up the equations we can limit ourselves to a co-planar motion and use only the $x$ and $y$ coordinates. But you should feel free to extend your equations to three dimensions, it is not very difficult and the data from NASA are all in three dimensions.

\href{{http://www.nasa.gov/index.html}}{NASA} has an excellent site at \href{{http://ssd.jpl.nasa.gov/horizons.cgi#top}}{\nolinkurl{http://ssd.jpl.nasa.gov/horizons.cgi\#top}}.
From there you can extract initial conditions in order to start your differential equation solver.
At the above website you need to change from \textbf{OBSERVER} to \textbf{VECTOR} and then write in the planet you are interested in.
The generated data contain the $x$, $y$ and $z$ values as well as their corresponding velocities. The velocities are in units of AU per day.
Alternatively they can be obtained in terms of km and km/s. 


Finally, using our Verlet solver, we carry out a real three-body calculation where all three systems, 
the Earth, Jupiter and the Sun are in motion. To do this, choose the center-of-mass position of the three-body system as 
the origin rather than the position of the sun. Give the Sun an initial velocity which makes the total momentum of the system exactly zero (the center-of-mass will remain fixed). Compare these results with those from the previous exercise and comment your results. Extend your program to include all planets in the solar system (if you have time, you can also include the various moons, but it is not required) and discuss your results. Use the above NASA link  to set up the initial positions and velocities for all planets. 



\paragraph{Part 3, Bonus part (30pt), testing conservation of angular momentum.}
Check also for the case of a circular orbit that both the kinetic and the potential energies are conserved.
Check also if the  angular momentum is conserved. Explain why these quantities
should be conserved.





\paragraph{Exercise 3 (20pt) Inverse-square force.}
Consider again the same effective potential as in 2. This leads to an attractive inverse-square-law force, $F=-\alpha/r^2$. Consider a particle of mass $m$ with angular momentum $L$. Taylor sections 8.4-8.7 are relevant background material.  See also the harmonic oscillator potential from hw7. The equation of motion for the radial degrees of freedom is (see also hw7) in the center-of-mass frame in two dimensions with $x=r\cos{(\phi)}$ and $y=r\sin{(\phi)}$ and
$r\in [0,\infty)$, $\phi\in [0,2\pi]$ and $r=\sqrt{x^2+y^2}$ are given by
\[
\ddot{r}=-\frac{1}{m}\frac{dV(r)}{dr}+r\dot{\phi}^2,
\]
and
\[
\dot{\phi}=\frac{L}{m r^2}.
\]
Here $V(r)$ is any central force which depends only on the relative coordinate.


\begin{itemize}
\item 3a (5pt)  Find the radius of a circular orbit by solving for the position of the minimum of the effective potential. 

\item 3b (5pt) At the minimum, the radial velocity is zero and it is only the \href{{https://en.wikipedia.org/wiki/Centripetal_force}}{centripetal velocity} which is nonzero. This implies that $\ddot{r}=0$.  What is the angular frequency, $\dot{\theta}$, of the orbit? Solve this by setting $\ddot{r}=0=F/m+\dot{\theta}^2r$.

\item 3c (5pt) Find the effective spring constant for the particle at the minimum.

\item 3d (5pt) What is the angular frequency for small vibrations about the minimum? How does this compare with the answer to (3b)?
\end{itemize}

\noindent

% ------------------- end of main content ---------------

% #ifdef PREAMBLE
\end{document}
% #endif

